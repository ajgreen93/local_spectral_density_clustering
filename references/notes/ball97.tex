\documentclass{article}
\usepackage{amsmath}
\usepackage{amsfonts, amsthm, amssymb}
\usepackage{bm}
\usepackage{graphicx}
\usepackage[colorlinks]{hyperref}
\usepackage[parfill]{parskip}
\usepackage{algpseudocode}
\usepackage{algorithm}
\usepackage{enumerate}

\usepackage{natbib}
\renewcommand{\bibname}{REFERENCES}
\renewcommand{\bibsection}{\subsubsection*{\bibname}}

\makeatletter
\newcommand{\leqnomode}{\tagsleft@true}
\newcommand{\reqnomode}{\tagsleft@false}
\makeatother

\newcommand{\eqdist}{\ensuremath{\stackrel{d}{=}}}
\newcommand{\Graph}{\mathcal{G}}
\newcommand{\Reals}{\mathbb{R}}
\newcommand{\Identity}{\mathbb{I}}
\newcommand{\distiid}{\overset{\text{i.i.d}}{\sim}}
\newcommand{\convprob}{\overset{p}{\to}}
\newcommand{\convdist}{\overset{w}{\to}}
\newcommand{\Expect}[1]{\mathbb{E}\left[ #1 \right]}
\newcommand{\Risk}[2][P]{\mathcal{R}_{#1}\left[ #2 \right]}
\newcommand{\Var}[1]{\mathrm{Var}\left( #1 \right)}
\newcommand{\Prob}[1]{\mathbb{P}\left( #1 \right)}
\newcommand{\iset}{\mathbf{i}}
\newcommand{\jset}{\mathbf{j}}
\newcommand{\myexp}[1]{\exp \{ #1 \}}
\newcommand{\norm}[1]{\left\lVert#1\right\rVert}
\newcommand{\dotp}[2]{\langle #1 , #2 \rangle}
\newcommand{\abs}[1]{\left \lvert #1 \right \rvert}
\newcommand{\restr}[2]{\ensuremath{\left.#1\right|_{#2}}}
\newcommand{\defeq}{\overset{\mathrm{def}}{=}}
\newcommand{\convweak}{\overset{w}{\rightharpoonup}}
\newcommand{\dive}{\mathrm{div}}

\newcommand{\emC}{C_n}
\newcommand{\emCpr}{C'_n}
\newcommand{\emCthick}{C^{\sigma}_n}
\newcommand{\emCprthick}{C'^{\sigma}_n}
\newcommand{\emS}{S^{\sigma}_n}
\newcommand{\estC}{\widehat{C}_n}
\newcommand{\hC}{\hat{C^{\sigma}_n}}
\newcommand{\vol}{\mathrm{vol}}
\newcommand{\vr}{\mathrm{vr}}
\newcommand{\Bal}{\textrm{Bal}}
\newcommand{\Cut}{\textrm{Cut}}
\newcommand{\Ind}{\textrm{Ind}}
\newcommand{\set}[1]{\left\{#1\right\}}
\newcommand{\seq}[1]{\set{#1}_{n \in \N}}
\newcommand{\Perp}{\perp \! \! \! \perp}


\newcommand{\Linv}{L^{\dagger}}
\newcommand{\tr}{\text{tr}}
\newcommand{\h}{\textbf{h}}
% \newcommand{\l}{\ell}
\newcommand{\x}{\textbf{x}}
\newcommand{\y}{\textbf{y}}
\newcommand{\bl}{\bm{\ell}}
\newcommand{\bnu}{\bm{\nu}}
\newcommand{\Lx}{\mathcal{L}_X}
\newcommand{\Ly}{\mathcal{L}_Y}
\newcommand{\wt}[1]{\widetilde{#1}}
\DeclareMathOperator*{\argmin}{argmin}


\newcommand{\emG}{\mathbb{G}_n}
\newcommand{\A}{\mathcal{A}}
\newcommand{\F}{\mathcal{F}}
\newcommand{\G}{\mathcal{G}}
\newcommand{\X}{\mathcal{X}}
\newcommand{\Rd}{\Reals^d}
\newcommand{\N}{\mathbb{N}}
\newcommand{\E}{\mathcal{E}}

\newtheoremstyle{alden}
{6pt} % Space above
{6pt} % Space below
{} % Body font
{} % Indent amount
{\bfseries} % Theorem head font
{.} % Punctuation after theorem head
{.5em} % Space after theorem head
{} % Theorem head spec (can be left empty, meaning `normal')

\theoremstyle{alden} 
\newtheorem{definition}{Definition}[section]

\newtheoremstyle{aldenthm}
{6pt} % Space above
{6pt} % Space below
{\itshape} % Body font
{} % Indent amount
{\bfseries} % Theorem head font
{.} % Punctuation after theorem head
{.5em} % Space after theorem head
{} % Theorem head spec (can be left empty, meaning `normal')

\theoremstyle{aldenthm}
\newtheorem{theorem}{Theorem}
\newtheorem{conjecture}{Conjecture}
\newtheorem{lemma}{Lemma}
\newtheorem{example}{Example}
\newtheorem{corollary}{Corollary}
\newtheorem{proposition}{Proposition}
\newtheorem{assumption}{Assumption}

\theoremstyle{remark}
\newtheorem{remark}{Remark}

\begin{document}
	
\title{Notes on `An Elementary Introduction to Modern Convex Geometry'}
\author{Alden Green}
\date{\today}
\maketitle

\section{Setup}

Let $K$ be a convex body in $\Rd$. Denote the ratio of the $d$-dimensional unit ball by $\nu_d$. Let $Q = [-1,1]^d$ be the unit cube in $\Rd$. 

Assuming symmetry of $K$ will often greatly ease proofs.

\begin{definition}
	We say $K$ is \textit{(centrally) symmetric} if $-x \in K$ whenever $x \in K$. This will also imply that $K$ is the unit ball of some norm $\norm{\cdot}_K$ on $\Rd$:
	\begin{equation*}
	K = \set{x: \norm{x}_{K} \leq 1}
	\end{equation*}
\end{definition}

\paragraph{Volume ratio.}

The volume ratio is used to prove reverse isoperimetric inequalities of the form we want. To define it, we first recall the concept of an \textit{ellipsoid}: given $(e_j)_{j=1}^{d}$ an orthonormal basis of $\Rd$ and $(a_j)$ positive numbers, the ellipsoid
\begin{equation*}
\set{x: \sum_{i = 1}^{d} \frac{\dotp{x}{e_j}^2}{\alpha_j^2} \leq 1}
\end{equation*}
has volume $\nu_d \prod \alpha_j$.
\begin{definition}
	Let $K$ be a convex body in $\Rd$. The \textit{volume ratio} of $K$ is
	\begin{equation*}
	\vr(K) := \left(\frac{\vol(K)}{\vol(\E)}\right)^{1/d}
	\end{equation*}
	where $\E$ is the ellipsoid of maximal volume in $K$. 
\end{definition}

Let $B_2^d$ be the $d$-dimensional unit Euclidean ball.
\begin{definition}
	The \textit{surface area} of a convex body $K \in \Rd$, $\partial K$, is defined by
	\begin{equation*}
	\vol(\partial K) = \lim_{\epsilon \to 0} \frac{\vol(K + \epsilon B_2^d) - \vol(K)}{\epsilon}
	\end{equation*}
\end{definition}

\section{Convolutions and Volume Ratios: The Reverse Isoperimetric Problem}

\begin{theorem}
	Let $K$ be a convex body and $T$ a regular solid simplex in $\Rd$. Then there is an affine image of $K$ whose volume is the same as that of $T$ and whose surface area is no larger than that of $T$. 
\end{theorem}

We will prove a related, by far simpler, result.

\begin{theorem}
	\label{thm: reverse_isoperimetry_symmetric_bodies}
	Let $K$ be a symmetric convex body, and $T$ the $d$-dimensional unit cube. Then there is an affine image of $K$ whose volume is the same as that of $T$ and whose surface area is no larger than that of $T$.
\end{theorem}

The result follows from a result showing that the unit cube $Q = [-1,1]^d$ has the largest volume ratio among convex bodies.
\begin{theorem}
	\label{thm: volume_ratio_symmetric_bodies}
	Among symmetric convex bodies the cube has largest volume ratio.
\end{theorem}

\begin{proof}[Proof of Theorem \ref{thm: reverse_isoperimetry_symmetric_bodies}]
	We begin by recall that for $Q$
	\begin{equation*}
	\vol(\partial Q) = 2d \vol(Q)^{(d-1)/d},
	\end{equation*}
	so we wish to show that any other convex body $K$ has an affine image $\wt{K}$ such that
	\begin{equation*}
	\vol(\partial \wt{K}) \leq 2d ~ \vol(\wt{K})^{(d-1)/d}.
	\end{equation*}
	
	Choose $\wt{K}$ so that its maximal volume ellipsoid is $B_2^n$, the Euclidean ball of radius $1$. Then, by Theorem \ref{thm: volume_ratio_symmetric_bodies}, we have that
	\begin{align*}
	\vr(\wt{K}) \leq \vr(Q)
	\end{align*}
	and since the maximal volume ellipsoid is the same for both $\wt{K}$ and $Q$, this implies $\vol(\wt{K}) \leq \vol(Q) = 2^d$.
	
	Now, since $\wt{K} \supset B_2^d$, the surface area can be upper bounded
	\begin{align*}
	\lim_{\epsilon \to 0} \frac{\vol(\wt{K} + \epsilon \wt{K}) - \vol(\wt{K})}{\epsilon} & = \vol(\wt{K}) \lim_{\epsilon \to 0} \frac{(1 + \epsilon)^d - 1}{\epsilon} \\
	& = d ~ \vol(\wt{K}) = d ~ \vol(\wt{K})^{1/d} \vol(\wt{K})^{(d-1)/d}  \\
	& \leq 2d \vol(\wt{K})^{(d-1)/d}.
	\end{align*}
\end{proof}

\end{document}