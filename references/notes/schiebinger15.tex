\documentclass{article}
\usepackage{amsmath}
\usepackage{amsfonts, amsthm, amssymb}
\usepackage{bm}
\usepackage{graphicx}
\usepackage[colorlinks]{hyperref}
\usepackage[parfill]{parskip}
\usepackage{algpseudocode}
\usepackage{algorithm}
\usepackage{enumerate}

\usepackage{natbib}
\renewcommand{\bibname}{REFERENCES}
\renewcommand{\bibsection}{\subsubsection*{\bibname}}

\makeatletter
\newcommand{\leqnomode}{\tagsleft@true}
\newcommand{\reqnomode}{\tagsleft@false}
\makeatother

\newcommand{\eqdist}{\ensuremath{\stackrel{d}{=}}}
\newcommand{\Graph}{\mathcal{G}}
\newcommand{\Reals}{\mathbb{R}}
\newcommand{\Identity}{\mathbb{I}}
\newcommand{\distiid}{\overset{\text{i.i.d}}{\sim}}
\newcommand{\convprob}{\overset{p}{\to}}
\newcommand{\convdist}{\overset{w}{\to}}
\newcommand{\Expect}[1]{\mathbb{E}\left[ #1 \right]}
\newcommand{\Risk}[2][P]{\mathcal{R}_{#1}\left[ #2 \right]}
\newcommand{\Var}[1]{\mathrm{Var}\left( #1 \right)}
\newcommand{\Prob}[1]{\mathbb{P}\left( #1 \right)}
\newcommand{\iset}{\mathbf{i}}
\newcommand{\jset}{\mathbf{j}}
\newcommand{\myexp}[1]{\exp \{ #1 \}}
\newcommand{\norm}[1]{\left\lVert#1\right\rVert}
\newcommand{\dotp}[2]{\langle #1 , #2 \rangle}
\newcommand{\abs}[1]{\left \lvert #1 \right \rvert}
\newcommand{\restr}[2]{\ensuremath{\left.#1\right|_{#2}}}
\newcommand{\defeq}{\overset{\mathrm{def}}{=}}
\newcommand{\convweak}{\overset{w}{\rightharpoonup}}
\newcommand{\dive}{\mathrm{div}}

\newcommand{\emC}{C_n}
\newcommand{\emCpr}{C'_n}
\newcommand{\emCthick}{C^{\sigma}_n}
\newcommand{\emCprthick}{C'^{\sigma}_n}
\newcommand{\emS}{S^{\sigma}_n}
\newcommand{\estC}{\widehat{C}_n}
\newcommand{\hC}{\hat{C^{\sigma}_n}}
\newcommand{\vol}{\mathrm{vol}}
\newcommand{\vr}{\mathrm{vr}}
\newcommand{\Bal}{\textrm{Bal}}
\newcommand{\Cut}{\textrm{Cut}}
\newcommand{\Ind}{\textrm{Ind}}
\newcommand{\set}[1]{\left\{#1\right\}}
\newcommand{\seq}[1]{\set{#1}_{n \in \N}}
\newcommand{\Perp}{\perp \! \! \! \perp}


\newcommand{\Linv}{L^{\dagger}}
\newcommand{\tr}{\text{tr}}
\newcommand{\h}{\textbf{h}}
% \newcommand{\l}{\ell}
\newcommand{\x}{\textbf{x}}
\newcommand{\y}{\textbf{y}}
\newcommand{\bl}{\bm{\ell}}
\newcommand{\bnu}{\bm{\nu}}
\newcommand{\Lx}{\mathcal{L}_X}
\newcommand{\Ly}{\mathcal{L}_Y}
\newcommand{\wt}[1]{\widetilde{#1}}
\DeclareMathOperator*{\argmin}{argmin}


\newcommand{\emG}{\mathbb{G}_n}
\newcommand{\A}{\mathcal{A}}
\newcommand{\F}{\mathcal{F}}
\newcommand{\G}{\mathcal{G}}
\newcommand{\X}{\mathcal{X}}
\newcommand{\Rd}{\Reals^d}
\newcommand{\N}{\mathbb{N}}
\newcommand{\E}{\mathcal{E}}
\newcommand{\PP}{\mathbb{P}}

\newtheoremstyle{alden}
{6pt} % Space above
{6pt} % Space below
{} % Body font
{} % Indent amount
{\bfseries} % Theorem head font
{.} % Punctuation after theorem head
{.5em} % Space after theorem head
{} % Theorem head spec (can be left empty, meaning `normal')

\theoremstyle{alden} 
\newtheorem{definition}{Definition}[section]

\newtheoremstyle{aldenthm}
{6pt} % Space above
{6pt} % Space below
{\itshape} % Body font
{} % Indent amount
{\bfseries} % Theorem head font
{.} % Punctuation after theorem head
{.5em} % Space after theorem head
{} % Theorem head spec (can be left empty, meaning `normal')

\theoremstyle{aldenthm}
\newtheorem{theorem}{Theorem}
\newtheorem{conjecture}{Conjecture}
\newtheorem{lemma}{Lemma}
\newtheorem{example}{Example}
\newtheorem{corollary}{Corollary}
\newtheorem{proposition}{Proposition}
\newtheorem{assumption}{Assumption}

\theoremstyle{remark}
\newtheorem{remark}{Remark}

\begin{document}
	
\title{Notes on `The Geometry of Kernelized Spectral Clustering'}
\author{Alden Green}
\date{\today}
\maketitle

\section{Analysis of normalized Laplacian embedding.}

For a given set of distributions $\mathbb{P}_1, \ldots, \mathbb{P}_m$ and weights $w_1, \ldots, w_K$ in the probability simplex, define the \textbf{mixture distribution}
\begin{equation*}
\bar{\mathbb{P}} := \sum_{m = 1}^{K} w_m \mathbb{P}_m.
\end{equation*}
Given a non-negative, continuous, symmetric kernel function $k(x,y)$ and a distribution $\mathbb{P}$ we introduce the \textbf{square root kernelized density} as the function $q \in L^2(\mathbb{P})$ given by
\begin{equation*}
q(x) := \sqrt{\int k(x,y) d\mathbb{P}(y)}.
\end{equation*}
In particular, we denote the square root kernelized density of the mixture distribution $\bar{P}$ by $\bar{q}$ and those of the mixture components $\set{\PP_m}_{m=1}^{K}$ by $\set{q_m}_{m = 1}^{K}$. 

\paragraph{Normalized densities and the coupling parameter.}

Because we typically deal with the matrix $L = D^{-1/2}AD^{-1/2}$ when performing spectral embedding, it is useful to define analogous continuum operators. In particular, we define the \textbf{normalized kernel function} $\bar{k}$ to be
\begin{equation*}
\bar{k}(x,y) = \frac{1}{\bar{q}(x)} k(x,y) \frac{1}{\bar{q}(y)}
\end{equation*}
and the normalized kernel for mixture component $k_m$ to be
\begin{equation*}
k_m(x,y) := \frac{k(x,y)}{q_m(x) q_m(y)} ~~ \text{for $m = 1,\ldots,K$.}
\end{equation*}

Now, we introduce the \textbf{coupling parameter}
\begin{equation*}
\mathcal{C}(\bar{P}) := \max_{m = 1,\ldots,K} \norm{k_m - w_m \bar{k}}_{\PP_m \bigotimes \PP_m}^2
\end{equation*}

This controls the maximum average connection between points generated by any one mixture component $\PP_i$ and those generated by a different mixture component $\PP_{j}$. 

\end{document}