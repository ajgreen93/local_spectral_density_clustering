\documentclass{article}
\usepackage{amsmath}
\usepackage{amsfonts, amsthm, amssymb}
\usepackage{bm}
\usepackage{graphicx}
\usepackage[colorlinks]{hyperref}
\usepackage[parfill]{parskip}
\usepackage{algpseudocode}
\usepackage{algorithm}
\usepackage{enumerate}
\usepackage{fullpage}

\usepackage{natbib}
\renewcommand{\bibname}{REFERENCES}
\renewcommand{\bibsection}{\subsubsection*{\bibname}}

\makeatletter
\newcommand{\leqnomode}{\tagsleft@true}
\newcommand{\reqnomode}{\tagsleft@false}
\makeatother

\newcommand{\eqdist}{\ensuremath{\stackrel{d}{=}}}
\newcommand{\Graph}{\mathcal{G}}
\newcommand{\Reals}{\mathbb{R}}
\newcommand{\Identity}{\mathbb{I}}
\newcommand{\distiid}{\overset{\text{i.i.d}}{\sim}}
\newcommand{\convprob}{\overset{p}{\to}}
\newcommand{\convdist}{\overset{w}{\to}}
\newcommand{\Expect}[1]{\mathbb{E}\left[ #1 \right]}
\newcommand{\Risk}[2][P]{\mathcal{R}_{#1}\left[ #2 \right]}
\newcommand{\Var}[1]{\mathrm{Var}\left( #1 \right)}
\newcommand{\Prob}[1]{\mathbb{P}\left( #1 \right)}
\newcommand{\iset}{\mathbf{i}}
\newcommand{\jset}{\mathbf{j}}
\newcommand{\myexp}[1]{\exp \{ #1 \}}
\newcommand{\norm}[1]{\left\lVert#1\right\rVert}
\newcommand{\dotp}[2]{\langle #1 , #2 \rangle}
\newcommand{\abs}[1]{\left \lvert #1 \right \rvert}
\newcommand{\restr}[2]{\ensuremath{\left.#1\right|_{#2}}}
\newcommand{\defeq}{\overset{\mathrm{def}}{=}}
\newcommand{\convweak}{\overset{w}{\rightharpoonup}}
\newcommand{\dive}{\mathrm{div}}
\newcommand{\Bin}{\mathrm{Bin}}

\newcommand{\emC}{C_n}
\newcommand{\emCpr}{C'_n}
\newcommand{\emCthick}{C^{\sigma}_n}
\newcommand{\emCprthick}{C'^{\sigma}_n}
\newcommand{\emS}{S^{\sigma}_n}
\newcommand{\estC}{\widehat{C}_n}
\newcommand{\hC}{\hat{C^{\sigma}_n}}
\newcommand{\Bal}{\textrm{Bal}}
\newcommand{\Cut}{\textrm{Cut}}
\newcommand{\Ind}{\textrm{Ind}}
\newcommand{\set}[1]{\left\{#1\right\}}
\newcommand{\seq}[1]{\set{#1}_{n \in \N}}
\newcommand{\Perp}{\perp \! \! \! \perp}
\newcommand{\Naturals}{\mathbb{N}}
\newcommand{\dist}{\mathrm{dist}}

\newcommand\independent{\protect\mathpalette{\protect\independenT}{\perp}}
\def\independenT#1#2{\mathrel{\rlap{$#1#2$}\mkern2mu{#1#2}}}


\newcommand{\Linv}{L^{\dagger}}
\newcommand{\tr}{\text{tr}}
\newcommand{\h}{\textbf{h}}
% \newcommand{\l}{\ell}
\newcommand{\x}{\textbf{x}}
\newcommand{\y}{\textbf{y}}
\newcommand{\bl}{\bm{\ell}}
\newcommand{\bnu}{\bm{\nu}}
\newcommand{\Lx}{\mathcal{L}_X}
\newcommand{\Ly}{\mathcal{L}_Y}
\DeclareMathOperator*{\argmin}{argmin}


\newcommand{\emG}{\mathbb{G}_n}
\newcommand{\A}{\mathcal{A}}
\newcommand{\F}{\mathcal{F}}
\newcommand{\G}{\mathcal{G}}
\newcommand{\X}{\mathcal{X}}
\newcommand{\Rd}{\Reals^d}
\newcommand{\N}{\mathbb{N}}
\newcommand{\E}{\mathcal{E}}

%%% Matrix related notation
\newcommand{\Xbf}{\mathbf{X}}
\newcommand{\Ybf}{\mathbf{Y}}
\newcommand{\Zbf}{\mathbf{Z}}
\newcommand{\Abf}{\mathbf{A}}
\newcommand{\Dbf}{\mathbf{D}}
\newcommand{\Wbf}{\mathbf{W}}
\newcommand{\Lbf}{\mathbf{L}}
\newcommand{\Ibf}{\mathbf{I}}
\newcommand{\Bbf}{\mathbf{B}}

%%% Vector related notation
\newcommand{\lbf}{\bm{\ell}}
\newcommand{\fbf}{\mathbf{f}}

%%% Set related notation
\newcommand{\Cset}{\mathcal{C}}
\newcommand{\Dset}{\mathcal{D}}
\newcommand{\Aset}{\mathcal{A}}
\newcommand{\Wset}{\mathcal{W}}
\newcommand{\Sset}{\mathcal{S}}

\newcommand{\Csig}{\Cset_{\sigma}}

%%% Distribution related notation
\newcommand{\Pbb}{\mathbb{P}}
\newcommand{\Qbb}{\mathbb{Q}}
% \newcommand{\Pr}{\mathrm{Pr}}}

%%% Functionals
\newcommand{\1}{\mathbf{1}}

%%% Functions over graphs
\newcommand{\cut}{\mathrm{cut}}
\newcommand{\vol}{\mathrm{vol}}


\newtheoremstyle{alden}
{6pt} % Space above
{6pt} % Space below
{} % Body font
{} % Indent amount
{\bfseries} % Theorem head font
{.} % Punctuation after theorem head
{.5em} % Space after theorem head
{} % Theorem head spec (can be left empty, meaning `normal')

\theoremstyle{alden} 
\newtheorem{definition}{Definition}[section]

\newtheoremstyle{aldenthm}
{6pt} % Space above
{6pt} % Space below
{\itshape} % Body font
{} % Indent amount
{\bfseries} % Theorem head font
{.} % Punctuation after theorem head
{.5em} % Space after theorem head
{} % Theorem head spec (can be left empty, meaning `normal')

\theoremstyle{aldenthm}
\newtheorem{theorem}{Theorem}
\newtheorem{conjecture}{Conjecture}
\newtheorem{lemma}{Lemma}
\newtheorem{example}{Example}
\newtheorem{corollary}{Corollary}
\newtheorem{proposition}{Proposition}
\newtheorem{assumption}{Assumption}

\theoremstyle{remark}
\newtheorem{remark}{Remark}

\begin{document}
	
\title{Notes on ``A nonlocal anisotropic model for phase transitions (Alberti 97)''}
\author{Alden Green}
\date{\today}
\maketitle

Let $\Csig = \Cset + B(0,\sigma)$ for some $\Cset \subset \Reals^d$. Define the nonlocal functional $F_{r}(v, \Csig)$ to be
\begin{equation*}
F_r(v, \Csig) := \frac{1}{4r} \iint_{\Csig} \frac{\1(\norm{x - y} \leq r)}{r^d} (v(x) - v(y))^2 dx dy + \frac{1}{r} \int_{\Csig} W(v(x)) dx
\end{equation*}
where $W:\Reals \to \Reals$ is a continuous function which vanishes at $\pm 1$ only and has at least \textcolor{red}{linear growth at infinity}.

\begin{theorem}[Theorem 3.1]
	Let $r_n \to 0$, $v_{n}: \Csig \to [-1,1]$ be such that $F_{r_n}(v_n, \Csig)$ is bounded. Then the sequence $\seq{v_n}$ is precompact in $L^1(\Csig)$. 
\end{theorem}
\begin{proof}
	Throughout, let  $J_r(\norm{x - y}) := \frac{\1(\norm{x - y} \leq r)}{r^d}$. 
	\item
	
	\textbf{Step 1:} At first, we will assume each $v_n$ takes values in $\pm1$ only. 
	
	Extend each function $v_n$ to $1$ in $\Reals^d \setminus \Csig$, and observe that our hypothesis implies (\textcolor{red}{omitting the proof})
	\begin{equation}
	\label{eqn: nonlocal_functional_bound}
	\iint_{\Reals^d \times \Reals^d} J_r(\norm{x - y}) \abs{v_n(x) - v_n(y)} dy dx = \mathcal{O}(r)
	\end{equation} 
	Moreover, it is an easily verifiable fact that for every non-negative $g \in L^1(\Reals^d)$,
	\begin{equation*}
	\iint_{\Reals^d \times \Reals^d} (g \ast g)(x - y) \abs{u(x) - u(y)} dy dx \leq 2 \norm{g}_1 \iint_{\Reals^d \times \Reals^d} g(x - y) \abs{u(x) - u(y)} dx dy
	\end{equation*}
	and in combination with \eqref{eqn: nonlocal_functional_bound} this implies
	\begin{equation}
	\label{eqn: nonlocal_functional_bound_1}
	\iint_{\Reals^d \times \Reals^d} (J_r \ast J_r)(x - y) \abs{u(x) - u(y)} dy dx = \mathcal{O}(r)
	\end{equation}
	
	Since $J \ast J$ is a nonnegative continuous function, we may find a nonnegative smooth function $\varphi$ with compact support such that
	\begin{equation}
	\label{eqn: varphi}
	\varphi \leq J_r \ast J_r,~ \abs{\nabla \varphi} \leq J \ast J.
	\end{equation}
	Letting $c = \int_{\Reals^d} \varphi(y) dy$, for $y \in \Reals^d$ define
	\begin{equation*}
	\varphi_r(y) := \frac{1}{cr^d} \varphi\left(\frac{y}{r}\right),~ w_n(y) := \varphi_{r_n} \ast v_n(y)
	\end{equation*}
	We claim that $\seq{w_n}$ is asymptotically equivalent to $\seq{v_n}$ in $L^1(\Reals^d)$ and that the gradients $\nabla{w_n}$ are uniformly bounded in $L^1(\Reals^d)$. Suppose this were true; then by Theorem \ref{thm: poincare}
	\begin{equation*}
	\norm{w_n}_{L^1(\Csig)} \leq \norm{w_n}_{L^{p^{\star}}(\Csig)} \leq \norm{w_n}_{L^{p*}(\Reals^d)} \leq C \norm{\nabla w_n}_{L^1(\Reals^d)} < \infty
	\end{equation*}
	and so $w_n \in W_{1,1}(\Csig)$. By the Rellich-Kondrachov Compactness Theorem (Theorem \ref{thm: arzela_ascoli}), $\seq{w_n}$ is precompact in $L^1(\Csig)$.
	
	It remains to prove the claim. We have
	\begin{align*}
	\int_{\Reals^d} \abs{w_n(x) - v_n(x)} dx & = \int_{\Reals^d} \abs{ \int_{\Reals^d} \varphi_{r_n}(v_n(y) - v_n(x)) dy} dx \\
	& \leq \iint_{\Reals^d \times \Reals^d} \abs{\varphi_{r_n}(\norm{x - y})} \abs{v_n(y) - v_n(x)} dy dx \\
	& \leq \frac{1}{c} \iint_{\Reals^d \times \Reals^d} (J \ast J)(\norm{x - y}) \abs{v_n(y) - v_n(x)} dy dx = \mathcal{O}(r).
	\end{align*}
	where the last equality follows from \eqref{eqn: nonlocal_functional_bound_1}. 
	
	Moreover, 
	\begin{align*}
	\int_{\Reals^d} \abs{\nabla w_n} dx & = \int_{\Reals^d} \abs{\int_{\Reals^d} \nabla \varphi_{r_n}(x - y) u_n(y) dy} dx \\
	& \overset{(i)}{=} \int_{\Reals^d} \abs{\int_{\Reals^d} \nabla \varphi_{r_n}(x - y) (u_n(y) - u_n(x)) dy} dx \\
	& \leq \iint_{\Reals^d \times \Reals^d} \abs{\nabla \varphi_{r_n}(x - y)} (u_n(y) - u_n(x)) dy dx \\
	& \overset{(ii)}{\leq} \frac{1}{cr}  \iint_{\Reals^d \times \Reals^d} (J_r \ast J_r)(x-y) (u_n(y) - u_n(x)) dy dx \\
	& \overset{(iii)}{=} \mathcal{O}(1),
	\end{align*}
	where $(i)$ follows from $\int_{\Reals^d} \nabla \varphi_{r_n}(x - y) dy = 0$, in the light of the fact that $\varphi$ has compact support; $(ii)$ follows from \eqref{eqn: varphi}; and $(iii)$ from \eqref{eqn: nonlocal_functional_bound_1}.
\end{proof}

\section{Additional Information}

For a function $J$, we write $J_r(x) := \frac{1}{r^d} J\left(\frac{x}{r}\right)$. 

The mollification of a function $f:\Reals^d \to \Reals$ by a mollifier $J_{r}$ is written $f_r: \Reals^d \to \Reals$, and is given by
\begin{equation*}
f_r(x) := J_r \ast f = \int_{\Reals^d} J_r(x - y) f(y) dy.
\end{equation*}

\begin{theorem}[Arzela-Ascoli]
	\label{thm: arzela_ascoli}
	Suppose $\seq{f_n}$ is a sequence of real-valued functions defined on $\Reals^d$ such that for all $k \in \Naturals$, $x \in \Reals^d$,
	\begin{equation*}
	\abs{f_n(x)} \leq M
	\end{equation*}
	for some constant $M$, and that moreover the functions $f_n$ are \emph{uniformly equicontinuous}, meaning that for each $\epsilon > 0$ there exists $\delta > 0$ such that 
	\begin{equation*}
	\abs{x - y} < \delta~ \Longrightarrow~ \abs{f_n(x) - f(n)}. \tag{for all $x,y \in \Reals^d$}
	\end{equation*}
	
	Then, there exists a subsequence $(f_{n_k})_{k \in \Naturals}$ and a continuous function $f$ such that
	\begin{equation*}
	f_{n_k} \to f~~ \textrm{uniformly on compact subsets of}~ \Reals^d
	\end{equation*}
\end{theorem}

\begin{definition}
	Let $X$ and $Y$ be Banach spaces, $X \subset Y$. We say that $X$ is \emph{compactly embedded} in $Y$, written
	\begin{equation*}
	X \subset \subset Y
	\end{equation*}
	if 
	\begin{enumerate}[(i)]
		\item $\norm{u}_Y \leq C \norm{u}_X$ for some constant $C$, and
		\item each bounded sequence in $X$ is precompact in $Y$. 
	\end{enumerate}
\end{definition}

Recall that the Sobolev conjugate is $p^*= \frac{pn}{n - p}$.

\begin{theorem}[Rellich-Kondrachov Compactness Theorem]
	\label{thm: compact_embedding}
	Assume $U$ is a bounded open subset of $\Reals^d$ and $\partial U$ is $C^1$. Suppose $1 \leq p < d$. Then
	\begin{equation*}
	W^{1,p}(U) \subset \subset L^q(U)
	\end{equation*}
	for $1 \leq q < p^{\star}$. In particular, since $p^{*} > p$, we have
	\begin{equation*}
	W^{1,p}(U) \subset \subset L^p(U)
	\end{equation*}
	for all $1 \leq p$.
\end{theorem}

\begin{theorem}[Gagliardo-Nirenberg-Sobolev]
	\label{thm: poincare}
	Assume $1 \leq p < d$. There exists constant $C$, depending only on $p$ and $d$, such that
	\begin{equation*}
	\norm{u}_{L^{p^*}(\Reals^d)} \leq C \norm{D u}_{L^{p}(\Reals)}
	\end{equation*}
	for all $u \in C_c^1(\Reals^d)$. 
\end{theorem}
\end{document}