\documentclass{article}
\usepackage{amsmath}
\usepackage{amsfonts, amsthm, amssymb}
\usepackage{bm}
\usepackage{graphicx}
\usepackage[colorlinks]{hyperref}
\usepackage[parfill]{parskip}
\usepackage{algpseudocode}
\usepackage{algorithm}
\usepackage{enumerate}

\usepackage{natbib}
\renewcommand{\bibname}{REFERENCES}
\renewcommand{\bibsection}{\subsubsection*{\bibname}}

\makeatletter
\newcommand{\leqnomode}{\tagsleft@true}
\newcommand{\reqnomode}{\tagsleft@false}
\makeatother

\newcommand{\eqdist}{\ensuremath{\stackrel{d}{=}}}
\newcommand{\Graph}{\mathcal{G}}
\newcommand{\Reals}{\mathbb{R}}
\newcommand{\Identity}{\mathbb{I}}
\newcommand{\distiid}{\overset{\text{i.i.d}}{\sim}}
\newcommand{\convprob}{\overset{p}{\to}}
\newcommand{\convdist}{\overset{w}{\to}}
\newcommand{\Expect}[1]{\mathbb{E}\left[ #1 \right]}
\newcommand{\Risk}[2][P]{\mathcal{R}_{#1}\left[ #2 \right]}
\newcommand{\Var}[1]{\mathrm{Var}\left( #1 \right)}
\newcommand{\Prob}[1]{\mathbb{P}\left( #1 \right)}
\newcommand{\iset}{\mathbf{i}}
\newcommand{\jset}{\mathbf{j}}
\newcommand{\myexp}[1]{\exp \{ #1 \}}
\newcommand{\norm}[1]{\left\lVert#1\right\rVert}
\newcommand{\dotp}[2]{\langle #1 , #2 \rangle}
\newcommand{\abs}[1]{\left \lvert #1 \right \rvert}
\newcommand{\restr}[2]{\ensuremath{\left.#1\right|_{#2}}}
\newcommand{\defeq}{\overset{\mathrm{def}}{=}}
\newcommand{\convweak}{\overset{w}{\rightharpoonup}}
\newcommand{\dive}{\mathrm{div}}
\newcommand{\Bin}{\mathrm{Bin}}

\newcommand{\emC}{C_n}
\newcommand{\emCpr}{C'_n}
\newcommand{\emCthick}{C^{\sigma}_n}
\newcommand{\emCprthick}{C'^{\sigma}_n}
\newcommand{\emS}{S^{\sigma}_n}
\newcommand{\estC}{\widehat{C}_n}
\newcommand{\hC}{\hat{C^{\sigma}_n}}
\newcommand{\vol}{\mathrm{vol}}
\newcommand{\Bal}{\textrm{Bal}}
\newcommand{\Cut}{\textrm{Cut}}
\newcommand{\Ind}{\textrm{Ind}}
\newcommand{\set}[1]{\left\{#1\right\}}
\newcommand{\seq}[1]{\set{#1}_{n \in \N}}
\newcommand{\Perp}{\perp \! \! \! \perp}
\newcommand{\Naturals}{\mathbb{N}}

\newcommand\independent{\protect\mathpalette{\protect\independenT}{\perp}}
\def\independenT#1#2{\mathrel{\rlap{$#1#2$}\mkern2mu{#1#2}}}


\newcommand{\Linv}{L^{\dagger}}
\newcommand{\tr}{\text{tr}}
\newcommand{\h}{\textbf{h}}
% \newcommand{\l}{\ell}
\newcommand{\x}{\textbf{x}}
\newcommand{\y}{\textbf{y}}
\newcommand{\bl}{\bm{\ell}}
\newcommand{\bnu}{\bm{\nu}}
\newcommand{\Lx}{\mathcal{L}_X}
\newcommand{\Ly}{\mathcal{L}_Y}
\DeclareMathOperator*{\argmin}{argmin}


\newcommand{\emG}{\mathbb{G}_n}
\newcommand{\A}{\mathcal{A}}
\newcommand{\F}{\mathcal{F}}
\newcommand{\G}{\mathcal{G}}
\newcommand{\X}{\mathcal{X}}
\newcommand{\Rd}{\Reals^d}
\newcommand{\N}{\mathbb{N}}
\newcommand{\E}{\mathcal{E}}

%%% Matrix related notation
\newcommand{\Xbf}{\mathbf{X}}
\newcommand{\Ybf}{\mathbf{Y}}
\newcommand{\Zbf}{\mathbf{Z}}
\newcommand{\Abf}{\mathbf{A}}
\newcommand{\Dbf}{\mathbf{D}}
\newcommand{\Wbf}{\mathbf{W}}
\newcommand{\Lbf}{\mathbf{L}}
\newcommand{\Ibf}{\mathbf{I}}
\newcommand{\Bbf}{\mathbf{B}}

%%% Vector related notation
\newcommand{\lbf}{\bm{\ell}}
\newcommand{\fbf}{\mathbf{f}}

%%% Set related notation
\newcommand{\Dset}{\mathcal{D}}
\newcommand{\Aset}{\mathcal{A}}
\newcommand{\Wset}{\mathcal{W}}

%%% Distribution related notation
\newcommand{\Pbb}{\mathbb{P}}
\newcommand{\Qbb}{\mathbb{Q}}
% \newcommand{\Pr}{\mathrm{Pr}}}

%%% Functionals
\newcommand{\1}{\mathbf{1}}


\newtheoremstyle{alden}
{6pt} % Space above
{6pt} % Space below
{} % Body font
{} % Indent amount
{\bfseries} % Theorem head font
{.} % Punctuation after theorem head
{.5em} % Space after theorem head
{} % Theorem head spec (can be left empty, meaning `normal')

\theoremstyle{alden} 
\newtheorem{definition}{Definition}[section]

\newtheoremstyle{aldenthm}
{6pt} % Space above
{6pt} % Space below
{\itshape} % Body font
{} % Indent amount
{\bfseries} % Theorem head font
{.} % Punctuation after theorem head
{.5em} % Space after theorem head
{} % Theorem head spec (can be left empty, meaning `normal')

\theoremstyle{aldenthm}
\newtheorem{theorem}{Theorem}
\newtheorem{conjecture}{Conjecture}
\newtheorem{lemma}{Lemma}
\newtheorem{example}{Example}
\newtheorem{corollary}{Corollary}
\newtheorem{proposition}{Proposition}
\newtheorem{assumption}{Assumption}

\theoremstyle{remark}
\newtheorem{remark}{Remark}

\begin{document}
	
\title{Notes on ``Random Walks and an $O^{\star}(n^5)$ Volume Algorithm for Convex Bodies''}
\author{Alden Green}
\date{\today}
\maketitle

Let $x,y$ be two points with $\abs{x - y} \leq \frac{\delta}{\sqrt{n}}$, and set
\begin{equation*}
C = (x + B') \cap (y + B'). \tag{$B' = \delta B$}
\end{equation*}
Consider the 'moons'
\begin{equation*}
M_x = (x + B') \setminus (y + B'),~ M_y = (y + B') \setminus (x + B')
\end{equation*}
and set
\begin{equation*}
R_x = M_x \cap (x - y + C),~ R_y = M_y \cap(y - x + C).
\end{equation*}
Finally, let $C'$ be obtained by blowing up $\frac{1}{2}(x + y)$ from its center by a factor of $1 + \frac{4n}{n - 1}$. 

\begin{lemma}
	\label{lem: large_intersection_1}
	\begin{equation*}
	M_x \setminus R_x \subseteq C'
	\end{equation*}
\end{lemma}


\begin{lemma}
	\label{lem: large_intersection_2}
	For every convex body $K$ containing $x$ and $y$,
	\begin{equation*}
	\vol(K \cap (M_x \setminus R_x)) \leq (e - 1)\vol(K \cap C)
	\end{equation*}
\end{lemma}
\begin{proof}
	Without loss of generality let $x - y = 0$, so that $C' = (1 + 4n/(n-1))C$. By Lemma \ref{lem: large_intersection_1}, we have
	\begin{equation*}
	K \cap (M_x \setminus R_x) \subseteq K \cap C'
	\end{equation*}
	Then, since $K$ is convex, $0 \in K$. So if $z \in K$, then $\alpha z$ is in $K$ for any $0 \leq \alpha \leq 1$. As a result, we have that $(K \cap C') \subseteq (1 + 4n/(n-1)) (K \cap C)$. Thus,
	\begin{equation*}
	\vol(K \cap (M_x \setminus R_x)) \leq \vol(K \cap C') \leq (1 + \frac{4n}{n - 1})^n \vol(K \cap C) \leq (e - 1) \vol(K \cap C).
	\end{equation*}
\end{proof}

\begin{lemma}
	\label{lem: large_intersection_3}
	For every convex body $K$,
	\begin{equation*}
	\vol(K \cap C)^2 \geq \vol(K \cap R_x) \vol(K \cap R_y)
	\end{equation*}
\end{lemma}
\begin{proof}
	By the Brunn-Minkowski Theorem,
	\begin{equation*}
	g(u) = \vol((u + C) \cap K)
	\end{equation*}
	is log-concave. Therefore
	\begin{equation*}
	g(0) \geq g(x - y)g(y - x)
	\end{equation*}
	and the statement follows.
\end{proof}

\begin{lemma}
	\label{lem: large_intersection_4}
	For every convex body $K$ containing $x$ and $y$,
	\begin{equation*}
	\vol(K \cap C) \geq \frac{1}{e + 1} \min \set{\vol((x + B') \cap K), \vol((y + B') \cap K)}
	\end{equation*}
\end{lemma}
\begin{proof}
	We have that
	\begin{equation*}
	\vol(K \cap R_x) + \vol(K \cap M_x \setminus R_x) = \vol(K \cap M_x)
	\end{equation*}
	and so by Lemma \ref{lem: large_intersection_2}, 
	\begin{equation*}
	\vol(K \cap R_x) \geq \vol(K \cap M_x) - (e - 1) \vol(K \cap C) = \vol((x + B') \cap K) - e(\vol(K \cap C))
	\end{equation*}
	with a corresponding lower bound holding for $\vol(K \cap R_y)$. We then apply Lemma \ref{lem: large_intersection_3} to obtain
	\begin{equation*}
	\vol(K \cap C) \geq \min\set{\vol((x + B') \cap K), \vol((y + B') \cap K)} - e(\vol(K \cap C))
	\end{equation*}
	from which the desired result is apparent.
\end{proof}

\end{document}