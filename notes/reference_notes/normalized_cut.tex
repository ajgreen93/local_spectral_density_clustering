\documentclass{article}
\usepackage{amsmath}
\usepackage{amsfonts, amsthm, amssymb}
\usepackage{graphicx}
\usepackage[colorlinks]{hyperref}
\usepackage[parfill]{parskip}
\usepackage{algpseudocode}
\usepackage{algorithm}
\usepackage{enumerate}
\usepackage[shortlabels]{enumitem}
\usepackage{fullpage}

\usepackage{natbib}
\renewcommand{\bibname}{REFERENCES}
\renewcommand{\bibsection}{\subsubsection*{\bibname}}

\newcommand{\eqdist}{\ensuremath{\stackrel{d}{=}}}
\newcommand{\Graph}{\mathcal{G}}
\newcommand{\Reals}{\mathbb{R}}
\newcommand{\Identity}{\mathbb{I}}
\newcommand{\distiid}{\overset{\text{i.i.d}}{\sim}}
\newcommand{\convprob}{\overset{p}{\to}}
\newcommand{\convdist}{\overset{w}{\to}}
\newcommand{\Expect}[1]{\mathbb{E}\left[ #1 \right]}
\newcommand{\Risk}[2][P]{\mathcal{R}_{#1}\left[ #2 \right]}
\newcommand{\Prob}[1]{\mathbb{P}\left( #1 \right)}
\newcommand{\iset}{\mathbf{i}}
\newcommand{\jset}{\mathbf{j}}
\newcommand{\myexp}[1]{\exp \{ #1 \}}
\newcommand{\norm}[1]{\left\lVert#1\right\rVert}
\newcommand{\abs}[1]{\left \lvert #1 \right \rvert}
\newcommand{\restr}[2]{\ensuremath{\left.#1\right|_{#2}}}
\newcommand{\ext}[1]{\widetilde{#1}}
\newcommand{\set}[1]{\left\{#1\right\}}
\newcommand{\seq}[1]{\set{#1}_{n \in \N}}
\newcommand{\dotp}[2]{\langle #1, #2 \rangle}
\newcommand{\floor}[1]{\left\lfloor #1 \right\rfloor}
\newcommand{\Var}{\mathrm{Var}}
\newcommand{\diam}{\mathrm{diam}}

\newcommand{\emC}{C_n}
\newcommand{\emCpr}{C'_n}
\newcommand{\emCthick}{C^{\sigma}_n}
\newcommand{\emCprthick}{C'^{\sigma}_n}
\newcommand{\emS}{S^{\sigma}_n}
\newcommand{\estC}{\widehat{C}_n}
\newcommand{\hC}{\hat{C^{\sigma}_n}}
\newcommand{\vol}{\text{vol}}
\newcommand{\spansp}{\mathrm{span}~}
\newcommand{\1}{\mathbb{I}}

\newcommand{\Linv}{L^{\dagger}}
\DeclareMathOperator*{\argmin}{argmin}
\DeclareMathOperator*{\argmax}{argmax}

\newcommand{\emF}{\mathbb{F}_n}
\newcommand{\emG}{\mathbb{G}_n}
\newcommand{\emP}{\mathbb{P}_n}
\newcommand{\F}{\mathcal{F}}
\newcommand{\D}{\mathcal{D}}
\newcommand{\R}{\mathcal{R}}
\newcommand{\Rd}{\Reals^d}

%%% Vectors
\newcommand{\thetast}{\theta^{\star}}

%%% Matrices
\newcommand{\X}{X} % no bold
\newcommand{\Y}{Y} % no bold
\newcommand{\Z}{Z} % no bold
\newcommand{\Lgrid}{L_{\grid}}
\newcommand{\Dgrid}{D_{\grid}}
\newcommand{\Linvgrid}{L_{\grid}^{\dagger}}

%%% Sets and classes
\newcommand{\Xset}{\mathcal{X}}
\newcommand{\Sset}{\mathcal{S}}
\newcommand{\Hclass}{\mathcal{H}}
\newcommand{\Pclass}{\mathcal{P}}

%%% Distributions and related quantities
\newcommand{\Pbb}{\mathbb{P}}
\newcommand{\Ebb}{\mathbb{E}}
\newcommand{\Qbb}{\mathbb{Q}}

%%% Vector spaces
\newcommand{\Zbb}{\mathbb{Z}}

%%% Operators
\newcommand{\Tadj}{T^{\star}}
\newcommand{\dive}{\mathrm{div}}
\newcommand{\dif}{\mathop{}\!\mathrm{d}}
\newcommand{\gradient}{\mathcal{D}}
\newcommand{\Hessian}{\mathcal{D}^2}

%%% Misc
\newcommand{\grid}{\mathrm{grid}}
\newcommand{\critr}{R_n}
\newcommand{\dx}{\,dx}
\newcommand{\dy}{\,dy}
\newcommand{\dr}{\,dr}
\newcommand{\wt}[1]{\widetilde{#1}}
\newcommand{\ol}[1]{\overline{#1}}

%%% Order of magnitude
\newcommand{\soom}{\sim}

% \newcommand{\span}{\textrm{span}}

\newtheoremstyle{alden}
{6pt} % Space above
{6pt} % Space below
{} % Body font
{} % Indent amount
{\bfseries} % Theorem head font
{.} % Punctuation after theorem head
{.5em} % Space after theorem head
{} % Theorem head spec (can be left empty, meaning `normal')

\theoremstyle{alden} 


\newtheoremstyle{aldenthm}
{6pt} % Space above
{6pt} % Space below
{\itshape} % Body font
{} % Indent amount
{\bfseries} % Theorem head font
{.} % Punctuation after theorem head
{.5em} % Space after theorem head
{} % Theorem head spec (can be left empty, meaning `normal')

\theoremstyle{aldenthm}
\newtheorem{theorem}{Theorem}
\newtheorem{conjecture}{Conjecture}
\newtheorem{lemma}{Lemma}
\newtheorem{example}{Example}
\newtheorem{corollary}{Corollary}
\newtheorem{proposition}{Proposition}
\newtheorem{assumption}{Assumption}
\newtheorem{remark}{Remark}


\theoremstyle{definition}
\newtheorem{definition}{Definition}[section]

\theoremstyle{remark}

\begin{document}
\title{Normalized cut}
\author{Alden Green}
\date{\today}
\maketitle

\section{The Normalized Graph Cut and Cheeger Constant: from Discrete to Continuous(Arias-Castro11)}

Let $G$ be a graph with weights $\set{\delta_{ij}}$, and $M$ be an open,connected subset of $\Rd$. We begin by defining the graph and continuous based functionals under consideration. The \emph{normalized cut} of a subset $S \subset G$ is defined as
\begin{equation*}
h(S;G) = \frac{\sigma(S)}{\min\set{\delta(S), \delta(S^c)}}, \quad \sigma(S) = \sum_{i \in S} \sum_{j \in S^c} \delta_{ij}, \quad \delta(S) = \sum_{i \in S} \sum_{j \neq i} \delta_{ij}.
\end{equation*}
The \emph{conductance} of $G$ is defined as $H(S) = \min\set{h(S;G): S \subset G, S \neq \emptyset}$. 

The \emph{normalized cut} of a subset $A \subset \Rd$ with respect to $M$ is
\begin{equation*}
h(A;M) = \frac{\vol_{d-1}(\partial A \cap M)}{\min\set{\vol_d(A \cap M), \vol_{d}(A^c \cap M)}},
\end{equation*}
where by $\vol_k(S)$ we mean the $k$-dimensional Hausdorff measure of $S$. The \emph{Cheeger constant} of $M$ is $H(M) = \min\set{h(A;M): A \subset M}$. 

Let $X_1\ldots,X_n$ be sampled independently from $\mu$, the uniform distribution over $M$, and let $G_{n,r}$ be the neighborhood graph of radius $r$ formed over $X_1,\ldots,X_n$. Our first result shows the pointwise consistency of the (suitably normalized) graph normalized cut to the continuous normalized cut. Let $\nu_d$ be the volume of the $d$-dimensional unit ball, and
\begin{equation*}
\gamma_d = \int_{\Rd} \max\set{\dotp{u}{z},0} \1(\norm{z} \leq 1) \,dz
\end{equation*}
be the volume of a spherical cap of uniformly random height of the $d$-dimensional unit ball.

\begin{theorem}
	Let $A$ be a fixed subset of $M$ with smooth relative boundary $\partial A \cap M$. Fix a sequence $r_n \to 0$ such that $\frac{nr_n^{d+1}}{\log n} \to \infty$. Letting $S_n = A \cap G_{n,r_n}$, with probability one
	\begin{equation*}
	\lim_{n \to \infty} \frac{\nu_d}{\gamma_d r_n} h(S_n; G_{n,r_n}) \to h(A;M)
	\end{equation*}
	and therefore
	\begin{equation*}
	\limsup_{n \to \infty}  \frac{\nu_d}{\gamma_d r_n} H(G_{n,r_N}) \leq H(M).
	\end{equation*}
\end{theorem}

\end{document}