\documentclass{article}
\usepackage{amsmath}
\usepackage{amsfonts, amsthm, amssymb}
\usepackage{bm}
\usepackage{graphicx}
\usepackage[colorlinks]{hyperref}
\usepackage[parfill]{parskip}
\usepackage{algpseudocode}
\usepackage{algorithm}
\usepackage{enumerate}
\usepackage{fullpage}

\usepackage{natbib}
\renewcommand{\bibname}{REFERENCES}
\renewcommand{\bibsection}{\subsubsection*{\bibname}}

\makeatletter
\newcommand{\leqnomode}{\tagsleft@true}
\newcommand{\reqnomode}{\tagsleft@false}
\makeatother

\newcommand{\eqdist}{\ensuremath{\stackrel{d}{=}}}
\newcommand{\Graph}{\mathcal{G}}
\newcommand{\Reals}{\mathbb{R}}
\newcommand{\Identity}{\mathbb{I}}
\newcommand{\distiid}{\overset{\text{i.i.d}}{\sim}}
\newcommand{\convprob}{\overset{p}{\to}}
\newcommand{\convdist}{\overset{w}{\to}}
\newcommand{\Expect}[1]{\mathbb{E}\left[ #1 \right]}
\newcommand{\Risk}[2][P]{\mathcal{R}_{#1}\left[ #2 \right]}
\newcommand{\Var}[1]{\mathrm{Var}\left( #1 \right)}
\newcommand{\Prob}[1]{\mathbb{P}\left( #1 \right)}
\newcommand{\iset}{\mathbf{i}}
\newcommand{\jset}{\mathbf{j}}
\newcommand{\myexp}[1]{\exp \{ #1 \}}
\newcommand{\norm}[1]{\left\lVert#1\right\rVert}
\newcommand{\dotp}[2]{\langle #1 , #2 \rangle}
\newcommand{\abs}[1]{\left \lvert #1 \right \rvert}
\newcommand{\restr}[2]{\ensuremath{\left.#1\right|_{#2}}}
\newcommand{\defeq}{\overset{\mathrm{def}}{=}}
\newcommand{\convweak}{\overset{w}{\rightharpoonup}}
\newcommand{\dive}{\mathrm{div}}
\newcommand{\Bin}{\mathrm{Bin}}

\newcommand{\emC}{C_n}
\newcommand{\emCpr}{C'_n}
\newcommand{\emCthick}{C^{\sigma}_n}
\newcommand{\emCprthick}{C'^{\sigma}_n}
\newcommand{\emS}{S^{\sigma}_n}
\newcommand{\estC}{\widehat{C}_n}
\newcommand{\hC}{\hat{C^{\sigma}_n}}
\newcommand{\Bal}{\textrm{Bal}}
\newcommand{\Cut}{\textrm{Cut}}
\newcommand{\Ind}{\textrm{Ind}}
\newcommand{\set}[1]{\left\{#1\right\}}
\newcommand{\seq}[1]{\set{#1}_{n \in \N}}
\newcommand{\Perp}{\perp \! \! \! \perp}
\newcommand{\Naturals}{\mathbb{N}}
\newcommand{\dist}{\mathrm{dist}}

\newcommand\independent{\protect\mathpalette{\protect\independenT}{\perp}}
\def\independenT#1#2{\mathrel{\rlap{$#1#2$}\mkern2mu{#1#2}}}


\newcommand{\Linv}{L^{\dagger}}
\newcommand{\tr}{\text{tr}}
\newcommand{\h}{\textbf{h}}
% \newcommand{\l}{\ell}
\newcommand{\x}{\textbf{x}}
\newcommand{\y}{\textbf{y}}
\newcommand{\bl}{\bm{\ell}}
\newcommand{\bnu}{\bm{\nu}}
\newcommand{\Lx}{\mathcal{L}_X}
\newcommand{\Ly}{\mathcal{L}_Y}
\DeclareMathOperator*{\argmin}{argmin}


\newcommand{\emG}{\mathbb{G}_n}
\newcommand{\A}{\mathcal{A}}
\newcommand{\F}{\mathcal{F}}
\newcommand{\G}{\mathcal{G}}
\newcommand{\X}{\mathcal{X}}
\newcommand{\Rd}{\Reals^d}
\newcommand{\N}{\mathbb{N}}
\newcommand{\E}{\mathcal{E}}

%%% Matrix related notation
\newcommand{\Xbf}{\mathbf{X}}
\newcommand{\Ybf}{\mathbf{Y}}
\newcommand{\Zbf}{\mathbf{Z}}
\newcommand{\Abf}{\mathbf{A}}
\newcommand{\Dbf}{\mathbf{D}}
\newcommand{\Wbf}{\mathbf{W}}
\newcommand{\Lbf}{\mathbf{L}}
\newcommand{\Ibf}{\mathbf{I}}
\newcommand{\Bbf}{\mathbf{B}}

%%% Vector related notation
\newcommand{\lbf}{\bm{\ell}}
\newcommand{\fbf}{\mathbf{f}}

%%% Set related notation
\newcommand{\Cset}{\mathcal{C}}
\newcommand{\Dset}{\mathcal{D}}
\newcommand{\Aset}{\mathcal{A}}
\newcommand{\Wset}{\mathcal{W}}
\newcommand{\Sset}{\mathcal{S}}

\newcommand{\Csig}{\Cset_{\sigma}}
\newcommand{\Asig}{\Aset_{\sigma}}

%%% Distribution related notation
\newcommand{\Pbb}{\mathbb{P}}
\newcommand{\Qbb}{\mathbb{Q}}
\newcommand{\Ebb}{\mathbb{E}}
% \newcommand{\Pr}{\mathrm{Pr}}}

%%% Functionals
\newcommand{\1}{\mathbf{1}}

%%% Functionals over graphs
\newcommand{\cut}{\mathrm{cut}}
\newcommand{\vol}{\mathrm{vol}}
% \newcommand{\deg}{\mathrm{deg}}

\newtheoremstyle{alden}
{6pt} % Space above
{6pt} % Space below
{} % Body font
{} % Indent amount
{\bfseries} % Theorem head font
{.} % Punctuation after theorem head
{.5em} % Space after theorem head
{} % Theorem head spec (can be left empty, meaning `normal')

\theoremstyle{alden} 
\newtheorem{definition}{Definition}[section]

\newtheoremstyle{aldenthm}
{6pt} % Space above
{6pt} % Space below
{\itshape} % Body font
{} % Indent amount
{\bfseries} % Theorem head font
{.} % Punctuation after theorem head
{.5em} % Space after theorem head
{} % Theorem head spec (can be left empty, meaning `normal')

\theoremstyle{aldenthm}
\newtheorem{theorem}{Theorem}
\newtheorem{conjecture}{Conjecture}
\newtheorem{lemma}{Lemma}
\newtheorem{example}{Example}
\newtheorem{corollary}{Corollary}
\newtheorem{proposition}{Proposition}
\newtheorem{assumption}{Assumption}

\theoremstyle{remark}
\newtheorem{remark}{Remark}

\begin{document}
	
\title{Notes for the week of 5/11/19 - 5/17/19}
\author{Alden Green}
\date{\today}
\maketitle

For an undirected graph $G = (V,E)$, the lazy random walk over $G$ is the Markov chain with transition probabilities given by $\Wbf := \frac{\Ibf + \Dbf^{-1}\Abf}{2}$, stationary distribution $\pi$. Denote the $m$-step probability distribution of this random walk originating from a particular $v \in V$ as $q^{(m)}: V \times V \to [0,1]$, $q^{(m)}(v,u) = e_v \Wbf^m e_u$. We wish to upper bound the total variation distance between the distributions $q_v^{(m)} := q^{(m)}(v,\cdot)$ and $\pi$,
\begin{equation*}
\norm{q_v^{(t + 1)} - \pi}_{TV} = \sum_{u \in V} \abs{q_v^{(m)}(u) - \pi(u)}
\end{equation*}
using geometric properties of the graph $\Wbf$. We introduce the \emph{degree, cut} and \emph{volume} functionals over a graph. For $u \in V$, $A \subseteq G$, 
\begin{equation*}
\cut(A;G) = \sum_{u \in A} \sum_{v \in A^c} \1((u,v) \in E), \quad \deg(u;G) = \sum_{v \in V} \1((u,v) \in E), \quad \vol(A;G) = \sum_{u \in A} \deg(u;G)
\end{equation*}

The \emph{local spread} is defined as
\begin{equation*}
s(G) := \frac{9}{10} \cdot \min_{u \in V} \set{\deg(u; G)} \cdot \min_{u \in V} \set{\pi(v)}
\end{equation*}

Letting \emph{normalized cut} of $A \subseteq G$, $\Phi(A;G)$ be defined as
\begin{equation*}
\Phi(A;G) = \frac{\cut(A, A^c; G)}{\min\set{\vol(A; G),\vol(A^c; G)}},
\end{equation*}
and the \emph{conductance} is
\begin{equation*}
\Phi(G) = \min_{A \subseteq V} \Phi(A;G).
\end{equation*}

\begin{theorem}
	\label{thm: tv_mixing_time}
	For any $v \in V$,
	\begin{equation*}
	\norm{q_v^{(t + 3)} - \pi}_{TV} \leq \set{as(G), \frac{1}{8} + \frac{1}{20} + \frac{1}{2 \min_{u \in V} \deg(u;G)}} + \left(\frac{1}{1 - 2s(G)/9}\right) \left(1 - \frac{\Phi^2(G)}{2}\right)^t
	\end{equation*}
\end{theorem}

As we will see, Theorem \ref{thm: tv_mixing_time} is an essential step to providing an upper bound on the uniform mixing time, which is what we want. We justify this statement next, before moving on to proving Theorem \ref{thm: tv_mixing_time}.

\paragraph{Uniform mixing time.}
Consider the \emph{uniform} distance \footnote{Note $d_{\textrm{unif}}$ is not a formally a distance as it is not symmetric.} between $q_v^{(t)}$ and $\pi$, given by
\begin{equation*}
d_{\textrm{unif}}(q_v^{(t)},\pi) = \max_{u \in V} \set{\frac{\pi(u) - q_v^{(t)}(u)}{\pi(u)}}.
\end{equation*}

\begin{theorem}
	\label{thm: tv_to_uniform_distance}
	Let $\norm{q_v^{(t)} - \pi}_{TV} \leq \frac{1}{14} \max \set{1, \frac{1}{s(G)}}$. Then,
	\begin{equation*}
	d_{\textrm{unif}}(q_v^{(t + 3)},\pi) \leq \frac{1}{4}
	\end{equation*}
\end{theorem}
\begin{proof}
	Fix $u \in V$ and let $m \geq t + 1$ be arbitrary. The stationarity of $\pi$ gives
	\begin{align}
	\frac{\pi(u) - q_v^{m}(u)}{\pi(u)} & = \sum_{y \in V} \left(\pi(y) - q^{(m-1)}(v,y)\right) \left(\frac{q^{(1)}(y,u) - \pi(u)}{\pi(u)}\right) \nonumber \\
	& \overset{(i)}{=} \sum_{y \neq u} \left(\pi(y) - q^{(m-1)}(v,y)\right) \left(\frac{q^{(1)}(y,u) - \pi(u)}{\pi(u)}\right) + \frac{\pi(u) - q^{(m - 1)}(v,u)}{\pi(u)} \left(\frac{1}{2} - \pi(u)\right) \nonumber \\
	& \leq \sum_{y \neq u} \left(\pi(y) - q^{(m-1)}(v,y)\right) \left(\frac{q^{(1)}(y,u) - \pi(u)}{\pi(u)}\right) + \frac{\pi(u) - q^{(m - 1)}(v,u)}{2 \pi(u)} \label{eqn: tv_to_uniform_distance_1}
	\end{align}
	where $(i)$ follows from $q^{(1)}(u,u) = \frac{1}{2}$. 
	
	Then 
	\begin{align*}
	\sum_{y \neq u} \left(\pi(y) - q^{(m-1)}(v,y)\right) \left(\frac{q^{(1)}(y,u) - \pi(u)}{\pi(u)}\right) & \leq \norm{q_v^{(m-1)} - \pi}_{TV} \max_{y \neq u} \abs{\frac{q^{(1)}(y,u) - \pi(u)}{\pi(u)}} \\
	& \leq \norm{q_v^{(m-1)} - \pi}_{TV} \max \set{1, \max_{y \neq u}\set{\frac{q^{(1)}(y,u)}{\pi(u)}}} \\
	& \leq \norm{q_v^{(m-1)} - \pi}_{TV} \max \set{1, \frac{1}{s(G)}} \\
	\end{align*}
	since for $y \neq u$, $q^{(1)}(y,u) \leq 1/\left(2 \min_{u \in V} \deg(u; G)\right)$. As $m - 1 \geq t$, it is well known \cite{lovasz} that the laziness  of the random walk guarantees  $\norm{q_v^{(m - 1)} - \pi}_{TV} \leq \norm{q_v^{(t)} - \pi}_{TV}$, and therefore
	\begin{equation*}
	\sum_{y \neq u} \left(\pi(y) - q^{(m-1)}(v,y)\right) \left(\frac{q^{(1)}(y,u) - \pi(u)}{\pi(u)}\right) \leq \frac{1}{14}.
	\end{equation*}
	
	Plugging this in to \eqref{eqn: tv_to_uniform_distance_1} and taking maximum on both sides, we obtain
	\begin{equation}
	d_{\textrm{unif}}(q_v^{(m)}, \pi) \leq \frac{2}{7} + \frac{d_{\textrm{unif}}(q_v^{(m - 1)}, \pi)}{2} \label{eqn: tv_to_uniform_distance_2}
	\end{equation}
	The recurrence relation of \eqref{eqn: tv_to_uniform_distance_2} along with the initial condition $d_{\textrm{unif}}(q_v^{(t)}, \pi) \leq 1$ yields
	\begin{equation*}
	d_{\textrm{unif}}(q_v^{(t + 1)}, \pi) \leq \frac{8}{14} \Rightarrow d_{\textrm{unif}}(q_v^{(t + 2)}, \pi) \leq \frac{10}{28} \Rightarrow  d_{\textrm{unif}}(q_v^{(t + 3)}, \pi) \leq \frac{1}{4}
	\end{equation*}
	and the claim is shown.
\end{proof}

\section{Proof of Theorem \ref{thm: tv_mixing_time}.}

For arbitrary starting distribution $q$ (meaning $\textrm{supp}(q) \subseteq V$ and $\sum_{u \in V}q(u) = 1$), and for $t \geq 0$ an integer, consider the distance function $h_q^{(t)}, t \geq 0$, 
\begin{equation*}
h_q^{(t)}(x) = \max \left\{ \sum_{u \in V} \left(q^{(m)}(u) - \pi(u)\right)w(u)\right\}
\end{equation*}
where the maximum is over all $w: V \to [0,1]$ such that $0 \leq w(u) \leq 1$ for all $u$, and $\sum_{u \in V} w(u) \pi(u) = x$. Writing $h_v^{(t)}(x) := h_{e_v}^{(t)}(x)$ in a small abuse of notation, in Theorem \ref{thm: lt_ub} and Lemma \ref{lem: mixing_time_small_sets}, we give an upper bound on $h_{e_v}^{(t)}(x)$ for all $0 \leq x \leq 1$. 

\vskip 0.1 in
\begin{remark}
	$h_q^{(t)}$ permits an equivalent relation. Order the elements of $V = \set{u_1, \ldots, u_N}$ ($N$ = $\abs{V}$), such that
	\begin{equation*}
	\frac{q^{(m)}(u_1)}{\pi(u_1)} \geq \frac{q^{(m)}(u_2)}{\pi(u_2)} \geq \ldots \geq \frac{q^{(m)}(u_N)}{\pi(u_N)}
	\end{equation*}
	and let $U_k = \set{u_1, \ldots, u_k}$. Then for any $x$, letting $k$ satisfy $\pi(U_{k - 1}) < x < \pi(U_k)$, it can be shown that,
	\begin{equation}
	\label{eqn: lovasz_simonovits_curve}
	h_q^{(t)}(x) = \sum_{j = 1}^{k - 1} (q^{(m)}(u_j) - \pi(u_j)) + \frac{x - \pi(U_{k - 1})}{\pi(u_k)} \left(q^{(m)}(u_k) - \pi(u_k) \right).
	\end{equation}
	The formulation on the right hand side of \eqref{eqn: lovasz_simonovits_curve} has come to be known as the \emph{Lovasz-Simonovits curve}.
\end{remark}

\paragraph{Mixing over large sets.}

For $0 \leq \mu \leq 1$ and $\mu \leq x \leq 1 - \mu$ let 
\begin{equation*}
\ell_{\mu}(x) = \frac{1 - \mu - x}{1 - 2\mu} h_q^{(0)}(\mu) + \frac{x - \mu}{1 - 2\mu}h_q^{(0)}(1 - \mu)
\end{equation*}
be the linear interpolator between $h_q^{(0)}(\mu)$ and $h_q^{(0)}(1 - \mu)$. 

\begin{theorem}
	\label{thm: lt_ub}
	For any $0 \leq \mu \leq 1/2$ and $\mu \leq x \leq 1 - \mu$ and $t \geq 0$, 
	\begin{equation*}
	h_q^{(t)}(x) \leq \ell_{\mu}(x) + \max \set{\frac{h_q^{(0)}(\mu)}{1 - 2\mu} + \frac{h_q^{(0)}(\mu)}{\mu} , \frac{h_q^{(0)}(1 - \mu) }{1 - 2\mu} + 1 }\left(1 - \frac{\Phi^2(G)}{2}\right)^t
	\end{equation*}
\end{theorem}

Theorem \ref{thm: lt_ub} is a direct consequence of Theorem \ref{thm: lovasz_simonovits_1993}. To state the latter, we introduce
\begin{equation*}
C_{\mu} = \max \set{\frac{h_q^{(0)}(x) - \ell_{\mu}(x)}{\sqrt{x - \mu}}, \frac{h_q^{(0)}(x) - \ell_{\mu}(x)}{\sqrt{1 - x - \mu}}: \mu < x < 1 - \mu}
\end{equation*}
\begin{theorem}[Theorem 1.2 of \textcolor{red}{Lovasz-Simonovits 1993}]
	\label{thm: lovasz_simonovits_1993}
	For any $0 \leq \mu \leq \frac{1}{2}$, $\mu \leq x \leq 1 - \mu$ and an integer $t \geq 0$,
	\begin{equation*}
	h_q^{(t)}(x) \leq \ell_{\mu}(x) + C_{\mu} \min \set{\sqrt{x - \mu}, \sqrt{1 - x - \mu}} \left(1 - \frac{\Phi^2(G)}{2}\right)^t
	\end{equation*}
\end{theorem}

\begin{proof}[Proof of Theorem \ref{thm: lt_ub}]
	Fix $0 \leq \mu \leq \frac{1}{2}$. We will show that for all $\mu \leq x \leq 1 - \mu$
	\begin{equation}
	\label{eqn: lt_ub_1}
	h_q^{(0)}(x) - \ell_{\mu}(x) \leq \max \set{\frac{h_q^{(0)}(\mu)}{1 - 2\mu} + \frac{h_q^{(0)}(\mu)}{\mu} , \frac{h_q^{(0)}(1 - \mu) }{1 - 2\mu} + 1 } \min \set{\sqrt{x - \mu},\sqrt{1 - x - \mu}}
	\end{equation}
	whence the claim follows by Theorem \ref{thm: lovasz_simonovits_1993}. 
	
	Note that $\ell_{\mu}(\mu) = h_q^{(0)}(\mu)$, and for $x \geq \mu$, 
	\begin{equation*}
	h_q^{(0)}(x) \leq h_q^{(0)}(\mu) + (x - \mu)\frac{h_q^{(0)}(\mu)}{\mu}
	\end{equation*}
	by the concavity of $h_0$ along with Lemma \ref{lem: lovasz_simonovits_subdifferential}. Some basic algebra then yields
	\begin{align*}
	h_q^{(0)}(x) - \ell_{\mu}(x) & \leq h_q^{(0)}(\mu) - \left(\frac{1 - \mu - x}{1 - 2\mu} h_q^{(0)}(\mu) + \frac{x - \mu}{1 - 2\mu} h_q^{(0)}(1 - \mu)\right) + \frac{h_q^{(0)}(\mu)}{\mu}(x - \mu) \\
	& =  \frac{x - \mu}{1 - 2\mu}h_q^{(0)}(\mu) + \frac{h_q^{(0)}(\mu)}{\mu} (x - \mu) - \frac{x - \mu}{1 - 2\mu} h_q^{(0)}(1 - \mu) \\
	& \leq \sqrt{x - \mu} \left(\frac{h_q^{(0)}(\mu)}{1 - 2\mu} + \frac{h_q^{(0)}(\mu)}{\mu} \right)
	\end{align*}
	On the other hand, $\ell_{\mu}(1 - \mu) = h_q^{(0)}(1 - \mu)$, and by the concavity of $h_q^{(0)}$ and Lemma \ref{lem: lovasz_simonovits_subdifferential},  for $x \leq 1 - \mu$
	\begin{equation*}
	h_q^{(0)}(x) \leq h_q^{(0)}(1 - \mu) + (1 - x - \mu).
	\end{equation*}
	Similar manipulations to above give the upper bound
	\begin{align*}
	h_q^{(0)}(x) - \ell_{\mu}(x) \leq \sqrt{1 - \mu - x}\left(\frac{h_q^{(0)}(1 - \mu) }{1 - 2\mu} + 1\right)
	\end{align*}
	and \eqref{eqn: lt_ub_1} follows.
\end{proof}

\begin{lemma}
	\label{lem: lovasz_simonovits_subdifferential}
	The subdifferential $v(x)$ of $h_q^{(0)}(x)$ satisfies
	\begin{equation*}
	-1 \leq v(x) \leq \frac{h_q^{(0)}(\mu)}{\mu}
	\end{equation*}
\end{lemma}

\paragraph{Mixing over small sets.}

\begin{lemma}
	\label{lem: mixing_time_small_sets}
	Let $0 \leq a \leq 1$, and $t \geq 1$ an integer. Then for any $x \leq as(G)$ or $x \geq 1 - as(G)$,
	\begin{equation*}
	h_v^{(t)}(x) \leq \max\set{as(G), \frac{1}{2^t} + \frac{9a}{20} + \frac{1}{2 \min_{u \in V}\deg(u;G)} }
	\end{equation*}
\end{lemma}
\begin{proof}
	First, we deal with the case $x \leq as(G)$. Letting $U_k$ be as in \eqref{eqn: lovasz_simonovits_curve}, we have
	\begin{equation}
	\label{eqn: mixing_time_small_sets_1}
	h_v^{(t)}(x) \leq q_v^{(m)}(U_{k - 1}) + q_v^{(m)}(u_k)
	\end{equation}
	
	We will rely on the key fact that for any $u \neq v, t \geq 1$,
	\begin{equation}
	\label{eqn: mixing_time_small_sets_2}
	q_v^{(t)}(u) \leq \frac{1}{2 \min_{u \in V}\deg(u;G)}
	\end{equation}
	
	On the other hand if $u = v$,
	\begin{equation}
	\label{eqn: mixing_time_small_sets_3}
	q_v^{(m)}(u) \leq \frac{1}{2^t} + \frac{1}{2\min_{u \in V}\deg(u;G)}.
	\end{equation}
	Therefore by \eqref{eqn: mixing_time_small_sets_1}, \eqref{eqn: mixing_time_small_sets_2}, and \eqref{eqn: mixing_time_small_sets_3}
	\begin{equation*}
	h_v^{(t)}(x) \leq \frac{1}{2^t} + \frac{\abs{U_k}}{2 \min_{u \in V}\deg(u;G)} + \frac{1}{2 \min_{u \in V}\deg(u;G)}.
	\end{equation*}
	Since $x \leq a s(G)$, 
	\begin{equation*}
	\abs{U_k} \leq \frac{x}{10 \min_{u \in V}(\pi(v))} \leq \frac{9a\min_{u \in V}\deg(u;G)}{10}.
	\end{equation*}
	
	For any $0 \leq b \leq 1$, $x \geq 1 - b$ implies $h_v^{(t)}(x) \leq b$. Taking $b = a s(G)$, the claim is shown.
\end{proof}

\paragraph{Proof of Theorem \ref{thm: tv_mixing_time}.}
For any $A \subseteq V$ and any integer $t \geq 0$,
\begin{equation*}
\max \set{h_v^{(t)}(\pi(A)), h_v^{(t)}(1 - \pi(A))} \geq \abs{q_v^{(t)}(A) - \pi(A)}
\end{equation*}
and taking max over both sides, we have
\begin{equation*}
\max_{0 \leq x \leq 1} h_v^{(t)}(x) \geq \norm{q_v^{(t)} - \pi}_{TV}.
\end{equation*}

Letting $q = e_v W^3$, observe that $h_v^{(t + 3)}(x) = h_v^{(t)}(x)$. Fix $\mu = \frac{1}{9}s(G)$. Then, for $\mu \leq x \leq 1 - \mu$, by Theorem \ref{thm: lt_ub},
\begin{align*}
h_v^{(t + 3)}(x) & = h_q^{(t)}(x) \\
& \leq \max \set{h_q^{(0)}(\mu), h_q^{(0)}(1 - \mu) }  + \left(\frac{1}{1 - 2 \mu} + \frac{1}{2\mu}\right) \left(1 - \frac{\Phi^2(G)}{2}\right)^t\\
& \leq \max \set{as(G), \frac{1}{8} + \frac{1}{20} + \frac{1}{2 \min_{u \in V} \deg(u;G)}} + \left(\frac{1}{1 - 2s(G)/9}\right) \left(1 - \frac{\Phi^2(G)}{2}\right)^t
\end{align*}
where the last inequality comes from application of Lemma \ref{lem: mixing_time_small_sets} to $h_q^{(t)} = h_v^{(t + 3)}$. 

For $x \leq \mu$ or $x \geq 1 - \mu$,
\begin{equation*}
h_v^{(t + 3)}(x) \leq \max \set{as(G), \frac{1}{8} + \frac{1}{20} + \frac{1}{2 \min_{u \in V} \deg(u;G)}}
\end{equation*}
and the proof is complete.

\end{document}