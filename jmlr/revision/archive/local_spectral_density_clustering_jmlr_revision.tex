\documentclass[11pt,twoside]{article}
\usepackage[preprint]{jmlr2e}
\usepackage{color}
\usepackage{fancyhdr}
\usepackage{amsfonts,epsfig,graphicx}
\usepackage{afterpage}
\usepackage{amsmath,amssymb} 
% \usepackage{fullpage}
\usepackage{epsf} 
\usepackage{graphics} 
\usepackage{amsfonts,amsmath}
% \usepackage[sort,numbers]{natbib} 
\usepackage{psfrag,xspace}
\usepackage{color,etoolbox}

\setlength{\textwidth}{\paperwidth}
\addtolength{\textwidth}{-6cm}
\setlength{\textheight}{\paperheight}
\addtolength{\textheight}{-4cm}
\addtolength{\textheight}{-1.1\headheight}
\addtolength{\textheight}{-\headsep}
\addtolength{\textheight}{-\footskip}
\setlength{\oddsidemargin}{0.5cm}
\setlength{\evensidemargin}{0.5cm}
\renewcommand{\floatpagefraction}{.8}%

%\newtheorem{theorem}{Theorem} 
%\newtheorem{lemma}{Lemma}
%\newtheorem{proposition}{Proposition}
%\newtheorem{corollary}{Corollary}
%\theoremstyle{definition}
%\newtheorem{definition}{Definition}
%\newtheorem{remark}{Remark}

\usepackage[utf8]{inputenc} % allow utf-8 input
\usepackage[T1]{fontenc}    % use 8-bit T1 fonts
\usepackage{hyperref}       % hyperlinks
\usepackage{url}            % simple URL typesetting
\usepackage{booktabs}       % professional-quality tables
\usepackage{amsfonts}       % blackboard math symbols
\usepackage{nicefrac}       % compact symbols for 1/2, etc.
\usepackage{microtype}      % microtypography

\usepackage{microtype}
\usepackage{graphicx}
\usepackage{float}
\usepackage[export]{adjustbox}
\usepackage{subcaption}
\usepackage{booktabs}
\usepackage{xcolor}
%\usepackage{xr-hyper}
%\usepackage{hyperref}
%\usepackage[reqno]{amsmath}
%\usepackage{amsfonts, amsthm, amssymb}
\usepackage{algorithm}
\usepackage{algorithmic}
%\usepackage[parfill]{parskip}
\usepackage{enumerate}
\usepackage[shortlabels]{enumitem}
\usepackage{bm}
\usepackage{mathtools}

%%%%%% Begin Alden


\newcommand{\diam}{\rho}
\newcommand{\set}[1]{\left\{#1\right\}}
\newcommand{\seq}[1]{\left\{#1\right\}_{n \in \mathbb{N}}}
\newcommand{\defeq}{\overset{\mathrm{def}}{=}}
\newcommand{\vol}{\mathrm{vol}}
\newcommand{\cut}{\mathrm{cut}}
\newcommand{\abs}[1]{\left \lvert #1 \right \rvert}
\newcommand{\N}{\mathbb{N}}
\newcommand{\Reals}{\mathbb{R}}
\newcommand{\Rd}{\Reals^d}
\newcommand{\norm}[1]{\left\lVert#1\right\rVert}
\usepackage{bbm}
\newcommand{\1}{\mathbbm{1}}
\newcommand{\Phibf}{\Phi_{u}}
\newcommand{\Psibf}{\Psi_{u}}
\newcommand{\taubf}{\tau_{u}}
\newcommand{\dist}{\mathrm{dist}}
\newcommand{\Err}{\mathrm{Err}}
\newcommand{\TV}{\mathrm{TV}}

%%% Vectors (no bold font)
\newcommand{\pbf}{p}        % removed bold font
\newcommand{\qbf}{q}
\newcommand{\ebf}[1]{{e}_{#1}}
\newcommand{\pibf}{\pi}

%%% Matrices (no bold font)
\newcommand{\Abf}{A}       % removed bold font 
\newcommand{\Xbf}{X}             
\newcommand{\Wbf}{W}
\newcommand{\Lbf}{L}
\newcommand{\Dbf}{D}
\newcommand{\Ibf}[1]{I_{#1}}

%%% Probability distributions (and related items)
\newcommand{\Pbb}{\mathbb{P}}
\newcommand{\Cbb}{\mathbb{C}}
\newcommand{\Ebb}{\mathbb{E}}

%%% Sets
\newcommand{\mc}[1]{\mathcal{#1}}
\newcommand{\Sset}{\mathcal{S}}
\newcommand{\Cset}{\mathcal{C}}
\newcommand{\Aset}{\mathcal{A}}
\newcommand{\Asig}{\Aset_{\sigma}}
\newcommand{\Csig}{\Cset_{\sigma}}
\newcommand{\Asigr}{\Aset_{\sigma,\sigma + r}}
\newcommand{\Csigr}{\Cset_{\sigma,\sigma + r}}

%%% Graph quantities
\newcommand{\Cest}{\widehat{C}}
\newcommand{\degminpr}{d_{\min}'}
\newcommand{\degminwt}{\widetilde{d}_{\min}}
\newcommand{\degmaxwt}{\widetilde{d}_{\max}}
\newcommand{\degmax}{d_{\max}}
\newcommand{\piminwt}{\widetilde{\pi}_{\min}}
\newcommand{\piminpr}{\pibf_{\min}'}
\newcommand{\degmin}{d_{\min}}

%%% Operators
\DeclareMathOperator*{\argmin}{arg\,min}
\newcommand{\dx}{\,dx}
\newcommand{\dy}{\,dy}
\newcommand{\dt}{\,dt}

%%% Algorithm notation
\newcommand{\ppr}{{\sc PPR}}
\newcommand{\pprspace}{{\sc PPR~}}

%%% Tilde notation for quantities over the expansion set 
\newcommand{\wt}[1]{\widetilde{#1}}
\newcommand{\wn}{\widetilde{n}}
\newcommand{\wX}{\widetilde{\Xbf}}
\newcommand{\wx}{\widetilde{x}}
\newcommand{\wz}{\widetilde{z}}
\newcommand{\wbz}{\widetilde{\bf{z}}}
\newcommand{\wu}{\widetilde{u}}
\newcommand{\wPbb}{\widetilde{\Pbb}}
\newcommand{\wf}{\widetilde{f}}
\newcommand{\wDbf}{\widetilde{\Dbf}}
\newcommand{\piwt}{\widetilde{\pi}}

\newcommand{\sbcomment}[1]{{\color{red} \bf{{{{SB --- #1}}}}}}
\newcommand{\ag}[1]{\textcolor{red}{#1}}

%\newtheoremstyle{aldenthm}
%{6pt} % Space above
%{6pt} % Space below
%{\itshape} % Body font
%{} % Indent amount
%{\bfseries} % Theorem head font
%{.} % Punctuation after theorem head
%{.5em} % Space after theorem head
%{} % Theorem head spec (can be left empty, meaning `normal')

%\theoremstyle{aldenthm}
%\newtheorem{theorem}{Theorem}
%\newtheorem{definition}{Definition}
%\newtheorem{lemma}{Lemma}
%\newtheorem{proposition}{Proposition}
%\newtheorem{corollary}{Corollary}

%\newtheoremstyle{aldenrmrk}
%{6pt} % Space above
%{6pt} % Space below
%{} % Body font
%{} % Indent amount
%{\itshape} % Theorem head font
%{.} % Punctuation after theorem head
%{.5em} % Space after theorem head
%{} % Theorem head spec (can be left empty, meaning `normal')
%
%\theoremstyle{aldenrmrk}
%\newtheorem{remark}{Remark}
%%%%%% End Alden

%\title{Local Spectral Clustering of Density Upper Level Sets}

% The \author macro works with any number of authors. There are two commands
% used to separate the names and addresses of multiple authors: \And and \AND.
%
% Using \And between authors leaves it to LaTeX to determine where to break the
% lines. Using \AND forces a line break at that point. So, if LaTeX puts 3 of 4
% authors names on the first line, and the last on the second line, try using
% \AND instead of \And before the third author name.
%
%\author{
%Alden Green \And
%Sivaraman Balakrishnan \And
%Ryan J. Tibshirani}
%
%\begin{document}
%\maketitle
%
%\vspace{0.1in}
%\begin{abstract}


\newcommand{\widgraph}[2]{\includegraphics[keepaspectratio,width=#1]{#2}}
\newcommand{\Like}{\ensuremath{\mathcal{L}}}
\newcommand{\Ball}[2]{\mathbb{B}_{#1}(#2)}
\newcommand{\Complement}[1]{\overline{#1}}

\newcommand{\xsam}{\ensuremath{x}}
\newcommand{\samind}{\ensuremath{\ell}}
\newcommand{\Xrv}{\ensuremath{X}}
\newcommand{\Event}{\ensuremath{\mathcal{E}}}
\newcommand{\Fevent}{\ensuremath{\mathcal{F}}}

\newcommand{\usedim}{\ensuremath{d}}
\newcommand{\mubold}{\ensuremath{\boldsymbol{\mu}}}
\newcommand{\lambold}{\ensuremath{\boldsymbol{\lambda}}}
\newcommand{\sep}{\ensuremath{\xi}}
\newcommand{\mixind}{\ensuremath{i}}
\newcommand{\mixtwo}{\ensuremath{j}}
\newcommand{\nummix}{\ensuremath{M}}
\newcommand{\mustar}{\ensuremath{\mu^*}}
\newcommand{\muboldstar}{\ensuremath{\mubold^*}}
\newcommand{\muboldt}{\ensuremath{\mubold^t}}

\newcommand{\numobs}{\ensuremath{n}}
\newcommand{\SamLike}{\ensuremath{\Like_\numobs}}
\newcommand{\PopLike}{\ensuremath{\Like}}
\newcommand{\Exs}{\E}
\newcommand{\thetanew}{\ensuremath{\theta^{\text{\small{new}} }}}

\newcommand{\stepsize}{\ensuremath{s}}


\newcommand{\QMAT}{\ensuremath{\mathbf{Q}}} 
\newcommand{\DMAT}{\ensuremath{\mathbf{D}}} 

\newcommand{\defn}{\ensuremath{: \, =}}
\newcommand{\muboldtilde}{\ensuremath{\tilde{\mubold}}}

%%% New version of \caption puts things in smaller type, single-spaced 
%%% and indents them to set them off more from the text.
\makeatletter
\long\def\@makecaption#1#2{
        \vskip 0.8ex
        \setbox\@tempboxa\hbox{\small {\bf #1:} #2}
        \parindent 1.5em  %% How can we use the global value of this???
        \dimen0=\hsize
        \advance\dimen0 by -3em
        \ifdim \wd\@tempboxa >\dimen0
                \hbox to \hsize{
                        \parindent 0em
                        \hfil 
                        \parbox{\dimen0}{\def\baselinestretch{0.96}\small
                                {\bf #1.} #2
                                %%\unhbox\@tempboxa
                                } 
                        \hfil}
        \else \hbox to \hsize{\hfil \box\@tempboxa \hfil}
        \fi
        }
\makeatother

\ShortHeadings{Local Spectral Clustering of Density Level Sets}{Green, Balakrishnan and Tibshirani}
\firstpageno{1}



%%%%%%%%%%%%%%%%%%%%%%%%%%%%%%%%%%%%%%%%%%%%%%%%%%%%%%%%%%%%%%%%%%%%%%%%%%%%%%

\begin{document}

\title{Local Spectral Clustering of Density Upper Level Sets}

\author{\name Alden Green \email ajgreen@stat.cmu.edu \\
	\addr Department of Statistics and Data Science\\
	Carnegie Mellon University\\
	Pittsburgh, PA 15213
	\AND
	\name Sivaraman Balakrishnan \email sbalakri@stat.cmu.edu \\
	\addr Department of Statistics and Data Science\\
	Carnegie Mellon University\\
	Pittsburgh, PA 15213
	\AND
	Ryan J. Tibshirani \email ryantibs@stat.cmu.edu \\
	\addr Department of Statistics and Data Science\\
	Carnegie Mellon University\\
	Pittsburgh, PA 15213}

\maketitle

\begin{abstract}
We analyze the Personalized PageRank (PPR) algorithm, a local spectral method
for clustering, which extracts clusters using locally-biased random walks around
a given seed node.  In contrast to previous work, we adopt a classical
statistical learning setup, where we obtain samples from an unknown
distribution, and aim to identify connected regions of high-density (density
clusters).  We prove that PPR, run on a neighborhood graph, extracts
sufficiently salient density clusters, that satisfy a set of natural geometric
conditions. We also show a converse result, that PPR can fail to recover
geometrically poorly-conditioned density clusters, even asymptotically. Finally,
we provide empirical support for our theory.
\end{abstract}

\begin{keywords}
	graphs, spectral clustering, density clustering, Personalized PageRank, unsupervised learning
\end{keywords}


\section{Introduction}
\label{sec: introduction}

In this paper, we consider the problem of clustering: splitting a given data set
into groups that satisfy some notion of within-group similarity and
between-group difference.  Our particular focus is on spectral clustering
methods, a family of powerful nonparametric clustering algorithms. Roughly
speaking, a spectral algorithm first constructs a geometric graph $G$, where
vertices correspond to samples, and edges correspond to proximities between
samples. The algorithm then estimates a feature embedding based on (an
appropriate) Laplacian matrix of $G$, and applies a simple clustering technique
(like $k$-means clustering) in the embedded feature space.

When applied to geometric graphs built from a large number of samples, global
spectral clustering methods can be computationally cumbersome and insensitive to
the local geometry of the underlying distribution
\citep{leskovec2010,mahoney2012}.  This has led to increased interest in
\emph{local} spectral clustering algorithms, which leverage locally-biased
spectra computed using random walks around some user-specified seed node.  A
popular local clustering algorithm is the Personalized PageRank (PPR) algorithm,
first introduced by \citet{haveliwala2003}, and then further developed by
several others
\citep{spielman2011,spielman2014,andersen2006,mahoney2012,zhu2013}.  

Local spectral clustering techniques have been practically very successful
\citep{leskovec2010,andersen2012,gleich2012,mahoney2012,wu2012}, which has led
many authors to develop supporting theory
\citep{spielman2013,andersen2009,gharan2012,zhu2013} that gives worst-case
guarantees on traditional graph-theoretic notions of cluster quality (such as
conductance).  In this paper, we adopt a classical statistical viewpoint, and
examine what the output of local clustering on a data set reveals about the
underlying density $f$ of the samples.  In particular, we examine the ability of
PPR to recover \emph{density clusters} of $f$, defined as the connected
components of the upper level set $\{x \in \Rd : f(x) \geq \lambda\}$ for some
$\lambda > 0$ (an object of central interest in the statistical clustering
literature, dating back to the work of \citet{hartigan1981}). 

\section{Background and related work}

We begin by providing some standard background on the PPR algorithm
and the density clustering setup, before turning our attention to related work
and a detailed summary of our contributions.

\subsection{PPR on a neighborhood graph} 

Let $\Xbf = \{x_1,\ldots, x_n\}$ be a sample drawn i.i.d.\ from a distribution
$\Pbb$ on $\Rd$, with density $f$.  For a radius $r > 0$, we define
$G_{n,r}=(V,E)$ to be the \emph{$r$-neighborhood graph} of $\Xbf$, an
unweighted, undirected graph with vertices $V=\Xbf$, and an edge $(x_i,x_j) \in
E$ if and only if $\norm{x_i - x_j} \leq r$, where $\norm{\cdot}$ is the
Euclidean norm. We denote by $\Abf \in \Reals^{n \times n}$ the adjacency
matrix, with entries $\Abf_{uv} = 1$ if $(u,v) \in E$ and $0$ otherwise.  We
also denote by $\Dbf$ the diagonal degree matrix, with entries $\Dbf_{uu} :=
\sum_{v \in V} \Abf_{uv}$, and by $\Ibf{}$ the $n \times n$ identity matrix.

First, we define the PPR vector $\pbf_v = \pbf(v,\alpha;G_{n,r})$, based on 
a given seed node $v \in V$ and a teleportation parameter $\alpha \in [0,1]$, to
be the solution of the following linear system:
\begin{equation}
\label{eqn: ppr_vector}
\pbf_v = \alpha \ebf{v} + (1 - \alpha) \pbf_v \Wbf,
\end{equation}
where $\Wbf = (\Ibf{} + \Dbf^{-1}\Abf)/2$ is the lazy random walk matrix over
$G_{n,r}$ and $e_{v}$ is the indicator vector for node $v$ (that has a 1 in
position $v$ and 0 elsewhere).  

Next, we define a \emph{$\beta$-sweep cut} of $\pbf_v = (\pbf_v(u))_{u \in V}$,
for a given level $\beta > 0$, as    
\begin{equation}
\label{eqn: sweep_cuts}
S_{\beta,v} := \set{u \in V: \frac{\pbf_v(u)}{\Dbf_{uu}} > \beta}.
\end{equation}
We will use the normalized cut metric to determine which sweep cut $S_{\beta}$
is the best cluster estimate. For a set $S \subseteq V$ with complement $S^c = V
\setminus S$, we define \smash{$\cut(S;G_{n,r}) := \sum_{u \in S, v \in S^c}
\Abf_{uv}$}, and \smash{$\vol(S; G_{n,r}) := \sum_{u \in S} \Dbf_{uu}$}.  We
then define the \emph{normalized cut} of $S$ as
\begin{equation}
\label{eqn: normalized_cut}
\Phi(S; G_{n,r}) := \frac{\cut(S;G_{n,r})}{\min \set{\vol(S; G_{n,r}), \vol(S^c; G_{n,r})}}.
\end{equation}
Having computed sweep cuts $S_{\beta}$ over \smash{$\beta \in (L, U)$} (where
the range $(L,U)$ is user-specified), we output the cluster estimate
\smash{$\Cest = S_{\beta^*}$} with minimum normalized cut.  For concreteness,
the PPR algorithm is summarized in Algorithm~\ref{alg: ppr}.   

\begin{algorithm}
\caption{PPR on a neighborhood graph}
\label{alg: ppr}	
{\bfseries Input:} data $\Xbf=\{x_1,\ldots,x_n\}$, radius $r > 0$, teleportation
parameter $\alpha \in [0,1]$, seed $v \in \Xbf$, sweep cut range $(L,U)$. \\     
{\bfseries Output:} cluster estimate $\Cest \subseteq V$.
\begin{algorithmic}[1]
  \STATE Form the neighborhood graph $G_{n,r}$.
  \STATE Compute the PPR vector $\pbf_v=\pbf(v, \alpha; G_{n,r})$ as in
  \eqref{eqn: ppr_vector}.  
  \STATE For \smash{$\beta \in (L,U)$}, compute sweep cuts $S_{\beta}$ as in
  \eqref{eqn: sweep_cuts}. 
  \STATE Return the cluster \smash{$\Cest = S_{\beta^*}$}, where  
  $$
  \beta^* = \argmin_{\beta \in (L,U)}~ \Phi(S_{\beta}; G_{n,r}).
  $$
\end{algorithmic}
\end{algorithm}

\subsection{Estimation of density clusters} 

Let \smash{$\Cbb_f(\lambda)$} denote the connected components of the density
upper level set $\{x \in \Rd: f(x) > \lambda\}$.  For a given density cluster
\smash{$\Cset \in \Cbb_f(\lambda)$}, we call $\Cset[\Xbf] = \Cset \cap \Xbf$ the
\emph{empirical density cluster}. The size of the symmetric set difference
between estimated and empirical cluster is a commonly used metric to quantify
cluster estimation error \citep{korostelev1993,polonik1995,rigollet2009}. We
will consider a related metric, the volume of the symmetric set difference,
which weights points according to their degree in $G_{n,r}$. To keep things simple, for a given set $S \subseteq \Xbf$ we write $\vol_{n,r}(S) := \vol(S;G_{n,r})$. 
\begin{definition}
  \label{def:volume_symmetric_set_difference}
  For an estimator \smash{$\Cest \subseteq \Xbf$} and a set 
  $\mathcal{S} \subseteq \Rd$, their symmetric set difference is 
  $$
  \Cest \vartriangle \mathcal{S}[\Xbf] :=
    \bigl(\Cest \setminus \mathcal{S}[\Xbf]\bigr) \cup
    \bigl(\mathcal{S}[\Xbf] \setminus \Cest\bigr).
  $$
  Furthermore, we denote the volume of the symmetric set difference by 
  $$
  \Delta(\Cest, \mathcal{S}[\Xbf]) := \vol_{n,r}(\Cest \vartriangle \mathcal{S}[\Xbf]). 
  $$
\end{definition}

Note that the symmetric set difference does not measure whether \smash{$\Cest$}
can (perfectly) distinguish any two distinct clusters \smash{$\Cset,\Cset' \in 
  \Cbb_f(\lambda)$}. We therefore also study a second notion of cluster 
estimation, first introduced by \citet{hartigan1981}, and defined
asymptotically.  

\begin{definition}
  \label{def: consistent_density_cluster_estimation}
  For an estimator \smash{$\Cest \subseteq \Xbf$} and cluster \smash{$\Cset \in
    \Cbb_f(\lambda)$}, we call \smash{$\Cest$} \emph{consistent} for
  \smash{$\Cset$} if for all \smash{$\Cset' \in \Cbb_f(\lambda)$} with  
  \smash{$\Cset \not= \Cset'$}, the following holds as $n \to \infty$: 
  \begin{equation}
    \label{eqn: consistent_density_cluster_recovery}
    \Cset[\Xbf] \subseteq \Cest \quad \text{and} \quad
    \Cest \cap \Cset'[\Xbf] = \emptyset,
  \end{equation}
  with probability tending to 1.
\end{definition}

Consistent cluster recovery roughly ensures that, for a given 
threshold $\lambda>0$, the estimated cluster \smash{$\Cest$} contains all points
in a true density cluster $\Cset \in \Cbb_f(\lambda)$, and simultaneously does 
not contain any points in any other density cluster $\Cset' \in
\Cbb_f(\lambda)$. 

With these definitions in place, our broad goal will be to understand the extent
to which the PPR algorithm is able to recover a cluster which either guarantees
a low symmetric set difference to a true density cluster, or which consistently
estimates a true density cluster.

\subsection{Related work}

In addition to the background on local spectral clustering given above, a
few related lines of work are worth highlighting. In the stochastic block model
(SBM), arguably one of the simplest models of network formation, edges between
nodes independently occur with probability based on a latent community
membership. In the SBM, the ability of spectral algorithms to perform
clustering---or community detection---is well-understood, dating 
back to \citet{mcsherry2001} who gives conditions under which the entire
community structure can be recovered. In more recent work, \citet{rohe2011}
upper bound the fraction of nodes misclassified by a spectral algorithm for the
high-dimensional (large number of blocks) SBM, and \citet{lei2015} extend these
results to the sparse (low average degree) regime. Relatedly,
\citet{clauset08,balakrishnan2011,li2018}, analyze the misclassification rate
when the block model exhibits some hierarchical structure. The framework we
consider, in which nodes correspond to data points sampled from an underlying 
density, and edges between nodes are formed based on geometric proximity, is
quite different than the SBM, and therefore so is our analysis.

In general, the study of spectral algorithms on neighborhood graphs has been
focused on establishing asymptotic convergence of eigenvalues and eigenvectors
of certain sample objects to the eigenvalues and eigenfunctions of corresponding
limiting operators. \citet{koltchinskii2000} establish convergence of spectral
projections of the adjacency matrix to a limiting integral operator, with
similar results obtained using simplified proofs in
\citet{rosasco10}. \citet{vonluxburg2008} studies convergence of eigenvectors of
the Laplacian matrix for a neighborhood graph of fixed radius. \citet{belkin07} and
\citet{garciatrillos18} extend these results to the regime where the radius $r
\to 0$ as $n \to \infty$.

These results are of fundamental importance; however, the behavior of the
spectra of these continuum operators can in general be hard to grasp. Therefore,
further work relating this spectra to the geometry of the underlying
distribution $\Pbb$ is of interest. In this spirit,
\citet{shi2009,schiebinger2015,garciatrillos19} examine the ability of spectral
algorithms to recover the latent labels in certain geometrically
well-conditioned nonparametric mixture models. Their results focus on global
rather than local methods, and thus impose global rather than local conditions
on the nature of the density. Moreover, they do not in general guarantee
recovery of density clusters, which is the focus in our work. Perhaps most
importantly, these works rely on general cluster saliency conditions, which
implicitly depend on many distinct geometric aspects of the cluster $\Cset$
under consideration. We make this dependence more explicit, and in doing so
expose the role each geometric condition plays in the clustering problem.

Our analysis naturally builds on a few of the aforementioned theoretical
analyses of PPR. For an arbitrary graph $G$ and subset $S \subseteq G$,
\citet{andersen2006} relate the quality of the PPR cluster \smash{$\Cest$} to
the normalized cut functional $\Phi(S;G)$. While this analysis is tight in a
worst-case sense, it fails to account for possible improvements when the cluster
$S$ is additionally assumed to be internally well-connected, which is an
intuitively more favorable case for clustering. Building on this intuition,  
\citet{zhu2013} assume that the subgraph $G[S]$ is internally
well-connected---as measured by a functional such as mixing time of a random
walk over $G[S]$---and prove upper bounds on \smash{$\vol(\Cest \vartriangle
  S;G)$}. Both of these analyses also hold with respect to an approximate form 
of PPR  (aPPR), which can be efficiently computed. 

We apply these results to our setting by carefully analyzing the normalized cut
and mixing time functionals in the particular case of $G = G_{n,r}$ and $S =
\Cset[X]$. One of our main challenges is to prove an upper bound on the mixing 
time of a random walk run only over the subset of nodes in $G_{n,r}$ which fall
within a density cluster $\Cset$. To do so, we rely on a series of seminal works
upper bounding the mixing time of \emph{geometric random walks} (see 
\citet{vempala2005} for a comprehensive review.)  This study was initiated by
\citet{dyer1991}, who used geometric random walks as a fundamental subroutine to
efficiently compute volumes of high-dimensional convex bodies. These results are
improved in \citet{lovasz1990,kannan97,kannan06}, who show, inter alia, that the
bounds on mixing time can be sharpened by avoiding so-called
``start-penalties''. As we discuss further in what follows, these improvements
are crucial to our work. Following the work of
\citet{abbasi-yadkori2016,abbasi-yadkori2016a}, we extend these results to hold
for Lipschitz deformations of convex sets. Additionally, we relate the mixing
time of these (continuous-space) geometric random walks to the mixing time of
random walks over (discrete) neighborhood graphs.

Finally, it is worth mentioning that density clustering and level set estimation
are themselves very well-studied problems in statistics. \citet{polonik1995,
  rigollet2009} study density clustering under the symmetric set difference
metric, \citet{tsybakov1997,singh2009} describe minimax optimal level-set
estimators under Hausdorff loss and
\citet{hartigan1981,chaudhuri2010,balakrishnan2013,kpotufe11} consider
consistent estimation of the cluster tree. We emphasize that our goal is
not to improve on these results, nor to offer a better algorithm for level set
estimation; indeed, seen as a density clustering algorithm, PPR has none 
of the optimality guarantees found in the aforementioned works. Instead, our
motivation is to start with a widely-used local spectral method, PPR, and to
better understand and characterize the distinctions between those density 
clusters which are well-conditioned for PPR, and those which are not.  

\subsection{Summary of results}

A summary of our results (and an outline for this paper) is as follows. 

\begin{enumerate}
\item In Section \ref{sec: consistent_cluster_estimation_with_ppr}, we introduce 
  a set of natural geometric conditions on the density cluster $\Cset$
  %formalize a measure of difficulty based on these geometric conditions, 
  and show that if Algorithm~\ref{alg: ppr} is properly initialized, then the size of
  the symmetric set difference between \smash{$\Cest$} and a thickened version
  of the density cluster $\Csig$ can be bounded in a meaningful way based on the
  geometric parameters. 
	
\item We further show in Section \ref{sec:
    consistent_cluster_estimation_with_ppr} that if the density cluster  
  $\Cset$ is particularly well-conditioned, then Algorithm~\ref{alg: ppr} 
  will consistently estimate a density cluster in the strong sense of
  \eqref{eqn: consistent_density_cluster_recovery}.   
	
\item In Section \ref{sec: analysis}, we detail some of the analysis required to 
  prove our main results, and expose the roles that various geometric quantities 
  play in the difficulty of the clustering problem.  
  
\item In Section \ref{sec:lower_bound}, we provide an accompanying lower bound,
  which demonstrates that when the cluster $\Cset$ is sufficiently poorly
  conditioned, it will not be recovered by Algorithm~\ref{alg: ppr}. 
	
\item In Section \ref{sec: experiments}, we empirically investigate the
  tightness of our analysis, and provide examples showing how violations of our  
  geometric conditions impact density cluster recovery by PPR.
\end{enumerate}

One of our main takeaways can be paraphrased as follows: PPR, run on a
neighborhood graph, recovers only \emph{geometrically compact} high-density
clusters. Our theoretical results make this takeaway precise, and provide a 
concrete way of quantifying the geometric compactness of a density 
cluster.  

\section{Main results}
\label{sec: consistent_cluster_estimation_with_ppr}

In this section, we present our main results on accuracy of the PPR algorithm
for recovering density clusters.  We begin by formally introducing various
geometric conditions, and use these to define a condition number
$\kappa(\Cset)$, which measures the difficulty PPR will have in estimating a
density cluster $\Cset$. With this condition in place our first main result
(Theorem~\ref{thm: volume_ssd}) provides a bound on the symmetric set difference
between the estimated cluster $\Cest$, obtained by an appropriately initialized 
version of the PPR algorithm, and the target cluster $\Cset$.  Our next main
result (Theorem~\ref{thm: consistent_recovery_of_density_clusters}) shows that
for sufficiently well-conditioned target clusters $\Cset$, the output of the PPR
algorithm $\Cest$ is consistent in the sense of Definition~\ref{def:
consistent_density_cluster_estimation}.

\subsection{Preliminaries}

At a high level, for PPR to be successful, the underlying density cluster must
be geometrically well-conditioned.  A basic requirement is that we need to avoid
clusters which contain arbitrarily thin bridges or spikes. As in the work of
\citet{chaudhuri2010}, we consider a thickened version of \smash{$\Cset \in
\Cbb_f(\lambda)$} defined as $\Csig := \set{x \in \Reals^d: \dist(x,\Cset) \leq
\sigma}$, which we call the \emph{$\sigma$-expansion} of $\Cset$. Here
\smash{$\dist(x,\Cset) := \inf_{y \in \Cset} \norm{y - x}$}.  We now list our
conditions on $\Csig$, which are defined with respect to $\sigma$ as well as positive numbers $\lambda_{\sigma}, \Lambda_{\sigma}, L$, $\theta \in (0,\infty)$, $\gamma \in [0,1]$, and $r \in (0,\frac{\sigma}{4d}]$. 

\begin{enumerate}[label=(A\arabic*)]
\item
  \label{asmp: bounded_density}
  \emph{Bounded density within cluster:} The density is bounded above and below in $\mc{C}_{\sigma}$:
  $$
  \lambda_{\sigma} \leq \inf_{x \in \Csig} f(x) \leq \sup_{x \in \Csig} f(x)
  \leq \Lambda_{\sigma}.
  $$

\item 
  \label{asmp: low_noise_density}
  \emph{Low noise density:} For any $x \in \Rd$ such that $0 < \dist(x, \Csig) \leq \sigma$,     
  $$
  \inf_{x' \in \Csig} f(x') - f(x) \geq  \theta \cdot \dist(x, \Csig)^{\gamma}.  
  $$
  Roughly, this assumption ensures that the density decays sufficiently quickly
  as we move away from the target cluster $\Csig$, and is a standard assumption
  in the level-set estimation literature (see for instance \citet{singh2009}).
  
\item
  \label{asmp: embedding}
  \emph{Lipschitz embedding:}
  There exists some $g: \Reals^d \to \Reals^d$ satisfying each of the following:
  \begin{enumerate}
  \item the thickened set $\Csig = g(\mathcal{K})$, where $\mathcal{K}
    \subseteq \Rd$ is a convex set with diameter $\mathrm{diam}(\mathcal{K}) = \sup_{x,y \in \mathcal{K}}\norm{x - y} \leq \diam$;
  \item $\det(\nabla g (x)) = 1$ for all $x \in \mc{K}$, where $\nabla g(x)$ is
    the Jacobian of $g$ evaluated at $x;$ and 
  \item The restriction of $g$ to $\mc{K}$ is $L$-Lipschitz, i.e.
    $$
    \norm{g(x) - g(y)} \leq L \norm{x - y} ~
    \text{for all $x,y \in \mathcal{K}$}. 
    $$
  \end{enumerate}
  Succinctly, we assume that $\Csig$ is the image of a convex set with finite
  diameter under a measure preserving, Lipschitz transformation. 

\item
  \label{asmp: bounded_volume}
  \emph{Bounded volume:}
  For a set $\mathcal{S} \subseteq \Reals^d$, define a $\Pbb$-weighted volume of 
  $\mathcal{S}$ to be  
  \begin{equation}
  \label{eqn:volume}
  \vol_{\Pbb,r}(\mathcal{S}) := \int_{\mathcal{S}} \Pbb(B(x,r)) f(x) \, dx.
  \end{equation}
  where $B(x,r)$ is the closed ball of radius $r$ centered at $x$. It holds that
  $$
  \vol_{\Pbb,r}(\Cset_\sigma) \leq \vol_{\Pbb,r}(\Reals^d \setminus \mc{C}_{\sigma}). 
  $$
\end{enumerate}

To motivate these conditions, we now give a brief high-level sketch
of our analysis (which we will return to more formally in Section~\ref{sec:  
  analysis}).  \citet{zhu2013} show that for an arbitrary graph $G = (V,E)$ and
subset of vertices $S \subseteq V$, the PPR algorithm (properly initialized
within $S$) will output an estimate \smash{$\Cest$} of $S$ satisfying, for a
constant $M>0$, 
\begin{equation}
\label{eqn: graph_symmetric_set_difference_1}
\vol(\Cest \vartriangle S; G) \leq M \cdot \Phi(S,G) \cdot \tau_{\infty}(G[S]),    
\end{equation}
where $\Phi(S;G)$ is the normalized cut of $S$ (as defined in \eqref{eqn:
  normalized_cut}), and \smash{$\tau_{\infty}(G[S])$} is  the \emph{mixing
  time} of a random walk over the induced subgraph $G[S]$ (to be defined
precisely later, in \eqref{eqn: mixing_time}).  The left-hand side in
\eqref{eqn: graph_symmetric_set_difference_1} is one of our principle metrics
of interest, the volume of the symmetric set difference, and our main goal will 
therefore be to upper bound the graph functionals $\Phi$ and $\tau_{\infty}$.
Towards this goal, as we will show in Section
\ref{sec: analysis}, the conditions \ref{asmp: bounded_density}--\ref{asmp: 
  bounded_volume} allow us to upper bound the normalized cut 
\smash{$\Phi(\Csig[\Xbf]; G_{n,r})$}, and the mixing time 
\smash{$\tau_{\infty}(G_{n,r}[\Csig[\Xbf]])$}. Specifically, assumption 
\ref{asmp: low_noise_density} yields an upper
bound on \smash{$\cut(\Csig[\Xbf]; G_{n,r})$}, and \ref{asmp: bounded_density}
yields a lower bound on \smash{$\vol_{n,r}(\Csig[\Xbf])$}; together with 
\ref{asmp: bounded_volume}, this gives an upper bound on the normalized cut.  On 
the other hand, \ref{asmp: bounded_density} and \ref{asmp: embedding} preclude
bottlenecks in the induced subgraph \smash{$G_{n,r}[\Csig[\Xbf]]$}, and combined
with the upper bound on diameter in \ref{asmp: embedding}, this leads to an upper bound on
the mixing time over this subgraph.

\subsubsection{Condition number} 

We will define the condition number $\kappa(\Cset)$ of a cluster $\Cset$ in 
terms of a suitable upper bound---expressed in terms of the geometric parameters
from \ref{asmp: bounded_density}--\ref{asmp: bounded_volume}---on the product of
normalized cut and mixing time, \smash{$\Phi(\Csig[\Xbf]; G_{n,r}) \cdot
  \tau_{\infty}(G_{n,r}[\Csig[\Xbf]])$}. Following \eqref{eqn:
  graph_symmetric_set_difference_1}, we see that the smaller the condition
number $\kappa(\Cset)$ is, the more success PPR will have in recovering the
target $\Cset$.  

\begin{definition}
  \label{def:well_conditioned_density_cluster}
  For $\lambda > 0$ and \smash{$\Cset \in \Cbb_f(\lambda)$}, let $\Cset$ satisfy  
  \ref{asmp: bounded_density}--\ref{asmp: bounded_volume} for some $\sigma, \lambda_{\sigma}, \Lambda_{\sigma}, L > 0$,  $\theta \in (0,\infty)$, $\gamma \in [0,1]$ and $r \in (0,\sigma/4d]$. Then, for positive universal constants $c_1$-$c_5$ to be specified later
  (in Theorems \ref{thm: conductance_upper_bound} and \ref{thm: mixing_time_upper_bound}), define
  \begin{align}
    \label{eqn: condition_number}
    \Phibf(\Cset) 
    & := c_1 r \frac{d}{\sigma} \frac{\lambda}{\lambda_{\sigma}}
    \frac{(\lambda_{\sigma} - \theta \frac{r^{\gamma}}{\gamma +
        1})}{\lambda_{\sigma}} \\
    \taubf(\Cset) & := c_2 \frac{\Lambda_{\sigma}^4 (d + 2)^3 
    	\rho^2 L^2}{\lambda_{\sigma}^4 r^2} \ln^2\Bigl(c_5^{1/d}\frac{\Lambda_{\sigma}^{2/d}L\rho}{2\lambda_{\sigma}^{2/d}r}\Bigr)
  \end{align}
  Letting \smash{$\kappa(\Cset) := \Phibf(\Cset) \cdot \taubf(\Cset)$},
  we call $\kappa(\Cset)$ the \emph{condition number} of $\Cset$.  We also call
  the set $\Cset$ a \emph{$\kappa$-well-conditioned} density cluster.  
\end{definition}

The condition number $\kappa(\Cset)$ succinctly captures the role of the various 
geometric parameters.\footnote{Technically speaking, the condition number depends on the numbers $\sigma, \lambda_{\sigma}, \Lambda_{\sigma},L > 0$, $\theta \in (0,\infty)$, $\gamma \in [0,1]$, and $r \in (0,\sigma/(4d))$ rather than on $\mc{C}$. However, we use the notation $\kappa(\mc{C})$ to make clear that the purpose of $\kappa$ is to measure the quality of the cluster $\mc{C}$.}  We note in passing that $\Phibf(\Cset)$ and
$\taubf(\Cset)$ are essentially the upper bounds  on \smash{$\Phi(\Csig[\Xbf];
  G_{n,r})$} and \smash{$\tau_{\infty}(G_{n,r}[\Csig[\Xbf]])$} that we derive in
our analysis later, in Section \ref{sec: analysis}. 

\subsubsection{Well-initialized algorithm} 

As is typical in the local clustering literature, our algorithmic results will
be stated with respect to specific ranges of each of the user-specified
parameters. In particular, for a well-conditioned density cluster $\Cset$, we
require that some of the tuning parameters of Algorithm~\ref{alg: ppr} are
chosen to fall within specific ranges,

\begin{equation}
\begin{gathered}
\label{eqn: initialization}
0 < r \leq \frac{\sigma}{4d}, \quad 
\alpha \in {\textstyle [\frac{1}{3}, \frac{5}{12})} \cdot
\frac{1}{\taubf(\mc{C})}, \\ 
(L,U) \subseteq {\textstyle(\frac{1}{50},\frac{1}{5})} \cdot 
\frac{1}{2{n \choose 2} \vol_{\Pbb,r}(\Csig)}.  
\end{gathered}
\end{equation}

\begin{definition}
If the input parameters to Algorithm \ref{alg: ppr} satisfy \eqref{eqn:
  initialization} (for some well-conditioned density cluster $\Cset$), then we  
say the algorithm is \emph{well-initialized}. 
\end{definition}

In practice of course, it is not feasible to set tuning parameters based on the 
underlying (unknown) density $f$. Typically, one runs PPR over some range of
tuning parameter values and selects the cluster which has the smallest
normalized cut.     

\subsection{Cluster recovery in symmetric set difference}

We now present our first main result: a bound on the volume of the symmetric set
difference between the estimated cluster \smash{$\Cest$} and empirical cluster 
$\Csig[\Xbf]$. In this theorem, and hereafter, we let $c_i > 0$ denote
universal constants, and $b_i > 0$ denote constants which may depend on  
$\Pbb,\lambda,r,d$ and so on, but \emph{not} on the sample size 
$n$.

\begin{theorem}
  \label{thm: volume_ssd}
  Fix $\lambda,\sigma > 0$, let \smash{$\Cset \in \Cbb_f(\lambda)$} be a   
  $\kappa$-well-conditioned density cluster, and assume Algorithm \ref{alg: ppr}
  is well-initialized with respect to $\Cset$. Then for any 
  \begin{equation*}
  n \geq b_0 \cdot 
  \begin{cases*}
  \log(n)^{3/2},& ~~\textrm{if $d = 2$,} \\
  \log(n),& ~~\textrm{if $d \neq 2$}
  \end{cases*}
  \end{equation*}
  there exists a set $\Csig[\Xbf]^g 
  \subseteq \Csig[\Xbf]$ of large volume, \smash{$\vol_{n,r}(\Csig[\Xbf]^g)
    \geq \vol_{n,r}(\Csig[\Xbf])/2$}, such that the following holds: if 
  Algorithm~\ref{alg: ppr} is run with any seed node $v \in \Csig[\Xbf]^g$, then
  the PPR estimated cluster \smash{$\Cest$} satisfies
  \begin{equation}
    \label{eqn: volume_ssd}
    \Delta(\Cest, \Csig[\Xbf]) \leq c_0 \cdot \kappa(\Cset) \cdot
    \vol_{n,r}(\Csig[\Xbf]),  
  \end{equation}
  with probability at least $1 - b_0/n - c_0n\exp\{-b_{11}n\}$. 
\end{theorem}

The proof of Theorem \ref{thm: volume_ssd}, as with all results in this paper,
is deferred to the appendix.  We reiterate that the primary technical 
work involved in proving Theorem~\ref{thm: volume_ssd} involves showing that
$\Phibf(\Cset)$ and $\taubf(\Cset)$ in \eqref{eqn: condition_number} are
valid upper bounds on the normalized cut and mixing time; once this has been 
established, the result follows more or less straightforwardly
from \citet{zhu2013}.  
% This result establishes that the volume of the symmetric
% set difference \smash{$\Delta(\Cest, \Csig[\Xbf])$} is upper-bounded by a  
% quantity proportional to the difficulty of the clustering problem, as measured 
% by the condition number $\kappa(\Cset)$. 

\subsection{Consistent cluster recovery}

The bound on symmetric set difference \eqref{eqn: volume_ssd} does not imply 
consistent density cluster estimation in the sense of \eqref{eqn:
  consistent_density_cluster_recovery}. This notion of consistency requires a
uniform bound over the PPR vector $\pbf_v$: as an example, suppose that we were
able to show that for all \smash{$\Cset' \in \Cbb_f(\lambda), \Cset' \neq
  \Cset$}, and each $u \in \Cset, w \in \Cset'$,  
\begin{equation}
\label{eqn: ppr_gap}
\frac{\pbf_v(w)}{D_{ww}} \leq \frac{1}{100 {n \choose 2} \vol_{\Pbb,r}(\Csig)} <
\frac{1}{10 {n \choose 2} \vol_{\Pbb,r}(\Csig)} \leq \frac{\pbf_v(u)}{D_{uu}}. 
\end{equation}
Then, any $(L,U)$ satisfying~\eqref{eqn: initialization} and any sweep cut
$S_{\beta}$ for $\beta \in (L,U)$ would fulfill both conditions laid out in
\eqref{eqn: consistent_density_cluster_recovery}. In Theorem 
\ref{thm: consistent_recovery_of_density_clusters}, we show that a sufficiently 
small upper bound on $\kappa(\Cset)$ ensures that with high probability the
uniform bound~\eqref{eqn: ppr_gap} is satisfied, and hence implies
\smash{$\Cest$} will be a consistent  estimator. We will need one additional 
regularity condition, to preclude arbitrarily low degree vertices for points $x
\in \Cset'[\Xbf]$.  

\begin{enumerate}[label=(A\arabic*)]
  \setcounter{enumi}{4}
\item 
  \label{asmp: C'_bounded_density}
  \emph{Bounded density in other clusters:} Letting $\sigma,\lambda_{\sigma}$ be   
  as in \ref{asmp: bounded_density}, for each $\Cset' \in \Cbb_f(\lambda)$ and
  for all $x \in \Csig'$, it holds that $\lambda_{\sigma} \leq f(x)$. 
\end{enumerate}

Next we give our main result on consistent cluster recovery by PPR.

\begin{theorem}
  \label{thm: consistent_recovery_of_density_clusters}
  Under the assumptions of Theorem \ref{thm: volume_ssd}, additionally assume 
  \ref{asmp: C'_bounded_density}, and 
  \begin{equation}
    \label{eqn: kappa_ub}
    \kappa(\Cset) \leq c \frac{(\lambda_{\sigma} r^d
      \nu_d)^2}{\vol_{\Pbb,r}(\Csig)}.
  \end{equation}
  Then for any
  $$
  n \geq b_1 (\log n)^{\max\{\frac{3}{d},1\}},
  $$
  there exists a set $\Csig[\Xbf]^g \subseteq \Csig[\Xbf]$ of large volume, 
  \smash{$\vol_{n,r}(\Csig[\Xbf]^g) \geq
    \vol_{n,r}(\Csig[\Xbf])/2$}, such that if Algorithm~\ref{alg:
    ppr} is run with any seed node $v \in \Csig[\Xbf]^g$, then the PPR estimated cluster \smash{$\Cest$} satisfies \eqref{eqn: consistent_density_cluster_recovery} with probability at least $1-b_2/n$.
\end{theorem}

Some remarks are in order.

\begin{remark}
We note that the restriction on $\kappa(\Cset)$ imposed by \eqref{eqn: kappa_ub} results in an upper bound on the symmetric set difference metric
  \smash{$\Delta(\Cest, \Csig[\Xbf])$} on the order of $r^d$. In plain terms, we 
  are able to recover a  density cluster $\Cset$ in the strong sense of
  \eqref{eqn: consistent_density_cluster_recovery} only when we can guarantee a
  very small fraction of points will be misclassified. This strong condition is
  the price we pay in order to obtain the uniform bound in~\eqref{eqn: ppr_gap}.  
\end{remark}

\begin{remark}
  Letting the radius of the neighborhood graph shrink, $r \to 0$ as $n \to  
  \infty$, would be computationally attractive (it would ensure that the graph 
  $G_{n,r}$ is sparse), but the presence of a factor of \smash{$\log^2(1/r)/r$}
  in $\kappa(\Cset)$ prevents us from making claims about the behavior of PPR in
  this regime. Although the restriction to a kernel function fixed in $n$ is
  common in spectral clustering theory \citep{schiebinger2015,vonluxburg2008},
  it is an interesting question whether PPR exhibits some degeneracy over 
  neighborhood graphs as $r \to 0$, or if this is merely looseness in our  
  upper bounds.
\end{remark}

\subsection{Approximate PPR vector} 

In practice, exactly solving the system of equations~\eqref{eqn:
  ppr_vector} to compute the PPR vector 
  may be too computationally expensive. To address this limitation,
\citet{andersen2006} introduced the \emph{$\varepsilon$-approximate} PPR vector
(aPPR), which we will denote by \smash{$\pbf^{(\varepsilon)}$}. We refer the
curious reader to \citet{andersen2006} for a formal algorithmic definition of
the aPPR vector, and limit ourselves to highlighting a few salient points: the
aPPR vector can be computed in order $\mathcal{O}(1/(\varepsilon\alpha))$ time,
while satisfying the following uniform error bound: 
\begin{equation}
\label{eqn: appr_error}
\textrm{for all $u \in V$}, \quad \pbf(u) - \varepsilon \Dbf_{uu}\leq
\pbf^{(\varepsilon)}(u) \leq \pbf(u).  
\end{equation}
For a sufficiently small choice of $\varepsilon$, the 
application of \eqref{eqn: appr_error} within the proofs of Theorems
\ref{thm: volume_ssd} and \ref{thm: consistent_recovery_of_density_clusters}  
leads to analogous results which hold for \smash{$\pbf^{(\varepsilon)}$}.

\begin{corollary}
  \label{cor: appr}
  Consider instead of
  Algorithm \ref{alg: ppr} using the approximate PPR vector from
  \citet{andersen2006} satisfying \eqref{eqn: appr_error}, and forming the 
  corresponding cluster estimate \smash{$\Cest$} in the same manner.  Then 
  provided we take 
  \begin{equation}
    \label{eqn: appr_parameter}
    \varepsilon = \frac{1}{25} \vol_{\Pbb,r}(\Csig),
  \end{equation}
  under the assumptions of Theorem~\ref{thm: volume_ssd} the upper bound on symmetric set difference in \eqref{eqn: volume_ssd} still
  holds, and under the assumptions of Theorem \ref{thm: consistent_recovery_of_density_clusters} the set inclusion and disjointedness statements in \eqref{eqn: consistent_density_cluster_recovery} still hold, each with probability at least $1 - b_2/n$ (under possibly different choices of the constants $c_i$ and $b_i$).
\end{corollary}

\section{Analysis overview}
\label{sec: analysis}

The primary technical contribution in our work is to show that the
geometric assumptions \ref{asmp: bounded_density}--\ref{asmp: bounded_volume} 
translate to  meaningful bounds on the normalized cut and mixing time of 
$\Csig[\Xbf]$ in the neighborhood graph $G_{n,r}$. In doing so, we elucidate 
how these geometric conditions contribute to the difficulty of the clustering 
problem.   

\subsection{Upper bound on normalized cut} 

We start with a finite-sample upper bound on the normalized cut \eqref{eqn:
  normalized_cut} of $\Cset_\sigma[\Xbf]$. For simplicity, we write
$\Phi_{n,r}(\Csig[\Xbf]) := \Phi(\Csig[\Xbf]; G_{n,r})$.  

\begin{theorem}
  \label{thm: conductance_upper_bound}
  Fix $\lambda,\sigma>0$, and assume $\Cset \in \Cbb_f(\lambda)$ satisfies    
  Assumptions \ref{asmp: bounded_density}, \ref{asmp: low_noise_density}, and   
  \ref{asmp: bounded_volume}. Then for any $\delta \in (0,1/2)$, it holds that
  \begin{equation}
    \label{eqn: conductance_additive_error_bound}
    \frac{\Phi_{n,r}(\Csig[\Xbf])}{r} \leq (1 + 4\delta) c_1\frac{dr}{\sigma}
    \frac{\lambda}{\lambda_{\sigma}} \frac{(\lambda_{\sigma} -
      \theta\frac{r^{\gamma}}{\gamma+1})}{\lambda_{\sigma}},
  \end{equation}
  with probability at least $1 - 3\exp\set{-b_{10} n \delta^2}$. 
\end{theorem}

\begin{remark}
  Observe that the diameter $\rho$ is absent from Theorem \ref{thm:
    conductance_upper_bound}, in contrast to the ultimate bound in Theorem    
  \ref{thm: volume_ssd} where the diameter enters through the condition number 
  $\kappa(\Cset)$, which worsens (increases) as $\rho$ increases. This reflects
  (what may be regarded as) established wisdom regarding spectral partitioning
  algorithms more generally \citep{guattery1995, hein2010}, albeit newly applied
  to the density clustering setting: if the diameter $\rho$ is large, then PPR
  may fail to recover $\Csig[\Xbf]$ even when $\Cset$ is sufficiently
  well-conditioned to ensure that $\Csig[\Xbf]$ has a small normalized cut in 
  $G_{n,r}$. This will be supported by simulations in Section \ref{sec:
    experiments}.   
\end{remark}

\subsection{Upper bound on mixing time} 

For $S \subseteq V$, denote by $G[S] = (S, E_S)$ the
subgraph induced by $S$ (where $E_S = E \cap (S \times S)$). Let 
$\Wbf_S$ be the lazy random walk matrix over $G[S]$, and write  
$$
q_{v}^{(t)}(u) = e_v\Wbf_S^t e_u
$$
for the $t$-step transition probability of the random walk over $G[S]$
originating at $v \in V$. Also let \smash{$\pi = (\pi(u))_{u \in S}$} be 
the stationary distribution of this random walk. (As $\Wbf_S$ is the transition 
matrix of a lazy random walk, it is well-known that a unique stationary
distribution exists and is given by  \smash{$\pi(u) = \deg(u;G[S])/\vol(S;
  G[S])$}, where we write $\deg(u;G[S]) = \sum_{w \in S} \1((u,w) \in E_S)$ for
the degree of $u$ in $G[S]$.) We define the \emph{mixing time} of $G[S]$ as 
\begin{equation}
\label{eqn: mixing_time}
\tau_{\infty}(G[S]) = \min\set{ t: \frac{\pi(u) - q_{v}^{(t)}(u)}
  {\pi(u)} \leq \frac{1}{4}, \; \text{for $u,v \in V$}}. 
\end{equation}
Next, we give an asymptotic (in the number of samples $n$) upper bound on
$\tau_{\infty}(G_{n,r}[\Csig[\Xbf]])$.  

\begin{theorem}
  \label{thm: mixing_time_upper_bound}
  Fix $\lambda, \sigma > 0$, and assume \smash{$\Cset \in \Cbb_f(\lambda)$} 
  satisfies Assumptions \ref{asmp: bounded_density} and \ref{asmp: embedding}.
  Also assume that \smash{$0 < r \leq \sigma/(4d)$}. Then for any \smash{$\delta \in \bigl(0,1/(8 + 16/d)\bigr)$}, and any
  \begin{equation}
  \label{eqn: mixing_time_sample_size}
  n \geq b_1 \cdot (1 + 4\delta) \cdot 
  \begin{cases*}
  \log(n)^{3/2},& ~~\textrm{if $d = 2$,} \\
  \log(n),& ~~\textrm{if $d \neq 2$}
  \end{cases*}
  \end{equation}
  the mixing time satisfies
  \begin{equation} 
    \label{eqn: mixing_time_upper_bound}
    \tau_{\infty}\bigl(G_{n,r}[\Csig[\Xbf]]\bigr) \leq (1 + c_3\delta) c_2 \frac{\Lambda_{\sigma}^4 (d + 2)^3 
      \rho^2 L^2}{\lambda_{\sigma}^4 r^2} \ln^2\Bigl(c_5^{1/d}\frac{\Lambda_{\sigma}^{2/d}L\rho}{2\lambda_{\sigma}^{2/d}r}\Bigr)
  \end{equation}
  with probability at least $1 - b_5/n - 4n\exp\{-\delta^2 b_2 n\}$.
\end{theorem}
The proof of Theorem \ref{thm: mixing_time_upper_bound} relies heavily on
analogous mixing time bounds developed for a continuous-space ``ball walk'' over
convex sets. To the best of our knowledge, our result is the first bound on the
mixing time of random walks over neighborhood graphs that is independent of $n$,
the number of vertices.    

\begin{remark}
  The embedding assumption \ref{asmp: embedding} and Lipschitz parameter $L$
  play an important role in proving the upper bound in Theorem \ref{thm:
    mixing_time_upper_bound}. There is some interdependence between $L$ and
  $\sigma,\rho$, which might lead one to hope that \ref{asmp: embedding} is
  non-essential. However, it is not possible to eliminate condition \ref{asmp:
    embedding} without incurring an additional factor of at least
  $(\rho/\sigma)^d$ in \eqref{eqn: mixing_time_upper_bound}, achieved, for
  instance, when $\Csig$ is a dumbbell-like set consisting of two balls of
  diameter $\rho$ linked by a cylinder of radius
  $\sigma$. \citet{abbasi-yadkori2016, abbasi-yadkori2016a}  
  develop theory regarding Lipschitz deformations of convex sets, wherein it
  is observed that star-shaped sets as well as half-moon shapes of the type we
  consider in Section \ref{sec: experiments} both satisfy \ref{asmp: embedding}
  for reasonably small values of $L$. 
\end{remark}

\section{Negative result}
\label{sec:lower_bound}

In this section, we exhibit a hard case for density clustering using PPR, that 
is, a distribution $\Pbb$ for which PPR is unlikely to recover a density
cluster. Let \smash{$\Cset^{(0)}, \Cset^{(1)}, \Cset^{(2)}$} be rectangles in
$\Reals^2$,    
$$
\Cset^{(0)} = \biggl[-\frac{\sigma}{2}, \frac{\sigma}{2}\biggr] \times 
\biggl[-\frac{\rho}{2}, \frac{\rho}{2}\biggr], \quad 
\Cset^{(1)} = \Cset^{(0)} - \set{(\sigma,0)}, \quad
\Cset^{(2)} = \Cset^{(0)} + \set{(\sigma,0)},
$$
where $0 < \sigma < \rho$, and let $\Pbb$ be the mixture distribution over
\smash{$\mathcal{X} = \Cset^{(0)} \cup \Cset^{(1)} \cup \Cset^{(2)}$} given by   
$$
\Pbb = \frac{1 - \epsilon}{2} \Psi_1 + \frac{1 - \epsilon}{2} \Psi_2 +
\frac{\epsilon}{2} \Psi_0, 
$$
where $\Psi_k$ is the uniform distribution over $\Cset^{(k)}$ for $k = 0,1,2$.  
The density function $f$ of $\Pbb$ is simply
\begin{equation}
\label{eqn:lb_density}
f(x) = \frac{1}{\rho\sigma}\left(\frac{1 - \epsilon}{2}\1(x \in
  \Cset^{(1)}) + \frac{1 - \epsilon}{2}\1(x \in \Cset^{(2)}) +
  \frac{\epsilon}{2}\1(x \in \Cset^{(0)})  \right), 
\end{equation}
so that for any $\epsilon < \lambda < (1 - \epsilon)/2$, we have 
\smash{$\Cbb_f(\lambda) = \set{\Cset^{(1)}, \Cset^{(2)}}$}. Figure~\ref{fig:
  hard_case} visualizes the density $f$ for two different choices of $\epsilon,
\sigma, \rho$.  

% PUT FIGURE BACK HERE

\subsection{Lower bound on symmetric set difference}

As the following theorem demonstrates, even when Algorithm~\ref{alg: ppr} is
reasonably initialized, if the density cluster \smash{$\Cset^{(1)}$} is 
sufficiently geometrically ill-conditioned (in words, tall and thin) the cluster 
estimator $\Cest$ will fail to recover \smash{$\Cset^{(1)}$}. Let
\begin{equation}
\label{eqn:lower_set}
\mathcal{L} = \set{(x_1,x_2) \in \mathcal{X}: x_2 < 0}.
\end{equation}

\begin{theorem}
  \label{thm:ppr_lb}
  Assume Algorithm~\ref{alg: ppr} is initialized using inputs $r < 
  \min\set{\frac{1}{40}\rho, \frac{1}{4}\sigma}$, $\alpha = 65  
  \Phi_{\Pbb}(\mathcal{L})$, and $(L,U) = (0,1)$.  Then, for any 
  $$
  n \geq \max\set{\frac{64}{\epsilon^2 \rho \sigma \pi r^2},
    \frac{8}{\epsilon}}, 
  $$
  there exists a set $\Cset[X]^g$ of large volume, \smash{$\vol_{n,r}(\Cset[X]^g \cap 
    \Cset^{(1)}[\Xbf]) \geq \vol_{n,r}(\Cset^{(1)}[\Xbf];G_{n,r})/10$}, such
  that for any seed node $v \in \Cset[X]^g$, the PPR estimated cluster
  \smash{$\Cest$} satisfies    
  \begin{equation}
    \label{eqn:ppr_lb}
    \frac{\sigma \rho}{r^2 n^2} \cdot \vol_{n,r}(\Cest \vartriangle
    \Cset^{(1)}[\Xbf]) \geq \frac{1}{4} -  c
    \frac{\sqrt{\sigma/\rho}}{\epsilon^2} \sqrt{ \log\left(\frac{\rho \sigma}
        {\epsilon^2 r^2}\right) \frac{\sigma}{r}},   
  \end{equation}
  with probability at least $1 - b_1 n \exp\set{-b_2 n}$.  Consequently, if
  $$
  \epsilon^2 > \frac{c}{8} \sqrt{\frac{\sigma}{\rho}} \cdot \sqrt{ 
    \log\left(\frac{\rho \sigma}{\epsilon^2 r^2}\right)\frac{\sigma}{r}}, 
  $$
  then with high probability \smash{$\frac{\sigma
    \rho}{r^2 n^2} \cdot \vol_{n,r}(\Cest \vartriangle \Cset^{(1)}[\Xbf]) 
  \geq 1/8$}.    
\end{theorem}

Note that $\vol_{n,r}(\Cset^{(1)}[\Xbf])$ for large enough $n$ will be of the order
$n^2 r^2/(\sigma\rho)$, and therefore the quantity \smash{$\frac{\sigma
    \rho}{r^2 n^2} \cdot \vol_{n,r}(\Cest \vartriangle \Cset^{(1)}[\Xbf])$} in 
\eqref{eqn:ppr_lb} is comparable to \smash{$\vol_{n,r}(\Cest
  \vartriangle \Cset^{(1)}[\Xbf]) / \vol_{n,r}(\Cset^{(1)}[\Xbf])$}, which
corresponds to the quantity we upper bound in Theorem~\ref{thm: volume_ssd}.   

Theorem~\ref{thm:ppr_lb} is stated with respect to a particular hard case, where
the density clusters are rectangular subsets of $\Reals^2$.  We chose this
setting to make the theorem simple to state, and our results are generalizable
to $\Reals^d$ and to non-rectangular clusters.  Moreover, although we state
our lower bound with respect to PPR run on a neighborhood graph, the conclusion is
likely to hold for a much broader class of spectral clustering algorithms. In
the proof of Theorem~\ref{thm:ppr_lb}, we rely heavily on the fact that when
$\epsilon^2$ is sufficiently greater than $\sigma/\rho$, the normalized cut of
$\Cset^{(1)}$ will be much larger than that of $\mathcal{L}$. In this case, not
merely PPR but any algorithm that approximates the minimum normalized cut is
unlikely to recover $\Cset^{(1)}$. In particular, local spectral clustering
algorithms based on truncated random walks \citet{spielman2013}, global spectral
clustering algorithms \citet{shi00}, and $p$-Laplacian based spectral embeddings
\citet{hein2010} all have provable upper bounds on the normalized cut of cluster
they output, and thus we expect that they would all fail to estimate
$\Cset^{(1)}$.

\subsection{Comparison \ag{between upper and lower bounds}.}

To better digest the implications of Theorem~\ref{thm:ppr_lb}, we translate the
results of our upper bound in Theorem~\ref{thm: volume_ssd} to the density $f$
given in \eqref{eqn:lb_density}. Observe that $\Cset^{(1)}$ satisfies each of
the Assumptions~\ref{asmp: bounded_density}--\ref{asmp: bounded_volume}:

\begin{enumerate}[label=(A\arabic*)]
\item The density $f(x) = \frac{1 - \epsilon}{2 \rho \sigma}$ for all $x \in
  \Cset^{(1)}$.  
\item The density $f(x) = \frac{\epsilon}{\rho\sigma}$ for all $x$ such
  that $0 < \dist(x,\Cset^{(1)}) \leq \sigma$. Therefore for all such $x$, 
  $$
  \inf_{x' \in \Cset^{(1)}} f(x') - f(x)  = \left\{\frac{1 - \epsilon}{2} -
    \epsilon \right\} \frac{1}{\rho \sigma},
  $$
  which meets the decay requirement with exponent $\gamma=0$. 
\item The set $\Cset^{(1)}$ is itself convex, and has diameter $\rho$.
\item By symmetry, \smash{$\vol_{\Pbb,r}(\Cset^{(1)}) =
    \vol_{\Pbb,r}(\Cset^{(2)})$}, and therefore
  \smash{$\vol_{\Pbb,r}(\Cset^{(1)}) \leq \frac{1}{2}\vol_{\Pbb,r}(\Reals^d)$}.   
\end{enumerate}

\begin{remark}
Technically, the rectangles $\Cset^{(0)},\Cset^{(1)},\Cset^{(2)}$ are not
$\sigma$-expansions due to their sharp corners. To fix this, one can   
%either modify the upper bound (specifically, Lemma~\ref{lem: expansion_volume}
%in the proof of Theorem~\ref{thm: conductance_upper_bound}) to hold with
%respect to rectangles of width $\sigma$, or  
simply modify these sets to have appropriately rounded corners, and our lower
bound arguments do not need to change significantly, subject to some
additional bookkeeping.  Thus we ignore this technicality in our subsequent
discussion. 
\end{remark}
 
If the user-specified parameters are initialized according to~\eqref{eqn:
  initialization}, we may apply Theorem~\ref{thm: volume_ssd}. This implies that 
there exists a set $\Cset^{(1)}[\Xbf] \subseteq \Cset^{(1)}$ with
\smash{$\vol_{n,r}(\Cset[\Xbf]^g) \geq \frac{1}{2}\vol_{n,r}(\Cset[\Xbf])$} such
that for any seed node $v \in \Cset^{(1)}[\Xbf]$, and for large enough $n$, the
PPR estimated cluster $\Cest$ satisfies with high probability
$$
\vol_{n,r}(\Cest \vartriangle \Cset^{(1)}[\Xbf]) \leq c \cdot
\kappa(\Cset^{(1)}) \cdot \vol_{n,r}(\Cset^{(1)}[\Xbf]), 
$$
where the condition number may be taken to be
$$
\kappa(\Cset^{(1)}) = c_1 \frac{\epsilon}{\sigma} 
\left(\frac{\rho^2}{r} \log^2\left(\frac{1}{r}\right) \right) + c_2. 
$$
for universal constants $c_1,c_2>0$. To facilitate comparisons between our upper
and lower bounds, assume $\sigma/4 \leq \rho/40$ and set $r = \sigma/4$.  Then
the following statements each hold with high probability. 

\begin{itemize}
\item If the user-specified parameters satisfy~\eqref{eqn: initialization}, and
  for some $c > 0$
  $$
  \epsilon < c \left(\frac{\sigma}{\rho \log(1/\sigma)}\right)^2,
  $$
  then \smash{$\Delta(\Cest, \Cset^{(1)}[\Xbf]) \leq c \cdot
    \vol_{n,r}(\Cset^{(1)}[\Xbf])$}.

\item If the user-specified parameters are as in Theorem~\ref{thm:ppr_lb},
  and for some $c>0$
  $$
  \epsilon > c \left({\frac{\sigma}{\rho}} \log^2 \left(\frac{\rho}
      {\epsilon^2 \sigma}\right)\right)^{1/4},
  $$
  then \smash{$\Delta(\Cest, \Cset^{(1)}[\Xbf]) \geq \frac{1}{8}
    \vol_{n,r}(\Cset^{(1)}[\Xbf])$}. 
\end{itemize}

Jointly, these upper and lower bounds give a relatively precise characterization
of what it means for a density cluster to be well- or poorly-geometrically
conditioned for recovery using PPR. 

\begin{remark}
It is worth pointing out that the above conclusions are reliant on specific
(albeit reasonable) ranges and choices of input parameters, which in some
instances differ between the upper and lower bounds. We suspect that our lower
bound continues to hold even when choosing input parameters as dictated by our
upper bound, but do not pursue the details.
\end{remark}

\begin{remark}
It is not hard to show that, in the example under consideration, classical
plug-in density cluster estimators can consistently recover the
$\sigma$-expansion $\Csig$ of a density cluster $\Cset$, even if $\epsilon$ is
large compared to $\sigma/\rho$. That PPR has trouble recovering density
clusters here (where standard plug-in approaches do not) is not meant to
be a knock on PPR. Rather, it simply reflects that while classical density
clustering approaches are specifically designed to identify high-density regions
regardless of their geometry, PPR relies on geometry as well as density when
forming the output cluster. 
\end{remark}

\section{Experiments}
\label{sec: experiments}

We provide numerical experiments to investigate the tightness of our bounds on
the normalized cut and mixing time of $\Csig[\Xbf]$, and examine the performance
of PPR on the ``two moons'' dataset. We defer details of the experimental
settings to the appendix.   

% PUT FIGURE BACK HERE

\paragraph{Validating theoretical bounds.}  We investigate the tightness of
Theorems \ref{thm: conductance_upper_bound} and \ref{thm:
  mixing_time_upper_bound} via simulation. Figure \ref{fig:bounds} compares our
upper bounds with the actual empirically-computed quantities \eqref{eqn:
  normalized_cut} and \eqref{eqn: mixing_time}, as we vary the diameter $\rho$
and thickness $\sigma$ of a cluster $\Cset$. The top left and top middle panels
display the resulting empirical clusters for two different values of
$\rho,\sigma$. 

The bottom left and bottom right panels assure that our mixing  
time upper bounds track closely the empirical mixing time, in both 2 and 3 
dimensions.\footnote{We rescaled all values of theoretical upper
  bounds by a constant, to mask the effect of large universal constants
  in these bounds. Therefore only the comparison of slopes, rather than
  intercepts, is meaningful.} This provides empirical evidence that Theorem
\ref{thm: mixing_time_upper_bound} has the right dependency on both expansion
parameter $\sigma$ and diameter $\rho$. The story for the normalized cut panels
is less obvious. We remark that while, broadly speaking, the trends do not
appear to match, this gap between theory and empirical results seems largest
when $\sigma $ and $\rho$ are approximately equal. As the ratio $\rho/\sigma$
grows, the slopes of empirical and theoretical curves become more similar.

\paragraph{Empirical behavior of PPR.} In Figure \ref{fig:moons}, to drive home 
the implications of Theorems \ref{thm: volume_ssd} and \ref{thm:
consistent_recovery_of_density_clusters}, we show the behavior of PPR,
normalized cut, and the density clustering algorithm of \citet{chaudhuri2010} on
the well-known ``two moons'' dataset (with added 2d Gaussian noise), considered
a prototypical success story for spectral clustering algorithms. The first
column shows the empirical density clusters $\Cset[\Xbf]$ and $\Cset'[\Xbf]$ for
a particular threshold $\lambda$ of the density function; the second column
shows the cluster recovered by PPR; the third column shows the global minimum
normalized cut, computed according to the algorithm of \citet{szlam2010}; and
the last column shows a cut of the density cluster tree estimator of
\citet{chaudhuri2010}.  We can see the degrading ability of PPR to recover
density clusters as the two moons become less well-separated. Of particular
interest is the fact that PPR fails to recover one of the moons even when
normalized cut still succeeds in doing so. Additionally, we note that the
Chaudhuri-Dasgupta algorithm succeeds even when both PPR and normalized cut
fail.  This supports our main message, which is that PPR recovers only
geometrically well-conditioned density clusters.

% PUT FIGURE BACK HERE

\section{Discussion}
\label{sec: discussion}

There are an almost limitless number of ways to define what the ``right''
clustering is. In this paper, we have considered one such notion---density upper
level sets---and have detailed a set of natural geometric criteria which, when 
appropriately satisfied, translate to provable bounds on estimation of the
cluster by PPR. We have also exhibited a hard case, showing that when a density
cluster is sufficiently geometrically ill-conditioned, PPR can fail to recover
it. Finally, we have empirically demonstrated the tightness of our analysis for
reasonable sample sizes.  

\section*{Acknowledgements}

SB is grateful to Peter Bickel, Martin Wainwright, and Larry Wasserman for
helpful and inspiring conversations. This work was supported in part by the NSF grant DMS-1713003.

\appendix

\section{Proof of Theorem \ref{thm: conductance_upper_bound}}
\label{sec: proof_of_theorem_1}

To ease notation, letting $S \subseteq X$ and $\Sset \subseteq \Reals^d$, we write
\begin{equation*}
\cut_{n,r}(S) = \cut(\Csig[\Xbf]; G_{n,r}), ~ \cut_{\Pbb,r}(\Sset)= \frac{\mathbb{E}[\cut_{n,r}(\Sset)]}{2{n \choose 2}}
\end{equation*}
for the random variable and mean of cut size, respectively.

With this notation in place, the goal of Theorem~\ref{thm: conductance_upper_bound} is to show that for a universal constant $c_1 > 0$,
\begin{align*}
\Phi_{n,r}(\Csig[\Xbf]) := \frac{\cut_{n,r}(\Csig[\Xbf])}{\min\{\vol_{n,r}(\Csig[\Xbf]), \vol_{n,r}((\Rd \setminus \Csig)[\Xbf])\}} \leq c_1 \frac{d}{\sigma}
    \frac{\lambda}{\lambda_{\sigma}} \frac{(\lambda_{\sigma} -
      \theta\frac{r^{\gamma}}{\gamma+1})}{\lambda_{\sigma}}
\end{align*}
with probability at least $1 - 3\exp\{-nb\}$.

The proof of this theorem follows essentially from two technical Lemmas. 
Lemma~\ref{lem:ball_bounds_in_probability} relates the terms in the numerator and denominator of $\Phi_{n,r}(\Csig[\Xbf])$ to their expected values. We restate the conclusions of this Lemma: for any $\delta > 0$,
\begin{align}
\frac{\cut_{n,r}(\Csig[\Xbf])}{2{n \choose 2}} \leq (1 + \delta)\cut_{\Pbb,r}(\Csig) &, \quad \frac{\vol_{n,r}(\Csig[\Xbf])}{2{n \choose 2}} \geq (1 - \delta)\vol_{\Pbb,r}(\Csig) \nonumber \\ \frac{\vol_{n,r}((\Rd \setminus \Csig)[\Xbf])}{2{n \choose 2}}& \geq (1 - \delta)\vol_{\Pbb,r}(\Rd \setminus \Csig) \label{eqn: conductance_upper_bound_pf1}
\end{align}
with probability at least 
\begin{align*}
1 - & \exp\set{-n \delta^2 (\cut_{\Pbb,r}(\Csig))^2} - \exp\set{-n \delta^2 (\vol_{\Pbb,r}(\Csig))^2} - \exp\set{-n \delta^2 (\vol_{\Pbb,r}(\Rd\setminus \Csig))^2} \\
& \geq 1 - 3\exp\set{-n \delta^2 (\cut_{\Pbb,r}(\Csig))^2}.
\end{align*} 
We note that as a consequence of~\ref{asmp: bounded_volume} we have that $\vol_{\Pbb,r}(\Rd \setminus \Csig)  \geq \vol_{\Pbb,r}(\Csig)$, so it will suffice to lower bound $\vol_{\Pbb,r}(\Csig)$ (since a lower bound for $\vol_{\Pbb,r}(\Rd \setminus \Csig)$ follows).
The following result provides upper and lower bounds on the expected values $\cut_{\Pbb,r}(\Csig)$ and $\vol_{\Pbb,r}(\Csig)$ respectively:
\begin{lemma}
	\label{lem: expected_density_cut}\label{lem: expected_density_volume}
	Under the setup and conditions of Theorem \ref{thm: conductance_upper_bound}, and for any $0 < r \leq \sigma/(2d)$,
	\begin{align}
	\cut_{\Pbb,r}(\Csig) &\leq \frac{ d \nu_d r^{d+1} \lambda}{\sigma - dr} \left(\lambda_{\sigma} - \theta\frac{r^{\gamma}}{\gamma + 1}\right) \nu(\Csig), \label{eqn:claim_one} \\
	\vol_{\Pbb,r}(\Csig) &\geq \biggl(1 + \frac{dr}{\sigma - dr}\biggr)^{-1} \lambda_{\sigma}^2 \nu_d r^d \nu(\Csig).\label{eqn:claim_two}
	\end{align}
\end{lemma}
Taking Lemma~\ref{lem: expected_density_cut} and \eqref{eqn: conductance_upper_bound_pf1} as given we can now complete the proof of the theorem. We lower bound $\Phi_{n,r}(\Csig[\Xbf])$ as follows:
\begin{equation*}
\Phi_{n,r}(\Csig[\Xbf]) \leq \frac{(1 + \delta)\cut_{\Pbb,r}(\Csig)}{(1 - \delta)\vol_{\Pbb,r}(\Csig)} \leq \frac{(1 + \delta)}{(1 - \delta)(1 - dr/\sigma)^2}\frac{d r \lambda \left(\lambda_{\sigma} - \theta\frac{r^{\gamma}}{\gamma + 1}\right)}{\sigma \lambda_{\sigma}^2}.
\end{equation*}
Noting that $(1 - dr/\sigma)^{-2} \leq 16/9$ for all $r \leq \sigma/(4d)$, the theorem is satisfied by choosing $c_1 = 16/9$ and $b = (\cut_{\Pbb,r}(\Csig))^2$. 

\subsection{Proof of Lemma~\ref{lem: expected_density_cut}}
We write $\Pbb(\Aset) = \int_{\Aset} f(x) dx$ for measurable $\Aset \subseteq \Rd$.
We 
let $\Csigr := \set{x: 0 < \dist(x, \Csig) < r}$, where $\Csig$ is as in Theorem \ref{thm: conductance_upper_bound}. 
Our goal will be to upper bound $\cut_{\Pbb,r}(\Csig)$ by a term that depends on the probability mass $\Pbb(\Csigr)$, 
and the bulk of our technical effort will be devoted to showing the following upper bound on $\Pbb(\Csigr)$:
\begin{lemma}
	\label{lem: expected_number_boundary_points}
	For any $0 < r \leq \sigma/(2d)$, under the conditions of Theorem \ref{thm: conductance_upper_bound} it holds that
	\begin{equation*}
	\Pbb(\Csigr) \leq \frac{dr}{\sigma - dr} \left(\lambda_{\sigma} - \theta\frac{r^{\gamma}}{\gamma + 1}\right) \nu(\Csig)
	\end{equation*}	
\end{lemma}
Taking this result as given we can now prove the claim~\eqref{eqn:claim_one}.
\paragraph{Proof of Claim~\eqref{eqn:claim_one}: } For each $i,j$ such that $i \neq j$, we can write 
\begin{equation*}
\cut_{\Pbb,r}(\Csig) =  \Pbb(x_i \not\in \Csig, x_j \in \Csig, \norm{x_i - x_j} \leq r).
\end{equation*}
Writing this as an integral, we have
\begin{align*}
\cut_{\Pbb,r}(\Csig) & = \int_{\Rd \setminus \Csig} f(x) \Pbb\bigl(B(x,r) \cap \Csig\bigr) \dx \\
& = \int_{\Csigr} f(x) \Pbb\bigl(B(x,r) \cap \Csig\bigr) \dx \\
& \leq \nu_d r^d \lambda  \int_{\Csigr} f(x) \dx = \nu_d r^d \lambda \Pbb(\Csigr).
\end{align*}
where the inequality follows from \ref{asmp: low_noise_density}, which implies $f(x) \leq \lambda$ for $x \in \Csig \setminus \Cset$. Then, upper bounding the integral using Lemma \ref{lem: expected_number_boundary_points} gives the final result.

\paragraph{Proof of Claim~\eqref{eqn:claim_two}: } On the other hand, the proof of Claim~\eqref{eqn:claim_two} follows from relating $\vol_{\Pbb,r}(\Csig)$ to the Lebesgue measure of $\mc{C}_{\sigma - r}$. In particular, since $f(y) \geq \lambda_{\sigma}$ for all $x \in \Csig$, and $B(x,r) \subset \Csig$ for all $x \in \mc{C}_{\sigma - r}$, it follows that
\begin{align*}
\vol_{\Pbb,r}(\Csig) & = \int_{\Csig} f(x) \Pbb(B(x,r)) \dx \\
& \geq \int_{\mc{C}_{\sigma - r}} f(x) \Pbb(B(x,r)) \dx \\
& \geq \lambda_{\sigma}^2 \nu_d r^d \nu(\mc{C}_{\sigma - r}).
\end{align*}
Then, noting that by the triangle inequality $\mc{C}_{\sigma} \subseteq \mc{C}_{\sigma - r} + \sigma B$, applying Lemma~\ref{lem: expansion_volume} (given in Appendix~\ref{sec: volume_estimates}) and the Bernoulli inequality (Lemma~\ref{lem: Taylor_series}) gives
\begin{equation*}
\nu(\Cset_{\sigma}) \leq \nu(\Cset_{\sigma - r} + \sigma B) \leq \biggl(1 + \frac{r}{\sigma - r}\biggr)^d \nu(\Cset_{\sigma - r}) \leq \biggl(1 + \frac{dr}{\sigma - dr}\biggr)\nu(\mc{C}_{\sigma - r}),
\end{equation*}
thus establishing the claim.

It remains to prove Lemma~\ref{lem: expected_number_boundary_points} and we turn our attention to this now.
\subsection{Proof of Lemma~\ref{lem: expected_number_boundary_points}}
We note that the proof of this Lemma relies on certain volume estimates whose statement and proof we defer to Appendix~\ref{sec: volume_estimates}.
	
We begin by partitioning $\Csigr$ into slices based on distance from $\Csig$ as follows: for $k \in \N$,
	\begin{equation*}
	\mathcal{T}_{i,k} = \set{x \in \Csigr: t_{i,k} < \frac{\dist(x, \Csig)}{r} \leq t_{i+1,k}}, ~~ \Csigr = \bigcup_{i = 0}^{k-1} \mathcal{T}_{i,k},
	\end{equation*}
	where $t_{i,k} = i/k$ for $i = 0, \ldots, k - 1$. As a result, for any $k \in \mathbb{N}$,
	\begin{equation}
	\label{eqn: partition_ub}
	\Pbb(\Csigr) = \int_{\Csigr} f(x) \dx = \sum_{i = 0}^{k-1} \int_{\mathcal{T}_{i,k}} f(x) \dx \leq \sum_{i = 0}^{k-1} \nu(\mathcal{T}_{i,k}) \max_{x \in \mathcal{T}_{i,k}} f(x).
	\end{equation}
	Assumptions~\ref{asmp: bounded_density} and \ref{asmp: low_noise_density} imply the upper bound
	\begin{equation*}
	\max_{x \in \mathcal{T}_{i,k}} f(x) \leq \lambda_{\sigma} - \theta(rt_{i,k})^{\gamma},
	\end{equation*}
	and writing
	\begin{equation*}
	\nu(\mathcal{T}_{i,k}) = \nu(\Csig + rt_{i+1,k}B) - \nu(\Csig + rt_{i,k}B) =: \nu_{i+1,k} - \nu_{i,k},
	\end{equation*}
	we have
	\begin{align}
	\label{eqn: telescoping_sum}
	\sum_{i = 0}^{k-1} \nu(\mathcal{T}_{i,k}) \max_{x \in \mathcal{T}_{i,k}} f(x) & \leq \sum_{i = 0}^{k-1} \biggl\{ \nu_{i+1,k} - \nu_{i,k} \biggr\} \biggl( \lambda_{\sigma} - \theta(rt_{i,k})^{\gamma} \biggr) \nonumber \\
	& = \underbrace{\sum_{i = 1}^{k} 
	\nu_{i,k} \biggl( \left[\lambda_{\sigma} - \theta(rt_{i-1,k})^{\gamma}\right] -  \left[\lambda_{\sigma} - \theta(rt_{i,k})^{\gamma}\right]\biggr)}_{:= \Sigma_k} + \underbrace{\biggl(\nu_{k,k}\left[\lambda_{\sigma} - \theta r^{\gamma}\right] - \nu_{1,k}\lambda_{\sigma} \biggr)}_{:= \xi}
	\end{align}
	where the second equality comes from rearranging terms in the sum.
	
	% AJG 5/14/19: This argument makes sense, correct?
	We first consider the term $\Sigma_k$. $\Cset$ has finite diameter by Assumption~\ref{asmp: bounded_density}, as otherwise $\int_{\Csig} f(x) dx = \infty$. Letting $\overline{\Cset}$ be the closure of $\Cset$, we observe that $\overline{\Csig} = \overline{\Cset} + \sigma B$, and moreover for any $\delta > 0$, $\nu(\overline{\Csig} + \delta B) = \nu(\Csig + \delta B)$ (as the boundary $\partial(\Csig + \delta B)$ is Lipschitz and therefore has measure zero). As a result, for each $t_{i,k}, i = 1, \ldots,k$ we may apply Lemma \ref{lem: expansion_volume} to $\overline{\Cset}$ and obtain
	\begin{equation}
	\label{eqn: slice_volume_bound}
	\nu_{i,k} = \nu(\Csig + rt_{i,k}B) \leq \nu(\Csig)\left(1 + \frac{rt_{i,k}}{\sigma}\right)^d
	\end{equation}
	which in turn gives
	\begin{align}
	\Sigma_k & \leq \theta\nu(\Csig) r^\gamma \sum_{i = 1}^{k} \left(1 + \frac{ rt_{i,k}}{\sigma}\right)^d \biggl( (t_{i,k})^{\gamma} - (t_{i-1,k})^{\gamma}\biggr) \nonumber \\
	& = \theta\nu(\Csig) r^\gamma \sum_{i = 1}^{k} \left(1 + \frac{ru_{i,k}^{1/\gamma}}{\sigma}\right)^d ( u_{i,k} - u_{i,k-1}).~~~~~~~~~~~~~~ (\text{substituting}~u_{i,k} := t_{i,k}^{\gamma}) \label{eqn: Sigmak_riemann_sum}
	\end{align}
	The expression in~\eqref{eqn: Sigmak_riemann_sum} is a Riemann sum, and taking the limit as $k \to \infty$ we obtain
	\begin{align}
	\lim_{k \to \infty} \theta\nu(\Csig) r^\gamma \sum_{i = 1}^{k} \left(1 + \frac{ru_{i,k}^{1/\gamma}}{\sigma}\right)^d ( u_{i,k} - u_{i,k-1}) & = \theta\nu(\Csig) r^\gamma \int_{0}^{1} \left(1 + \frac{r u^{1/\gamma}}{\sigma}\right)^{d} du \nonumber \\
	& \overset{\text{(i)}}{\leq} \theta\nu(\Csig) r^\gamma \int_{0}^{1} \left(1 + \frac{ d r u^{1/\gamma}}{\sigma - dr}\right) du \nonumber \\
	& = \theta\nu(\Csig) r^\gamma \left(1 + \gamma \frac{d r}{(\gamma + 1)(\sigma - dr)}\right). \label{eqn: Sigmak_integral}
	\end{align}
	where $\text{(i)}$ follows from the upper bound in Lemma \ref{lem: Taylor_series} in light of the fact $r \leq \sigma/(4d)$. 
	
	An upper bound on $\xi$ follows from largely the same logic, although it does not involve integration:
	\begin{align}
	\xi & \overset{\text{(ii)}}{\leq} \nu(\Csig) \biggl\{ \left(1 + \frac{ r}{\sigma}\right)^d(\lambda_{\sigma} - \theta r^{\gamma}) - \lambda_{\sigma} \biggr\} \nonumber \\
	& \overset{\text{(iii)}}{\leq} \nu(\Csig) \biggl\{ \left(1 + \frac{d r}{\sigma - dr}\right)(\lambda_{\sigma} - \theta r^{\gamma}) - \lambda_{\sigma} \biggr\} = \nu(\Csig) \biggl\{ \frac{dr}{\sigma - dr}(\lambda_{\sigma} - \theta r^{\gamma}) - \theta r^{\gamma} \biggr\}. \label{eqn: xi_ub}
	\end{align}
	where $\text{(ii)}$ follows from \eqref{eqn: slice_volume_bound}, and $\text{(iii)}$ from Lemma \ref{lem: Taylor_series}. As the bounds in \eqref{eqn: partition_ub} and \eqref{eqn: telescoping_sum} hold for all $k$, these along with \eqref{eqn: Sigmak_integral} and \eqref{eqn: xi_ub} imply the desired result.


\subsection{Volume Estimates}
\label{sec: volume_estimates}
We begin by recalling some notation. We let $\Aset \subseteq \Reals^d$, and for $\sigma > 0$, write $\sigma B := B(0,\sigma) = \set{x \in \Rd: \norm{x} \leq \sigma}$ for the closed ball of radius $\sigma$ centered at the origin (and let $B^{\circ}(0,\sigma)$ denote the corresponding open ball). Let $\Asig = \Aset + \sigma B$ be the direct sum of $\Aset$ and $\sigma B$, $\Asig = \set{z = x + y: x \in \Aset, y \in \sigma B}$. Recall that we use $\nu$ for Lebesgue measure, and $\nu_d = \nu(B)$ for $B = (0,1)$. 

Lemma \ref{lem: expansion_volume} provides a bound on the ratio $\nu(\Csig + r B) / \nu(\Csig)$, an important intermediate quantity in bounding the ratio $\cut(\Csig[\Xbf]; G_{n,r})/\vol(\Csig[\Xbf]; G_{n,r})$. 

\begin{lemma}
	\label{lem: expansion_volume}
	If $\Aset$ is closed and bounded, then for any $\sigma$ and $\delta > 0$,
	\begin{equation}
	\label{eqn: expansion_volume}
	\nu(\Asig + \delta B) \leq \left(1 + \frac{\delta}{\sigma}\right)^d \nu(\Asig).
	\end{equation}
\end{lemma}
\begin{proof}
	We will show that for any $\epsilon > 0$, 
	\begin{equation}
	\label{eqn: ratio_of_volume}
	\frac{\nu(\Asig + \delta B)}{\nu(\Asig)} \leq \frac{(\sigma + \delta + \epsilon)^d}{\sigma^d}
	\end{equation}
	Taking the limit as $\epsilon \to 0$ results in \eqref{eqn: expansion_volume}.
	
	
	Fix $\epsilon > 0$. Our first goal is to find a finite collection $x_1, \ldots, x_N \in \Rd$ (where $N$ is a finite number that may implicitly depend on $\epsilon$) such that
	\begin{equation*}
	\bigcup_{i = 1}^{N} B(x_i, \sigma) \subseteq \Asig \subset \bigcup_{i = 1}^{N} B(x_i, \sigma + \epsilon).
	\end{equation*}
	Note that $\Asig$ is the direct sum of two compact sets, and is therefore itself compact. Moreover, for any $\epsilon > 0$,
	\begin{equation*}
	\Asig \subset \bigcup_{x \in \Aset} B^{\circ}(x,\sigma + \epsilon)
	\end{equation*}
	so by compactness there exists a finite subcover $x_1, \ldots,x_N \in \Aset$ such that
	\begin{equation}
	\label{eqn: finite_subcover}
	\Asig \subset \bigcup_{i = 1}^{N} B^{\circ}(x_i,\sigma + \epsilon).
	\end{equation}
	As a direct consequence of \eqref{eqn: finite_subcover}, $\Asig + \delta B \subset \bigcup_{i = 1}^{N} B^{\circ}(x_i,\sigma + \epsilon + \delta)$, and by definition for every $x_i \in \Aset$, $B(x_i,\sigma) \in \Asig$. Summarizing our findings, we have
	\begin{equation}
	\label{eqn: finite_subcover-1}
	\bigcup_{i = 1}^{N} B(x_i,\sigma) \subseteq \Asig  ,~\Asig + \delta B \subset \bigcup_{i = 1}^{N} B^{\circ}(x_i,\sigma + \delta + \epsilon).
	\end{equation}
\noindent	We next show a lower bound on $\nu(\Asig)$. Partition $\Asig$ using the balls $B(x_i,\sigma)$, meaning let $\Aset_{\sigma}^{(1)} := B(x_1,\sigma)$, $\Aset_{\sigma}^{(2)} := B(x_2,\sigma) \setminus B(x_1,\sigma)$, and so on, so that
	\begin{equation*}
	\Aset_{\sigma}^{\text{(i)}} := B(x_i,\sigma) \setminus \bigcup_{j = 1}^{i - 1} \Aset_{\sigma}^{(j)}. \tag{$i = 1,\ldots,N$}
	\end{equation*}
	Observe that $\bigcup_{i = 1}^{N} \Asig^{\text{(i)}} = \bigcup_{i = 1}^{N} B(x_i,\sigma)$, so by \eqref{eqn: finite_subcover} $\Asig \supseteq \bigcup_{i = 1}^{N} \Asig^{\text{(i)}}$. As $\Asig^{(1)},\ldots, \Asig^{(N)}$ are non-overlapping,
	\begin{align*}
	\nu(\Asig) & \geq \sum_{i = 1}^{N} \nu(\Asig^{\text{(i)}}) \\
	& = \sigma^d \nu_d \sum_{i = 1}^{N}  \frac{\nu(\Asig^{\text{(i)}})}{\nu(B(x_i,\sigma))}
	\end{align*}
	We turn to proving an analogous upper bound on $\nu(\Asig + \delta B)$. Let $\Aset_{\sigma + \epsilon + \delta}^{(1)} := B(x_1,\sigma + \delta + \epsilon)$ and
	\begin{equation*}
	\Aset_{\sigma + \delta + \epsilon}^{\text{(i)}} := B(x_i,\sigma + \delta + \epsilon) \setminus \bigcup_{j = 1}^{i - 1} \Aset_{\sigma + \delta + \epsilon}^{(j)}. \tag{$i = 2,\ldots,N$}
	\end{equation*}
As $\bigcup_{i = 1}^{N} \Aset_{\sigma + \delta + \epsilon}^{\text{(i)}} = \bigcup_{i = 1}^{N} B(x_i,\sigma + \delta + \epsilon)$, by \eqref{eqn: finite_subcover}
	\begin{equation*}
	\Aset_{\sigma} + \delta B \subset \bigcup_{i =1}^{N} \Aset_{\sigma + \delta + \epsilon}^{\text{(i)}}
	\end{equation*}
	and therefore 
	\begin{align*}
	\nu(\Aset_\sigma + \delta B) & \leq \sum_{i = 1}^{N} \nu\bigl(\Aset_{\sigma + \delta + \epsilon}^{\text{(i)}}\bigr) \\
	& = \sum_{i = 1}^{N} \nu_d (\sigma + \delta + \epsilon)^d \frac{\nu(\Aset_{\sigma + \delta + \epsilon}^{\text{(i)}})}{\nu(B(x_i, \sigma + \delta + \epsilon))} \\
	& \leq \nu_d (\sigma + \delta + \epsilon)^d \sum_{i = 1}^{N} \frac{\nu(\Asig^{\text{(i)}})}{\nu(B(x_i,\sigma))}
	\end{align*}
	where the last inequality follows from Lemma \ref{lem: covering}. We have shown \eqref{eqn: ratio_of_volume}, and thus the claim.
\end{proof}
\noindent We require Lemma \ref{lem: covering} to prove Lemma \ref{lem: expansion_volume}.
\begin{lemma}
	\label{lem: covering}
	For $i = 1, \ldots, N$ and  $\Aset_{\sigma}^{\text{(i)}}, \Aset_{\sigma + \delta + \epsilon}^{\text{(i)}}$ as in Lemma \ref{lem: expansion_volume},
	\begin{equation*}
	\frac{\nu(\Aset_{\sigma + \delta + \epsilon}^{\text{(i)}})}{\nu(B(x_i, \sigma + \delta + \epsilon))} \leq \frac{\nu(\Aset_{\sigma}^{\text{(i)}})}{\nu(B(x_i, \sigma))}
	\end{equation*}
\end{lemma}
\begin{proof}
	Let $\delta' := \delta + \epsilon$. It will be sufficient to show that
	\begin{equation*}
	\biggl(\Aset_{\sigma + \delta'}^{\text{(i)}} - \set{x_i}\biggr) \subseteq \left(1 + \frac{\delta'}{\sigma}\right)\cdot\biggl(\Asig^{\text{(i)}} - \set{x_i}\biggr) 
	\end{equation*}
	since then
	\begin{equation*}
	\nu(\Aset_{\sigma + \delta'}^{\text{(i)}}) \leq \left(1 + \frac{\delta'}{\sigma}\right)^d \nu(\Aset_{\sigma}^{\text{(i)}}) = \frac{\nu(B(x_i, \sigma + \delta'))}{\nu(B(x_i, \sigma))} \nu(\Aset_{\sigma}^{\text{(i)}}).
	\end{equation*}
	
	Assume without loss of generality that $x_i = 0$, and let $x \in \Aset_{\sigma + \delta'}^{\text{(i)}}$, meaning
	\begin{equation}
	\norm{x} \leq \sigma + \delta',~ \norm{x - x_j} > \sigma + \delta'~ \textrm{for $j = 1, \ldots, i - 1$}.
	\end{equation}
	Letting $x' = \frac{\sigma}{\sigma + \delta'} x$, since $\norm{x} \leq \sigma + \delta'$, $\norm{x'} \leq \sigma$ and therefore $x' \in B(0,\sigma)$. Additionally observe that for any $j = 1, \ldots, i - 1$, by the triangle inequality
	\begin{equation*}
	\norm{x' - x_j} \geq \norm{x - x_j} - \norm{x - x'} > \sigma + \delta' - \frac{\delta'}{\sigma + \delta'}\norm{x} \geq \sigma
	\end{equation*}
	and therefore $x' \not\in B(x_j,\sigma)$ for any $j = 1,\ldots, i - 1$. So $x' \in \Asig^{\text{(i)}}$.
\end{proof}

We will need to carefully control the volume of expansion sets using the estimate in Lemma~\ref{lem: expansion_volume}; Lemma~\ref{lem: Taylor_series} (Bernoulli's inequality) serves this purpose (see also, Lemma~23 in \cite{balakrishnan2013}). 
\begin{lemma}
	\label{lem: Taylor_series}
	For any $-1 \leq x \leq 1/(d - 1)$, it holds that
	\begin{equation*}
	(1 + x)^{d} \leq 1 + \frac{dx}{1 - (d - 1)x}
	\end{equation*}
	and
	\begin{equation*}
	(1 - x)^d \geq 1 - d x.
	\end{equation*}
	If additionally $0 \leq x \leq 1/(2d - 2)$, then
	\begin{align*}
	(1 + x)^d & \leq 1 + 2d x.
	\end{align*}
\end{lemma}

\section{Proof of Theorem \ref{thm: mixing_time_upper_bound}}

Let $\widetilde{G}_{n,r} := G_{n,r}[\Csig[\Xbf]]$; in general, we will use tilde notation to refer to quantities computed over $\Csig$ or over the induced subgraph $\widetilde{G}_{n,r}$. The goal of Theorem \ref{thm: mixing_time_upper_bound} is to show that with high probability,
\begin{equation*}
\tau_{\infty}(\widetilde{G}_{n,r}) \leq c_2 \frac{\Lambda_{\sigma}^4 d^3 \rho^2 L^2}{\lambda_{\sigma}^4 r^2} \log^2\left(\frac{ \Lambda_{\sigma}}{\lambda_{\sigma}^2 r}\right) + c_3.
\end{equation*}
We follow a two-step approach typically used to establish upper bounds on the mixing time of Markov chains. 

In the first step, we relate the mixing time $\tau_{\infty}(G)$ of an arbitrary undirected graph $G = (V,E)$ to expansion properties of subsets $U \subseteq V$. We now build to a formal definition of these expansion properties. First, we recall the the \emph{cut} and \emph{volume} functionals over a graph, and introduce the \emph{degree} functional as well. For $u \in V$, $S \subseteq V$, and $S^c = V \setminus S$,
\begin{equation*}
\cut(S;G) = \sum_{u \in S} \sum_{v \in S^c} \1((u,v) \in E), \quad \deg(u;G) = \sum_{v \in V} \1((u,v) \in E), \quad \vol(S;G) = \sum_{u \in S} \deg(u;G).
\end{equation*}
Additionally, we recall the \emph{normalized cut} $\Phi(S;G)$, defined (as in \eqref{eqn: normalized_cut}) as
\begin{equation*}
\Phi(S;G) = \frac{\cut(S; G)}{\min\set{\vol(S; G),\vol(S^c; G)}}.
\end{equation*}
We can now formally define the expansion parameters \emph{local spread} $s(G)$ and \emph{conductance} $\Phi(G)$,
\begin{equation*}
s(G) := \frac{9}{10} \cdot \min_{u \in V} \set{\deg(u; G)} \cdot \min_{u \in V} \set{\pi(u)}, \quad \Phi(G) := \min_{S \subseteq V} \Phi(S;G).
\end{equation*}

The following proposition establishes an upper bound on the mixing time $\tau_{\infty}(G)$ in terms of the local spread $s(G)$ and conductance $\Phi(G)$.
\begin{proposition}
	\label{prop: pointwise_mixing_time}
	Assume $\min_{u \in V} \deg(u; G) \geq 10$. Then,
	\begin{equation*}
	\tau_{\infty}(G) \leq \frac{2}{\Phi^2(G)} \ln \left(\frac{320}{s(G)}\right)\log_2 \left(\frac{14}{s(G)}\right)  + 3 \log_2 \left(\frac{14}{s(G)}\right) + 3
	\end{equation*}
\end{proposition}

The second step in our approach is to lower bound the local spread and conductance over the neighborhood graph $\widetilde{G}_{n,r}$. In the following result we give lower bounds for both quantities. 

\begin{proposition}
	\label{prop: local_spread_conductance}
	Fix $\delta \in (0,1)$. Under the setup and conditions of Theorem \ref{thm: mixing_time_upper_bound}, there exist constants $b_3$, $b_4$, and $b_5$ independent of $n$ such that the following statement holds true: for any $n$ such that
	\begin{equation*}
	n \geq \max\left\{\frac{(1 + \delta)^2}{(1 - \delta)^2} \cdot (\log n)^{dp_d}\cdot b_3^d, b_4\right\}
	\end{equation*}
	the following inequalities:
	\begin{equation}
	\label{eqn:min_degree}
	\min_{u \in \Csig[\Xbf]} \deg(u;\widetilde{G}_{n,r}) \geq 10,
	\end{equation}
	and
	\begin{equation}
	\label{eqn: local_spread}
	s(\widetilde{G}_{n,r}) \geq \frac{9}{40} \cdot \biggl(1 - \sqrt{\frac{d + 2}{2\pi}}\frac{r}{\sigma}\biggr)^2 \cdot \frac{(1 - \delta)^2}{(1 + \delta)} \cdot \frac{\lambda_{\sigma}^2(2r)^d}{\Lambda_{\sigma}^2(L\rho)^d},
	\end{equation}
	and
	\begin{equation}
	\label{eqn: conductance}
	\Phi(\widetilde{G}_{n,r}) \geq \Bigl(1 - \delta\Bigr)\Bigl(1 - \frac{r}{4\rho L}\Bigr) \Bigl(1 - \frac{r}{\sigma}\sqrt{\frac{d + 2}{2\pi}}\Bigr)^2 \cdot \frac{\sqrt{2\pi}}{36} \cdot \frac{r\lambda_{\sigma}^2}{\Lambda_{\sigma}^2\rho L \sqrt{d + 2}}
	\end{equation}
	hold with probability at least $1 - \frac{b_5}{n} - 2n\exp\Bigl\{-\delta^2 \lambda_{\sigma} \nu_dr^dn/(6 + 2\delta)\Bigr\} - 2\exp\set{-n\delta^2(\widetilde{\vol}_{\Pbb,r}(\Csig))^2}$.
\end{proposition}

Taking these results as given, the proof of Theorem \ref{thm: mixing_time_upper_bound} is more or less complete. Since the condition $\min_{u \in \Csig[\Xbf]} \deg(u;\widetilde{G}_{n,r}) \geq 10$ is satisfied, we may apply Proposition \ref{prop: pointwise_mixing_time}. Noting that: 
\begin{itemize}
	\item By assumption $r/\sigma \leq 1/(4d)$. Thus 
	\begin{equation*}
	1 - \frac{r}{4\rho L} \geq 1 - \frac{r}{4\sigma} \geq \frac{15}{16}
	\end{equation*}
	and
	\begin{equation*}
	\Bigl(1 - \frac{r}{\sigma}\sqrt{\frac{d + 2}{2\pi}}\Bigr) \geq 1 - \frac{1}{4}\sqrt{\frac{d + 2}{2d^2\pi}} \geq 1 - \frac{1}{4\sqrt{2}}.
	\end{equation*}
	
	Thus from~\eqref{eqn: conductance}, we obtain
	\begin{align*}
	\frac{1}{\Phi^2(\wt{G}_{n,r})} & \leq \frac{36^2 \cdot 16^2}{15^2(1 - \delta)^2(1 - 1/(4\sqrt{2}))^4 2\pi} \cdot \frac{\Lambda_{\sigma}^4\rho^2 L^2 (d + 2)}{r^2 \lambda_{\sigma}^4} \\
	& \leq \frac{511}{(1 - \delta)^2} \frac{\Lambda_{\sigma}^4\rho^2 L^2 (d + 2)}{r^2 \lambda_{\sigma}^4}.
	\end{align*}
	\item by~\eqref{eqn: local_spread}, for any constant $c \geq 1$,
	\begin{align*}
	\log\biggl(\frac{c}{s(\wt{G}_{n,r})}\biggr) & \leq d \log\biggl(c^{1/d}\frac{40^{1/d}}{9^{1/d}\bigl(1 - \sqrt{\frac{d + 2}{2\pi}}\frac{r}{\sigma}\bigr)^{2/d}} \cdot \frac{(1 + \delta)^{2/d}}{(1 - \delta)^{2/d}} \cdot \frac{\lambda_{\sigma}^{2/d}L\rho}{\Lambda_{\sigma}^{2/d} 2r}\biggr) \\
	& \leq d \frac{(1 + \delta)^{2/d}}{(1 - \delta)^{2/d}} \log\biggl(c^{1/d}\frac{60^{1/d}}{9^{1/d} } \cdot \frac{\lambda_{\sigma}^{2/d} L\rho}{\Lambda_{\sigma}^{2/d} 2r}\biggr)
	\end{align*}
\end{itemize}
Hence, by Proposition~\ref{prop: pointwise_mixing_time} it holds that
\begin{align*}
\tau_{\infty}(\widetilde{G}_{n,r}) & \leq \frac{2}{\Phi^2(\widetilde{G}_{n,r})} \ln \left(\frac{320}{s(\widetilde{G}_{n,r})}\right)\log_2 \left(\frac{14}{s(\widetilde{G}_{n,r})}\right)  + 3 \log_2 \left(\frac{14}{s(\widetilde{G}_{n,r})}\right) + 3 \\
& \leq \frac{1022(1 + \delta)^{4/d}}{(1 - \delta)^{4/d + 2}} \frac{\Lambda_{\sigma}^4\rho^2 L^2 (d + 2)^3}{r^2 \lambda_{\sigma}^4} \log_2\biggl(c_4^{1/d}\frac{\lambda_{\sigma}^{2/d} L\rho}{\Lambda_{\sigma}^{2/d} 2r}\biggr) \ln\biggl(c_5^{1/d}\frac{\lambda_{\sigma}^{2/d} L\rho}{\Lambda_{\sigma}^{2/d} 2r}\biggr) + \\
& ~~ 3d\frac{(1 + \delta)^{2/d}}{(1 - \delta)^{2/d}} \log_2\biggl(c_4^{1/d}\frac{\lambda_{\sigma}^{2/d} L\rho}{\Lambda_{\sigma}^{2/d} 2r}\biggr) + 3 \\
& \leq 1022\biggl(1 + \Bigl(\frac{32}{d} + 8\Bigr)\delta\biggr) \frac{\Lambda_{\sigma}^4\rho^2 L^2 (d + 2)^3}{r^2 \lambda_{\sigma}^4} \log_2\biggl(c_4^{1/d}\frac{\lambda_{\sigma}^{2/d} L\rho}{\Lambda_{\sigma}^{2/d} 2r}\biggr) \ln\biggl(c_5^{1/d}\frac{\lambda_{\sigma}^{2/d} L\rho}{\Lambda_{\sigma}^{2/d} 2r}\biggr) + \\
& ~~ 3d(1 + 32\delta)\log_2\biggl(c_4^{1/d}\frac{\lambda_{\sigma}^{2/d} L\rho}{\Lambda_{\sigma}^{2/d} 2r}\biggr) + 3 \\
& \leq 1028 \biggl(1 + 40\delta\biggr) \frac{\Lambda_{\sigma}^4\rho^2 L^2 (d + 2)^3}{r^2 \lambda_{\sigma}^4} \ln^2\biggl(c_5^{1/d}\frac{\lambda_{\sigma}^{2/d} L\rho}{\Lambda_{\sigma}^{2/d} 2r}\biggr)
\end{align*}
where $c_4 := 94, c_5 := 2134$, and the second to last inequality holds whenever $\delta < 1/(8 + 16/d)$. Thus the Theorem holds upon proper choice of universal constants $c_2 = 1028$, $c_3 = 40$, and constants $b_1 = b_3^d + b_4$, $b_2 = \min\{\wt{\vol}_{\Pbb,r}(\Csig), \lambda_{\sigma}\nu_dr^d/8\}$.

In the rest of this section we proceed to proving Propositions \ref{prop: pointwise_mixing_time} and \ref{prop: local_spread_conductance}.

\subsection{Proof of Proposition \ref{prop: pointwise_mixing_time}}
\label{sec: pointwise_mixing_time}

We first generalize some notation from the main text. Let $A$ be the adjacency matrix of an undirected graph $G = (V,E)$, and $D$ the associated diagonal degree matrix. The lazy random walk over $G$ is the Markov chain with transition probabilities given by $\Wbf := \frac{\Ibf{} + \Dbf^{-1} \Abf}{2}$ and stationary distribution $\pi$ with elements $\pi(u) = D_{uu}/\vol(V;G)$.  Denote the $t$-step probability distribution of this random walk, originating from a  vertex $v \in V$, as $q^{(t)}: V \times V \to [0,1]$, $q^{(t)}(v,u) = e_v \Wbf^t e_u$.

Consider the uniform distance\footnote{Note $d_{\textrm{unif}}$ is not formally a distance as it is not symmetric.} between the distributions $q_v^{(t)} := q^{(t)}(v,\cdot)$ and $\pi$, given by
\begin{equation*}
d_{\textrm{unif}}(q_v^{(t)},\pi) = \max_{u \in V} \set{\frac{\pi(u) - q_v^{(t)}(u)}{\pi(u)}}.
\end{equation*}
Our goal is to demonstrate that for a sufficiently large $t$, $d_{\textrm{unif}}(q_v^{(t)},\pi) \leq \frac{1}{4}$ for every $v \in V$ (see~\eqref{eqn: mixing_time} to recall the definition of $\tau_{\infty}(G)$). To do so, we introduce the \emph{total variation distance} between the distributions $q_v^{(t)}$ and $\pi$, given by
\begin{equation*}
\norm{q_v^{(t)} - \pi}_{\mathrm{TV}} = \sum_{u \in V} \abs{q_v^{(t)}(u) - \pi(u)}.
\end{equation*}
As we will see, an upper bound on the uniform distance can be obtained from an analogous upper bound on the total variation distance. First, however, we upper bound the total variation distance $||q_v^{(t)} - \pi||_{\mathrm{TV}}$ as a function of the local spread $s(G)$, the conductance $\Phi(G)$ and the number of steps $t$.
\begin{lemma}
	\label{lem: tv_mixing_time}
	For any $v \in V$, and any $0 < a < 1/4$,
	\begin{align}
	\norm{q_v^{(t + 3)} - \pi}_{\mathrm{TV}} \leq \max\set{as(G), \frac{1}{8} + \frac{9a}{20} + \frac{1}{2 \min_{u \in V} \deg(u;G)}} 
	\nonumber \\
	~~~~~~~~~~~~~~~~~~~~~ + \left(\frac{1}{1 - 2as(G)} +  \frac{1}{2 as(G)}\right) \left(1 - \frac{\Phi^2(G)}{2}\right)^t. \label{eqn: tv_mixing_time}
	\end{align}
\end{lemma}

As mentioned, a bound on the total variation distance implies a corresponding bound on the uniform distance $d_{\mathrm{unif}}$, given by the following result:
\begin{lemma}
	\label{lem: tv_to_uniform_distance}
	Let $\norm{q_v^{(t)} - \pi}_{\mathrm{TV}} \leq \frac{s(G)}{14}$. Then,
	\begin{equation*}
	d_{\mathrm{unif}}(q_v^{(t + 3)},\pi) \leq \frac{1}{4}.
	\end{equation*}
\end{lemma}

\noindent Taking these Lemmas as given, we proceed that show that for
\begin{equation*}
\tau_1 = \frac{2}{\Phi^2(G)} \log \left(\frac{1440}{s(G)}\right)\log \left(\frac{14}{s(G)}\right)  + 3 \log \left(\frac{14}{s(G)}\right) + 3
\end{equation*}
the uniform distance $d_{\textrm{unif}}(q_v^{(\tau_1 + 3)}, \pi) \leq \frac{1}{4}$, which proves the claim of Proposition \ref{prop: pointwise_mixing_time}. Fix $a = \frac{1}{12}$, and let $\tau_0 = \frac{2}{\Phi^2(G)} \ln \left(\frac{320}{ s(G)}\right)$, so that
\begin{equation*}
\left(1 - \frac{\Phi^2(G)}{2}\right)^{\tau_0} \leq \exp(- \tau_0\Phi^2(G)/2) \leq \frac{s(G)}{320}.
\end{equation*}
Recall that by assumption, $\min_{u \in V} \deg(u;G) \geq 10$. Moreover note that since $a = 1/12 \leq 1/4$ and $s(G) \leq 9/10 < 1$, we have $2as(G) < 1/2 < 1 - 2as(G)$. By Lemma \ref{lem: tv_mixing_time} we therefore obtain
\begin{equation*}
\norm{q_v^{(\tau_0 + 3)} - \pi}_{\mathrm{TV}} \leq \max \set{\frac{1}{20}, \frac{1}{8} + \frac{9}{240} + \frac{1}{20}} + \left(\frac{12}{s(G)}\right) \frac{s(G)}{320} = \frac{1}{4}.
\end{equation*}
It is a well-known fact (see e.g.~\cite{montenegro2002}) that if $\norm{q_v^{(t)} - \pi}_{\mathrm{TV}} \leq \frac{1}{4}$, then for any $0 < \epsilon < 1$, $\norm{q_v^{(t\log_2(1/\epsilon))} - \pi}_{\mathrm{TV}} \leq \epsilon$. Therefore, letting $\tau_1 = (\tau_0 + 3) \log_2(\frac{14}{s(G)})$, 
\begin{equation*}
\norm{q_v^{(\tau_1)} - \pi}_{\mathrm{TV}} \leq \frac{s(G)}{14}
\end{equation*}
and so by Lemma \ref{lem: tv_to_uniform_distance}, $d_{\textrm{unif}}(q_v^{(\tau_1 + 3)}, \pi) \leq \frac{1}{4}$. The proof of Proposition \ref{prop: pointwise_mixing_time} is therefore complete once we have proved Lemmas \ref{lem: tv_mixing_time} and \ref{lem: tv_to_uniform_distance}.


\subsection{Proof of Lemma \ref{lem: tv_mixing_time}}
\label{sec: mixing_time_on_graphs}

We generalize the notation of the previous subsection. For a starting distribution $z$ to be specified later, and for $t \geq 0$ an integer, let $q_z^{(t)}$ be the $t$-step probability distribution of the lazy random walk with starting distribution $z$.\footnote{We say $z$ is a starting distribution over a graph $G$ when $\textrm{supp}(z) \subseteq V$ and $\sum_{u \in V}z(u) = 1$. Then, $q_z^{(t)} = zW^t$.}

We will find it useful to introduce the \emph{Lovasz-Simonovits curve}, defined for any $t \in \mathbb{N}$ and starting distribution $z$ to be $h_z^{(t)}: [0,1] \to [0,1]$,
\begin{equation*}
h_z^{(t)}(x) = \max_{w \in \mathcal{W}_x} \left\{ \sum_{u \in V} \left(q_z^{(t)}(u) - \pi(u)\right)w(u)\right\}.
\end{equation*}
The maximum in the preceding display is over the set of weight functions $\mathcal{W}_{x}$
\begin{align*}
\mathcal{W}_x = \left\{w: V \to [0,1] \Big\vert 0 \leq w(u) \leq 1~~\forall~u,~~\text{and}~~\sum_{u \in V} w(u) \pi(u) = x\right\}.
\end{align*}
The utility of the Lovasz-Simonovits curve is demonstrated by the following observations. First, note that
\begin{equation*}
\norm{q_v^{(t)} - \pi}_{\mathrm{TV}} = \sup_{S \subseteq V} \abs{q_v^{(t)}(S) - \pi(S)}
\end{equation*}
where we use the standard notation $\pi(S) := \sum_{u \in S} \pi(u)$, and likewise for $q_v^{(t)}(S)$. Moreover, observe that for any $S \subseteq V$ and any integer $t \geq 0$,
\begin{equation}
\label{eqn: lovasz_simonovits_1}
\abs{q_v^{(t)}(S) - \pi(S)} \leq \max \set{h_v^{(t)}(\pi(S)), h_v^{(t)}(1 - \pi(S))}
\end{equation}
(To see this, consider the weight functions $w(u) = \1(u \in S)$ and $w'(u) = 1 - w(u)$.) Taking the maximum on both sides of \eqref{eqn: lovasz_simonovits_1}  over all $S \subseteq V$, we have
\begin{equation*}
\norm{q_v^{(t)} - \pi}_{\mathrm{TV}} \leq \max_{0 \leq x \leq 1} h_v^{(t)}(x).
\end{equation*}
Now, take $z = e_vW^3$, and trivially observe that $h_v^{(t+3)}(x) = h_z^{(t)}(x)$. To prove Lemma \ref{lem: tv_mixing_time}, it is therefore sufficient to show the desired upper bound \eqref{eqn: tv_mixing_time} holds with respect to $h_z^{(t)}(x)$ for all $x \in [0,1]$, and all starting distributions $z = e_vW^3$. 

Let $\mu = as(G)$, and note under the condition $a < 1/4$, $\mu < 1 - \mu$. To show the desired upper bound, we split the interval $[0,1]$ into the subinterval $[\mu, 1 - \mu]$ and the remainder $[0,\mu) \cup (1 - \mu, 1]$. Let $\ell_{\mu}(x)$ be the linear interpolator between $h_z^{(0)}(\mu)$ and $h_z^{(0)}(1 - \mu)$, 
\begin{equation*}
\ell_{\mu}(x) = \frac{1 - \mu - x}{1 - 2\mu} h_z^{(0)}(\mu) + \frac{x - \mu}{1 - 2\mu}h_z^{(0)}(1 - \mu).
\end{equation*}

The following technical Lemma collects the upper bounds which hold over $[\mu, 1 - \mu]$ and $[0,\mu) \cup (1 - \mu, 1]$, respectively:
\begin{lemma}
	\label{lem: mixing_all_sets}
	For any $\mu \leq x \leq 1 - \mu$ and any starting distribution $z = e_vW^3$,
	\begin{equation}
	\label{eqn: mixing_large_sets}
	h_z^{(t)}(x) \leq \ell_{\mu}(x) + \left(\frac{1}{1 - 2\mu} + \frac{1}{\mu}\right)\left(1 - \frac{\Phi^2(G)}{2}\right)^t.
	\end{equation}
	For any $0 \leq x < \mu$ or $1 - \mu < x \leq 1$
	\begin{equation}
	\label{eqn: mixing_small_sets}
	h_z^{(t)}(x) \leq \max\set{as(G), \frac{1}{2^{t+3}} + \frac{9a}{20} + \frac{1}{2 \min_{u \in V}\deg(u;G)} }
	\end{equation}
\end{lemma}
\noindent Taking Lemma \ref{lem: mixing_all_sets} as given, we have nearly completed our proof of Lemma \ref{lem: tv_mixing_time}. Note that for any $\mu \leq x \leq 1 - \mu$,
\begin{align*}
\ell_{\mu}(x) & \leq \max\set{h_z^{(t)}(\mu),h_z^{(t)}(1 - \mu)} \\
& \leq \max\set{as(G), \frac{1}{2^{t + 3}} + \frac{9a}{20} + \frac{1}{2 \min_{u \in V}\deg(u;G)} }
\end{align*}
with the latter inequality following from \eqref{eqn: mixing_small_sets}. Therefore, for any $x \in [0,1]$,
\begin{equation*}
h_z^{(t)}(x) \leq \max\set{as(G), \frac{1}{8} + \frac{9a}{20} + \frac{1}{2 \min_{u \in V}\deg(u;G)} } + \left(\frac{1}{1 - 2as(G)} + \frac{1}{as(G)}\right)\left(1 - \frac{\Phi^2(G)}{2}\right)^t
\end{equation*}
which is exactly the claimed result of Lemma \ref{lem: tv_mixing_time}.

\subsection{Proof of Lemma \ref{lem: mixing_all_sets}}

We first prove \eqref{eqn: mixing_large_sets}, and then \eqref{eqn: mixing_small_sets}.
\paragraph{Proof of \eqref{eqn: mixing_large_sets}:}
The desired result is a consequence of Theorem 1.2 of \cite{lovasz1990}. To state this theorem, we introduce the notation
\begin{equation*}
C_{\mu} = \max \set{\frac{h_z^{(0)}(x) - \ell_{\mu}(x)}{\sqrt{x - \mu}}, \frac{h_z^{(0)}(x) - \ell_{\mu}(x)}{\sqrt{1 - x - \mu}}: \mu < x < 1 - \mu},
\end{equation*}
and then the theorem itself.
\begin{theorem}[Theorem 1.2 of \cite{lovasz1990}]
	\label{thm: lovasz_simonovits_1993}
	For any $\mu \leq x \leq 1 - \mu$, $z$ an arbitrary starting distribution, and an integer $t \geq 0$,
	\begin{equation}
	\label{eqn: lovasz_simonovits_1993}
	h_z^{(t)}(x) \leq \ell_{\mu}(x) + C_{\mu} \min \set{\sqrt{x - \mu}, \sqrt{1 - x - \mu}} \left(1 - \frac{\Phi^2(G)}{2}\right)^t
	\end{equation}
\end{theorem}
In order to show \eqref{eqn: mixing_large_sets}, we must therefore provide an appropriate bound on the quantity $C_{\mu}$. Precisely, we will show that for any $\mu < x < 1 - \mu$,
\begin{equation}
\label{eqn: lt_ub_1}
h_z^{(0)}(x) - \ell_{\mu}(x) \leq \max \set{\frac{h_z^{(0)}(\mu)}{1 - 2\mu} + \frac{h_z^{(0)}(\mu)}{\mu} , \frac{h_z^{(0)}(1 - \mu) }{1 - 2\mu} + 1 } \min \set{\sqrt{x - \mu},\sqrt{1 - x - \mu}}
\end{equation}
which will in turn imply
\begin{align*}
C_{\mu} & \leq \max \set{\frac{h_z^{(0)}(\mu)}{1 - 2\mu} + \frac{h_z^{(0)}(\mu)}{\mu} , \frac{h_z^{(0)}(1 - \mu) }{1 - 2\mu} + 1 } \\
& \leq \frac{1}{\mu} + \frac{1}{1 - 2\mu}.
\end{align*}
Plugging this upper bound into \eqref{eqn: lovasz_simonovits_1993}, we obtain
\begin{align*}
h_z^{(t)}(x) & \leq \ell_{\mu}(x) +  \left(\frac{1}{\mu} + \frac{1}{1 - 2\mu}\right)\min \set{\sqrt{x - \mu}, \sqrt{1 - x - \mu}} \left(1 - \frac{\Phi^2(G)}{2}\right)^t \\
& \leq \ell_{\mu}(x) +  \left(\frac{1}{\mu} + \frac{1}{1 - 2\mu}\right)\left(1 - \frac{\Phi^2(G)}{2}\right)^t
\end{align*}
which is the desired result.

It remains to show \eqref{eqn: lt_ub_1}. To do so, we make use of an equivalent representation of the Lovasz-Simonovits curve $h_z^{(t)}$. Order the elements of $V = \set{u_1, \ldots, u_N}$, where $N = \abs{V}$, such that
\begin{equation}
\label{eqn: lovasz_simonovits_curve_1}
\frac{q_z^{(t)}(u_1)}{\pi(u_1)} \geq \frac{q_z^{(t)}(u_2)}{\pi(u_2)} \geq \ldots \geq \frac{q_z^{(t)}(u_N)}{\pi(u_N)},
\end{equation}
and for each $k = 1,\ldots, N$, let $U_k = \set{u_1, \ldots, u_k}$. Then for any $x \in [0,1]$, letting $k$ satisfy $\pi(U_{k - 1}) < x < \pi(U_k)$, it can be shown that\footnote{See \cite{lovasz1990} for a formal justification.},
\begin{equation}
\label{eqn: lovasz_simonovits_curve}
h_z^{(t)}(x) = \sum_{j = 1}^{k - 1} (q_z^{(t)}(u_j) - \pi(u_j)) + \frac{x - \pi(U_{k - 1})}{\pi(u_k)} \left(q_z^{(t)}(u_k) - \pi(u_k) \right).
\end{equation}
The representation of the Lovasz-Simonovits curve given by~\eqref{eqn: lovasz_simonovits_curve} makes it clear that $h_z^{(t)}(x)$ is a piecewise linear curve, with knots at the points $x_k = \pi(U_k)$ for $k = 1,\ldots,N$, where the $k$th linear segment has slope $v_k = q_z^{(t)}(u_{k})/\pi(u_{k}) - 1$. By the ordering of~\eqref{eqn: lovasz_simonovits_curve_1}, we have that $v_1(x) > v_2(x) > \ldots > v_{N-1}(x)$, and the curve is therefore concave. 

In fact, letting $v(x) = \sum_{k = 1}^{N} v_k \1(x \in [U_k, U_{k+1}) )$ be the slope of the Lovasz-Simonovits curve, for any $x \geq \mu$ it can be shown that
\begin{align}
v(x) & \leq \min_{k: u_k \leq x} v_k \nonumber  \\
& \leq \frac{h_z^{(0)}(\mu)}{\mu}. \label{eqn: lovasz_simonovits_slope}
\end{align}
From the upper bound in~\eqref{eqn: lovasz_simonovits_slope} along with the concavity of $h_z^{(0)}(x)$, we have that for any $x \geq \mu$,
\begin{equation*}
h_z^{(0)}(x) \leq h_z^{(0)}(\mu) + (x - \mu)\frac{h_z^{(0)}(\mu)}{\mu}.
\end{equation*}
Some algebra then yields that for any $x \geq \mu$,
\begin{align*}
h_z^{(0)}(x) - \ell_{\mu}(x) & \leq h_z^{(0)}(\mu) - \left(\frac{1 - \mu - x}{1 - 2\mu} h_z^{(0)}(\mu) + \frac{x - \mu}{1 - 2\mu} h_z^{(0)}(1 - \mu)\right) + \frac{h_z^{(0)}(\mu)}{\mu}(x - \mu) \\
& =  \frac{x - \mu}{1 - 2\mu}h_z^{(0)}(\mu) + \frac{h_z^{(0)}(\mu)}{\mu} (x - \mu) - \frac{x - \mu}{1 - 2\mu} h_z^{(0)}(1 - \mu) \\
& \leq \sqrt{x - \mu} \left(\frac{h_z^{(0)}(\mu)}{1 - 2\mu} + \frac{h_z^{(0)}(\mu)}{\mu} \right)
\end{align*}
On the other hand, $\ell_{\mu}(1 - \mu) = h_z^{(0)}(1 - \mu)$, and by the concavity of $h_z^{(0)}$ and \eqref{eqn: lovasz_simonovits_slope},  for $x \leq 1 - \mu$
\begin{equation*}
h_z^{(0)}(x) \leq h_z^{(0)}(1 - \mu) + (1 - x - \mu).
\end{equation*}
Similar manipulations to above give the upper bound
\begin{align*}
h_z^{(0)}(x) - \ell_{\mu}(x) \leq \sqrt{1 - \mu - x}\left(\frac{h_z^{(0)}(1 - \mu) }{1 - 2\mu} + 1\right)
\end{align*}
and \eqref{eqn: lt_ub_1} follows.

\paragraph{Proof of \eqref{eqn: mixing_small_sets}:}
We let $z = e_vW^3$ for an arbitrary $v \in V$. First, we deal with the case $x \leq as(G)$. Letting $\pi(U_{k-1}) \leq x \leq \pi(U_{k})$, we have
\begin{equation}
\label{eqn: mixing_time_small_sets_1}
h_z^{(t)}(x) \leq q_z^{(t)}(U_{k - 1}) + q_z^{(t)}(u_k)
\end{equation}
We observe a few facts about the random walk defined by $q_z^{(t)}$. By the laziness of the random walk, for $u \neq v, t \geq 1$
\begin{equation}
\label{eqn: mixing_time_small_sets_2}
q_z^{(t)}(u) \leq \frac{1}{2 \min_{u \in V}\deg(u;G)}
\end{equation}
On the other hand if $u = v$,
\begin{equation}
\label{eqn: mixing_time_small_sets_3}
q_z^{(t)}(u) \leq \frac{1}{2^{t + 3}} + \frac{1}{2\min_{u \in V}\deg(u;G)}.
\end{equation}
Therefore by \eqref{eqn: mixing_time_small_sets_1}, \eqref{eqn: mixing_time_small_sets_2}, and \eqref{eqn: mixing_time_small_sets_3}
\begin{equation*}
h_z^{(t)}(x) \leq \frac{1}{2^{t + 3}} + \frac{\abs{U_{k - 1}}}{2 \min_{u \in V}\deg(u;G)} + \frac{1}{2 \min_{u \in V}\deg(u;G)}.
\end{equation*}
Finally, since $x \leq a s(G)$ and $x \geq \pi(U_{k - 1})$,
\begin{align*}
\abs{U_{k - 1}} & \leq \frac{\pi(U_{k-1})}{\min_{u \in V} \pi(v) } \\
& \leq \frac{as(G)}{\min_{u \in V} \pi(v) } \\
& \leq \frac{9a\min_{u \in V}\deg(u;G)}{10}.
\end{align*}
and the desired result is shown.

Now, we turn to the case where $x \geq 1 - as(G)$. Noting that the slope $v(x)$ of the Lovasz-Simonovits curve $h_z^{(t)}$ is bounded below by $-1$ for all $x \in [0,1]$, by the concavity of $h_z^{(t)}$ we have
\begin{align*}
h_z^{(t)}(x) & \leq h_z^{(t)}(1) + (1 - x) \\
& = 1 - x \leq as(G).
\end{align*}
and the proof of Lemma~\ref{lem: tv_mixing_time} is complete.

\subsection{Proof of Lemma \ref{lem: tv_to_uniform_distance}}

The proof of Lemma \ref{lem: tv_to_uniform_distance} follows from straightforward algebraic manipulation. Fix $u \in V$ and let $m \geq t + 1$ be arbitrary. The stationarity of $\pi$ gives (see~(16) of \citep{morris2005})
\begin{align}
\frac{\pi(u) - q_v^{m}(u)}{\pi(u)} & = \sum_{y \in V} \left(\pi(y) - q^{(m-1)}(v,y)\right) \left(\frac{q^{(1)}(y,u) - \pi(u)}{\pi(u)}\right) \nonumber \\
& \overset{\text{(i)}}{=} \sum_{y \neq u} \left(\pi(y) - q^{(m-1)}(v,y)\right) \left(\frac{q^{(1)}(y,u) - \pi(u)}{\pi(u)}\right) + \frac{\pi(u) - q^{(m - 1)}(v,u)}{\pi(u)} \left(\frac{1}{2} - \pi(u)\right) \nonumber \\
& \leq \sum_{y \neq u} \left(\pi(y) - q^{(m-1)}(v,y)\right) \left(\frac{q^{(1)}(y,u) - \pi(u)}{\pi(u)}\right) + \frac{\pi(u) - q^{(m - 1)}(v,u)}{2 \pi(u)} \label{eqn: tv_to_uniform_distance_1}
\end{align}
where $\text{(i)}$ follows from $q^{(1)}(u,u) = \frac{1}{2}$. Then,
\begin{align}
\sum_{y \neq u} \left(\pi(y) - q^{(m-1)}(v,y)\right) \left(\frac{q^{(1)}(y,u) - \pi(u)}{\pi(u)}\right) & \leq \norm{q_v^{(m-1)} - \pi}_{\mathrm{TV}} \max_{y \neq u} \abs{\frac{q^{(1)}(y,u) - \pi(u)}{\pi(u)}} \nonumber \\
& \leq \norm{q_v^{(m-1)} - \pi}_{\mathrm{TV}} \max \set{1, \max_{y \neq u}\set{\frac{q^{(1)}(y,u)}{\pi(u)}}} \nonumber \\
& \overset{(ii)}{\leq} \norm{q_v^{(m-1)} - \pi}_{\mathrm{TV}} \max \set{1, \frac{1}{2 \pi(u) \min_{u' \in V} \deg(u';G) }} \nonumber \\
& \leq \norm{q_v^{(m-1)} - \pi}_{\mathrm{TV}} \frac{1}{s(G)} \label{eqn: tv_to_uniform_distance_3}
\end{align}
where $(ii)$ follows from the fact that for $y \neq u$, $q^{(1)}(y,u) \leq 1/\left(2 \min_{u \in V} \deg(u; G)\right)$. As $m - 1 \geq t$, it is well-known (see e.g. \cite{lovasz1990}) that the laziness of the random walk guarantees $\norm{q_v^{(m - 1)} - \pi}_{\mathrm{TV}} \leq \norm{q_v^{(t)} - \pi}_{\mathrm{TV}}$, and therefore by \eqref{eqn: tv_to_uniform_distance_3} and the hypothesis of Lemma \ref{lem: tv_to_uniform_distance},
\begin{equation*}
\sum_{y \neq u} \left(\pi(y) - q^{(m-1)}(v,y)\right) \left(\frac{q^{(1)}(y,u) - \pi(u)}{\pi(u)}\right) \leq \frac{1}{14}.
\end{equation*}
Plugging this in to \eqref{eqn: tv_to_uniform_distance_1} and taking maximum on both sides, we obtain
\begin{equation}
d_{\textrm{unif}}(q_v^{(m)}, \pi) \leq \frac{1}{14} + \frac{d_{\textrm{unif}}(q_v^{(m - 1)}, \pi)}{2} \label{eqn: tv_to_uniform_distance_2}
\end{equation}
The recurrence relation of \eqref{eqn: tv_to_uniform_distance_2} along with the initial condition $d_{\textrm{unif}}(q_v^{(t)}, \pi) \leq 1$ yields
\begin{equation*}
d_{\textrm{unif}}(q_v^{(t + 1)}, \pi) \leq \frac{8}{14} \Rightarrow d_{\textrm{unif}}(q_v^{(t + 2)}, \pi) \leq \frac{5}{14} \Rightarrow  d_{\textrm{unif}}(q_v^{(t + 3)}, \pi) \leq \frac{1}{4}
\end{equation*}
and the claim is shown. We have proved Lemmas \ref{lem: tv_mixing_time} and \ref{lem: tv_to_uniform_distance}, and therefore Proposition \ref{prop: pointwise_mixing_time}.

\subsection{Proof of Proposition \ref{prop: local_spread_conductance}}

We prove~\eqref{eqn:min_degree} and~\eqref{eqn: local_spread} immediately before turning our attention to~\eqref{eqn: conductance}, which will require the bulk of our attention. First, however, we introduce some notation. For $S \subseteq \Csig[\Xbf]$ and $u \in \Csig[\Xbf]$, we will write
\begin{equation*}
\widetilde{\cut}_{n,r}(S) := \cut(S; \widetilde{G}_{n,r}), \quad \widetilde{\deg}_{n,r}(u) := \deg(u; \widetilde{G}_{n,r}), \quad 
\widetilde{\vol}_{n,r}(S) := \vol(S; \widetilde{G}_{n,r})
\end{equation*}
and let $\widetilde{\pi}_{n,r}(u) = \widetilde{\deg}_{n,r}(u)/\widetilde{\vol}_{n,r}(\Csig[\Xbf])$ be the stationary distribution of the lazy random walk over $\widetilde{G}_{n,r}$. We also let $\widetilde{\Phi}_{n,r}(S) := \Phi(S; \widetilde{G}_{n,r})$ denote the normalized cut functional over $\widetilde{G}_{n,r}$, and $\degminwt := \min_{u \in \Csig[\Xbf]} \widetilde{\deg}_{n,r}(u)$.

\paragraph{Proof of~\eqref{eqn:min_degree}:}

Applying Lemma~\ref{lem:graph_functional_concentration} with $\delta = 1/2$, we see that
\begin{align*}
\wt{d}_{\min} & \geq \frac{n}{4}\biggl(1 - \sqrt{\frac{d + 2}{2\pi}}\frac{r}{\sigma}\biggr)\lambda_{\sigma} \nu_d r^d \\
& \geq \frac{n}{8}\lambda_{\sigma} \nu_d r^d,
\end{align*}
with probability at least $1 - 2n\exp\Bigl\{-\lambda_{\sigma} \nu_dr^dn/28\Bigr\}$. Therefore for any
\begin{equation*}
n \geq \frac{80}{\lambda_{\sigma}\nu_dr^d} =:b_4
\end{equation*}
the minimum degree $\wt{d}_{\min} \geq 10$.

\paragraph{Proof of \eqref{eqn: local_spread}:}

We rewrite $s(\widetilde{G}_{n,r}) = \frac{9 \degminwt^2}{10\widetilde{\vol}_{n,r}(\Csig[\Xbf])}$. Then applying Lemma \ref{lem:ball_bounds_in_probability} to upper bound the minimum degree and lower bound the volume, we have that with probability at least $1 - 2n\exp\Bigl\{-\delta^2 \lambda_{\sigma} \nu_dr^dn/(6 + 2\delta)\Bigr\} - 2\exp\set{-n\delta^2(\widetilde{\vol}_{\Pbb,r}(\Csig))^2}$
\begin{align*}
s(\widetilde{G}_{n,r}) & \geq \frac{9}{40} \cdot \biggl(1 - \sqrt{\frac{d + 2}{2\pi}}\frac{r}{\sigma}\biggr)^2 \cdot \frac{(1 - \delta)^2}{(1 + \delta)} \cdot \frac{\nu_d^2 r^{2d} \lambda_{\sigma}^2}{\wt{\vol}_{\Pbb,r}(\mc{C}_{\sigma})} \\
& \geq \frac{9}{40} \cdot \biggl(1 - \sqrt{\frac{d + 2}{2\pi}}\frac{r}{\sigma}\biggr)^2 \cdot \frac{(1 - \delta)^2}{(1 + \delta)} \cdot \frac{\nu_d r^d \lambda_{\sigma}^2}{\Lambda_{\sigma}^2 \nu(\mc{C}_{\sigma})} \\
& \geq \frac{9}{40} \cdot \biggl(1 - \sqrt{\frac{d + 2}{2\pi}}\frac{r}{\sigma}\biggr)^2 \cdot \frac{(1 - \delta)^2}{(1 + \delta)} \cdot \frac{\lambda_{\sigma}^2(2r)^d}{\Lambda_{\sigma}^2(L\rho)^d},
\end{align*}
where the last inequality follows by Assumption~\ref{asmp: embedding}, which ensures that $\mc{C}_{\sigma}$ is contained in a ball of radius $L\rho/2$. 

\paragraph{Proof of \eqref{eqn: conductance}:}
Roughly speaking, our goal is to show that for sufficiently large $n$, with probability at least $1 - \frac{b_5}{n}$,
\begin{equation*}
\min_{S \subseteq \Csig[\Xbf]} \widetilde{\Phi}_{n,r}(S) \geq \Bigl(1 - \delta\Bigr)\Bigl(1 - \frac{r}{4\rho L}\Bigr) \Bigl(1 - \frac{r}{\sigma}\sqrt{\frac{d + 2}{2\pi}}\Bigr)^2 \cdot \frac{\sqrt{2\pi}}{36} \cdot \frac{r\lambda_{\sigma}^2}{\Lambda_{\sigma}^2\rho L \sqrt{d + 2}}
\end{equation*}
In order to show this bound holds uniformly over all sets $S \subseteq \Csig[\Xbf]$, we will split the analysis into two cases based on the size of $S \subseteq \Csig[\Xbf]$. To do so, we introduce $\mathcal{L}(G) := \set{S \subseteq V: \pi(S), \pi(S^c) \geq s(G)}$ (where as usual $\pi$ denotes the stationary distribution of a lazy random walk over $G$.)

Lemma \ref{lem: graph_conductance_profile_lb_1} shows that for any subset $S \subseteq \Csig[\Xbf]$ not in $\mathcal{L}(\widetilde{G}_{n,r})$, the graph normalized cut of $S$ is at least $1/10$. In fact, this statement holds for any graph $G$.
\begin{lemma}
	\label{lem: graph_conductance_profile_lb_1}
	Let $G = (V,E)$ be an arbitrary undirected graph. Then, for non-empty subsets $S \subseteq V$, 
	\begin{equation*}
	\min_{S \not\in \mathcal{L}(G)}\Phi(S; G) \geq \frac{1}{10}.
	\end{equation*}
\end{lemma}
\begin{proof}
	The claim follows by direct manipulations:
	\begin{align*}
	\Phi(S;G) & \geq \frac{\cut(S;G)}{\vol(S;G)} \\
	& \geq \sum_{u \in S} \frac{\deg(u;G) - \abs{S}}{\vol(S;G)} \\
	& \geq \sum_{u \in S} \frac{\deg(u;G) - \pi(S)/(\min_{u \in V}\pi(u))}{\vol(S;G)} \\
	& \geq \sum_{u \in S} \frac{\deg(u;G) - \frac{9}{10} \min_{u \in V}\deg(u;G)}{\vol(S;G)} \tag{since $S \not\in \mc{L}(G)$} \\
	& \geq \frac{1}{10} \sum_{u \in S} \frac{\deg(u;G)}{\vol(S;G)} = \frac{1}{10}.
	\end{align*}
\end{proof}

In Lemma \ref{lem: graph_conductance_profile_lb}, we establish a uniform lower bound on the normalized cut for the remaining sets $S \in \mathcal{L}(\widetilde{G}_{n,r})$. Let $p_d = 3/4$ if $d = 2$, and $p_d = 1/d$ if $d \neq 2$.

\begin{lemma}
	\label{lem: graph_conductance_profile_lb}
	Fix $\delta \in (0,1)$. Under the setup and conditions of Theorem \ref{thm: mixing_time_upper_bound}, there exist constants $b_3$ and $b_5$ independent of $n$ such that the following statement holds true: for any $n$ such that
	\begin{equation*}
	\frac{n}{(\log n)^{d p_d}} \geq \frac{(1 + \delta)^2}{(1 - \delta)^2}b_3^d
	\end{equation*}
	the following upper bound holds
	\begin{equation}
	\label{eqn: graph_conductance_profile_lb}
	\min_{S \in \mathcal{L}(\widetilde{G}_{n,r})} \widetilde{\Phi}_{n,r}(S) \geq \Bigl(1 - \delta\Bigr)\Bigl(1 - \frac{r}{4\rho L}\Bigr) \Bigl(1 - \frac{r}{\sigma}\sqrt{\frac{d + 2}{2\pi}}\Bigr)^2 \cdot \frac{\sqrt{2\pi}}{36} \cdot \frac{r\lambda_{\sigma}^2}{\Lambda_{\sigma}^2\rho L \sqrt{d + 2}}.
	\end{equation}
	with probability at least $1 - \frac{b_5}{n} - 2n\exp\Bigl\{-\delta^2 \lambda_{\sigma} \nu_dr^dn/(6 + 2\delta)\Bigr\} - 2\exp\set{-n\delta^2(\widetilde{\vol}_{\Pbb,r}(\Csig))^2}$. 
\end{lemma}

The desired upper bound on graph conductance \eqref{eqn: conductance} follows from Lemma \ref{lem: graph_conductance_profile_lb}, along with Lemma \ref{lem: graph_conductance_profile_lb_1}, in light of the fact $\frac{\lambda_{\sigma}^2 r}{\Lambda_{\sigma}^2 2^{13} \rho L \sqrt{d}} < \frac{1}{10}$. We turn now to the proof of Lemma \ref{lem: graph_conductance_profile_lb}.

\subsection{Proof of Lemma \ref{lem: graph_conductance_profile_lb}.}

The proof of Lemma~\ref{lem: graph_conductance_profile_lb} will essentially follow from a pair of technical results. The first of these will demonstrate that the functional $\widetilde{\Phi}_{n,r}(S)$ can be lower bounded by a population analogue $\widetilde{\Phi}_{\Pbb,r}(\Sset)$, for an appropriately chosen $\Sset \subseteq \Csig$; we term this latter functional the continuous normalized cut. This lower bound will hold uniformly over sets $S \subseteq \Csig[X]$. The second technical Lemma will build on known continuous space isoperimetric inequalities to lower bound the continuous normalized cut $\widetilde{\Phi}_{\Pbb,r}(\Sset)$ uniformly over sets $\Sset \subseteq \Csig$.

We will now build slowly toward a formal definition of the continuous normalized cut, before establishing a relation between it and it's discrete counterpart. Let $\Sset \subseteq \Csig$ be measurable. We introduce the \emph{$r$-ball walk}, a Markov chain over $\Csig$ with transition probability at $x \in \Csig$ given by 
\begin{equation*}
\widetilde{P}_{\Pbb,r}(x; \Sset) := \frac{\Pbb(\Sset \cap B(x,r))}{\Pbb(\Csig \cap B(x,r))}.
\end{equation*}

Denote the stationary distribution for this Markov chain by $\piwt_{\Pbb,r}$, which is defined by the relation
\begin{equation*}
\int_{\Csig} \widetilde{P}_{\Pbb,r}(x; \Sset) \,d \piwt_{\Pbb,r}(x) = \piwt_{\Pbb,r}(\Sset).
\end{equation*}
Letting the \emph{$\Pbb$-local conductance} be given by
\begin{equation*}
\ell_{\Pbb,r}(x) := \Pbb\bigl(\Csig \cap B(x,r)\bigr)
\end{equation*}
a bit of algebra verifies that
\begin{equation*}
\piwt_{\Pbb,r}(\Sset) = \frac{\int_{\Sset} \ell_{\Pbb,r}(x) \,d \Pbb(x) }{\int_{\Csig} \ell_{\Pbb,r}(x) \,d \Pbb(x)}.
\end{equation*}

We next introduce the \emph{ergodic flow}, $\widetilde{Q}_{\Pbb,r}$. Let $\Sset \cap \Sset^c = \Csig$ be a partition of $\Csig$. Then the ergodic flow between $\Sset$ and $\Sset$ is given by 
\begin{equation*}
\widetilde{Q}_{\Pbb,r}(\Sset, \Sset^c) := \int_{\Sset} \widetilde{P}_{\Pbb,r}(x; \Sset^c) \,d\piwt_{\Pbb,r}(x), 
\end{equation*}
and the \emph{($\Pbb$) continuous normalized cut} by
\begin{equation*}
\widetilde{\Phi}_{\Pbb,r}(\Sset) := \frac{\widetilde{Q}_{\Pbb,r}(\Sset, \Sset^c)}{\min \set{\piwt_{\Pbb,r}(\Sset),\piwt_{\Pbb,r}(\Sset^c)}},
\end{equation*}

As the functionals $\widetilde{\Phi}_{n,r}$ and $\widetilde{\Phi}_{\Pbb,r}$ act in the different spaces $\Csig[\Xbf]$ and $\Csig$, respectively, it is not obvious how to relate them. To do so, following the lead of \cite{garciatrillos16}, we introduce transportation maps between the space $\Csig$ and the sample points $\Csig[\Xbf]$. We note that by assumption~\ref{asmp: bounded_density}, $\Pbb(\Csig) > 0$, and therefore with probability one as $n \to \infty$, the number of sample points $\abs{\Csig[\Xbf]}$ will be non-zero as well. We may therefore define the conditional probability measures
\begin{equation*}
\widetilde{\Pbb}(\Sset) = \frac{\Pbb(\Sset)}{\Pbb(\Csig)}, ~ \widetilde{\Pbb}_{n}(\Sset) := \frac{1}{\abs{\Csig[\Xbf]}} \sum_{x_i \in \Csig[\Xbf]} \1(x_i \in \Sset).
\end{equation*} 
We then define a \emph{transportation map} between $\widetilde{\Pbb}$ and $\widetilde{\Pbb}_n$ to be any measurable map $T: \Csig \to \Csig[\Xbf]$ such that for every $S \subseteq \Csig[\Xbf]$,
\begin{equation*}
\widetilde{\Pbb}(T^{-1}(S)) = \widetilde{\Pbb}_n(S),
\end{equation*}
where $T^{-1}(S) = \set{x \in \Csig: T(x) \in S}$ is the preimage of $T$. Observe that by the definition of the transportation map $T$, for any $g \in L^1\bigl(\widetilde{\Pbb}_n\bigr)$ the following change of variables formula holds
\begin{equation*}
\int_{\Csig} g(x) \,d\widetilde{\Pbb}_{n}(x) = \int_{\Csig} g\bigl(T(x)\bigr) \,d\widetilde{\Pbb}(x)
\end{equation*}
Using the change of variables formula with an appropriate choice of $g$, after suitable rescaling we can relate $\widetilde{\cut}_{n,r}(S)$ to $\widetilde{Q}_{\Pbb,r}(T^{-1}(S), T^{-1}(S)^c)$. Similarly, we can relate $\widetilde{\vol}_{n,r}(S^c)$ to $\widetilde{\pi}_{\Pbb,r}(T^{-1}(S))$. Working along these lines, we obtain the following lower bound on $\widetilde{\Phi}_{n,r}(S)$, stated in terms of the transportation distance $\norm{\mathrm{Id} - T}_{L^{\infty}(\Pbb)}$, where $\mathrm{Id}(x) = x$ is the identity mapping over $\Csig$.

\begin{lemma}
	\label{lem: graph_to_continuous_conductance}
	Let $T:\Csig \to \Csig[\Xbf]$ be a transportation map between $\widetilde{\Pbb}$ and $\widetilde{\Pbb}_n$. Suppose $\norm{\mathrm{Id} - T}_{L^{\infty}(\widetilde{\Pbb})} < \min \set{s(\widetilde{G}_{n,r}), \lambda_{\sigma}r/(2^{d+3}d\Lambda_{\sigma})}$. Then there exists a constant $b_6 > 0$ which does not depend on the sample size $n$, such that for all $S \in \mathcal{L}(\widetilde{G}_{n,r})$, 
	\begin{equation}
	\label{eqn: graph_to_continuous_conductance}
	\widetilde{\Phi}_{n,r}(S) \geq \widetilde{\Phi}_{\Pbb,r}(T^{-1}(S)) - \frac{b_6\norm{\mathrm{Id} - T}_{L^{\infty}(\widetilde{\Pbb})} }{s(\widetilde{G}_{n,r}) - b_6 \norm{\mathrm{Id} - T}_{L^{\infty}(\widetilde{\Pbb})}  }
	\end{equation}
\end{lemma}

Clearly, Lemma \ref{lem: graph_to_continuous_conductance} is useful only when combined with an upper bound on the transportation distance $\norm{\mathrm{Id} - T_n}_{L^{\infty}(\widetilde{\Pbb})}$. Theorem 1.1 of \cite{garciatrillos16b} establishes such an upper bound, with respect to transportation maps on measures supported on open, connected and bounded domains with Lipschitz boundaries. The following result is a restatement of this Theorem with respect to the domain $\Csig$, and the measure $\widetilde{\Pbb}$. Although $\Csig$ is closed rather than open, as $\nu(\partial \Csig) = 0$ we may apply Theorem 1.1 of \cite{garciatrillos16} to the interior $\Csig^o$ of $\Csig$, and the desired result will hold for any arbitrary extension of $T_n$ to $\Csig$. Let $\widetilde{n} = \abs{\Csig[\Xbf]}$. 

\begin{theorem}[Restatement of Theorem 1.1 of \cite{garciatrillos16b}]
	\label{thm: stagnating_transportation_maps}
	There exists a constant $b_5$ which do not depend on $n$ such that with probability at least $1 - \frac{b_5}{n}$:
	\begin{equation*}
	\norm{\mathrm{Id} - T_n}_{L^{\infty}(\widetilde{\Pbb})} \leq 
	b_{5} \begin{cases*}
	\frac{\log(n)^{3/4}}{n^{1/2}},&~~\textrm{if $d = 2$,} \\
	\frac{\log(n)^{1/d}}{n^{1/d}},&~~\textrm{if $d \neq 2$.}
	\end{cases*} 
	\end{equation*}
\end{theorem}
Lemma~\ref{lem: graph_to_continuous_conductance} and Theorem~\ref{thm: stagnating_transportation_maps} show that with high probability, the discrete normalized cut $\widetilde{\Phi}_{n,r}(S)$ is lower bounded by the continuous normalized cut $\widetilde{\Phi}_{\Pbb,r}(T_n^{-1}(S))$ over all sufficiently large sets $S \subseteq \Xbf$. The following result then supplies the last step, a uniform lower bound on the continuous normalized cut $\widetilde{\Phi}_{\Pbb,r}(\Sset)$ for all sets $\Sset \subseteq \Csig$. Let the \emph{$\Pbb$-continuous conductance} be given by
\begin{equation*}
\widetilde{\Phi}_{\Pbb,r} := \min_{\Sset \subseteq \Csig} \widetilde{\Phi}_{\Pbb,r}(\Sset).
\end{equation*} 
\begin{lemma}
	\label{lem: nonuniform_continuous_conductance}
	Under the setup and conditions of Theorem \ref{thm: mixing_time_upper_bound}, the $\Pbb$-continuous conductance of the $r$-ball walk satisfies
	\begin{equation*}
	\widetilde{\Phi}_{\Pbb,r} \geq \Bigl(1 - \frac{r}{4\rho L}\Bigr) \Bigl(1 - \frac{r}{\sigma}\sqrt{\frac{d + 2}{2\pi}}\Bigr)^2 \cdot \frac{\sqrt{2\pi}}{36} \cdot \frac{r\lambda_{\sigma}^2}{\Lambda_{\sigma}^2\rho L \sqrt{d + 2}}.
	\end{equation*}
\end{lemma}
With Lemmas \ref{lem: graph_to_continuous_conductance} and \ref{lem: nonuniform_continuous_conductance}, as well as Theorem \ref{thm: stagnating_transportation_maps}, in hand, we proceed to the proof of Lemma \ref{lem: graph_conductance_profile_lb}. 
\begin{proof}[Lemma~\ref{lem: graph_conductance_profile_lb}]
To invoke Lemma~\ref{lem: graph_to_continuous_conductance}, we must establish an appropriate upper bound on $\|\mathrm{Id} - T_n\|_{L^{\infty}(\wt{P})}$. We do so using Theorem~\ref{thm: stagnating_transportation_maps}, which implies that if
\begin{equation*}
n^{1/d} > b_5\log(n)^{p_d} \cdot \frac{(1 + \delta)^2}{(1 - \delta)^2}  \max\Biggl\{\frac{b_6 (1 + \delta)^2 \Lambda_{\sigma}^2 (L\rho)^d}{9(1 - \delta)^2 \lambda_{\sigma}^2 (2r)^d \wt{\Phi}_{\Pbb,r}}, \frac{2^{d + 3}d \Lambda_{\sigma}}{r \lambda_{\sigma}}\Biggr\} =: \log(n)^{p_d} \cdot b_3 \cdot \frac{(1 + \delta)^2}{(1 - \delta)^2} 
\end{equation*}
then
\begin{equation*}
\|\mathrm{Id} - T_n\|_{L^{\infty}(\wt{P})} \leq \min\biggl\{\frac{\delta \cdot s(\wt{G}_{n,r}) \wt{\Phi}_{\Pbb,r}}{b_6(1 + \delta \wt{\Phi}_{\Pbb,r})}, \frac{\lambda_\sigma r}{2^{d + 3}d \Lambda_{\sigma}}\biggr\}
\end{equation*}
(In the above, we take as given that the inequality~\eqref{eqn: local_spread} is satisfied.) As a result we may apply Lemma~\ref{lem: graph_to_continuous_conductance}, from which we obtain
\begin{align*}
\min_{S \in \mathcal{L}(\widetilde{G}_{n,r})} \widetilde{\Phi}_{n,r}(S) & \geq \min_{S \in \mathcal{L}(\widetilde{G}_{n,r})} \widetilde{\Phi}_{\Pbb,r}(T_n^{-1}(S)) - \frac{b_6 \norm{\mathrm{Id} - T_n}_{L^{\infty}(\widetilde{\Pbb})} }{s(\widetilde{G}_{n,r}) - b_6\norm{\mathrm{Id} - T_n}_{L^{\infty}(\widetilde{\Pbb})}} \\
& \geq  \widetilde{\Phi}_{\Pbb,r} -  \frac{b_6 \norm{\mathrm{Id} - T_n}_{L^{\infty}(\widetilde{\Pbb})} }{s(\widetilde{G}_{n,r}) - b_6\norm{\mathrm{Id} - T_n}_{L^{\infty}(\widetilde{\Pbb})}} \\
& \geq (1- \delta)\widetilde{\Phi}_{\Pbb,r},
\end{align*}
where the last inequality follows since $\delta < \frac{\widetilde{\Phi}_{\Pbb,r}\cdot s(\widetilde{G}_{n,r})}{2b_6 + \widetilde{\Phi}_{\Pbb,r}}$. Then, Lemma~\ref{lem: nonuniform_continuous_conductance} yields the desired result. 
\end{proof}
It remains to prove Lemmas \ref{lem: graph_to_continuous_conductance} and \ref{lem: nonuniform_continuous_conductance}, which we do in Appendices~\ref{subsec:pf_graph_to_continuous_conductance}-\ref{subsec:pf_uniform_continuous_conductance}.

\subsection{Proof of Lemma \ref{lem: graph_to_continuous_conductance}}
\label{subsec:pf_graph_to_continuous_conductance}
Let $S \subseteq \Csig[\Xbf]$ belong to $\mathcal{L}(\widetilde{G}_{n,r})$, and denote $\Sset := T^{-1}(S)$. Further let $\varDelta := \norm{\mathrm{Id} - T}_{L^{\infty}(\widetilde{\Pbb})}$, and $r^{+} := r + \varDelta$ and $r^{-} := r - \varDelta$. Our goal is to lower bound
\begin{equation*}
\frac{\widetilde{\cut}_{n,r}(S,S^c)}{\widetilde{\vol}_{n,r}(S)} \geq \widetilde{\Phi}_{\Pbb,r}(\Sset) - \frac{b_6\varDelta}{s(\widetilde{G}_{n,r}) - b_6 \varDelta}
\end{equation*}
for some constant $b_6 > 0$. We first state the relationships between the discrete functionals $\widetilde{\cut}_{n,r}$ and $\widetilde{\vol}_{n,r}$, and the continuous functionals $\widetilde{Q}_{\Pbb,r}$ and $\widetilde{\pi}_{\Pbb,r}$, alluded to in the previous section.
\begin{lemma}
	\label{lem: cut_volume}
	For any set $S \subseteq \Csig[\Xbf]$ and $\mathcal{S} = T^{-1}(S)$, 
	\begin{equation}
	\label{eqn: cut}
	\frac{1}{\widetilde{n}^2}\widetilde{\cut}_{n,r}(S) \geq \frac{\int_{\Csig} \ell_{\Pbb,r^-}(x) \,d\Pbb(x)}{\Pbb(\Csig)^2} \widetilde{Q}_{\Pbb,r^-}(\Sset, \Sset^c)
	\end{equation}
	and 
	\begin{equation}
	\label{eqn: volume}
	\frac{1}{\widetilde{n}^2}\widetilde{\vol}_{n,r}(S) \leq \frac{\int_{\Csig} \ell_{\Pbb,r^+}(x) \,d\Pbb(x)}{\Pbb(\Csig)^2} \widetilde{\pi}_{\Pbb,r^+}(\Sset)
	\end{equation}
\end{lemma}
To make use of Lemma \ref{lem: cut_volume}, we provide deviation inequalities on $\int_{\Csig} \ell_{\Pbb,r^+}(x) \,d\Pbb(x)$, $\int_{\Csig} \ell_{\Pbb,r^-}(x) \,d\Pbb(x)$, $\widetilde{Q}_{\Pbb,r^-}(\Sset, \Sset^c)$, and $\widetilde{\pi}_{\Pbb,r^+}(\Sset)$ in terms of the transportation distance $\varDelta$.
\begin{lemma}
	\label{lem: deviation_transportation_distance}
	Suppose $\Delta \leq r$. Then there exist constants $b_7,b_8 \geq 0$ which do not depend on the sample size $n$, such that for every $\Sset \subseteq \Csig$,
	\begin{align*}
	\int_{\Csig} \ell_{\Pbb,r^+}(x) \,d\Pbb(x) & \leq \int_{\Csig} \ell_{\Pbb,r}(x) \,d\Pbb(x) + b_7 \varDelta \\
	\int_{\Csig} \ell_{\Pbb,r^-}(x) \,d\Pbb(x) & \geq \int_{\Csig} \ell_{\Pbb,r}(x) \,d\Pbb(x) - b_7 \varDelta \\
	\widetilde{Q}_{\Pbb,r^-}(\Sset, \Sset^c) & \geq \widetilde{Q}_{\Pbb,r}(\Sset, \Sset^c) - b_8 \varDelta \\
	\widetilde{\pi}_{\Pbb,r^+}(\Sset) & \leq \widetilde{\pi}_{\Pbb,r}(\Sset) + b_8 \varDelta.
	\end{align*}
\end{lemma}

The combination of Lemmas \ref{lem: cut_volume} and \ref{lem: deviation_transportation_distance} brings us close to our goal, as demonstrated by the following manipulations:
\begin{align}
\frac{\widetilde{\cut}_{n,r}(S,S^c)}{\widetilde{\vol}_{n,r}(S)} & \geq \frac{\widetilde{Q}_{\Pbb,r^-}(\Sset,\Sset^c)}{\widetilde{\pi}_{\Pbb,r^+}(\Sset)} \frac{\int_{\Csig} \ell_{\Pbb,r^-}(x) \,d\Pbb(x)}{\int_{\Csig} \ell_{\Pbb,r^+}(x) \,d\Pbb(x)} \tag{Lemma \ref{lem: cut_volume}} \\
& \geq \frac{\widetilde{Q}_{\Pbb,r}(\Sset,\Sset^c) - b_8 \varDelta }{\widetilde{\pi}_{\Pbb,r^+}(\Sset) + b_8 \varDelta} \left(1 - \frac{2 b_7 \varDelta}{\int_{\Csig}\ell_{\Pbb,r}(x)\,d\Pbb(x)}\right) \tag{Lemma \ref{lem: deviation_transportation_distance} } \\
& \geq \frac{\widetilde{Q}_{\Pbb,r}(\Sset,\Sset^c)}{\widetilde{\pi}_{\Pbb,r}(\Sset)} - \left(\frac{2b_8}{\widetilde{\pi}_{\Pbb,r}(\Sset)} + \frac{2b_7}{\int_{\Csig}\ell_{\Pbb,r}(x)\,d\Pbb(x)}\right)\varDelta \label{eqn: proof_of_lemma_graph_to_continuous_1}
\end{align}
where the last line follows from some basic algebra. The following result relates the unknown stationary distribution $\widetilde{\pi}_{\Pbb,r}(\Sset)$ to $\widetilde{\pi}_{n,r}(S)$.
\begin{lemma}
	\label{lem: continuous_pi_lb}
	Suppose $\varDelta \leq \lambda_{\sigma}r/(2^{d+3}d\Lambda_{\sigma})$. Then there exists a constant $b_9 > 0$ which does not depend on the sample size $n$ such that for every $S \subseteq \Csig[\Xbf]$,
	\begin{equation*}
	\widetilde{\pi}_{\Pbb,r}(\Sset) \geq \widetilde{\pi}_{n,r}(S) - b_9 \varDelta
	\end{equation*}
\end{lemma}
Combining this result with~\eqref{eqn: proof_of_lemma_graph_to_continuous_1}, we obtain
\begin{align*}
\frac{\widetilde{\cut}_{n,r}(S,S^c)}{\widetilde{\vol}_{n,r}(S)} & \geq \widetilde{\Phi}_{\Pbb,r}(\Sset) - \left(\frac{2b_8}{\widetilde{\pi}_{n,r}(S) - b_9\varDelta} + \frac{2b_7}{\int_{\Csig}\ell_{\Pbb,r}(x)\,d\Pbb(x)}\right)\varDelta \\
& \geq \widetilde{\Phi}_{\Pbb,r}(\Sset) - \left(\frac{2b_8}{s(\widetilde{G}_{n,r}) - b_9\varDelta} + \frac{2b_7}{\int_{\Csig}\ell_{\Pbb,r}(x)\,d\Pbb(x)}\right)\varDelta
\end{align*}
where the latter inequality follows as we assumed $S \in \mathcal{L}(\widetilde{G}_{n,r})$. Choosing the constant $b_6$ in Lemma \ref{lem: graph_to_continuous_conductance} to be $b_6 = \max\set{2b_7/\bigl(\int_{\Csig}\ell_{\Pbb,r}(x)\,d\Pbb(x) \bigr) + 2b_8, b_9}$, we achieve our desired result. It remains to show Lemmas~\ref{lem: cut_volume}, \ref{lem: deviation_transportation_distance}, and \ref{lem: continuous_pi_lb}, which we now turn to.

\paragraph{Proof of Lemma~\ref{lem: cut_volume}: }

We begin with $\widetilde{\cut}_{n,r}(S)$.
\begin{align*}
\frac{1}{\widetilde{n}^2} \widetilde{\cut}_{n,r}(S) & = \frac{1}{ \widetilde{n}^2} \sum_{x_i, x_j \in \Csig[\Xbf]} \1(\norm{x_i - x_j} \leq r) \1(x_i \in S) \1(x_j \not\in S)\\
& = \iint_{\Csig \times \Csig} \1(\norm{x - y} \leq r) \1(x \in S) \1(y \not\in S) \,d\widetilde{\Pbb}_n(x) \,d\widetilde{\Pbb}_n(y) \\
& =  \iint_{\Csig \times \Csig} \1(\norm{T(x) - T(y)} \leq r) \1(T(x) \in S) \1(T(y) \not\in S)\,d\widetilde{\Pbb}(x) \,d\widetilde{\Pbb}(y) \tag{change of variables}  \\
& \geq \iint_{\Csig \times \Csig} \1(\norm{x - y} \leq r^-) \1(T(x) \in S) \1(T(y) \not\in S) \,d\widetilde{\Pbb}(x) \,d\widetilde{\Pbb}(y) \\
& = \int_{\Sset} \int_{\Sset^c \cap B(x,r^-)} 1 \,d\widetilde{\Pbb}(y) \,d\widetilde{\Pbb}(x)
\end{align*}
By definition we have $\frac{\,d\Pbb(x)}{\,d\widetilde{\Pbb}(x)} = \Pbb(\Csig)$. Therefore,
\begin{align}
\int_{\Sset} \int_{\Sset^c \cap B(x,r^-)} d\widetilde{\Pbb}(y) d\widetilde{\Pbb}(x) & = \frac{1}{\Pbb(\Csig)^2} \int_{\Sset} \int_{\Sset^c \cap B(x,r^-)} d\Pbb(y) d\Pbb(x) \nonumber \\
& = \frac{\int_{\Csig} \ell_{\Pbb,r^-}(x) d\Pbb(x)}{\Pbb(\Csig)^2} \int_{\Sset} \frac{\Pbb\bigl(\Sset^c \cap B(x,r^-)\bigr)}{\ell_{\Pbb,r^-}(x)} d\piwt_{\Pbb,r^-}(x) \nonumber \\
& = \frac{\int_{\Csig} \ell_{\Pbb,r^-}\,d\Pbb(x)}{\Pbb(\Csig)^2} \widetilde{Q}_{\Pbb,r^-}(\Sset, \Sset^c). \label{eqn: cut_bound}
\end{align}	
We obtain an upper bound on $\widetilde{\vol}_{n,r}(S)$ by similar manipulations, as follows:
\begin{align}
\frac{1}{\widetilde{n}^2} \widetilde{\vol}_{n,r}(S) & \leq \frac{1}{ \widetilde{n}^2} \sum_{x_i, x_j \in \Csig[\Xbf]} \1(\norm{x_i - x_j} \leq r) \1(x_i \in S) \nonumber \\
& = \iint_{\Csig \times \Csig} \1(\norm{x - y} \leq r) \1(x \in S) \,d\widetilde{\Pbb}_n(x) \,d\widetilde{\Pbb}_n(y) \nonumber \\
& = \iint_{\Csig \times \Csig} \1(\norm{T(x) - T(y)} \leq r) \1(T(x) \in S) \,d\widetilde{\Pbb}(x) \,d\widetilde{\Pbb}(y) \nonumber \\
& \leq \iint_{\Csig \times \Csig} \1(\norm{x - y} \leq r^+) \1(x \in S) \,d\widetilde{\Pbb}(x) \,d\widetilde{\Pbb}(y) \nonumber \\
& = \int_{\Sset} \int_{\Csig \cap B(x,r^+)} 1 \,d\widetilde{\Pbb}(y) \,d\widetilde{\Pbb}(x) \nonumber \\
& = \frac{1}{\Pbb(\Csig)^2}\int_{\Sset} \int_{\Csig \cap B(x,r^+)} 1 \,d\Pbb(y) \,d\Pbb(x) \nonumber \\
& = \frac{1}{\Pbb(\Csig)^2}\int_{\Sset} \ell_{\Pbb,r^+}(x) \,d\Pbb(x) \nonumber \\
& = \frac{\int_{\Csig} \ell_{\Pbb,r}(x) \,d\Pbb(x)}{\Pbb(\Csig)^2} \piwt_{\Pbb,r^+}(\Sset). \label{eqn: vol_bound}
\end{align}

\paragraph{Proof of Lemma~\ref{lem: deviation_transportation_distance}: }

Consider the set
\begin{equation*}
\mathcal{R}(x) := \set{y \in \Csig: y \in B(x,r^+), y \not\in B(x,r^-)}
\end{equation*} 
for $x \in \Csig$, and observe that
\begin{equation*}
\int_{\mathcal{R}(x)} \,d\Pbb(y) \leq \Lambda_{\sigma} \nu_d \left((r + \delta)^d - (r - \varDelta)^d\right)
\end{equation*}
and therefore
\begin{equation}
\label{eqn: deviation_transportation_distance_2}
\int_{\Csig} \int_{\mathcal{R}(x)} \,d\Pbb(y) \,d\Pbb(x) \leq \Pbb(\Csig )\Lambda_{\sigma} \nu_d \left((r + \varDelta)^d - (r - \varDelta)^d\right).
\end{equation}
A first-order Taylor expansion of $(r + \varDelta)^d$ results in the upper bound $(r + \varDelta)^d \leq r^d + d\varDelta(r + \varDelta)^{d - 1}$, and similarly $(r - \varDelta)^d \geq r^d - d\varDelta(r + \varDelta)^{d - 1}$. Plugging these bounds into~\eqref{eqn: deviation_transportation_distance_2} yields
\begin{align}
\int_{\Csig} \int_{\mathcal{R}(x)} \,d\Pbb(y) \,d\Pbb(x)  & \leq \Pbb(\Csig) \Lambda_{\sigma} \nu_d \bigl( 2d(r + \Delta)^{d - 1}\varDelta \bigr) \tag{1st-order Taylor expansion of $(r + \Delta)^d$ } \\
& \leq \Pbb(\Csig) \Lambda_{\sigma} \nu_d  2^{d}d r^{d-1}\varDelta =: b_7 \varDelta. \label{eqn: deviation_transportation_distance_1}
\end{align}
where the second line follows from the condition $\Delta \leq r$. Using~\eqref{eqn: deviation_transportation_distance_1}, we proceed to obtain each of the four bounds stated in Lemma~\ref{lem: deviation_transportation_distance}. First, as $\ell_{\Pbb,r^{+}}(x) \leq 1$ for all $x \in \Csig$, we have
\begin{align*}
\int_{\Csig} \ell_{\Pbb,r^+}(x) \,d\Pbb(x) & \leq \int_{\Csig} \ell_{\Pbb,r}(x) \,d\Pbb(x) + \int_{\Csig} \int_{\mathcal{R}(x)} \,d\Pbb(y) \,d\Pbb(x) \\
& \leq \int_{\Csig} \ell_{\Pbb,r}(x) \,d\Pbb(x) + b_7\varDelta.
\end{align*}
An equivalent bound for $\int_{\Csig} \ell_{\Pbb,r^-}(x) \,d\Pbb(x)$ is obtained by similar reasoning. We now lower bound $\widetilde{Q}_{\Pbb,r^{-}}(\Sset, \Sset^c)$,
\begin{align*}
\widetilde{Q}_{\Pbb,r^{-}}(\Sset, \Sset^c) & = \int_{\Sset} \widetilde{P}_{\Pbb,r^-}(x; \Sset) \,d\widetilde{\pi}_{\Pbb,r}(x) \\
& = \frac{\int_{\Sset} \Pbb(\Sset^c \cap B(x,r^-)) d\Pbb(x)}{\int_{\Csig} \Pbb(\Csig \cap B(x,r^-)) d\Pbb(x)} \\
& \geq \frac{\int_{\Sset} \Pbb(\Sset^c \cap B(x,r)) d\Pbb(x) - \int_{\Csig} \int_{\mathcal{R}(x)} \,d\Pbb(y) \,d\Pbb(x)}{\int_{\Csig} \Pbb(\Csig \cap B(x,r)) d\Pbb(x)} \\
& \geq \widetilde{Q}_{\Pbb,r}(\Sset, \Sset^c) - \frac{b_7 \varDelta}{\int_{\Csig}\ell_{\Pbb,r}(x) \,d\Pbb(x)} =:\widetilde{Q}_{\Pbb,r}(\Sset, \Sset^c) - b_8 \varDelta.
\end{align*}
Finally, we upper bound $\widetilde{\pi}_{\Pbb,r^{+}}(\Sset)$,
\begin{align*}
\widetilde{\pi}_{\Pbb,r^{+}}(\Sset) & = \frac{\int_{\Sset} \ell_{\Pbb,r^+}(x) \,d\Pbb(x)}{\int_{\Csig} \ell_{\Pbb,r^+}(x) \,d\Pbb(x)} \\
& \leq \frac{\int_{\Sset} \ell_{\Pbb,r}(x) \,d\Pbb(x) + \int_{\Sset} \int_{\mathcal{R}(x)} \,d\Pbb(y) \,d\Pbb(x)}{\int_{\Csig} \ell_{\Pbb,r}(x) \,d\Pbb(x)} \\
& \leq \widetilde{\pi}_{\Pbb,r}(\Sset) + b_8\varDelta.
\end{align*}

\paragraph{Proof of Lemma~\ref{lem: continuous_pi_lb}: }
The proof of Lemma~\ref{lem: continuous_pi_lb} will not be too different from the proof of Lemma~\ref{lem: deviation_transportation_distance}. From the change of variables formula, we have
\begin{align*}
\widetilde{\pi}_{n,r}(S) & = \frac{\int_{\Sset}\int_{\Csig} \1(\norm{x - y} \leq r) \,d\widetilde{\Pbb}_n(y) \,d\widetilde{\Pbb}_n(x)}{\int_{\Csig}\int_{\Csig} \1(\norm{x - y} \leq r) \,d\widetilde{\Pbb}_n(y) \,d\widetilde{\Pbb}_n(x)} \\
& \leq \frac{\int_{\Sset} \ell_{\Pbb,r^+}(x) \,d\Pbb(x) }{\int_{\Csig}\ell_{\Pbb,r^-}(x) \,d\Pbb(x)} \\
& \leq \frac{\int_{\Sset} \ell_{\Pbb,r}(x) \,d\Pbb(x) + b_7\varDelta}{\int_{\Csig} \ell_{\Pbb,r}(x) \,d\Pbb(x) - b_7\varDelta} \\
& \leq \widetilde{\pi}_{\Pbb,r}(\Sset) + \frac{2b_7\varDelta}{\int_{\Csig} \ell_{\Pbb,r}(x) \,d\Pbb(x) - b_7\varDelta} \\
& \leq \widetilde{\pi}_{\Pbb,r}(\Sset) + \frac{4b_7\varDelta}{\int_{\Csig} \ell_{\Pbb,r}(x) \,d\Pbb(x)} =: \widetilde{\pi}_{\Pbb,r}(\Sset) + b_9 \varDelta.
\end{align*}
where the last inequality follows from the assumption $\varDelta \leq \lambda_{\sigma}r/(2^{d + 3} d\Lambda_{\sigma})$, which implies $b_7\varDelta \leq \int_{\Csig}\ell_{\Pbb,r}(x) \,d\Pbb(x) / 2$.

\subsection{Proof of Lemma \ref{lem: nonuniform_continuous_conductance}. }
We recall that our goal is to lower bound the continuous conductance $\widetilde{\Phi}_{\Pbb,r}$. Results along these lines are already known (see e.g. \cite{kannan04}) when the density $f$ is uniform (or log-concave) and $\Csig$ is itself a convex set -- indeed, in this case stronger versions of it exist, though we will not require them. In \cite{abbasi-yadkori2016a}, a statement of this sort is made with respect to uniform density $f$ and $\Csig$ a Lipschitz deformation of a convex set, but for completeness we produce all proofs here.

\paragraph{Population-level conductance with uniform density.}
For $x \in \mc{C}_{\sigma}$ and measurable $\Sset \subseteq \mc{C}_{\sigma}$, let $\ell_{\nu,r}(x) = \nu(B(x,r) \cap \mc{C}_{\sigma})$ and ${J}_{\nu,r}(\mc{S}) = \int_{\mc{S}} \ell_{\nu,r}(x) \,dx$. Further define
\begin{equation*}
\widetilde{P}_{\nu,r}(x; \Sset) := \frac{\nu(\Sset \cap B(x,r))}{\nu(\mc{C}_{\sigma} \cap B(x,r))}, \quad \widetilde{\pi}_{\nu,r}(\Sset) = \frac{J_{\nu,r}(\mc{S})}{J_{\nu,r}(\mc{C}_{\sigma})}, \quad  \widetilde{Q}_{\nu,r}(\Sset_1,\Sset_2) := \int_{\Sset_1} \widetilde{P}_{\nu,r}(x;\Sset_2) \,d\widetilde{\pi}_{\nu,r}(x).
\end{equation*}
The uniform continuous normalized cut and conductance are then
\begin{equation*}
\widetilde{\Phi}_{\nu,r}(\Sset) := \frac{\widetilde{Q}_{\nu,r}(\Sset, \Sset^c)}{\min\set{\widetilde{\pi}_{\nu,r}(\Sset),\widetilde{\pi}_{\nu,r}(\Sset^c)}}, \quad \widetilde{\Phi}_{\nu,r} := \min_{\Sset \subseteq \mc{C}_{\sigma}} \widetilde{\Phi}_{\nu,r}(\Sset),
\end{equation*}
where the minimum is taken over all measurable sets $\Sset \subseteq \mc{C}_{\sigma}$.

\begin{lemma}
	\label{lem: uniform_continuous_conductance}
	Let $\mc{C}_{\sigma}$ satisfy Assumption \textcolor{red}{(A3)} for some convex set $\mathcal{K}$ with diameter $\rho$, and measure-preserving mapping $g: \mathcal{K} \to \mc{C}_{\sigma}$ with Lipschitz constant $L$. Let $0 < \ell < 1/2$ be a number that satisfies $\ell_{\nu,r}(x) \geq \nu_d r^d \ell$ for all $x \in \mc{C}_{\sigma}$.  Then the uniform conductance $\widetilde{\Phi}_{\nu,r}$ is lower bounded:
	\begin{equation*}
	\widetilde{\Phi}_{\nu,r} \geq \Bigl(1 - \frac{r}{4\rho L}\Bigr) \cdot\frac{\ell^2\sqrt{2\pi}}{9} \cdot \frac{r}{\rho L \sqrt{d + 2}}
	\end{equation*}
\end{lemma}
Lemma~\ref{lem: overlap_balls} gives the lower bound $\ell \geq 1/2 \cdot (1 - r/\sigma \cdot \sqrt{(d + 2)/(2\pi)})$ leading to the following corollary.
\begin{corollary}
	\label{cor: uniform_continuous_conductance}
	Let $\mc{C}_{\sigma}$ satisfy Assumption \textcolor{red}{(A3)} for some convex set $\mathcal{K}$ with diameter $\rho$, and measure-preserving mapping $g: \mathcal{K} \to \mc{C}_{\sigma}$ with Lipschitz constant $L$. Then the uniform conductance $\widetilde{\Phi}_{\nu,r}$ is lower bounded:
	\begin{equation*}
	\widetilde{\Phi}_{\nu,r} \geq \Bigl(1 - \frac{r}{4\rho L}\Bigr) \Bigl(1 - \frac{r}{\sigma}\sqrt{\frac{d + 2}{2\pi}}\Bigr)^2 \cdot \frac{\sqrt{2\pi}}{36} \cdot \frac{r}{\rho L \sqrt{d + 2}}
	\end{equation*}
\end{corollary}

Most of the technical work needed to show Lemma~\ref{lem: nonuniform_continuous_conductance}---and in turn, Corollary~\ref{cor: uniform_continuous_conductance}---involves proving Lemma~\ref{lem: uniform_continuous_conductance}. We first show that  Lemma~\ref{lem: nonuniform_continuous_conductance} is a simple consequence of Corollary~\ref{cor: uniform_continuous_conductance}---and \ref{asmp: bounded_density}---before turning to prove Lemma~\ref{lem: uniform_continuous_conductance}. 

\begin{proof}[of Lemma~\ref{lem: nonuniform_continuous_conductance}]
To relate the uniform- and $\Pbb$-continuous conductances, observe that by \ref{asmp: bounded_density} we obtain
\begin{align*}
\frac{\widetilde{Q}_{\Pbb,r}(\Sset,\Sset^c)}{\piwt_{\Pbb,r}(\Sset)} & = \frac{\int_{\Sset} \Pbb(\Sset^c \cap B(x,r)) \,d\Pbb(x)}{\int_{\Sset} \Pbb(\Csig \cap B(x,r)) \,d\Pbb(x)} \\
& \geq \frac{\lambda_{\sigma}^2 \int_{\Sset} \nu(\Sset^c \cap B(x,r)) \dx}{\Lambda_{\sigma}^2 \int_{\Sset} \nu(\Csig \cap B(x,r)) \dx} \\
& \geq \frac{\lambda_{\sigma}^2}{\Lambda_{\sigma}^2} \frac{\widetilde{Q}_{\nu,r}(\Sset,\Sset^c)}{\pi_{\nu,r}(\Sset)}.
\end{align*}
Exchanging $\mc{S}$ and $\mc{S}^c$ in the above, we also have $\widetilde{Q}_{\Pbb,r}(\Sset,\Sset^c)/\piwt_{\Pbb,r}(\Sset^c) \geq (\lambda_{\sigma}^2/\Lambda_{\sigma}^2) \widetilde{Q}_{\nu,r}(\Sset,\Sset^c)/\piwt_{\nu,r}(\Sset^c)$, and the claim follows by~\ref{cor: uniform_continuous_conductance}. 
\end{proof}

Lemma~\ref{lem: nonuniform_continuous_conductance} therefore follows from Lemma~\ref{lem: uniform_continuous_conductance} (by way of~\ref{cor: uniform_continuous_conductance}), and it remains to prove this Lemma.

\subsection{Proof of Lemma~\ref{lem: uniform_continuous_conductance}}
\label{subsec:pf_uniform_continuous_conductance}
Lower bounds on the conductance of geometric random walks follow a typical pattern, in which one supplies an isoperimetric inequality, uses this to lower bound the ergodic flow $\wt{Q}$, and thereby derives a lower bound on the conductance. The proof of Lemma~\ref{lem: uniform_continuous_conductance} proceeds along these lines, and we start by sketching the chain of inequalities we will use. For a given partition $\{\mc{S},\mc{R}\}$ of $\mc{C}_{\sigma}$, write
\begin{equation*}
\mc{S}^{\delta} := \biggl\{u \in \mc{S}: \wt{P}(u,\mc{R}) \leq \delta \biggr\}~~\textrm{and}~~\mc{R}^{\delta} := \biggl\{u \in \mc{R}: \wt{P}(u,\mc{S}) \leq \delta \biggr\} 
\end{equation*}
where $\delta$ is a small number we will determine later on. Let $\mc{S}_{\delta} := \mc{C}_{\sigma} \setminus (\mc{S}^{\delta} \cup \mc{R}^{\delta})$. 
\begin{enumerate}
	\item[1.] The ergodic flow $\wt{Q}$ between $\mc{S}$ and $\mc{R}$ satisfies the following lower bound:
	\begin{equation*}
	\wt{Q}_{\nu,r}(\mc{S},\mc{R}) \geq \frac{1}{2} \wt{\pi}_{\nu,r}(\mc{S}_{\delta}) \cdot \delta.
	\end{equation*}
	\item[2.] For any $\mc{S} \subseteq \mc{C}_{\sigma}$, it holds that
	\begin{equation*}
	\frac{\nu_dr^d\nu(\mc{S})}{J(\mc{C}_{\sigma})} \cdot \ell \leq \wt{\pi}_{\nu,r}(\mc{S}) \leq \frac{\nu_d r^d \nu(\mc{S})}{J(\mc{C}_{\sigma})}
	\end{equation*}
	\item[3.] From \textcolor{red}{Abbasi-Yadkori} (see also the seminal works of \textcolor{red}{(Lovasz and Simonovits)} and \textcolor{red}{(Dyer)}), we have the following isoperimetric inequality.
	\begin{lemma}[Isoperimetry of Lipschitz embeddings of convex sets.]
		\label{lem: nonconvex_isoperimetry}
		Let $\mc{C}_{\sigma}$ satisfy Assumption \textcolor{red}{(A3)} for some convex set $\mathcal{K}$ with diameter $\rho$, and measure-preserving mapping $g: \mathcal{K} \to \mc{C}_{\sigma}$ with Lipschitz constant $L$. Then, for any partition $(\Omega_1,\Omega_2,\Omega_3)$ of $\mc{C}_{\sigma}$, 
		\begin{equation*}
		\nu(\Omega_3) \geq 2\frac{\dist(\Omega_1, \Omega_2)}{\rho L} \min(\nu(\Omega_1), \nu(\Omega_2)).
		\end{equation*}
	\end{lemma}
	\item[4.] 
	The following Lemma relates the Euclidean distance $\|x - y\|$ to the the total variation distance
	\begin{equation*}
	\|\wt{P}(x,\cdot) - \wt{P}(y,\cdot)\|_{\mathrm{TV}} = \sup_{\mc{S} \subseteq \mc{C}_{\sigma}} |\wt{P}(x,\mc{S}) - \wt{P}(y,\mc{S}) |
	\end{equation*}
	\begin{lemma}
		\label{lem: one_step_distributions}
		For all $x,y \in \mc{C}_{\sigma}$ and any $\mc{S} \in \mc{C}_{\sigma}$, it holds that
		\begin{equation}
		\label{eqn:one_step_distributions_1}
		\|x - y\| \geq \Bigl(\bigl|\wt{P}(x,\mc{S}) - \wt{P}(y,\mc{S})\bigr|\Bigr)\sqrt{\frac{2\pi}{d + 2}}r\ell
		\end{equation}
		Therefore, for any $\mc{S}_1,\mc{S}_2 \subseteq \mc{C}_{\sigma}$, it holds that
		\begin{equation}
		\label{eqn:one_step_distributions_2}
		\dist(\mc{S}_1,\mc{S}_2) \geq \Bigl(\inf_{\substack{x \in \mc{S}_1 \\ y \in \mc{S}_2}}\|\wt{P}(x,\cdot) - \wt{P}(y,\cdot)\|_{\mathrm{TV}}\Bigr)\sqrt{\frac{2\pi}{d + 2}}r\ell.
		\end{equation}
	\end{lemma}
	Note that for any $x \in \mc{S}^{\delta}, y \in \mc{R}^{\delta}$, we have that
	\begin{equation*}
	\wt{P}(x,\mc{S}) - \wt{P}(y,\mc{S}) = 1 - \bigl(\wt{P}(x,\mc{R}) + \wt{P}(y,\mc{S})\bigr) \geq 1 - 2\delta,
	\end{equation*}
	Thus, applying Lemma~\ref{lem: one_step_distributions} to $\mc{S}^{\delta}$ and $\mc{R}^{\delta}$ gives
	\begin{equation*}
	\dist(\mc{S}^{\delta},\mc{R}^{\delta}) \geq \bigl(1 - 2\delta\bigr) \cdot \sqrt{\frac{2\pi}{d + 2}}r\ell
	\end{equation*}
\end{enumerate}
The claim of Lemma~\ref{lem: uniform_continuous_conductance}---i.e. a lower bound on the continuous normalized cut $\wt{\Phi}_{\nu,r}(\mc{S})$ that holds for any measurable $\mc{S} \subseteq \mc{C}_{\sigma}$---directly follows from these inequalities, as we now show. In particular, the ergodic flow $\wt{Q}(\mc{S},\mc{R})$ can be lower bounded
\begin{equation*}
\wt{Q}_{\nu,r}(\mc{S},\mc{R}) \geq \frac{1}{2}\wt{\pi}_{\nu,r}(\mc{S}_{\delta}) \cdot \delta. \tag{Step 1}
\end{equation*}
Now, suppose $\wt{\pi}_{\nu,r}(\mc{S}^{\delta}) \leq (1 - a) \wt{\pi}_{\nu,r}(\mc{S})$ or $\wt{\pi}_{\nu,r}(\mc{R}^{\delta}) \leq (1 - a) \wt{\pi}_{\nu,r}(\mc{R})$, for some $0 < a < 1$. Then $\wt{\pi}_{\nu,r}(\mc{S}_{\delta}) \geq a \min\{\wt{\pi}_{\nu,r}(\mc{S}),\wt{\pi}_{\nu,r}(\mc{R})\}$, and consequently,
\begin{equation*}
\wt{\Phi}_{\nu,r}(\mc{S}) = \frac{\wt{Q}_{\nu,r}(\mc{S},\mc{S}^c)}{\min\{\wt{\pi}_{\nu,r}(\mc{S}),\wt{\pi}_{\nu,r}(\mc{R})\}} \geq \frac{1}{2}a \delta.
\end{equation*}
Otherwise, we deduce that
\begin{align*}
\wt{Q}_{\nu,r}(\mc{S},\mc{R}) & \geq \frac{\nu(\mc{S}_{\delta})\nu_dr^d}{2J(\mc{C}_{\sigma})} \cdot \ell \delta \tag{Step 2} \\
& \geq \frac{\dist(\mc{S}^{\delta},\mc{R}^{\delta})}{\rho L} \frac{\nu_dr^d\min\{\nu(\mc{S}^{\delta}),\nu(\mc{R}^{\delta})\}}{J(\mc{C}_{\sigma})}  \ell^2 \delta \tag{Step 3}\\
& \geq \frac{r (1 - 2\delta)\sqrt{2\pi}}{\rho L\sqrt{d + 2}} \frac{\nu_dr^d \min\{\nu(\mc{S}^{\delta}),\nu(\mc{R}^{\delta})\}}{J(\mc{C}_{\sigma})}  \ell^2 \delta \tag{Step 4} \\
& \geq  \frac{r (1 - 2\delta)\sqrt{2\pi}}{\rho L\sqrt{d + 2}} \min\bigl\{\wt{\pi}_{\nu,r}(\mc{S}^{\delta}),\wt{\pi}_{\nu,r}(\mc{R}^{\delta})\bigr\} \ell^2 \delta \tag{Step 2} \\
& \geq \frac{r (1 - 2\delta)\sqrt{2\pi}}{\rho L\sqrt{d + 2}} \min\bigl\{\wt{\pi}_{\nu,r}(\mc{S}),\wt{\pi}_{\nu,r}(\mc{R})\bigr\}   \ell^2 \delta (1 - a). \tag{by hypothesis}
\end{align*}
Setting $\delta = 1/3$, we have
\begin{equation*}
\wt{\Phi}_{\nu,r}(\mc{S}) \geq \min\biggl\{\frac{\sqrt{2\pi} r}{9\sqrt{d + 2} \rho L} \ell^2(1 - a), \frac{a\ell}{6}\biggr\}
\end{equation*}
and putting $a = r\ell/(2\rho L)$ yields the claim.

Now, we turn to establishing the inequalities used in the above chain.

\paragraph{Step 1: Lower bound the ergodic flow.}
By symmetry, $\wt{Q}_{\nu,r}(\mc{S}_1,\mc{S}_2) = \wt{Q}_{\nu,r}(\mc{S}_1,\mc{S}_2)$, whence a direct computation yields
\begin{equation*}
2\wt{Q}_{\nu,r}(\mc{S}_1,\mc{S}_2) = \wt{Q}_{\nu,r}(\mc{S}_1,\mc{S}_2) + \wt{Q}_{\nu,r}(\mc{S}_2,\mc{S}_1) \geq \delta \wt{\pi}_{\nu,r}(\mc{S}_{\delta}).
\end{equation*}

\paragraph{Step 2: Estimate the stationary distribution.}

Both the upper and lower  bounds on $\wt{\pi}_{\nu,r}(\mc{S})$ are a direct consequence of the following inequalities:
\begin{equation*}
\ell \nu_d r^d \nu(\mc{S})  \leq {J}_{\nu,r}(\mc{S}) \leq \nu_d r^d \nu(\mc{S}).
\end{equation*}
\paragraph{Step 3: Isoperimetric inequality.}
NB: the proof of Lemma \ref{lem: nonconvex_isoperimetry} from first principles is non-trivial, even when the domain $\mc{C}_{\sigma}$ is itself convex, and is a primary technical contribution of the seminal work \textcolor{red}{(Lovasz 1990)}, extended by \textcolor{red}{(Dyer 1991)} among others. However, once the result is shown in the convex setting, it is not hard to show that it applies to Lipschitz transformations of convex sets as well.
\begin{proof}[Proof of Lemma \ref{lem: nonconvex_isoperimetry}]
	For $\Omega_i, i = 1,2,3$, denote the preimage
	\begin{equation*}
	R_i = \set{x \in \mathcal{K}: g(x) \in \Omega_i}
	\end{equation*}
	For any $x \in R_1, y \in R_2$, 
	\begin{equation*}
	\norm{x - y} \geq \frac{1}{L}\norm{g(x) - g(y)} \geq \frac{1}{L} \dist(\Omega_1, \Omega_2). 
	\end{equation*}
	Since $x \in \Omega_1$ and $y \in \Omega_2$ were arbitrary, we have
	\begin{equation*}
	\dist(R_1, R_2) \geq \frac{1}{L} \dist(\Omega_1, \Omega_2).
	\end{equation*}
	By Theorem 2.2 of \textcolor{red}{(Lovasz 1990)},
	\begin{align*}
	\nu(R_3) & \geq 2\frac{\dist(R_1, R_2)}{\rho} \min \{\nu(R_1), \nu(R_2)\} \\
	& \geq \frac{2}{\rho L} \dist(\Omega_1, \Omega_2) \min\{\nu(R_1), \vol(R_2)\}
	\end{align*}
	and by the measure-preserving property of $g$, this implies
	\begin{equation*}
	\nu(\Omega_3) \geq\frac{2}{\rho L} \dist(\Omega_1, \Omega_2) \min\{\nu(\Omega_1), \nu(\Omega_2)\}.
	\end{equation*}
\end{proof}

\paragraph{Step 4: Total variation distance.}

\begin{proof}[Proof of Lemma~\ref{lem: one_step_distributions}]
	Note that~\eqref{eqn:one_step_distributions_2} follows from~\eqref{eqn:one_step_distributions_1} by first taking the supremum over all $\mc{S} \subseteq \mc{C}_{\sigma}$, and then the infimum over all $x \in \mc{S}_1$ and $y \in \mc{S}_2$. 
	
	It remains to show~\eqref{eqn:one_step_distributions_1}. Let $\mc{R} = \mc{C}_{\sigma} \setminus \mc{S}$ and $\mc{I} = B(x,y) \cap B(y,r)$, and suppose without loss of generality that $\ell_{\nu,r}(x) > \ell_{\nu,r}(y)$. We can relate $\wt{P}(x,\mc{S}) - \wt{P}(y,\mc{S})$ to the volume of the intersection $\mc{I}$ as follows:
	\begin{align*}
	\wt{P}(x,\mc{S}) - \wt{P}(y,\mc{S}) & = 1 - \bigl(\wt{P}(x,\mc{R}) + \wt{P}(y,\mc{S})\bigr) \\
	& = 1 - \biggl(\frac{\nu\bigl(B(x,r) \cap \mc{R}\bigr)}{\ell_{\nu,r}(x)} + \frac{\nu\bigl(B(y,r) \cap \mc{S}\bigr)}{\ell_{\nu,r}(y)}\biggr) \\
	& \leq 1 - \frac{1}{\ell_{\nu,r}(x)}\Bigl(\nu\bigl(B(x,r) \cap \mc{R}\bigr) + \nu\bigl(B(y,r) \cap \mc{S}\bigr)\Bigr) \\
	& \leq 1 - \frac{1}{\ell_{\nu,r}(x)}\nu\bigl(\mc{I} \cap \mc{C}_{\sigma}\bigr) \\
	& \leq 1 - \frac{1}{\ell_{\nu,r}(x)}\Bigl(\nu(\mc{I}) - \nu\bigl(B(x,r)\bigr) + \ell_{\nu,r}(x)\Bigr) \\
	& = \frac{1}{\ell_{\nu,r}(x)}\Bigl(\nu_dr^d - \nu(\mc{I})\Bigr).
	\end{align*}
	Now, by hypothesis $\ell_{\nu,r}(x) \geq \ell \nu_dr^d$. On the other hand, Lemma~\ref{lem: volume_of_spherical_cap} gives a lower bound on the volume of $\mc{I}$, and from this we obtain
	\begin{equation*}
	\wt{P}(x,\mc{S}) - \wt{P}(y,\mc{S}) \leq \frac{\|x - y\|}{\ell r} \sqrt{\frac{d + 2}{2\pi}};
	\end{equation*}
	solving for $\|x - y\|$ yields~\eqref{eqn:one_step_distributions_1}.
\end{proof}

We have completed our proof of (in turn) Lemmas~\ref{lem: uniform_continuous_conductance}, \ref{lem: nonuniform_continuous_conductance}, and \ref{lem: graph_conductance_profile_lb}. As a result, Proposition \ref{prop: local_spread_conductance} is proved, and the proof of Theorem \ref{thm: mixing_time_upper_bound} is complete.

\section{Volume Estimates}
In this section, we establish some inequalities regarding the volume of (intersections of) balls, which we used in Appendices A and B. 
\begin{lemma}
	\label{lem: overlap_balls}
	For any $x,y \in \Rd$, it holds that
	\begin{equation}
	\label{eqn:overlap_balls_1}
	\nu\bigl(B(x,r) \cap B(y,r)\bigr) \geq \nu_d r^d\biggl(1 - \frac{\|x - y\|}{r} \sqrt{\frac{d + 2}{2\pi}}\biggr).
	\end{equation}
	If additionally $\|x - y\| \leq \sigma$, then
	\begin{equation}
	\label{eqn:overlap_balls_2}
	\nu\bigl(B(x,r) \cap B(y,\sigma)\bigr) \geq \frac{1}{2} \nu_d r^d\biggl(1 - \frac{r}{\sigma}\sqrt{\frac{d + 2}{2\pi}}\biggr).
	\end{equation}
	An immediate implication of~\eqref{eqn:overlap_balls_2} is that for any $x \in \mc{C}_{\sigma}$:
	\begin{equation}
	\label{eqn:uniform_local_conductance}
	\ell_{\nu,r}(x) \geq \frac{1}{2} \nu_d r^d\biggl(1 - \frac{r}{\sigma}\sqrt{\frac{d + 2}{2\pi}}\biggr).
	\end{equation}
\end{lemma}
\begin{proof}
	We first prove~\eqref{eqn:overlap_balls_1}. It is not hard to see that $\mathcal{I} := B(x,r) \cap B(y,r)$ consists of two symmetric spherical caps, each with height
	\begin{equation*}
	h = r - \frac{\|x - y\|}{2}
	\end{equation*} 
	As a result, by Lemma \ref{lem: volume_of_spherical_cap} we have
	\begin{equation*}
	\nu\bigl(\mathcal{I}\bigr) = \nu_d r^d I_{1 - \alpha}(\frac{d + 1}{2}; \frac{1}{2})
	\end{equation*}
	where
	\begin{equation*}
	\alpha = 1 - \frac{2rh - h^2}{r^2} = \frac{\|x - y\|^2}{4r^2}.
	\end{equation*}
	Expanding the incomplete beta function in integral form, we obtain
	\begin{align*}
	\nu\bigl(\mathcal{I}\bigr) & = \nu_d r^d \frac{\Gamma\bigl(\frac{d}{2}+ 1\bigr)}{\Gamma\bigl(\frac{d + 1}{2}\bigr) \Gamma\bigl(\frac{1}{2}\bigr)} \int_{0}^{1 - \alpha}u^{(d-1)/2}(1 - u)^{-1/2}du \\
	& \overset{\text{(i)}}{\geq} \nu_d r^d \left(1 - \frac{\Gamma\bigl(\frac{d}{2}+ 1\bigr)}{\Gamma\bigl(\frac{d + 1}{2}\bigr) \Gamma\bigl(\frac{1}{2}\bigr)} \frac{ \|x - y\|}{r} \right) \\
	& \overset{\text{(ii)}}{\geq} \nu_d r^d \left(1 - \frac{\|x - y\|}{r} \sqrt{\frac{d + 2}{2\pi}} \right),
	\end{align*}
	where $\text{(i)}$ follows from Lemma \ref{lem: beta_integral}, and $\text{(ii)}$ from Lemma \ref{lem: beta_function}.
	
	We now establish~\eqref{eqn:overlap_balls_2}. Assume that $\|x - y\| = \sigma$, as otherwise if $\|x - y\| < \sigma$ the volume of the overlap will only grow. Then $B(x,r) \cap B(y,\sigma)$ contains a spherical cap of radius $r$ and height $r - \frac{r^2}{2\sigma}$, and similar derivations to those used to show~\eqref{eqn:overlap_balls_1} imply that
	\begin{equation*}
	\nu(\mathrm{cap}_r(r - r^2/(2\sigma))) \geq \frac{1}{2}\nu_dr^d\biggl(1 - \frac{r}{\sigma}\sqrt{\frac{d + 2}{2\pi}}\biggr).
	\end{equation*}
\end{proof}

\subsection{Spherical caps and associated estimates}
\label{subsec:caps}
In this section, we state a result for the volume of a spherical cap and derive some 
useful upper bounds. 
\begin{lemma}
	\label{lem: volume_of_spherical_cap}
	Let $\mathrm{cap}_r(h)$ denote a spherical cap of radius $r$ and height $h$. Then, 
	\begin{equation*}
	\nu\bigl( \mathrm{cap}_r(h)  \bigr) = \frac{1}{2} \nu_d r^d I_{1 - \alpha}\left(\frac{d + 1}{2}; \frac{1}{2}\right)
	\end{equation*}
	where
	\begin{equation*}
	\alpha := 1 - \frac{2 r h - h^2}{r^2}
	\end{equation*}
	and
	\begin{equation*}
	I_{1 - \alpha}(z,w) = \frac{\Gamma(z + w)}{\Gamma(z) \Gamma(w)} \int_{0}^{1 - \alpha} u^{z - 1} (1 - u)^{w - 1} du.
	\end{equation*}
	is the cumulative distribution function of a $\mathrm{Beta}(z,w)$ distribution, evaluated at $1 - \alpha$. 
\end{lemma}
The following result provides a lower bound on the Beta integral, and the result in Lemma~\ref{lem: beta_function} provides
an upper bound on the ratio of Gamma functions. 
\begin{lemma}
	\label{lem: beta_integral}
	For any $0 \leq \alpha \leq 1$,
	\begin{equation*}
	\int_{0}^{1 - \alpha}u^{(d-1)/2}(1 - u)^{-1/2}du \geq \frac{\Gamma\bigl(\frac{d + 1}{2}\bigr)\Gamma\bigl(\frac{1}{2}\bigr)}{ \Gamma\bigl(\frac{d}{2}+ 1\bigr)} - 2\sqrt{\alpha}
	\end{equation*}
\end{lemma}
\begin{proof}
	We can write 
	\begin{equation*}
	\int_{0}^{1 - \alpha}u^{(d-1)/2}(1 - u)^{-1/2}du = \int_{0}^{1}u^{(d-1)/2}(1 - u)^{-1/2}du - \int_{1 - \alpha}^{1}u^{(d-1)/2}(1 - u)^{-1/2}du
	\end{equation*}
	The first integral is simply the beta function, with
	\begin{equation*}
	B(\frac{d+1}{2},\frac{1}{2}) := \frac{\Gamma\bigl(\frac{d + 1}{2}\bigr)\Gamma\bigl(\frac{1}{2}\bigr)}{ \Gamma\bigl(\frac{d}{2}+ 1\bigr)}.
	\end{equation*}
	Noting that for all $u \in [0,1]$ and $d \geq 1$, $u^{(d - 1)/2} \leq 1$, the second integral can be upper bounded as follows:
	\begin{equation*}
	\int_{1 - \alpha}^{1}u^{(d-1)/2}(1 - u)^{-1/2}du \leq \int_{1 - \alpha}^{1}(1 - u)^{-1/2}du = \int_{0}^{\alpha} u^{-1/2}du = 2\sqrt{\alpha}.
	\end{equation*}
\end{proof}

\begin{lemma}
	\label{lem: beta_function}
	\begin{equation*}
	\frac{\Gamma\bigl(\frac{d}{2}+ 1\bigr)}{\Gamma\bigl(\frac{d + 1}{2}\bigr) \Gamma\bigl(\frac{1}{2}\bigr)} \leq \sqrt{\frac{d + 2}{2\pi}}
	\end{equation*}
\end{lemma}
\noindent The proof of Lemma \ref{lem: beta_function} is straightforward and follows from the fact that $\Gamma(1/2) = \sqrt{\pi}$ and the upper bound $\Gamma(x + 1)/ \Gamma(x+s) \leq (x + 1)^{1-s}$ for $s \in [0,1]$.


\section{Proof of Theorem~\ref{thm: volume_ssd}}
\label{sec:proof_of_volume_ssd}

In this section, we prove Theorem \ref{thm: volume_ssd}. To do so, we apply the following Lemma, which upper bounds the volume of the symmetric set difference in terms of normalized cut and mixing time on an arbitrary graph $G$. 
It is essentially identical to Lemma 3.4 of~\citep{zhu2013}, except with tighter constants.
\begin{lemma}
	\label{lem:zhu}
	Let $G = (V,E)$ be a undirected, unweighted, connected graph and let $p_v^{(\varepsilon)}$ be an $\varepsilon$-approximation to the PPR vector $p_v := p(v,\alpha;G)$. For $\beta \in (0,1)$,  the sweep cut $S_{\beta,v}$ is
	\begin{equation*}
	S_{\beta,v} = \set{u \in V: \frac{p_v^{(\varepsilon)}(u)}{\deg(u;G)} \geq \beta}.
	\end{equation*} 
	For some $A \subseteq V$, suppose that 
	\begin{equation*}
	\alpha \leq \min\Bigl\{\frac{1}{2000}, \frac{1}{2\tau_{\infty}(G[A])}\Bigr\},~~ \beta \leq \frac{1}{5\vol(A;G)},~~ \varepsilon \leq \frac{1}{25\vol(A;G)}
	\end{equation*}
	Then there exists a set $A^g \subset A$ with $\vol(A^g;G) \geq \frac{1}{2}\vol(A^g;G)$ such that for any $v \in A^g$, the sweep cut $S_{\beta,v}$ satisfies
	\begin{equation*}
	\vol(A \vartriangle S_{\beta,v};G) \leq 6\frac{\Phi(A;G)}{\alpha \beta}.
	\end{equation*}
\end{lemma}
\begin{proof}
	We adopt the notation $\wt{p}_v$ for the (exact) PPR vector computed over the subgraph $G[A]$, and 
	\begin{equation*}
	\wt{\pi}(u) = \frac{\deg(u;G[A])}{\vol(A;G[A])}
	\end{equation*}
	for the stationary distribution of a random walk over $G[A]$. The proof of this Lemma proceeds along very similar lines to that of~\citep[Lemma 3.4]{zhu2013}. In fact, we directly use the following three inequalities, derived in that work:
	\begin{itemize}
		\item \citep[Lemma 3.2]{zhu2013} For any seed node $v \in A$, the PPR vector is lower bounded,
		\begin{equation}
		\label{pf:zhu1}
		\wt{p}_v(u) \geq \frac{3}{4}\bigl(1 - \alpha \cdot \tau_{\infty}(G[A])\bigr) \cdot \wt{\pi}(u),
		\end{equation}
		for every $u \in A$.
		\item \citep[Corollary 3.3]{zhu2013} For any seed node $v \in A$, there exists a non-negative distribution vector $\ell$ supported on $A$ for which
		\begin{equation}
		\label{pf:zhu2}
		p_v(u) \geq \wt{p}_v(u) - \wt{p}_{\ell}(u) - \varepsilon \cdot \deg(u;G),
		\end{equation}
		for all $u \in A$. Furthermore $\|\ell\|_1 \leq 2\frac{\Phi(A;G)}{\alpha}$.
		\item \citep[Lemma 3.1][Theorem 5.1]{zhu2013,andersen2006} There exists a set $A^g \subset A$ with $\vol(A^g;G) \geq \frac{1}{2}\vol(A;G)$ such that for every seed node $v \in A^g$, it holds that
		\begin{equation}
		\label{pf:zhu3}
		\sum_{u \neq A} p_v(u) \leq 2\frac{\Phi(A;G)}{\alpha}.
		\end{equation}
	\end{itemize}
	From~\eqref{pf:zhu1}-\eqref{pf:zhu3}, we can upper bound the volumes of $S_{\beta,v} \setminus A$ and $A \setminus S_{\beta,v}$. First of all, observe that any $u \in S_{\beta,v} \setminus A$, $p_v^{(\varepsilon)}(u) > \beta \cdot \deg(u;G)$. Summing up over all such vertices, from~\eqref{pf:zhu3} we may conclude that
	\begin{equation}
	\label{pf:zhu3.5}
	\vol(S_{\beta,v} \setminus A; G) \leq \frac{1}{\beta} \sum_{u \not\in A}p_v^{(\varepsilon)}(u) \leq 2\frac{\Phi(A;G)}{\beta \cdot \alpha}.
	\end{equation} 
	Next, from~\eqref{pf:zhu1} and~\eqref{pf:zhu2} we obtain a lower bound on $p_v^{(\varepsilon)}(u)$ for any $u \in A$:
	\begin{equation*}
	p_v^{(\varepsilon)}(u) \geq \frac{3}{4}\bigl(1 - \alpha \cdot \tau_{\infty}(G[A])\bigr) \cdot \wt{\pi}(u) - \wt{p}_{\ell}(u) - \varepsilon \cdot \deg(u;G).
	\end{equation*}
	If additionally $u \not\in S_{\beta,v}$ then $p_v(u) < \beta \deg(u;G)$, and thus for all $u \in A \setminus S_{\beta,v}$ it must be that
	\begin{equation}
	\label{pf:zhu4}
	\frac{3}{4}\bigl(1 - \alpha \cdot \tau_{\infty}(G[A])\bigr)\wt{\pi}(u) - (\varepsilon + \beta)\deg(u;G) \leq \wt{p}_{\ell}(u).
	\end{equation}
	Let us restrict our attention for the moment to those vertices $u \in A$ with sufficiently large degree in $G[A]$. Defining
	\begin{equation*}
	A^{\mathrm{int}} := \Bigl\{u \in A: \deg(u;G[A]) > \bigl(1 - \alpha \cdot \beta \cdot \vol(A;G)\bigr) \deg(u;G) \Bigr\},
	\end{equation*}
	we note that for any $u \in A^{\mathrm{int}}$,
	\begin{equation*}
	\wt{\pi}(u) > \frac{(1 - \alpha \beta \vol(A;G))\deg(u;G)}{\vol(A;G)},
	\end{equation*}
	plugging this in to~\eqref{pf:zhu4}, we have that
	\begin{equation*}
	\biggl(\frac{3(1 - \alpha \beta \vol(A;G))\cdot\bigl(1 - \alpha \tau_{\infty}(G[A])\bigr)}{4\vol(A;G)} - (\beta + \varepsilon)\biggr) \cdot \deg(u;G) \leq \wt{p}_{\ell}(u)
	\end{equation*}
	for any $u \in A^{\mathrm{int}} \setminus S_{\beta,v}$. Summing over all such u, we obtain
	\begin{equation*}
	\biggl(\frac{3(1 - \alpha \beta \vol(A;G))\cdot\bigl(1 - \alpha \tau_{\infty}(G[A])\bigr)}{4\vol(A;G)} - (\beta + \varepsilon)\biggr) \cdot \vol(A^{\mathrm{int}} \setminus S_{\beta,v}; G) \leq \sum_{u \in A^{\mathrm{int}} \setminus S_{\beta,v}} \wt{p}_{\ell}(u) \leq 2\frac{\Phi(A;G)}{\alpha}
	\end{equation*}
	where the second upper bound follows from the upper bound on $\|\ell\|_1$, since $\sum_{u \in A} \wt{p}_{\ell}(u) = \|\ell\|_1$. 
	Then, from the assumed upper bounds on $\alpha, \beta$ and $\varepsilon$ we obtain the inequality
	\begin{equation*}
	\biggl(\frac{3(1 - \alpha \beta \vol(A;G))\cdot\bigl(1 - \alpha \tau_{\infty}(G[A])\bigr)}{4\vol(A;G)} - (\beta + \varepsilon)\biggr) \geq \frac{2}{3}\beta,
	\end{equation*}
	and we conclude that
	\begin{equation}
	\label{pf:zhu5}
	\vol(A^{\mathrm{int}} \setminus S_{\beta,v}; G) \leq \frac{3\Phi(A;G)}{\alpha\beta}.
	\end{equation}
	Otherwise $u \in A \setminus A^{\mathrm{int}}$, and so $\mathrm{cut}(u, A^C;G[A]) \geq \alpha \cdot \beta \cdot \vol(A;G) \cdot \deg(u;G)$. Summing over all such vertices, we obtain
	\begin{equation}
	\label{pf:zhu6}
	\vol(A \setminus A^{\mathrm{int}};G) \leq \frac{\Phi(A;G)}{\alpha \cdot \beta}.
	\end{equation} 
	The claim follows upon summing the upper bounds in~\eqref{pf:zhu3.5}, \eqref{pf:zhu5} and~\eqref{pf:zhu6}.
	\end{proof}

\paragraph{Proof of Theorem~\ref{thm: volume_ssd}.}
To prove Theorem~\ref{thm: volume_ssd}, we simply apply Lemma~\ref{lem:zhu} to the specific graph $G = G_{n,r}$ and cluster $A = \mc{C}_{\sigma}[\Xbf]$. We must first establish that the conditions of Lemma~\ref{lem:zhu} are met in this setting. Applying Theorem~\ref{thm: mixing_time_upper_bound} with $\delta = 1/200$ and recalling \eqref{eqn: initialization}, we have that for all $n$ sufficiently large, the following upper bound on $\alpha$:
\begin{equation*}
\alpha < \frac{5}{12\taubf(\mc{C})} \leq \min\biggl\{\frac{1}{2000},\frac{1}{2\tau_\infty(G_{n,r}(\Cset_{\sigma}[\Xbf])} \biggr\}
\end{equation*}
holds with probability at least $1 - \frac{b_5}{n} - 4n\exp\{-b_2n/200^2\}$. Additionally, in Algorithm~\ref{alg: ppr} we use an exact PPR vector $p_v = p_v^{(0)}$ to construct the sweep cut sets. Therefore by Lemma~\ref{lem:zhu}, there exists a set $\Csig[\Xbf]^g \subset \Csig[\Xbf]$ with $\vol_{n,r}(\Csig[\Xbf]^g) \geq \frac{1}{2}\vol_{n,r}(\Csig[\Xbf])$ such that for any $\beta \leq  \frac{1}{5\vol_{n,r}(\Csig[\Xbf])}$, the sweep cut $S_{\beta,v}$ of $p(\alpha, v; G_{n,r})$ satisfies
\begin{align}
\vol_{n,r}(\Csig[\Xbf] \vartriangle S_{\beta,v}) & \leq \frac{6}{\alpha \beta}\Phi_{n,r}(\mc{C}_{\sigma}) \\
& \leq \frac{20}{\beta} (1 + 4\delta) \kappa(\mc{C}) \label{eqn:volume_ssd_pf1}
\end{align}
where the latter inequality holds with probability at least $1 - 3\exp\set{-1/(36)^2b_{10}n}$, and follows from setting $\delta = 1/36$ in Theorem~\ref{thm: conductance_upper_bound} and the lower bound $\alpha \geq \frac{1}{3 \tau_u(\mc{C})}$ given in \eqref{eqn: initialization}. 

Finally, note that $\Cest = S_{\beta,v}$ for some $\beta \in \left(\frac{1}{7},\frac{1}{6}\right)\cdot\frac{1}{2{n \choose 2}\vol_{\Pbb,r}(\Csig)}$. Applying  Lemma~\ref{lem:ball_bounds_in_probability} with $\delta = 1/5$, we obtain
\begin{equation}
\label{eqn:volume_ssd_pf2}
\frac{4}{5}\vol_{\Pbb,r}(\Csig) \leq \frac{\vol_{n,r}(\Csig[\Xbf])}{2{n \choose 2}} \leq \frac{6}{5}\vol_{\Pbb,r}(\Csig)
\end{equation}
with probability at least $1 - 2\exp\{-\frac{1}{25} n \vol_{\Pbb,r}(\Csig)^2\}$; thus $\Cest = S_{\beta,v}$ for some $\beta \in \frac{1}{\vol_{n,r}(\Csig[\Xbf])}\left(\frac{4}{35},\frac{1}{5}\right)$, and from \eqref{eqn:volume_ssd_pf1} we have that
\begin{equation*}
\vol_{n,r}(\Cest \vartriangle \Csig[\Xbf]) \leq 175 \cdot \kappa(\mc{C}) \cdot \vol_{n,r}(\mc{C}_{\sigma}[\Xbf]).
\end{equation*}
To summarize, we have shown that when $n$ is sufficiently large (as determined by setting $\delta = 1/200$ in ~\eqref{eqn: mixing_time_sample_size})
then
\begin{equation*}
\vol_{n,r}(\Csig[\Xbf] \vartriangle \Cest) \leq 175 \cdot\kappa(\Cset)\cdot\vol_{n,r}(\Csig[\Xbf])
\end{equation*} 
with probability at least $1 - \frac{b_5}{n} - 4n\exp\{-b_2n/200^2\} - 3\exp\set{-(b_{10} + \frac{1}{5}\vol_{\Pbb,r}(\Csig)^2)n}$. We have proved Theorem~\ref{thm: volume_ssd} (upon appropriate choice of universal constant $c_0 = 175$, and constants $b_0 = \max\{2b_1, b_5\}$ and $b_{11} = \min\{b_2/200^2, b_{10} + \frac{1}{5}\vol_{\Pbb,r}(\Csig)^2\}$ in the theorem statement).

\section{Proof of Theorem~\ref{thm: consistent_recovery_of_density_clusters}}
\label{sec: proof_of_consistent_cluster_recovery}

As a reminder, to prove Theorem~\ref{thm: consistent_recovery_of_density_clusters} it will be sufficient to show that
\begin{equation}
\label{eqn: PPR_gap}
\max_{u' \in \Cset'[\Xbf]} \frac{p_v(u')}{\deg_{n,r}(u')} \leq \frac{1}{100 {n \choose 2} \vol_{\Pbb,r}(\Csig)} < \frac{1}{10 {n \choose 2} \vol_{\Pbb,r}(\Csig)} \leq \min_{u \in \Cset[\Xbf]} \frac{p_v(u)}{\deg_{n,r}(u)}
\end{equation}
with probability at least $1 - \frac{b_2}{n}$. 

To show this, we first establish in Lemma~\ref{lem: setup} that when Algorithm \ref{alg: ppr} is properly initialized, the resulting PPR vector is large for every vertex $u \in \Cset$, and small for every vertex $u' \in \Cset'$. Let $\overline{\pibf}_{n,r}: \Csig[\Xbf] \to [0,1]$ be the vector given by\footnote{Note that $\overline{\pibf}_{n,r}$ is distinct from $\widetilde{\pibf}_{n,r}$, as we normalize by $\vol_{n,r}(\Csig[\Xbf])$ rather than $\widetilde{\vol}_{n,r}(\Csig[\Xbf])$.} 
\begin{equation*}
\overline{\pibf}_{n,r}(u) := \frac{\widetilde{\deg}_{n,r}(u)}{\vol_{n,r}(\Csig[\Xbf])}.
\end{equation*} Additionally, let $\degminpr = \min_{u \in \Cset'[\Xbf]} \deg_{n,r}(u')$. 

\begin{lemma} 
	\label{lem: setup}
	Let $0 < r < \sigma$ and $\alpha \leq \frac{1}{9 \tau_{\infty}(\widetilde{G}_{n,r})}$. Then the following statement holds: there exists a good set $\Csig[\Xbf]^g \subseteq \Csig[\Xbf]$ with $\vol_{n,r}(\Csig[\Xbf]^g) \geq \vol_{n,r}(\Csig[\Xbf])/2$ such that the following bounds hold with respect to $\pbf_v := \pbf(v,\alpha;G_{n,r})$ for any $v \in \Csig[\Xbf]^g$:
	\begin{itemize}
		\item For each $u \in \Cset[\Xbf]$,
		\begin{equation}
		\label{eqn: lower_bound_PPR_in_cluster}
		\pbf_v(u) \geq \frac{4}{5} \overline{\pibf}_{n,r}(u) - \frac{20 \Phi_{n,r}(\Csig[\Xbf])/\alpha}{\degminwt}
		\end{equation}
		\item Let $\Cset' \neq \Cset \in \Cbb_f(\lambda)$ be another $\lambda$-density cluster. Then for each $u' \in \Cset'[\Xbf]$,
		\begin{equation}
		\label{eqn: upper_bound_PPR_in_other_cluster}
		\pbf_v(u') \leq \frac{20 \Phi_{n,r}(\Csig[\Xbf])/\alpha}{\degminpr}.
		\end{equation}
	\end{itemize}
\end{lemma}

We immediately note that by assumption, Algorithm \ref{alg: ppr} is well-initialized, and therefore the seed node $v$ is chosen in $\Csig[\Xbf]^g$. Since we additionally assume $\Cset$ is a $\kappa$-well-conditioned density cluster, we have that $r < \sigma/(2d) < \sigma$, and the upper bound 
\begin{equation*}
\alpha < \frac{1}{9\taubf(\theta)} \leq \frac{1}{(9 \tau_{\infty}(G_{n,r}[\Csig[\Xbf]]))}
\end{equation*}
holds with probability at least $1 - \frac{b_2}{n}$.
Therefore, all the conditions of Lemma \ref{lem: setup} are met. 

We now collect the estimates on graph functionals we will use to complete the proof of Theorem~\ref{thm: consistent_recovery_of_density_clusters}. 
\begin{itemize}
	\item By Theorem~\ref{thm: conductance_upper_bound},
	\begin{equation*}
	\Phi(\Csig[\Xbf]; G_{n,r}) \leq \Phibf(\mc{C}) 
	\end{equation*}
	with probability at least $1 - 3\exp\{-nb\}$. By \eqref{eqn: initialization}, $\alpha \geq 1/(10 \taubf(\theta))$ and therefore $2 \Phi(\Csig[\Xbf]; G_{n,r})/\alpha \leq 20 \kappa(\Cset)$.
	\item By Lemma~\ref{lem:ball_bounds_in_probability}, for any $\delta \in (0,1)$,
	\begin{equation*}
	\degminwt \geq \frac{6}{25}(1 - \delta) \lambda_{\sigma} r^d \nu_d n,
	\end{equation*}
	with probability at least $1 - n\exp\set{-\frac{2 \delta^2 \lambda_{\sigma} \nu_d r^d n}{75(1 + \frac{\delta}{3})}}$.
	\item By Lemma~\ref{lem:ball_bounds_in_probability}, for any $\delta \in (0,1)$,
	\begin{equation*}
	\degminpr \geq (1 - \delta) \lambda_{\sigma} r^d \nu_d n,
	\end{equation*}
	with probability at least $1 - n\exp\set{-\frac{\delta^2 \lambda_{\sigma} \nu_d r^d n}{3(1 + \frac{\delta}{3})}}$.
	
	\item By Lemma~\ref{lem:ball_bounds_in_probability}, for any $\delta \in (0,1)$,
	\begin{equation*}
	d_{\max} \leq (1 + \delta) \Lambda_{\sigma} \nu_d r^d
	\end{equation*}
	with probability at least $1 - n\exp\set{-\frac{2\delta^2 \lambda_{\sigma} \nu_d r^d n}{75(1 + \frac{\delta}{3})}}$.
	
	\item 
	\begin{equation*}
	\widetilde{\vol}_{n,r}(\Csig[\Xbf]) \leq 2(1 + \delta) \vol_{\Pbb,r}(\Csig){n \choose 2}
	\end{equation*}
	with probability at least $1 - \exp\set{-\delta^2 (\vol_{\Pbb,r}(\Csig))^2}$.
\end{itemize}

In the rest of this proof we assume the above bounds hold. For all $u \in \Cset[\Xbf]$, $\widetilde{\deg}_{n,r}(u) = \deg_{n,r}(u)$, so by \eqref{eqn: lower_bound_PPR_in_cluster},
\begin{align*}
\frac{p_v(u)}{\deg_{n,r}(u)} & \geq \frac{4}{5 \vol_{n,r}(\Csig[\Xbf])} - \frac{20 \kappa(\Cset)}{\widetilde{d}_{\min}^2} \\
& \geq \frac{4}{5(1+\delta)\vol_{\Pbb,r}(\Csig)n^2} - \frac{20\kappa(\Cset)}{\left(\frac{6}{25}(1 - \delta)\lambda_{\sigma}\nu_dr^dn\right)^2} \\
& \geq \frac{1}{10{n \choose 2}\vol_{\Pbb,r}(\Csig)},
\end{align*}
where the last inequality follows by choosing $\delta = 1/2$ and \eqref{eqn: kappa_ub}. For all $u' \in \Cset'[\Xbf]$, 
\begin{align}
\frac{p_v(u')}{\deg_{n,r}(u')} & \leq \frac{20\kappa(\Cset)}{(d_{\min}')^2} \nonumber \\
& \leq \frac{20\kappa(\Cset)}{\left((1 - \delta)\lambda_{\sigma}\nu_dr^dn\right)^2} \nonumber \\
& \leq \frac{1}{100{n \choose 2}\vol_{\Pbb,r}(\Csig)} \label{eqn:consistent_recovery_of_density_clusters_pf1}
\end{align}
where the last inequality again follows by choosing $\delta = 1/2$ and \eqref{eqn: kappa_ub}. We have shown~\eqref{eqn: PPR_gap} and therefore proved Theorem~\ref{thm: consistent_recovery_of_density_clusters} (upon an appropriate choice of constants $b_2$ and $c$ in the statement of the theorem).

% (choosing $b_2 = c_4(\Pbb,r) + \frac{6}{b} + \frac{900}{\lambda_{\sigma}\nu_dr^d} + \frac{1}{4}\vol_{\Pbb,r}(\Csig)^2$).

We defer the proof of Lemma \ref{lem: setup}, and first extend Theorems~\ref{thm: volume_ssd} and \ref{thm: consistent_recovery_of_density_clusters} to apply with respect to the aPPR vector.

\subsection{Proof of Corollary~\ref{cor: appr}.}
\label{sec:appr}
To prove the first claim of Corollary~\ref{cor: appr}, it will be sufficient to show that the conditions of Lemma~\ref{lem:zhu} are still met when we use the approximate PPR vector $p^{(\varepsilon)}_v$ rather than the exact PPR vector $p_v$. In particular, we must show that $\varepsilon \leq \frac{1}{20}\vol_{n,r}(\Csig)$. However by Lemma~\ref{lem:ball_bounds_in_probability},
\begin{equation}
\label{eqn:appr_pf1}
\vol_{n,r}(\Csig[\Xbf]) \geq (1 - \delta)\vol_{\Pbb,r}(\Csig)
\end{equation}
with probability at least $1 - \exp\set{-n \delta^2 (\vol_{\Pbb,r}(\Csig))^2}$, and by setting $\delta = 1/5$ the desired claim follows.

We now prove the second claim, by showing that both bounds in \eqref{eqn: PPR_gap} hold with respect to the aPPR vector. The upper bound follows immediately. Since $p^{(\varepsilon)}_v(x) \leq p_v(x)$ for all $x \in \Xbf$, the upper bound on $\max_{u' \in \Cset'[\Xbf]} p_v(u')$ also applies to $\max_{u' \in \Cset'[\Xbf]} p^{(\varepsilon)}_v(u')$. 

To show the desired lower bound on $\min_{u \in \Cset[\Xbf]} p^{(\varepsilon)}_v(u)$, we use the uniform approximation guarantee in \eqref{eqn: appr_error}. As previously observed, since $r < \sigma$, for any $u \in \Csig$ the ball $B(u,r) \subseteq \Csig$, and so $\widetilde{\deg}_{n,r}(u) = \deg_{n,r}(u)$. Along with \eqref{eqn: appr_parameter} and \eqref{eqn:appr_pf1}, this means
\begin{equation*}
\varepsilon \deg_{n,r}(u) \leq \frac{\overline{\pi}_{n,r}(u)}{25(1 - \delta)}
\end{equation*}
and therefore by \eqref{eqn: lower_bound_PPR_in_cluster}, for every $u \in \Cset[\Xbf]$,
\begin{equation*}
p^{(\varepsilon)}_v(u) \geq \left\{\frac{4}{5} - \frac{1}{25(1 - \delta)}\right\} \overline{\pi}_{n,r}(u) - \frac{20 \Phi_{n,r}(\Csig[\Xbf])/\alpha}{\degminwt}.
\end{equation*}
\noindent Using the arguments given in the proof of Theorem~\ref{thm: consistent_recovery_of_density_clusters}, and choosing the universal constant in \eqref{eqn: kappa_ub} to be sufficiently small, the upper bound in \eqref{eqn:consistent_recovery_of_density_clusters_pf1} follows.

\subsection{Proof of Lemma~\ref{lem: setup}}

In this subsection, we will let $\Dbf$ and $\Wbf$ be the degree and lazy random walk matrices over $G_{n,r}$. Additionally we let $\widetilde{\Dbf}$ and $\widetilde{\Wbf}$ be the degree and lazy random walk matrices over the induced subgraph $\widetilde{G}_{n,r}$. Given a starting distribution $q$ with $\mathrm{supp}(q) \subseteq \Csig[\Xbf]$, we let
	\begin{equation}
	\widetilde{\pbf}_q = \alpha q + (1 - \alpha) \widetilde{\pbf}_{q} \widetilde{\Wbf}
	\end{equation}
	be the \pprspace vector originating from $q$ over $\widetilde{G}_{n,r}$. (When the starting distribution $q = e_v$ is a point mass at a seed node $v \in \Csig[\Xbf]$, we write $\widetilde{\pbf}_v := \widetilde{\pbf}_{e_v}$ in a slight abuse of notation).
	
	Our analysis will involve \emph{leakage} and \emph{soakage} vectors, defined by
	\begin{align}
	\label{eqn: leakage_soakage}
	\ell_t & := e_v (\Wbf \widetilde{I} )^t (I - \Dbf^{-1} \wDbf),~ \ell := \sum_{t = 0}^{\infty} (1 - \alpha)^t \ell_t, \nonumber \\
	s_t & := e_v (\Wbf \widetilde{I} )^t (\Wbf \widetilde{I}^c),~ s := \sum_{t = 0}^{\infty} (1 - \alpha)^{t} s_t.
	\end{align}
	where $I$ is the $n \times n$ identity matrix, $\widetilde{I}$ is an $n \times n$ diagonal matrix with $\widetilde{I}_{uu} = 1$ if $u \in \Csig[\Xbf]$ and $0$ otherwise, and $\widetilde{I}^c = I - \widetilde{I}$.
	
	In words, for $u \in \Csig[\Xbf]$, $\ell_t(u)$ is the probability that a random walk over $G_{n,r}$ originating from $v \in \Csig[\Xbf]$ stays within $\widetilde{G}_{n,r}$ for $t$ steps, arriving at $u$ on the $t$th step, and then ``leaks out'' of $\Csig[\Xbf]$ on the $t + 1$th step. For $w \in \Xbf \setminus \Csig[\Xbf]$, $\ell_t(w) = 0$. By contrast, for $w$ again in $\Xbf \setminus \Csig[\Xbf]$, $s_t(w)$ is the probability that a random walk originating from $v$ stays within $\Csig[\Xbf]$ for $t$ steps, and then is ``soaked up'' into $w$ on the $t + 1$ step, while $s_t(u) = 0$ for all $u \in \Csig[\Xbf]$. The vectors $\ell$ and $s$ then give the total mass leaked and soaked, respectively, by the \pprspace vector. 
	
	We first prove \eqref{eqn: lower_bound_PPR_in_cluster}, and begin by restating some results of \citep{zhu2013}, adapted to our notation. By Lemma 3.1 of \citep{zhu2013}, there exists a good set $\Csig[\Xbf]^g \subseteq \Csig[\Xbf]$ with $\vol(\Csig[\Xbf]^g; G_{n,r}) \geq \vol(\Csig[\Xbf]; G_{n,r})/2$ such that for every $v \in \Csig[\Xbf]^g$
	\begin{equation}
	\label{eqn: zhulemma3.1}
	\pbf_v(u) \geq \widetilde{\pbf}_v(u) - \widetilde{\pbf}_{\ell}(u), \quad \textrm{and} \quad \norm{\ell}_1 \leq \frac{2 \Phi_{n,r}(\Csig[\Xbf])}{\alpha}.
	\end{equation}
	(The result $\norm{\ell}_1 \leq \frac{2 \Phi_{n,r}(\Csig[\Xbf])}{\alpha}$ is the only result in the proof of Theorem \ref{thm: consistent_recovery_of_density_clusters} which relies on the restriction $v \in \Csig[\Xbf]^g$.)
	
	If additionally $\alpha \leq \frac{1}{9\tau_{\infty}(\widetilde{G}_{n,r})}$, then by Corollary 3.3 of \citep{zhu2013}, for every $u \in \Csig[\Xbf]$
	\begin{equation*}
	\widetilde{\pbf}_v(u) \geq \frac{4}{5} \overline{\pibf}_{n,r}(u)
	\end{equation*}
	and along with \eqref{eqn: zhulemma3.1}, we obtain
	\begin{equation*}
	\pbf_v(u) \geq \frac{4}{5} \overline{\pibf}_{n,r}(u) - \widetilde{\pbf}_{\ell}(u).
	\end{equation*}
	
	We proceed to show the upper bound $\widetilde{\pbf}_{\ell}(u) \leq \norm{\ell}_1 / \degminwt$, whence \eqref{eqn: lower_bound_PPR_in_cluster} follows by \eqref{eqn: zhulemma3.1}. We note two facts regarding $\widetilde{\pbf}_{\ell}(u)$, which hold for all $u \in \Cset[\Xbf]$. 
	\begin{enumerate}
		\item Since $r < \sigma$, $(u,w) \not\in G_{n,r}$ for any $w \not\in \Csig$. As a result, for all $t \geq 1$, $\ell_t(u) = 0$ and by extension, $\ell(u) = 0$ as well.
		\item For any $q$ such that $\sum_{w \in \Csig[\Xbf]} q(w) \leq 1$ and $u \not\in \mathrm{supp}(q)$, and any $t \geq 1$,
		\begin{align}
		\label{eqn: one_step_bound}
		q \widetilde{\Wbf}^t (u) & \leq \norm{q}_1 \max_{v \neq u} W_{vu} \nonumber \\
		& \leq \frac{1}{2\degminwt} 
		\end{align}
		where last inequality follows from the fact $(u,w) \in \widetilde{G}_{n,r}$ implies $w \in \Csig$, and therefore $\deg(w; \widetilde{G}_{n,r}) \geq \degminwt$.
	\end{enumerate}
	
	These facts, along with some basic algebra, lead to the desired lower bound on $\widetilde{\pbf}_{\ell}(u)$ for every $u \in \Cset[\Xbf]$:
	\begin{align*}
	\widetilde{\pbf}_{\ell}(u) & = \alpha \sum_{t = 0}^{\infty} (1 - \alpha)^t \left(\ell \widetilde{\Wbf}^t \right)(u)  \nonumber \\
	& = \norm{\ell}_1 \alpha \sum_{t = 0}^{\infty} (1 - \alpha)^t \left(\frac{\ell}{\norm{\ell}_1}  \widetilde{\Wbf}^t \right)(u)\nonumber \\
	& = \norm{\ell}_1 \alpha \sum_{t = 1}^{\infty} (1 - \alpha)^t \left(\frac{\ell}{\norm{\ell}_1}  \widetilde{\Wbf}^t \right)(u)\nonumber \\
	& \leq \frac{\norm{\ell}_1 }{\degminwt}. \tag{since $u \not\in \mathrm{supp}(\ell)$}
	\end{align*}
	and \eqref{eqn: lower_bound_PPR_in_cluster} is proved.
	
	We turn to showing \eqref{eqn: upper_bound_PPR_in_other_cluster}. By Lemma \ref{lem: soakage_ppr_lb}, for all $u' \not\in \Csig[\Xbf]$,
	\begin{equation*}
	\pbf_v(u') \leq \pbf_s(u').
	\end{equation*}
	
	Note that by \ref{asmp: low_noise_density}, $\dist(\Csig,\Cset') > r$. Therefore for every $u \in \Csig[\Xbf]$ and $u' \in \Cset'[\Xbf]$ $(u',u) \not\in E$ and so $s(u') = 0$. Some manipulations, similar to those in the preceding part of the proof, yield a lower bound on $\pbf_v(u')$ in terms of $\norm{s}_1$:
	\begin{align*}
	\pbf_s(u') & = \alpha \sum_{t = 0}^{\infty} (1 - \alpha)^t \left(s \Wbf^t\right)(u') \\
	& = \norm{s}_1 \alpha \sum_{t = 0}^{\infty} (1 - \alpha)^t \left(\frac{s}{\norm{s}_1} \Wbf^t\right)(u') \\
	& = \norm{s}_1 \alpha \sum_{t = 1}^{\infty} (1 - \alpha)^t \left(\frac{s}{\norm{s}_1} \Wbf^t\right)(u') \\
	& \leq \frac{\norm{s}_1}{2\degminpr}
	\end{align*}
	where the last inequality follows from precisely the same reasoning as \eqref{eqn: one_step_bound}. The claim follows in light of Lemma \ref{lem: total_leakage_equal_total_soakage}, along with \eqref{eqn: zhulemma3.1}.

\subsection{Linear Algebra Facts}
\label{sec: linalg}

We state here a number of basic facts which follow from matrix manipulations, which are used in the proof of Theorem \ref{thm: consistent_recovery_of_density_clusters}. 

\begin{lemma}
	\label{lem: soakage_ppr_lb}
	For any $v \in \Csig[\Xbf]$ and $u \not\in \Csig[\Xbf]$,
	\begin{equation*}
	\pbf_{v}(u) \leq \pbf_{s}(u)
	\end{equation*}
	where $s$ is defined as in \eqref{eqn: leakage_soakage} and depends implicitly upon $v$.
\end{lemma}
\begin{proof}
	The statement follows from Lemma \ref{lem: sum_of_soakages} along with a series of algebraic manipulations,
	\begin{align*}
	\pbf_v(u) & = \alpha\sum_{T = 0}^{\infty} (1 - \alpha)^t \left(e_v \Wbf^T\right)(u) \\
	& = \alpha\sum_{T = 1}^{\infty} (1 - \alpha)^t \left(e_v \Wbf^T\right)(u) \\
	& \leq \alpha \sum_{T=1}^{\infty} (1 - \alpha)^T \left( \sum_{t = 0}^{T - 1} s_t \Wbf^{T - t - 1} \right)(u) \tag{Lemma \ref{lem: sum_of_soakages}}\\
	& = \alpha \sum_{t=0}^{\infty} \sum_{T = t + 1}^{\infty} (1 - \alpha)^T \left( s_t \Wbf^{T - t - 1} \right)(u)\\
	& = \alpha \sum_{t=0}^{\infty} \sum_{\Delta = 0}^{\infty} (1 - \alpha)^{\Delta + t + 1} \left( s_t \Wbf^{\Delta} \right)(u) \\
	& \leq \alpha \sum_{t=0}^{\infty} \sum_{\Delta = 0}^{\infty} (1 - \alpha)^{\Delta + t } \left( s_t \Wbf^{\Delta} \right)(u) \\
	& = \alpha \sum_{\Delta = 0}^{\infty} (1 - \alpha)^{\Delta} \left(s \Wbf^{\Delta}\right)(u) \\
	& = \pbf_s(u)
	\end{align*}
\end{proof}

Let $s_t := q(\Wbf \Ibf{S})^t(\Wbf(\Ibf{S^c}))$ be the soakage vector out of $S \subseteq V$, where $\Ibf{S}$ is a $\abs{V} \times \abs{V}$ diagonal matrix with $(\Ibf{S})_{uu} = 1$ if $u \in S$ and 0 otherwise, and $\Ibf{S^c} := \Ibf{} - \Ibf{S}$. 
\begin{lemma}
	\label{lem: sum_of_soakages}
	Let $G = (V,E)$ be an unweighted, undirected graph with associated random walk matrix $\Wbf$. For any $T \in \mathbb{N}, T \geq 1$, $q \in \Reals^{\abs{V}}$, and $S \subseteq V$
	\begin{equation}
	\label{eqn: sum_of_soakages}
	q\Wbf^T = \sum_{t = 0}^{T - 1} s_t \Wbf^{T - t - 1} + q(\Wbf \Ibf{S})^T
	\end{equation}
	In particular, if $u \in V\setminus S$, then
	\begin{equation}
	\label{eqn: sum_of_soakages_2}
	q\Wbf^T(u) = \sum_{t = 0}^{T - 1} \left(s_t \Wbf^{T - t - 1}\right)(u)
	\end{equation}
\end{lemma}
\begin{proof}	
	We show \eqref{eqn: sum_of_soakages}, from which \eqref{eqn: sum_of_soakages_2} is an immediate consequence.
	To show \eqref{eqn: sum_of_soakages}, we proceed by induction on $T$. When $T = 1$,
	\begin{equation*}
	q\Wbf = q\Wbf\Ibf{S} + q\Wbf\Ibf{S^c} =  q\Wbf\Ibf{S} + s_0. 
	\end{equation*}
	
	Then, for $T \in \mathbb{N},~ T \geq 2$,
	\begin{align*}
	q\Wbf^{T} & = q\Wbf^{T - 1}{\Wbf} \\
	& = \left\{\sum_{t = 0}^{T - 2} s_t \Wbf^{T - 2 - t} + q(\Wbf \Ibf{S})^{T - 1} \right\} \Wbf \tag{by the inductive hypothesis}\\
	& = \sum_{t = 0}^{T - 2} s_t \Wbf^{T - 1 - t} + q(\Wbf \Ibf{S})^{T - 1} (\Wbf \Ibf{S} + \Wbf \Ibf{S^c}) \\
	& = \sum_{t = 0}^{T - 1} s_t \Wbf^{T - 1 - t} + q(\Wbf \Ibf{S})^{T - 1} (\Wbf \Ibf{S})
	\end{align*}
	and the proof is complete.
\end{proof}

\begin{lemma}
	\label{lem: total_leakage_equal_total_soakage}
	Letting $s_t$, $\ell_t$ $s$ and $\ell$ be as in \eqref{eqn: leakage_soakage},
	\begin{equation*}
	\norm{s_t}_1 = \norm{\ell_t}_1,~ \textrm{for each $t \geq 0$}
	\end{equation*}
	and therefore $\norm{s}_1 = \norm{\ell}_1$. 
\end{lemma}
\begin{proof}
	By the definition of $s_t$ and $\ell_t$, we have
	\begin{align*}
	\norm{s_t}_1 & = \norm{q_t (\Wbf \widetilde{I}^c)}_1 \\
	& = \sum_{u \in \Xbf} \sum_{u' \in \Xbf} q_t(u) (\Wbf \widetilde{I}^c)(u, u') \\
	& = \sum_{u \in\Csig[\Xbf]} \sum_{u' \in \Csig[\Xbf]^c} \frac{q_t(u)}{(\Dbf)_{uu}} \1((u,u') \in G_{n,r}) \\
	& = \sum_{u \in \Csig[\Xbf]} \frac{q_t(u) \left((\Dbf)_{uu} - (\widetilde{\Dbf})_{uu} \right)}{(\Dbf)_{uu}} \\
	& = \norm{q_t (\Ibf{} - \Dbf^{-1} \widetilde{\Dbf})}_1 = \norm{\ell_t}_1.
	\end{align*}
\end{proof}

\section{Proof of Lower Bound.}

To prove Theorem~\ref{thm:ppr_lb}, we will proceeding according to the following steps:
\begin{enumerate}
	\item We study the spectral partitioning properties of PPR on an arbitrary graph $G$, and show that when suitably initialized inside a subset $S \subset V$, the normalized cut of the PPR sweep cut is upper bounded by (a function of) $\Phi(S;G)$. 
	\item We specialize to the graph $G = G_{n,r}$ and the subset $\mathcal{L}[\Xbf] \subset \Xbf$, and show that the normalized cut $\Phi_{n,r}(\mathcal{L}[\Xbf])$ is small (with high probability) when the diameter $\rho$ is large.
	\item We reason that for the input parameters given in Theorem~\ref{thm:ppr_lb}, the output of Algorithm~\ref{alg: ppr} $\Cest$ must therefore also have small normalized cut.
	\item On other hand, we show that when the noise parameter $\epsilon$ is not too small, the empirical density cluster $\Cset^{(1)}[\Xbf]$ will have large normalized cut $\Phi_{n,r}(\Cset^{(1)}[\Xbf])$. In fact, we generalize this to hold for any set $A \subset \Xbf$ for which the symmetric set distance metric $\Delta(A,\Cset_1[\Xbf])$ is small.
	\item We conclude that the symmetric set distance metric $\Delta(\Cest,\Cset^{(1)}[\Xbf])$ must not be small.
\end{enumerate}

We devote the subsequent sections to proving each of the aforementioned steps.

\subsection{Spectral partitioning properties of PPR.}

Let $G = (V,E)$ be an undirected, unweighted graph with $m = \abs{E}$ total edges, defined on vertices $V = \set{v_1, \ldots, v_n}$. Let $C$ be a subset of the vertices $V$, Recall that for a given $\beta \in (0,1)$ the sweep cut 
\begin{equation*}
S_{\beta,v} = \set{u \in V: \frac{p_v(u)}{\deg(u;G)} > \beta}
\end{equation*} 
The following theorem relates the normalized cut of the sweep sets $\Phi(S_{\beta};G)$ to the normalized cut of $C$; it is stated with respect to the graph functionals 
\begin{equation*}
d_{\max} := \max_{u \in V} \deg(u;G), ~~\textrm{and}~~ d_{\min} := \min_{u \in V} \deg(u;G).
\end{equation*}
\begin{theorem}
	\label{thm:conductance_ppr}
	Let $C \subseteq V$ satisfy the following conditions:
	\begin{itemize}
		\item $\vol(C;G) \leq \frac{2}{3}\vol(G)$,
		\item $\abs{C} \geq \frac{d_{\max}}{d_{\min}}$, and
		\item $\frac{20\Phi(C;G)}{1 + 10\Phi(C;G)} + \frac{d_{\max}}{2d_{\min}^2} \leq \frac{1}{10}$.
	\end{itemize}
	Suppose $60\Phi(C;G) \leq \alpha \leq 70\Phi(C;G)$, and let $(L,U) = (0,1)$. Then, there exists a subset $C^g \subset C$ with $\vol(C^g;G) \geq \frac{5}{6}\vol(C;G)$ such that for any $v \in C^g$ the following statement holds: For the PPR vector $p_v := p(v,\alpha;G)$, the minimum conductance sweep cut set satisfies 
	\begin{equation*}
	\min_{\beta \in (0,1)}\Phi(S_{\beta,v};G) \leq \sqrt{11200\left\{\log\left(\frac{m}{d_{\min}^2}\right) + \log 20\right\} \Phi(C;G)}
	\end{equation*}
\end{theorem}
Although this theorem appears quite similar to standard results in the PPR literature -- for instance, Theorem 6 of \citet{andersen2006} -- crucially the above bound depends on $\log\left(\frac{m}{d_{\min}^2}\right)$ rather than $\log m$. In the case where $d_{\min} \asymp n$, this amounts to replacing a factor of $O(\log m)$ by a factor of ${O}(1)$, and therefore allows us to obtain meaningful results in the limit as $m \to \infty$. 

Notwithstanding these improvements, the proof of Theorem~\ref{thm:conductance_ppr} follows the same general outline as the proof of Theorem~6 of \citet{andersen2006}. We now walk through this outline step by step, modifying the results of \citet{andersen2006} as needed. As with their work, we begin by proving a mixing time bound on the PPR vector $p_v$.

\subsubsection{Mixing time of PPR.}

To quantify the mixing of a PPR vector $p_v$, we introduce the function $p[\cdot]: [0,2m] \to [0,1]$. For $j = 1,\ldots,n$, let $\beta_j$ be the smallest value of $\beta \in (0,1)$ such that $S_{\beta_j}$ contains at least $j$ vertices. (For notational ease, we will write $S_{i} := S_{\beta_i}$, so that $S_1,S_2,\ldots,S_n$ comprise the $n$ unique sweep cuts of $p_v$.)
For each $j = 1,\ldots,n$, we let $p[\vol(S_j)] =  \sum_{u \in S} p_v(u)$. Additionally, we let $p[0] = 0$ and $p[2m] = 1$. Finally, we extend $p[\cdot]$  by piecewise interpolation to be defined everywhere on its domain. The mixedness of the PPR vector is then measured by the function $h:[0,2m] \to [0,1]$, defined as 
\begin{equation*}
h(k) = p[k] - \frac{k}{2m}.
\end{equation*}
Next, for a given $0 \leq K_0 \leq m$, let 
\begin{equation*}
L_{K_0}(k) = \frac{2m - K_0 - k}{2m - 2K_0}h(K_0) + \frac{k - K_0}{2m - 2K_0}h(2m - K_0)
\end{equation*}
be the linear interpolator of $h(K_0)$ and $h(2m - K_0)$, and additionally let
\begin{equation*}
C(K_0) = \max\set{\frac{h(k) - L_{K_0}(k)}{\sqrt{\overline{k}}}: K_0 < k < 2m - K_0}.
\end{equation*}
where we use the notation $\overline{k} := \min\{k, 2m - k\}$.

Theorem~\ref{thm:mixing_time_PPR} implies that if the PPR random walk is not well mixed, then some sweep cut of $p_v$ must have small normalized cut.
\begin{theorem}
	\label{thm:mixing_time_PPR}
	Let $p_v = p(v,\alpha;G)$ be a PPR vector, and let $\phi$ be any constant in $[0,1]$. Then, either the following bound holds for any integer $t$, any $0 < K_0 < m$, and any $k \in [K_0,2m - K_0]$:
	\begin{equation}
	\label{eqn:mixing_time_PPR}
	h(k) \leq \alpha t + L_{K_0}(k) + C(K_0)\sqrt{\overline{k}}\left(1 - \frac{\phi^2}{8}\right)^t
	\end{equation}
	or else there exists some sweep cut $S_j$ of $p_v$ such that $\Phi(S_j;G) < \phi$.
\end{theorem}

\begin{proof}[Proof (of Theorem~\ref{thm:mixing_time_PPR}).]
	The proof of Theorem~\ref{thm:mixing_time_PPR} is essentially a combination of the proofs of Theorem~3 in \citet{andersen2006} and Theorem 1.2 in \citet{lovasz1990}. We will show that if $\Phi(S_j) > \phi$ for each $j = 1,\ldots,n$, then \eqref{eqn:mixing_time_PPR} holds for all $t$ and any $k \in (K_0,2m - K_0)$.
	
	We proceed by induction on $t$. Our base case will be $t = 0$. Observe that $C(K_0) \cdot \sqrt{\overline{k}} \geq  h(k) - L_{K_0}(k)$ for all $k \in [K_0,2m - K_0]$, which implies
	\begin{equation*}
	L_{K_0}(k) + C(K_0) \cdot \sqrt{\overline{k}} \geq h(k).
	\end{equation*}
	
	Now, we proceed with the inductive step. By the definition of $L_{K_0}$, the inequality~\eqref{eqn:mixing_time_PPR} holds when $k = K_0$ or $k = 2m - K_0$. We will additionally show that~\eqref{eqn:mixing_time_PPR} holds for every $k_j = \vol(S_j), j = 1,2,\ldots,n$ such that $k_j \in [K_0, 2m - K_0]$. Once this is shown, the concavity of the expression on the right-hand side of~\eqref{eqn:mixing_time_PPR} implies that the inequality holds for all $k \in [K_0,2m - K_0]$.
	
	By Lemma 5 of \citet{andersen2006}, we have that
	\begin{align}
	p[k_j] & \leq \alpha + \frac{1}{2}  \left(p[k_j - \abs{\partial(S_j)}] + p[k_j + \abs{\partial{S_j}}]  \right) \nonumber\\
	& \leq \alpha + \frac{1}{2} \left(p[k_j - \Phi(S_j) \overline{k}_j] + p[k_j + \Phi(S_j) \overline{k}_j]  \right) \nonumber \\
	& \leq \alpha + \frac{1}{2} \left(p[k_j - \phi \overline{k}_j] + p[k_j + \phi \overline{k}_j]\right) \nonumber
	\end{align}
	and subtracting $k_j/2m$ from both sides, we get
	\begin{equation}
	\label{eqn:mixing_time_PPR_pf1}
	h(k_j) \leq \alpha + \frac{1}{2} \bigl(h(k_j - \phi \overline{k}_j) + h(k_j +  \phi \overline{k}_j) \bigr)
	\end{equation}
	From this point, we divide our analysis into cases. 
	
	\textbf{Case 1.}
	Assume $k_j - 2 \phi \overline{k}_j$ and $k_j + 2 \phi \overline{k}_j$ are both in $[K_0,2m  - K_0]$. We are therefore in a position to apply our inductive hypothesis to \eqref{eqn:mixing_time_PPR_pf1}, yielding
	\begin{align*}
	h(k_j) & \leq \alpha + \alpha(t-1) \frac{1}{2}\biggl(L_{K_0}(k_j - \phi \overline{k}_j) + L_{K_0}(k_j + \phi \overline{k}_j) + C(K_0)\bigl(\sqrt{\overline{k_j - \phi \overline{k}_j}} + \sqrt{\overline{k_j + \phi \overline{k}_j}}\bigr)\left(1 - \frac{\phi^2}{8}\right)^{t-1} \biggr) \\
	& \leq \alpha t + L_{K_0}(k) + \frac{1}{2}\biggl(C(K_0)\bigl(\sqrt{\overline{k_j - \phi \overline{k}_j}} + \sqrt{\overline{k_j + \phi \overline{k}_j}}\bigr)\left(1 - \frac{\phi^2}{8}\right)^{t-1} \biggr) \\
	& \leq \alpha t + L_{K_0}(k) + \frac{1}{2}\biggl(C(K_0)\bigl(\sqrt{\overline{k}_j - \phi \overline{k}_j} + \sqrt{\overline{k}_j + \phi \overline{k}_j}\bigr)\left(1 - \frac{\phi^2}{8}\right)^{t-1} \biggr).
	\end{align*}
	A Taylor expansion of $\sqrt{1 + \phi}$ around $\phi = 0$ yields the following bound:
	\begin{equation*}
	\sqrt{1 + \phi} + \sqrt{1 - \phi} \leq 2 - \frac{\phi^2}{4},
	\end{equation*}
	and therefore
	\begin{equation*}
	h(k_j) \leq  \alpha t + L_{K_0}(k) + \frac{C(K_0)}{2}\cdot \sqrt{\overline{k}_j}\cdot\left(2 - \frac{\phi^2}{4}\right)\left(1 - \frac{\phi^2}{8}\right)^{t-1} = \alpha t + L_{K_0}(k) + C(K_0)\sqrt{\overline{k}_j}\left(1 - \frac{\phi^2}{8}\right)^{t}.
	\end{equation*}
	
	\textbf{Case 2.}
	
	Now, assume one of $k_j - 2 \phi \overline{k}_j$ or $k_j + 2 \phi \overline{k}_j$ is not in $[K_0,2m  - K_0]$. Without loss of generality assume $k_j < m$, so that (i) we have $k_j - 2 \phi \overline{k}_j < K_0$ and (ii) $k_j + (k_j - K_0) \leq 2m - K_0$. By the concavity of $h$, and applying the inductive hypothesis to $h(2k_j - K_0)$, we have
	\begin{align*}
	h(k_j) & \leq \alpha + \frac{1}{2}\Bigl(h(K_0) + h\bigl(k_j + (k_j - K_0)\bigr)\Bigr) \\
	& \leq\alpha + \frac{\alpha(t - 1)}{2} + \frac{1}{2}\Bigl(L_{K_0}(K_0) + L_{K_0}(2k_j - K_0\bigr) + C(K_0)\sqrt{\overline{2k_j - K_0}}\left(1 - \frac{\phi^2}{8}\right)^{t - 1}\Bigr) \\
	& \leq \alpha t + L_{K_0}(k_j) + C(K_0) \frac{\sqrt{2\overline{k}_j}}{2} \left(1 - \frac{\phi^2}{8}\right)^{t - 1} \\
	& \leq \alpha t + L_{K_0}(k_j) + C(K_0) \sqrt{\overline{k}_j} \cdot \left(1 - \frac{\phi^2}{8}\right)^{t}
	\end{align*}
\end{proof}

As a sanity check, we confirm that Theorem~\ref{thm:mixing_time_PPR} is no weaker than Theorem~3 of \citet{andersen2006}. It is not hard to show that $h(k) \leq \min\{1,\sqrt{k}\}$, and therefore that $C(K_0) \leq 1$ for any $K_0$. Setting $K_0 = 0$ in Theorem~\ref{thm:mixing_time_PPR}, we therefore recover Theorem~3 of \citet{andersen2006}.

We now proceed to identify when Theorem~\ref{thm:mixing_time_PPR} may offer some improvement on Theorem~3 of \citet{andersen2006}, by showing when we can upper bound $C(K_0) << 1$. The critical point is that since $h(k)$ is concave and $L_{K_0}(K_0) = h(K_0)$ the upper bound
\begin{equation*}
\frac{h(k) - L_{K_0}(k)}{\sqrt{\overline{k}}} \leq h'(K_0) \sqrt{k}
\end{equation*}
holds whenever $k < m$. For similar reasons, when $k > m$, 
\begin{equation*}
\frac{h(k) - L_{K_0}(k)}{\sqrt{\overline{k}}} \leq -h'(2m - K_0) \sqrt{2m - k}.
\end{equation*} 
(Since $h$ is not differentiable at points $k = \vol(S_j)$, here we use $h'$ to denote the left derivative of $h$ whenever $k < m$, and the right derivative of $h$ whenever $k \geq m$)  

The following Lemma gives good estimates for $h'(K_0)$ and $h'(2m - K_0)$, and a resulting upper bound on $C(K_0)$.
\begin{lemma}
	\label{lem:linearization_bound}
	There exists $K_0 \in \set{0,\deg(v;G)}$ such that
	\begin{equation}
	\label{eqn:left_derivative}
	h'(K_0) \leq  \frac{1}{2d_{\min}^2}.
	\end{equation}
	Additionally, for all $K_0 \in [0,2m]$,
	\begin{equation}
	\label{eqn:right_derivative}
	h'(2m - K_0) \geq -\frac{d_{\max}}{d_{\min}\vol(G)}.
	\end{equation}
	As a result,
	\begin{equation*}
	C(K_0) \leq \frac{\sqrt{m}}{d_{\min}^2}.
	\end{equation*}
\end{lemma}
\begin{proof}[Proof (of Lemma~\ref{lem:linearization_bound}).]
	The result of the Lemma is obvious once we show \eqref{eqn:left_derivative} and \eqref{eqn:right_derivative}. To show either inequality, it will be useful to work with an alternative representation of $h$. In particular,  whenever $\vol(S_j) \leq k < \vol(S_{j + 1})$ (where we let $S_0 = \emptyset$), the function $h(k)$ may be written as
	\begin{equation}
	\label{eqn:lovasz_simonovits}
	h(k) = \sum_{i = 0}^{j} \left(p_v(u_{(i)}) - \pi(u_{(i)};G)\right) + \frac{\bigl(k - \vol(S_j;G)\bigr)}{\deg(u_{(j + 1)};G)} \left(p_v(u_{(j+1)}) - \pi(u_{(j+1)};G)\right) 
	\end{equation}
	where the vertices are ordered $\frac{p_v(u_{(1)})}{\deg(u_{(1)};G)} \geq \frac{p_v(u_{(2)})}{\deg(u_{(2)};G)} \geq \cdots \geq \frac{p_v(u_{(n)})}{\deg(u_{(n)};G)}$, and as usual $\pi(u;G) = \frac{\deg(u;G)}{\vol(G)}$. 
	
	From this representation, it is not hard to verify that the left derivative $h'(k)$ can be upper bounded
	\begin{equation}
	\label{eqn:linearization_bound_pf1}
	h'(k) \leq \frac{p(v_{(j + 1)})}{\deg(v_{(j + 1)};G)}
	\end{equation}
	
	We now upper bound $p(u)$ uniformly over all $u$ except the seed node $v$. For any $u \in V$ besides the seed node $v$, we can show by induction that
	\begin{equation*}
	e_v W^t(u) \leq \frac{1}{2 d_{\min}}
	\end{equation*} 
	for any $t \geq 0$, and therefore
	\begin{equation}
	\label{eqn:linearization_bound_pf2}
	p(\alpha,\chi_v)(u) = \alpha \sum_{t = 0}^{\infty} (1 - \alpha)^t \chi_v W^t(u) \leq  \frac{1}{2d_{\min}}.
	\end{equation}
	As a result, by \eqref{eqn:linearization_bound_pf1}, for either $K_0 = \deg(v;G)$ (in the case where $v_{(1)} = v$) or otherwise for $K_0 = 0$, the inequality $h'(K_0) \leq \frac{1}{2d_{\min}^2}$ holds, proving \eqref{eqn:left_derivative}. The inequality \eqref{eqn:right_derivative} follows immediately from the representation \eqref{eqn:lovasz_simonovits}, since
	\begin{equation*}
	h'(k) \geq -\frac{\pi(v_{(j+1)})}{d(v_{(j + 1)})} \geq -\frac{\pi_{\max}}{d_{\min}},
	\end{equation*}
	and the proof of the Lemma is therefore complete.
\end{proof}

To apply Theorem~\ref{thm:mixing_time_PPR}, we must also upper bound the linear interpolator $L_{K_0}(k)$. Of course, trivially $L_{K_0}(k) \leq \max\set{h(K_0), h(2m - K_0)}$ for all $k$. As it happens, this observation will lead to a sufficient upper bound on $L_{K_0}$.
\begin{lemma}
	\label{lem:interpolator_bound}
	Assume $s = \chi_v$ for some $v \in V$. Let $K_0 = \vol(S_j)$ for some $j = 0,\ldots,n$. Then, 
	\begin{equation*}
	h(2m - K_0) \leq \frac{K_0}{2m} ~\mathrm{and}~ h(K_0) \leq \frac{K_0}{2d_{\min}^2} + \frac{2\alpha}{1 + \alpha}.
	\end{equation*}
	and as a result for any $k \in \Reals$,
	\begin{equation*}
	L_{K_0}(k) \leq \frac{2\alpha}{1 + \alpha} + \frac{K_0}{2d_{\min}^2}.
	\end{equation*}
\end{lemma}
\begin{proof}[Proof (of Lemma~\ref{lem:interpolator_bound})]
	We make use of the representation~\eqref{eqn:lovasz_simonovits} to prove the desired upper bounds on $h(2m - K_0)$ and $h(K_0)$. We first upper bound $h(2m - K_0)$,
	\begin{align*}
	h(2m - K_0) & = \sum_{i = 1}^{j} p(v_{(i)}) - \pi(v_{(i)}) \\
	& \leq 1 - \sum_{i = 1}^{j} \pi(v_{(i)}) \\
	& = 1 - \sum_{i = 1}^{j} \frac{d(v_{i})}{2m} = \frac{K_0}{2m}.
	\end{align*}
	
	We will upper bound $h(K_0)$ by $p[\vol(S_j)] \leq p_v(v) + \sum_{u \in S_j \setminus \set{v}}p_v(u)$. In the proof of Lemma~\ref{lem:linearization_bound} we have already given an upper bound on $p_v(u)$ when $u \neq v$. Now, we additionally observe that for all $t$,
	\begin{equation*}
	e_vW^t(v) \leq \frac{1}{2d_{\min}} + \left(\frac{1}{2}\right)^t
	\end{equation*}
	and therefore $p_v(v) \leq \frac{1}{2d_{\min}} + \frac{2\alpha}{1 + \alpha}$.
	As a result,
	\begin{equation}
	h(K_0) \leq \frac{2\alpha}{1 + \alpha} + \frac{\abs{S_j}}{2d_{\min}} \leq \frac{2\alpha}{1 + \alpha} + \frac{K_0}{2d_{\min}^2},
	\end{equation}
	where the latter inequality follows since $K_0 = \vol(S_j) \geq \abs{S_j}\cdot d_{\min}$.
\end{proof}

Combining Theorem~\ref{thm:mixing_time_PPR}, Lemma~\ref{lem:linearization_bound} and Lemma~\ref{lem:interpolator_bound}, we have the following result.
\begin{corollary}
	\label{cor:mixing_time_PPR}
	Let $p_v = p(v,\alpha;G)$ be a PPR vector with seed node $v \in V$, and let $\phi$ be any constant in $[0,1]$. Then, either the following bound holds for any integer $t$ and any $k \in [d_{\max},2m - d_{\min}]$:
	\begin{equation*}
	h(k) \leq \alpha t + \frac{2\alpha}{1 + \alpha} + \frac{d(v)}{2d_{\min}^2} + \frac{\sqrt{m}}{d_{\min}^2} \cdot \sqrt{\overline{k}} \left(1 - \frac{\phi^2}{8}\right)^{t}
	\end{equation*}
	or there exists some sweep cut $S_j$ of $p_v$ such that $\Phi(S_j;G) < \phi$.
\end{corollary}

We arrive now at the main result of this section. It is similar in form to Theorem 2 of \citet{andersen2006} but reflects the improvements due to using Corollary~\ref{cor:mixing_time_PPR}. To simplify notation, we will write the total mass placed by $p_v$ on a subset $S \subset V$ as $p_v(S) := \sum_{u \in S} p_v(u)$.
\begin{theorem}
	\label{thm:mixing_time_PPR_contrapositive}
	Let $p_v = p(v,\alpha;G)$ be a PPR vector with seed node $v \in V$. Suppose there exists some $\delta > \frac{2\alpha}{1 + \alpha} + \frac{d_{\max}}{2d_{\min}^2}$, such that
	\begin{equation}
	\label{eqn:mixing_time_PPR_contrapositive_1}
	p_v(S) - \frac{\vol(S;G)}{\vol(G)} > \delta
	\end{equation}
	for a set $S$ with cardinality $\abs{S} \geq \frac{d_{\max}}{d_{\min}}$. Then there exists a sweep cut $S_j$ of $p$, such that
	\begin{equation*}
	\Phi(S_j) < \sqrt{\frac{16\alpha\left\{\log\left(\frac{m}{d_{\min}^2}\right) + \log\left(\frac{2}{\delta'}\right)\right\}}{\delta'}}
	\end{equation*}
	where $\delta' = \delta - \frac{2\alpha}{1 + \alpha} + \frac{d(v)}{2d_{\min}^2}$. 
\end{theorem}
\begin{proof}
	Suppose the assumption of the theorem is satisified, that is there exists a set $S \subset V$ with cardinality $\abs{S} \geq \frac{d_{\max}}{d_{\min}}$ which satisfies \eqref{eqn:mixing_time_PPR_contrapositive_1}. Then for $j = \abs{S}$ the sweep cut $S_j$ has volume at least $d_{\max}$, and by hypothesis $h(\vol(S_j)) >  \delta$.
	
	Now, letting
	\begin{equation*}
	t = \frac{8}{\phi^2}\left\{\log\left(\frac{m}{d_{\min}^2}\right) + \log\left(\frac{2}{\delta'}\right)\right\}, \quad \phi^2 = \frac{16\alpha\set{\log\left(\frac{m}{d_{\min}^2}\right) + \log(\frac{2}{\delta'})}}{\delta'}
	\end{equation*}
	we have that
	\begin{equation*}
	\alpha t + \frac{2\alpha}{1 + \alpha} + \frac{d(v)}{2d_{\min}^2} + \frac{\sqrt{m}}{d_{\min}^2} \cdot \sqrt{\overline{k}} \left(1 - \frac{\phi^2}{8}\right)^{t} \leq \frac{\delta'}{2} + \frac{2\alpha}{1 + \alpha} + \frac{d(v)}{2d_{\min}^2} + \frac{\delta'}{2} < \delta,
	\end{equation*}
	and the Theorem follows by Corollary~\ref{cor:mixing_time_PPR}.
\end{proof}

\subsubsection{Improved Local Partitioning with PPR.}

As in \citet{andersen2006}, the mixing time results of the previous section lead to an upper bound on the normalized cut $\Phi(\Cest;G)$. First, we restate a theorem of \citet{andersen2006} which lower bounds the probability mass $p(v,\alpha;G)(C)$ as a function of the normalized cut $\Phi(C)$. 

\begin{theorem}
	\label{thm:acl_3}
	For any set $C$ and any constant $\alpha$, there exists a subset $C^g \subset C$ with $\vol(C^g;G) \geq \frac{5}{6}\vol(C;G)$, such that for any vertex $v \in C^g$, the PPR vector $p(v,\alpha;G)$ satisfies
	\begin{equation*}
	p(v,\alpha;G) \geq 1 - 6\frac{\Phi(C;G)}{\alpha}.
	\end{equation*}
\end{theorem}
We are now in a position to prove Theorem~\ref{thm:conductance_ppr} by combining Corollary~\ref{cor:mixing_time_PPR} and Theorem~\ref{thm:acl_3}.

\begin{proof}[Proof (of Theorem~\ref{thm:conductance_ppr})]
	Since $\alpha \geq 60\Phi(C)$ and $v \in C^g$, by Theorem~\ref{thm:acl_3},
	\begin{equation*}
	p_v(C) \geq \frac{9}{10}.
	\end{equation*}
	This inequality along with the assumption $\vol(C) \leq \frac{2}{3}\vol(G)$ implies that $p_v(C) - \frac{\vol(C)}{\vol(G)} \geq \frac{1}{5}$. Since we assume $\abs{C} \geq \frac{d_{\max}}{d_{\min}}$, the hypothesis of Theorem~\ref{thm:mixing_time_PPR_contrapositive} is satisfied with $\delta = 1/5$. Therefore, the minimum conductance sweep cut satisfies
	\begin{equation*}
	\min_{j = 1,\ldots,n} \Phi(S_j;G) \leq \sqrt{\frac{1120\cdot \Phi(C;G)\left\{\log\left(\frac{m}{d_{\min}^2}\right) + \log\left(\frac{2}{\delta'}\right)\right\}}{\delta'}}
	\end{equation*}
	Finally, we assume $\frac{20\Phi(C)}{1 + 10\Phi(C)} + \frac{d_{\max}}{2d_{\min}^2} \leq \frac{1}{10}$ which implies that
	\begin{equation*}
	\delta' = \delta - \frac{20\alpha}{1 + 10\alpha} + \frac{d_{\max}}{2d_{\min}^2} \geq \frac{1}{10} 
	\end{equation*}
	completing the proof of the theorem.
\end{proof}

\subsection{Normalized cut of $\mathcal{L}[\Xbf]$.}

Recall that for any set $\mathcal{A} \subset \mathcal{X}$, the $\Pbb$-weighted \emph{cut} and \emph{volume} functionals can be written as
\begin{equation*}
\cut_{\Pbb,r}(\Aset) = \int_{\Aset} \int_{\mathcal{X} \setminus \Aset} \1(\norm{x - y} \leq r) \,d\Pbb(x) \,d\Pbb(y), \vol_{\Pbb,r}(\Aset):= \int_{\Aset} \int_{\mathcal{X}} \1(\norm{x - y} \leq r) \,d\Pbb(x) \,d\Pbb(y),
\end{equation*}
and the continuous \emph{normalized cut} is
\begin{equation*}
\Phi_{\Pbb,r}(\Aset) := \frac{\cut_{\Pbb,r}(\Aset)}{\min\{\vol_{\Pbb,r}(\Aset),\vol_{\Pbb,r}(\mathcal{X} \setminus \Aset)\}}.
\end{equation*}

We now upper bound the normalized cut $\Phi_{\Pbb,r}(\mathcal{L})$ as a function of the diameter $\rho$, and the neighborhood graph radius $r$. Our bounds will be simple and not tight, but will display the right dependence on these parameters, and so will be sufficient for our purposes.

To upper bound $\cut_{\Pbb,r}(\mathcal{L})$, note that for any $x = (x_1,x_2) \in \mathcal{L}$, if $x_2 \leq -r$ the ball $B(x,r)$ and the set $\mathcal{X}\setminus\mathcal{L}$ are disjoint. This implies
\begin{align*}
\cut_{\Pbb,r}(\mathcal{L}) & \leq \Pbb(\set{x \in \mathcal{X}: -r < x_2 < 0}) \cdot \max_{x \in \mathcal{X}} \Pbb(B(x,r)) \\
& \leq \frac{r}{2 \rho} \cdot \frac{\pi r^2}{2 \sigma \rho}.
\end{align*}
By symmetry, $\vol_{\Pbb,r}(\mathcal{L}) = \vol_{\Pbb,r}(\mathcal{X} \setminus \mathcal{L})$, and therefore to upper bound $\Phi_{\Pbb,r}(\mathcal{L})$, it is sufficient to lower bound $\vol_{\Pbb,r}(\mathcal{L})$. We have
\begin{align*}
\vol_{\Pbb,r}(\mathcal{L}) & \geq \Pbb(\set{x \in \Cset_1 \cap \mathcal{L}: \dist(x, \partial \Cset_1) > r}) \cdot \frac{\pi r^2}{2 \sigma \rho} \\
& = \frac{(\sigma - 2r)(\rho - r)}{2 \sigma \rho} \cdot \frac{\nu_d r^d}{2 \sigma \rho}  \\
& \geq \frac{3}{16} \cdot \frac{\pi r^2}{2 \sigma \rho}
\end{align*}
where the last inequality follows since $r \leq \frac{1}{4}\sigma < \frac{1}{4}\rho$. Therefore, $\Phi_{\Pbb,r}(\mathcal{L}) \leq \frac{8r}{3\rho}.$

Then, Lemma~\ref{lem:graph_functional_concentration} implies that the graph functionals $\cut_{n,r}(\mathcal{L}[\Xbf])$ and $\vol_{n,r}(\mathcal{L}[\Xbf])$--and in turn $\Phi_{n,r}(\mathcal{L}[\Xbf])$-- concentrate around their expectations. Precisely, we have that 
\begin{align}
\Phi_{n,r}(\mathcal{L}[\Xbf]) & = \frac{\cut_{n,r}(\mathcal{L}[\Xbf])}{\min\set{\vol_{n,r}(\mathcal{L}[\Xbf]),\vol_{n,r}((\mathcal{X}\setminus\mathcal{L})[\Xbf])}} \nonumber \\
& \leq \frac{3}{2}\Phi_{\Pbb,r}(\Aset) \leq \frac{4 r}{\rho} \label{eqn:lb_pf2}
\end{align}
with probability at least $1 - 3\exp\{-\frac{1}{25}n(\cut_{\Pbb,r}(\mathcal{L}))^2\}$. 

\subsection{Normalized cut of $\Cest$.}

To upper bound $\Phi_{n,r}(\Cest)$, we need to show that the conditions of Theorem~\ref{thm:conductance_ppr} are met with respect to the graph $G = G_{n,r}$ and subset $C = \mathcal{L}[\Xbf]$. To do, we require the following inequalities, which are satisfied with probability at least $1 - 2n\exp\set{-\frac{\pi\epsilon r^2\delta^2n}{8 \rho \sigma(1 + \frac{\delta}{3})}} - 6 \exp\set{-n\delta^2(\cut_{\Pbb,r}(\mathcal{L}[\Xbf]))^2}$: 
\begin{itemize}
	\item For any $r \in (0,\frac{1}{4}\sigma)$ and any $x \in \mathcal{X}$,
	\begin{equation*}
	\frac{\epsilon}{4\rho\sigma}\pi r^2 \leq \Pbb((B(x,r))) \leq  \frac{1}{2\rho\sigma} \pi r^2.
	\end{equation*}
	Therefore by Lemma~\ref{lem:bernstein_union}, $\frac{(1 - \delta)\epsilon \pi r^2}{4\rho \sigma}n \leq d_{\min} \leq d_{\max} \leq \frac{(1 + \delta)\pi r^2}{2\rho \sigma }n$. 
	\item $\frac{(1 - \delta)}{2} n \leq \abs{\mathcal{L}[\Xbf]} \leq \frac{(1 + \delta)}{2} n$, 
	\item $\vol_{n,r}(\mathcal{L}[\Xbf]) \leq (1 + \delta)\vol_{\Pbb,r}(\mathcal{L}) = \frac{(1 + \delta)}{2} \vol_{\Pbb,r}(\mathcal{X}) \leq \frac{(1 + \delta)}{2} \vol(G_{n,r})$, and
	\item $(1 - \delta) \cut_{\Pbb,r}(\mathcal{L}[\Xbf]) \leq \cut_{n,r}(\mathcal{L}[\Xbf]) \leq (1 + \delta) \cut_{\Pbb,r}(\mathcal{L}[\Xbf])$.
\end{itemize}
We now condition on these inequalities, and letting $\delta = \frac{2}{67}$ we verify that under the setup of Theorem~\ref{thm:ppr_lb}, each of the conditions of Theorem~\ref{thm:conductance_ppr} are met:
\begin{itemize}
	\item $\vol(\mathcal{L}[X]) \leq \frac{(1 + \delta)}{2(1 - \delta)}\vol(G_{n,r}) \leq \frac{2}{3}\vol(G_{n,r})$ since $\delta < 1/7$,
	\item $\abs{\mathcal{L}[X]} \geq \frac{n(1 - \delta)}{2} \geq \frac{2(1 + \delta)}{(1 - \delta)\epsilon} \geq \frac{d_{\max}}{d_{\min}}$ and $\frac{d_{\max}}{2d_{\min}^2} \leq \frac{8(1 + \delta)}{(1 - \delta)^2 \epsilon^2 \rho \sigma \pi r^2} \cdot \frac{1}{n} \leq \frac{1}{10}$ by the assumed lower bound on the sample size,
	\item $\Phi_{n,r}(\mathcal{L}[X]) \leq \frac{4r}{\rho} \leq \frac{1}{10}$, by assumption on $r$ and $\rho$, and
	\item $60\Phi_{n,r}(\mathcal{L}[X]) \leq \frac{60(1 + \delta)}{1 - \delta}\Phi_{\Pbb,r}(\mathcal{L}) \leq \alpha \leq \frac{65(1 + \delta)}{1 - \delta}\Phi_{n,r}(\mathcal{L}[X]) \leq 70\Phi_{n,r}(\mathcal{L}[X])$ since $\delta < 2/67$. 
\end{itemize}

We may therefore apply Theorem~\ref{thm:conductance_ppr}, which allow us to upper bound the minimum conductance sweep cut $\min_{\beta \in (0,1)}\Phi(S_{\beta,v};G)$ or equivalently the output of Algorithm~\ref{alg: ppr}.

To be precise, we have that there exists a set $\mathcal{L}[\Xbf]^g \subset \mathcal{L}[\Xbf]$ with $\vol_{n,r}(\mathcal{L}[\Xbf]^g) \geq \frac{5}{6} \vol_{n,r}(\mathcal{L}[\Xbf])$, such that the following statement holds for any $v \in \mathcal{L}[\Xbf]^g$: when Algorithm~\ref{alg: ppr} is run with inputs $\Xbf, r < \frac{1}{4}\sigma,\alpha = 65 \Phi_{\Pbb,r}(\mathcal{L}[\Xbf]),v \in \mathcal{L}[\Xbf]^g$ and $(L,U) = (0,1)$, the resulting PPR cluster estimate $\Cest$ satisfies
\begin{align}
\Phi_{n,r}(\Cest) & \leq \sqrt{11200\left\{\log\left(\frac{m}{d_{\min}^2}\right) + \log 20\right\} \Phi_{n,r}(\mathcal{L[X]})} \nonumber \\
& \leq \sqrt{89600\left\{\log\left(\frac{\rho \sigma}{\epsilon^2 \pi r^2}\right) + \log 20\right\} \frac{r}{\rho}} \label{eqn:lb_pf4} 
\end{align}
with probability at least $1 - 2n\exp\set{-\frac{\pi\epsilon r^2 n}{8978 \rho \sigma}} - 6 \exp\set{-\frac{1}{1123}(\cut_{\Pbb,r}(\mathcal{L}))^2n}$ (where the latter inequality follows from \eqref{eqn:lb_pf2} and Lemma~\ref{lem:graph_functional_concentration}.)

\subsection{Lower bound on normalized cut.}

The precise statement we will prove is contained in the following Lemma.
\begin{lemma}
	\label{lem:normalized_cut_lb}
	The normalized cut $\Phi_{n,r}(A)$ is upper bounded
	\begin{equation*}
	\Phi_{n,r}(A) \geq \frac{1}{12\pi} \left(1 - 4\frac{\sigma \rho}{r^2 n^2} \vol_{n,r}(A \vartriangle \Cset^{(1)}[\Xbf]) \right) \frac{\epsilon^2 r}{\sigma}
	\end{equation*}
	uniformly over all $A \subset X$ with probability at least $1 - \exp\set{-2n\delta^2(\vol_{\Pbb,r}(\mathcal{X}))^2} - \frac{12\sigma \rho}{r^2} \exp\set{-\frac{\delta^2\epsilon r^2 n}{\rho \sigma(3 + \delta)}} - \frac{2\rho}{r}\exp\set{-\frac{\delta^2\pi r^3n}{2\sigma\rho^2(3 + \delta)}}$.
\end{lemma}
\begin{proof}
	To lower bound the normalized cut $\Phi_{n,r}(A)$, we must lower bound $\cut_{n,r}(A)$ and upper bound $\vol_{n,r}(A)$. A naive upper bound on the volume is simply 
	\begin{equation}
	\label{eqn:normalized_cut_lb_pf4}
	\vol_{n,r}(A) \leq \vol_{n,r}(G_{n,r}) \overset{(i)}{\leq} (1 + \delta)  \vol_{\Pbb,r}(\mathcal{X}) n^2 \leq (1 + \delta)\frac{\pi r^2}{\rho \sigma} n^2
	\end{equation}
	where $(i)$ holds with probability at least $1 - \exp\set{-2n\delta^2(\vol_{\Pbb,r}(\mathcal{X}))^2}$, and it turns out this will suffice for our purposes. (Here and in the rest of this proof we take $\delta = 1/2$.)
	
	We turn to lower bounding $\cut_{n,r}(A)$. We will approximate the cut of $A$ by discretizing the space $\mathcal{X}$ into bins, relate the cut of $A$ to the boundary of the binned set $\overline{A}$, and then lower bound the size of the boundary of $\overline{A}$.
	
	Let $(k_1,k_2)$ for $k_1 \in \bigl[\frac{6\sigma}{r}\bigr], k_2 \in \bigl[\frac{2\rho}{r}\bigr]$ be the upper right  corner of the cube
	\begin{equation*}
	Q_{(k_1,k_2)} = \biggl[-\frac{3\sigma}{2} + \frac{(k_1 - 1)}{2}r, -\frac{3\sigma}{2} + \frac{k_1}{2}r\biggr] \times \biggl[-\frac{\rho}{2} + \frac{(k_2 - 1)}{2}r, -\frac{\rho}{2} + \frac{k_2}{2}r\biggr]
	\end{equation*}
	and let $\overline{Q} = \set{Q_{(k_1,k_2)}: k_1 \in \left[\frac{6\sigma}{r}\right], k_2 \in \bigl[\frac{2\rho}{r}\bigr]}$ be the collection of such cubes. For a set $A \subset X$ we define the binned set $\overline{A} \subset \overline{Q}$ as follows
	\begin{equation*}
	\overline{A} := \set{Q \in \overline{Q}: \Pbb_n(A \cap Q) \geq \frac{1}{2}\Pbb_n(Q)},
	\end{equation*}
	and we let 
	\begin{equation*}
	\partial \overline{A} := \set{Q_{(k_1,k_2)} \in \overline{A}: \exists (\ell_1,\ell_2) \in \Bigl[\frac{3\sigma}{r}\Bigr] \times \Bigl[\frac{\rho}{r}\Bigr]~~\textrm{such that}~~Q_{(\ell_1,\ell_2)} \not\in \overline{A}, \norm{k - \ell}_1 = 1}.
	\end{equation*}
	be the boundary set of $\overline{A}$ in $\overline{Q}$. Intuitively, every point $x_i \in A$ in the boundary set of $\overline{A}$ will have many edges to $X\setminus A$. Formally, letting $Q_{\min} := \min_{Q \in \overline{Q}} \Pbb_n(Q)$, we have
	\begin{equation}
	\label{eqn:normalized_cut_lb_pf1}
	\cut_{n,r}(A) \geq \cut_{n,r}(A \cap \set{x_i \in \overline{A}}) \geq \frac{1}{4} \abs{\partial \overline{A}} Q_{\min}^2,
	\end{equation}
	where the last inequality follows since for every cube $Q_k \in \partial\overline{A}$, there exists a cube $Q_\ell \not\in \overline{A}$ such that $\norm{i - j}_1 \leq 1$, and since each cube has side length $r/2$, this implies that for every $x_i \in Q_k$ and $x_j \in Q_\ell$ the edge $(x_i,x_j)$ belongs to $G_{n,r}$. 
	
	Now we move on lower bounding the size of the boundary $\abs{\partial\overline{A}}$. To do so, we divide $\mathcal{X}$ into slices horizontally. Let $R_k = \set{(x_1,x_2) \in \mathcal{X}: x_2 \in \bigl[-\frac{\rho}{2} + \frac{(k - 1)}{2}r, -\frac{\rho}{2} + \frac{k}{2}r\bigr]}$ be the $k$th horizontal slice, and $\overline{R}_k = \set{Q_{(k_1,k)} \in \overline{Q}:k_1 \in [\frac{6\sigma}{r}]}$ be the binned version of $R_k$. For each $k$, either
	\begin{enumerate}
		\item $\overline{R}_k \cap \overline{A} = \emptyset$, in which case
		\begin{equation*}
		\vol_{n,r}\Bigl( \bigl(A \vartriangle C_1[\Xbf]\bigr) \cap R_k \Bigr) \geq \frac{1}{2}\vol_{n,r}(C_1[\Xbf] \cap R_k), ~~ \textrm{or}
		\end{equation*}
		\item $\overline{R}_k \cap \overline{A} = \overline{R}_k$, in which case
		\begin{equation*}
		\vol_{n,r}\Bigl( \bigl(A \vartriangle C_1[\Xbf]\bigr) \cap R_k \Bigr) \geq \frac{1}{2}\vol_{n,r}(C_2[\Xbf] \cap R_k), ~~\textrm{or}
		\end{equation*}
		\item $\overline{R}_k \cap \partial \overline{A} \neq \emptyset$.
	\end{enumerate}
	Let $N(R)$ be the number of slices for which $\overline{R}_k \cap \partial \overline{A} \neq \emptyset$. By the cases elucidated above, letting
	\begin{equation*} R_{\min} := \min_{k}\Bigl\{\vol_{n,r}(C_1[\Xbf] \cap R_k) \wedge \vol_{n,r}(C_2[\Xbf] \cap R_k)\Bigr\}
	\end{equation*} 
	we obtain the following lower bound on the volume of the symmetric set difference,
	\begin{equation}
	\label{eqn:normalized_cut_lb_pf2}
	\vol_{n,r}(A \vartriangle C_1[\Xbf]) \geq \frac{1}{2} R_{\min}  \left[\frac{2\rho}{r} - N(R)\right].
	\end{equation}
	Finally note that $\abs{\partial\overline{A}} \geq N(R)$. Therefore combining \eqref{eqn:normalized_cut_lb_pf1} and \eqref{eqn:normalized_cut_lb_pf2}, we have that
	\begin{align}
	\cut_{n,r}(A) & \geq \frac{1}{4}N(R) Q_{\min}^2 \nonumber \\
	& \geq \frac{1}{2}\left(\frac{\rho}{r} - \frac{\vol_{n,r}(A \vartriangle C_1[\Xbf])}{R_{\min}}\right) Q_{\min}^2 \label{eqn:normalized_cut_lb_pf3}
	\end{align}
	for all $A \subset \Xbf$.
	
	It remains to lower bound the random quantities $R_{\min}$ and $Q_{\min}$.
	To do so, we first lower bound the expected probability of any cell $Q$,
	\begin{align*}
	\min_{Q \in \overline{Q}} \Pbb(Q) \geq \frac{\epsilon r^2}{\rho \sigma}.
	\end{align*}
	and the expected volume of $\Cset^{(1)}[\Xbf] \cap R_k$ and $\Cset^{(2)}[\Xbf] \cap R_k$,
	\begin{equation}
	\vol_{\Pbb,r}(\Cset^{(1)} \cap R_k) = \vol_{\Pbb,r}(\Cset^{(2)} \cap R_k) \geq \frac{\pi r^3}{2 \sigma \rho^2} \label{eqn:normalized_cut_lb_pf5}
	\end{equation}
	
	Since $Q_{\min}$ and $R_{\min}$ are obtained by taking the minimum of functionals over a fixed number of sets in $n$, they concentrate tightly around their means. Specifically, note that the total number of cubes is $\abs{\overline{Q}} = \frac{12 \sigma \rho}{r^2}$, and the total number of horizontal slices is $\frac{2\rho}{r}$. Along with \eqref{eqn:normalized_cut_lb_pf3} and \eqref{eqn:normalized_cut_lb_pf5}, by Lemma~\ref{lem:bernstein_union} 
	\begin{equation*}
	Q_{\min} \geq (1 - \delta)\frac{\epsilon r^2}{\rho \sigma} ~~\textrm{and}~~ R_{\min} \geq \frac{(1 - \delta)}{2}\frac{\pi r^3}{\sigma \rho^2 },
	\end{equation*}
	with probability at least $1 - \frac{12\sigma \rho}{r^2} \exp\set{-\frac{\delta^2\epsilon r^2 n}{\rho \sigma(3 + \delta)}} - \frac{2\rho}{r}\exp\set{-\frac{\delta^2\pi^2 r^6n}{4\sigma^2\rho^4}}$. Combining these lower bounds with \eqref{eqn:normalized_cut_lb_pf4} and \eqref{eqn:normalized_cut_lb_pf3}, we obtain
	\begin{equation*}	
	\Phi_{n,r}(A) \geq \frac{(1 - \delta)^2}{2(1 + \delta)\pi} \left(1 - 2 \frac{\sigma \rho}{(1 - \delta) r^2 n^2} \vol_{n,r}(A \vartriangle \Cset^{(1)}[\Xbf]) \right) \frac{\epsilon^2 r}{\sigma},
	\end{equation*}
	and plugging in $\delta = 1/2$ yields the claim of Lemma~\ref{lem:normalized_cut_lb}.
\end{proof}

\paragraph{Conclusion.}
Combining \eqref{eqn:lb_pf4} and Lemma~\ref{lem:normalized_cut_lb}, we have that there exists a set $\mathcal{L}[\Xbf]^g \subset \mathcal{L}[\Xbf]$ with $\vol_{n,r}(\mathcal{L}[\Xbf]^g) \geq \frac{5}{6}\vol_{n,r}(\mathcal{L}[\Xbf])$ such that for any seed node $v \in \mathcal{L}[\Xbf]^g$, the following bounds hold:
\begin{equation*}
\frac{1}{12\pi} \left(1 - 4\frac{\sigma \rho}{r^2 n^2} \vol_{n,r}(\Cest \vartriangle \Cset^{(1)}[\Xbf]) \right) \frac{\epsilon^2 r}{\sigma} \leq \Phi_{n,r}(\Cest) \leq \sqrt{89600\left\{\log\left(\frac{\rho \sigma}{\epsilon^2 \pi r^2}\right) + \log 20\right\} \frac{r}{\rho}},
\end{equation*}
with probability at least $1 - b_2/n$ for an appropriate choice of constant $b_2$. Solving for $\vol_{n,r}(\Cest \vartriangle \Cset^{(1)}[\Xbf])$ in the previous equation, we obtain \eqref{eqn:ppr_lb} (for an appropriate choice of constant $c$).
Finally, we show that the volume of $\mathcal{L}[\Xbf]^g$ is sufficiently large to ensure that it includes many points in $\Cset^{(1)}[\Xbf]$:
\begin{align*}
\vol_{n,r}(\mathcal{L}[\Xbf]^g \cap \Cset^{(1)}[\Xbf]) & \geq  \vol_{n,r}(\mathcal{L}[\Xbf]^g) - \vol_{n,r}((\mathcal{L}[\Xbf]^g \cap (\Cset^{(0)} \cup \Cset^{(2)})[\Xbf])[\Xbf]) \\
& \geq \frac{5}{6}\vol_{n,r}(\mathcal{L}[\Xbf]) - \vol_{n,r}((\mathcal{L} \cap (\Cset^{(0)} \cup \Cset^{(1)})[\Xbf])[\Xbf]) \\
& \geq \frac{5}{6}\vol_{n,r}((\mathcal{L} \cap \Cset^{(1)})[\Xbf]) - \frac{1}{6}\vol_{n,r}(\mathcal{L}[\Xbf]) \\
& \geq \left(\frac{5}{6}(1 - \delta) - \frac{1}{2}(1 + \delta)\right)\vol_{\Pbb,r}( \mathcal{L} \cap \Cset^{(1)}) \\
& \geq \frac{(1 - \delta)}{2}\left(\frac{5}{6}(1 - \delta) - \frac{1}{2}(1 + \delta)\right)\vol_{n,r}(\Cset^{(1)}[\Xbf])
\end{align*}
where the final two inequalities follow from Lemma~\ref{lem:graph_functional_concentration} and hold with probability at least $1 - 3\exp\set{-n\delta^2(\vol_{\Pbb,r}( \mathcal{L} \cap \Cset^{(1)}))^2}$. Setting $\delta = 1/13$, we have that $\vol_{n,r}(\mathcal{L}[\Xbf]^g \cap \Cset^{(1)}[\Xbf]) \geq \frac{1}{10}\vol_{n,r}(\Cset^{(1)}[\Xbf])$.

\section{Bounding the misclassification rate.}
\label{sec: proof_of_misclassification_rate}

A common loss function for clustering is the misclassification rate.
\begin{definition}[Misclassification rate.]
	\label{def: symmetric_set_diff}
	For an estimator \smash{$\Cest \subseteq \Xbf$} and set
	$\mathcal{S} \subseteq \Reals^d$, we define   
	\begin{equation}
	\label{eqn: misclassification_rate}
	\Delta_{\textrm{mc}}(\Cest, \mathcal{S}) := \abs{\Cset \vartriangle \mathcal{S}[\Xbf]},
	\end{equation}
	the cardinality of the symmetric set difference between 
	\smash{$\Cest$} and $\mathcal{S} \cap \Xbf = \mathcal{S}[\Xbf]$. 
\end{definition}

In order to upper bound the misclassification rate of a PPR cluster estimate, we will need to slightly modify our approach to computing sweep cuts, and no longer normalize by degree; we formally define this modified algorithm in Algorithm~\ref{alg:ppr2}. Intuitively, this change helps us avoid including many low-degree vertices $w \not\in \Csig[\Xbf]$ in our estimated cluster.
\begin{algorithm}
	\caption{Unnormalized PPR on a neighborhood graph}
	\label{alg:ppr2}	
	{\bfseries Input:} data $\Xbf=\{x_1,\ldots,x_n\}$, radius $r > 0$, teleportation
	parameter $\alpha \in [0,1]$, seed $v \in \Xbf$, target stationary probability
	$\pi_0 > 0$, range $(L,U)$. \\     
	{\bfseries Output:} cluster $\Cest \subseteq V$.
	\begin{algorithmic}[1]
		\STATE Form the neighborhood graph $G_{n,r}$.
		\STATE Compute the PPR vector $p_v=\pbf(v, \alpha; G_{n,r})$ as in \eqref{eqn: 
			ppr_vector}. 
		\STATE For \smash{$\beta \in (\frac{1}{40}, \frac{1}{11})$} compute sweep cuts 
		$S_{\beta}$
		\begin{equation}
		\label{eqn:sweep_cuts2}
		S_\beta := \set{u \in V: p_v(u) > \beta \pi_0}.
		\end{equation}
		\STATE Return as a cluster \smash{$\Cest_{\textrm{un}} = S_{\beta^*}$}, where  
		$$
		\beta^* = \argmin_{\beta \in (L,U)} \Phi(S_{\beta}; G_{n,r}).
		$$
	\end{algorithmic}
\end{algorithm}

We say Algorithm \ref{alg:ppr2} is well-initialized if $r,\alpha$ and $v$ satisfy~\eqref{eqn: initialization}, and additionally
\begin{equation}
\label{eqn:initialization_2}
\pi_0 = \frac{\lambda_{\sigma}}{\Lambda_{\sigma}\Pbb(\Csig)n}, ~~\textrm{and}~~ (L,U) \in \left(\frac{1}{100},\frac{1}{50}\right).
\end{equation}

\begin{theorem}
	\label{thm:misclassification_rate}
	Fix $\lambda > 0$ let \smash{$\Cset \in \Cbb_f(\lambda)$} be a
	$\kappa$-well-conditioned density cluster, and assume If Algorithm
	\ref{alg:ppr2} is well-initialized with respect to $\Cset$. Then for any
	\begin{equation*}
	n \geq b_1(\log n)^{\max\{\frac{3}{d},1\}}
	\end{equation*}
	there exists a set $\Csig[\Xbf]^g \subseteq \Csig[\Xbf]$ of large volume, $\vol_{n,r}(\Csig[\Xbf]^g) \geq \vol_{n,r}(\Csig[\Xbf])/2$, such that the following holds: if Algorithm~\ref{alg:ppr2} is run with any seed node $v \in \Csig[\Xbf]^g$, then the PPR estimated cluster $\Cest$ satisfies
	\begin{equation}
	\label{eqn: misclassification_rate_ub}
	\Delta(\Csig[\Xbf], \Cest) \leq c \kappa(\Cset)
	\frac{\Lambda_{\sigma}}{\lambda_{\sigma}}, 
	\end{equation}
	with probability at least $1 - \frac{b_2}{n}$.
\end{theorem}

\subsection{Proof of Theorem~\ref{thm:misclassification_rate}.}
The proof of Theorem~\ref{thm:misclassification_rate} follows from Corollary 3.3 of \citet{zhu2013}.

\paragraph{Corollary 3.3 of Zhu.}
Let $G = (V,E)$ be an undirected, unweighted graph, let $p_v := p(v,\alpha;G)$ be a PPR vector with seed node $v \in V$ and teleportation parameter $\alpha \in (0,1)$. 
\begin{lemma}[Corollary 3.3 of \citet{zhu2013}.]
	\label{cor:zhu}
	Let $A \subseteq G$, and suppose $\alpha \leq \frac{1}{9\tau_{\infty}(G[A])}$. Then, there exists a set $A^g \subseteq A$ with $\vol(A^g;G) \geq \frac{1}{2}\vol(A;G)$ such that the following statement holds: for any $v \in A$; the PPR vector $p_v$ satisfies
	\begin{equation}
	\label{eqn:zhu_1}
	p_v(V \setminus A) \leq 2\frac{\Phi(A;G)}{\alpha}
	\end{equation}
	and additionally there exists a residual vector $p_{\ell} \in [0,1]^V$ with $\norm{p_{\ell}}_1 \leq \frac{2\Phi(A;G)}{\alpha}$ such that
	\begin{equation}
	\label{eqn:zhu_2}
	\textrm{for all $u \in A$}, \quad p_v(u) \geq \frac{4}{5} \frac{\deg(u;G[A])}{\vol(A;G)} - p_{\ell}(u).
	\end{equation}
\end{lemma}

\paragraph{Upper bound on $\abs{S_{\beta} \vartriangle A}$.}

For given $\pi_0$ and $\beta \in (0,1)$, consider the sweep cut
\begin{equation*}
S_{\beta,v} := \set{u \in V: p_v(u) \geq \beta \pi_0}.
\end{equation*}
Suppose the conditions $\alpha \leq \frac{1}{9\tau_{\infty}(G[A])}$ and $v \in A^g$ are met. Then by~\eqref{eqn:zhu_1},
\begin{equation}
\label{eqn:zhu_3}
\abs{S_{\beta,v} \setminus A} \leq \frac{p_v(V \setminus A)}{\min_{u \in S_{\beta}}p_v(u)} \leq \frac{2 \Phi(A;G)}{\alpha \beta \pi_0}. 
\end{equation}
To upper bound $\abs{A \setminus S_{\beta}}$, note that for every $u \in A \setminus S_{\beta}$, $p_v(u) \leq \beta \pi_0$. Therefore by~\eqref{eqn:zhu_2},
\begin{equation*}
p_{\ell}(u) \geq \frac{4\deg(u;G[A])}{5\vol(A;G)} - \beta\pi_0, \textrm{for all $u \in A \setminus S_{\beta}$}
\end{equation*}
Since additionally $\norm{p_{\ell}}_1 \leq 2\Phi(S;G)/\alpha$, we have
\begin{align}
\sum_{u \in A \setminus S_{\beta,v}} \left(\frac{4\deg(u;G[A])}{5\vol(A;G)} - \beta\pi_0\right) & \leq \frac{2\Phi(S;G)}{\alpha} \nonumber \Longrightarrow\\
\abs{A \setminus S_{\beta,v}}\underbrace{\left(\frac{4\min_{u \in A}\deg(u;G[A])}{\abs{A}5\max_{u \in A}\deg(u;G)} - \beta\pi_0\right)}_{:=T_1(A;G)} & \leq \frac{2\Phi(S;G)}{\alpha} \Longrightarrow \nonumber \\
\abs{A \setminus S_{\beta,v}} \leq \frac{2\Phi(S;G)}{\alpha T_1(A;G)} .
\label{eqn:zhu_4}
\end{align}

\paragraph{Bounds on graph functionals for $G_{n,r}$.}

To apply~\eqref{eqn:zhu_3} and~\eqref{eqn:zhu_4} when $G = G_{n,r}$ and $A = \Csig[\Xbf]$, we need to verify the condition $\alpha \leq \frac{1}{9\tau_{\infty}(\widetilde{G}_{n,r})}$, and additionally upper bound $\Phi_{n,r}(\Csig[\Xbf])$ and $T_1(G_{n,r}; \Csig[\Xbf])$. We now state the necessary upper bounds, and the probability with which they hold:
\begin{itemize}
	\item Since $\alpha < \frac{1}{9\tau_u(\Csig)}$, by Theorem~\ref{thm: mixing_time_upper_bound} 
	\begin{equation*}
	\alpha \leq \frac{1}{9\tau_{\infty}(\widetilde{G}_{n,r})},
	\end{equation*}
	with probability at least $1 - \frac{b_2}{n} - 2n\exp\set{-b_3n} - 2\exp\set{-b_4n}$. 
	\item  By Theorem~\ref{thm: conductance_upper_bound}, 
	\begin{equation*}
	\Phi_{n,r}(\Csig[\Xbf]) \leq \Phibf(\Csig) 
	\end{equation*}
	with probability at least $1 - 3\exp\{-nb\}$. 
	\item By Lemma~\ref{lem:ball_bounds_in_probability},
	\begin{equation*}
	\frac{\min_{u \in \Csig[\Xbf]}\widetilde{\deg}_{n,r}(u)}{\max_{u \in \Csig[\Xbf]}\deg_{n,r}(u)}= \frac{\degminwt}{\degmax} \geq \frac{(1 - \delta)6\lambda_{\sigma}}{(1+\delta)25\Lambda_{\sigma}},
	\end{equation*}
	and
	\begin{equation*}
	\abs{\Csig[\Xbf]} \leq (1 + \delta)n\Pbb(\Csig[\Xbf])
	\end{equation*}
	all with probability at least $1 - 2n\exp\set{-\frac{2 \delta^2 \lambda_{\sigma} \nu_d r^d n}{75(1 + \frac{\delta}{3})}} - 2\exp\set{-2\delta^2 \Pbb(\Csig)^2}$. These bounds along with the initialization conditions~\eqref{eqn:initialization_2} imply
	\begin{align}
	T_1(G_{n,r}, \Csig[\Xbf]) & \geq \frac{\lambda_{\sigma}}{\Lambda_{\sigma}\abs{\Csig[\Xbf]}}\left(\frac{(1 - \delta)6}{(1+\delta)25} - (1 + \delta)\beta\right) \nonumber \\
	& \geq \frac{\lambda_{\sigma}}{\Lambda_{\sigma}\abs{\Csig[\Xbf]}}\left(\frac{(1 - \delta)6}{(1+\delta)25} - \frac{(1 + \delta)}{50}\right). \label{eqn:misclassification_rate_pf1}
	\end{align}
\end{itemize}
We assume these high probability bounds are satisfied with $\delta = \frac{1}{2}$ and move on to upper bounding $\Delta_{\textrm{mc}}(\Cest_{\textrm{un}}, \Csig[\Xbf])$. 

\paragraph{Upper bound on misclassification rate.}
Since $\alpha \leq \frac{1}{9\tau_{\infty}(\widetilde{G}_{n,r})}$, we may apply Lemma~\ref{cor:zhu} to the graph $G_{n,r}$ and subset $\Csig[\Xbf]$. Combined with the inequalities we've already derived, this implies the following: there exists a subset $\Csig[\Xbf]^g \subset \Csig[\Xbf]$ with $\vol_{n,r}(\Csig[\Xbf]) \leq \frac{1}{2}\vol_{n,r}(\Csig[\Xbf])$ such that the following bounds hold,
\begin{equation*}
\abs{S_{\beta,v} \setminus \Csig[\Xbf]} \leq \frac{2\Phi_{n,r}(\Csig[\Xbf])}{\alpha \beta \pi_0} \leq \frac{2000 \kappa(\Cset)}{\pi_0} \leq \frac{4000 \kappa(\Cset) \Lambda_{\sigma}}{\lambda_{\sigma}}\abs{\Csig[\Xbf]},
\end{equation*}
and
\begin{equation*}
\abs{\Csig[\Xbf] \setminus S_{\beta,v}} \leq \frac{2\Phi_{n,r}(\Csig[\Xbf])}{\alpha T_1(\Csig[\Xbf];G_{n,r})} \leq \frac{1000 \kappa(\Cset) \Lambda_{\sigma}}{\lambda_{\sigma}}\abs{\Csig[\Xbf]}.
\end{equation*}
This completes the proof of Theorem~\ref{thm:misclassification_rate}.

\section{Probabilistic bounds}
\label{sec: concentration}

In our theory, we frequently appeal to concentration of graph functionals such as degree, volume, and cut size around their means. In this section, we establish such concentration inequalities. We begin with Lemma~\ref{lem:hoeffding_2}, Hoeffding's inequality, which we will use to bound the empirical probability of a fixed set.
\begin{lemma}[Hoeffding's Inequality.]
	\label{lem:hoeffding_2}
	Let $\mathcal{A}$ be a subset of $\Reals^d$. Then,
	\begin{equation*}
	(1 - \delta) \Pbb(\mathcal{A}) \leq \Pbb_n(\mathcal{A}) \leq (1 + \delta)\Pbb(\mathcal{A})
	\end{equation*}
	with probability at least $1 - \exp\set{-2\delta^2(\Pbb(\mathcal{A}))^2}$. 
\end{lemma}

To bound minimum and maximum degrees, we will use Lemma~\ref{lem:bernstein_union}, which is a combination of Bernstein's inequality and a union bound.

\begin{lemma}[Bernstein's inequality + Union bound.]
	\label{lem:bernstein_union}
	For $M \geq 1$, let $\mathcal{A}_1,\ldots,\mathcal{A}_M$ be subsets of $\Reals^d$. Denote the minimum probability mass among these sets as $p_{\min} := \min_{m = 1,\ldots,M} \Pbb(\mathcal{A}_m)$, and likewise let $p_{\max} := \max_{m = 1,\ldots,M} \Pbb(\mathcal{A}_m)$. Then
	\begin{equation*}
	(1 - \delta)p_{\min} \leq \min_{m = 1,\ldots,M} \Pbb_n(\mathcal{A}_m) \leq \max_{m = 1,\ldots,M} \Pbb_n(\mathcal{A}_m) \leq (1 + \delta)p_{\max}
	\end{equation*}
	with probability at least $1 - 2 M \exp\left\{-\frac{\frac{1}{3}\delta^2p_{\min}n}{1 + \frac{\delta}{3}}\right\}$.  
\end{lemma}
The above Lemma will allow us to upper and lower bound the minimum and maximum degrees within the neighborhood graphs $G_{n,r}$ and $\widetilde{G}_{n,r}$. To bound cut size and volume functionals, we will use Hoeffding's inequality for U-statistics. Recall that $U_n$ is an order-2 U-statistic with kernel $h: \Rd \times \Rd \to \Reals$ if 
\begin{equation*}
U_n = \frac{1}{2{n \choose 2}}\sum_{i = 1}^{n} \sum_{j \neq i} h(x_i,x_j).
\end{equation*}
\begin{lemma}[Hoeffding's Inequality for U-statistics.]
	\label{lem:hoeffding}
	Assume $\norm{h}_{\infty} \leq 1$. Then,
	\begin{equation*}
	(1 - \delta) \mathbb{E}(U_m) \leq U_n \leq (1 + \delta) \mathbb{E}(U_m).
	\end{equation*}
	with probability at least $1 - 2 \exp\left(-\delta^2 (\mathbb{E}(U_m))^2 n\right)$.
\end{lemma}

We collect the bounds on graph functionals we need in Lemma~\ref{lem:ball_bounds_in_probability}. Let $S \subset X$ and $\mathcal{S} \subseteq \Rd$. As a reminder, for notational ease we write $\widetilde{G}_{n,r} = G_{n,r}[\Csig[\Xbf]]$ and
\begin{alignat*}{2}
\widetilde{d}_{\min} & = \min_{u \in \Csig[\Xbf]} \widetilde{\deg}_{n,r}(u), \quad && d_{\min}' = \min_{u' \in \Cset'[\Xbf]} {\deg}_{n,r}(u') \\
d_{\max} & = \max_{u \in \Csig[\Xbf]} \deg_{n,r}(u), \quad && \widetilde{\vol}_{n,r}(S) = \vol(S; \widetilde{G}_{n,r}). 
\end{alignat*}
Additionally, let $\widetilde{\vol}_{\Pbb,r}(\mathcal{S}) := \frac{1}{2{n \choose 2}}\mathbb{E}[\widetilde{\vol}_{n,r}(\mathcal{S}[\Xbf])]$.
\begin{lemma}
	\label{lem:ball_bounds_in_probability} \label{lem:graph_functional_concentration}
	\item[]
	The following statements hold for any $\delta \in (0,1)$:
	
	\noindent With probability at least $1 - 2\exp\{-n\delta^2(\widetilde{\vol}_{\Pbb,r}(\mathcal{S}))^2\}$,
	\begin{equation}
	\label{eqn:volwt_bound}
	(1 - \delta) \widetilde{\vol}_{\Pbb,r}(\mathcal{S}) \leq \frac{\widetilde{\vol}_{n,r}(\mathcal{S}[\Xbf])}{2{n \choose 2}} \leq (1 + \delta) \widetilde{\vol}_{\Pbb,r}(\mathcal{S}).
	\end{equation}
	With probability at least $1 - 2\exp\set{-n \delta^2 \vol_{\Pbb,r}(\Sset)^2}$,
	\begin{equation}
	\label{eqn:vol_bound}
	(1 - \delta)\vol_{\Pbb,r}(\Sset) \leq \frac{\vol_{n,r}(\Sset[\Xbf])}{2{n \choose 2}}\leq 2(1 + \delta)\cut_{\Pbb,r}(\Sset).
	\end{equation}
	With probability at least $1 - 2\exp\set{-n \delta^2 \cut_{\Pbb,r}(\Sset)^2}$,
	\begin{equation}
	\label{eqn:cut_bound}
	(1 - \delta)\cut_{\Pbb,r}(\Sset) \leq \frac{\cut_{n,r}(\Sset[\Xbf])}{2{n \choose 2}}\leq (1 + \delta)\vol_{\Pbb,r}(\Sset).
	\end{equation}
	With probability at least $1 - 2\exp\{-n\delta^2\Pbb(\Csig)^2\}$, 
	\begin{equation}
	\label{eqn:cardinality}
	(1 - \delta) \Pbb(\Csig) \leq \abs{\Csig[\Xbf]} \leq (1 + \delta) \Pbb(\Csig).
	\end{equation}
	
	\noindent The following statements hold for any $0 < r \leq \frac{\sigma}{2d}$:
	
	\noindent Assuming $\Csig$ satisfies~\ref{asmp: bounded_density}, with probability at least $1 - 2n\exp\set{-\frac{ \delta^2 \lambda_{\sigma} \nu_d r^d n}{6(1 + \frac{\delta}{3})}}$,
	\begin{equation}
	\label{eqn:degwt_bound}
	\frac{1}{2}\biggl(1 - \frac{r}{\sigma}\sqrt{\frac{d + 2}{2 \pi}}\biggr)(1 - \delta)\lambda_{\sigma}\nu_dr^dn \leq \degminwt \leq d_{\max} \leq (1 + \delta) \Lambda_{\sigma} \nu_dr^dn.
	\end{equation}
	Assuming $\Cset'$ satisfies~\ref{asmp: C'_bounded_density}, with probability at least $1 - n\exp\set{-\frac{\delta^2 \lambda_{\sigma} \nu_d r^d n}{3(1 + \frac{\delta}{3})}}$,
	\begin{equation}
	\label{eqn:degpr_bound}
	\degminpr \geq (1 - \delta)\lambda_{\sigma}\nu_dr^dn.
	\end{equation}
\end{lemma}
\begin{proof}{(of Lemma~\ref{lem:ball_bounds_in_probability})}
	\\
	\\
	\hfill 
	\textit{Proof of \eqref{eqn:degwt_bound}, \eqref{eqn:degpr_bound}:}
	Under~\ref{asmp: bounded_density}, we have that for every $x_j \in \Csig[\Xbf]$
	\begin{equation*}
	\mathbb{E}(\widetilde{\deg}_{n,r}(x_j)) \leq \Lambda_{\sigma} \nu_d r^d;
	\end{equation*}
	furthermore, recalling the weighted local conductance $\ell_{\Pbb,r}(x) = \Pbb(B(x,r) \cap \Csig)$, we have,
	\begin{equation*}
	\mathbb{E}(\widetilde{\deg}_{n,r}(x_j) | x_j) \geq n \cdot \ell_{\Pbb,r}(x_j) \geq n \lambda_{\sigma} \cdot \biggl\{\min_{x \in \mc{C}_{\sigma}} \ell_{\nu,r}(x)\biggr\} \geq \frac{n}{2}\biggl(1 - \frac{r}{\sigma}\sqrt{\frac{d + 2}{2 \pi}}\biggr)\lambda_{\sigma}\nu_dr^d 
	\end{equation*}
	where the last inequality follows from Lemma~\ref{lem: overlap_balls}.  Under~\ref{asmp: C'_bounded_density} we have for every $x_i \in \Cset'[\Xbf]$
	\begin{equation*}
	\mathbb{E}({\deg}_{n,r}(x_i)|x_i) \geq n \lambda_{
		\sigma} \nu_d r^d.
	\end{equation*} 
	Then \eqref{eqn:degwt_bound} and \eqref{eqn:degpr_bound} each follow from Lemma~\ref{lem:bernstein_union}.
	\\
	\\
	\noindent \textit{Proof of \eqref{eqn:volwt_bound}, \eqref{eqn:vol_bound}, \eqref{eqn:cut_bound}:}
	We have that $\frac{1}{2{n \choose 2}}\widetilde{\vol}_{n,r}(\Csig[\Xbf])$, $\frac{1}{2{n \choose 2}}\vol_{n,r}(\Sset[\Xbf])$, $\frac{1}{2{n \choose 2}}\cut_{n,r}(\Sset[\Xbf])$ are each order-2 U statistics, with respective kernels
	\begin{align*}
	h_{\widetilde{\vol}}(x_i,x_j) & = \1(x_i \in \Csig)\1(x_j \in \Csig)\1(\norm{x_i - x_j} \leq r), \\ h_{\cut}(x_i,x_j) & = \1(x_i \in \Csig)\1(x_j \notin \Csig)\1(\norm{x_i - x_j} \leq r) \\
	h_{\vol}(x_i,x_j) & = \1(x_i \in \Csig)\1(\norm{x_i - x_j} \leq r).
	\end{align*}
	We may therefore apply Lemma~\ref{lem:hoeffding} in each case to obtain the stated bounds.
	\\
	\\
	\noindent
	\textit{Proof of \eqref{eqn:cardinality}:}
	Apply Lemma~\ref{lem:hoeffding_2} to $\Csig$.
\end{proof}

\section{Experiments}
\label{sec: experimental_setting}

In Section \ref{sec: experimental_setting}, we detail the experimental settings of Section \ref{sec: experiments} in the main text, and include an additional figure. 

\subsection{Experimental settings for Figure \ref{fig:bounds}}
We sample points according to the density function $q$, where for $x \in \Rd$
\begin{equation}
q(x) :=
\begin{cases}
\lambda,~ & x \in [0,\sigma] \times \rho^{d-1} =: \Cset, \\
\lambda - \dist(x,\Cset)\eta,~ & x \in \Csig \setminus \Cset, \\
(\lambda - \sigma \eta) - \dist(x,\Csig)^{\gamma}, & x \in (\Csig + \sigma B) \setminus \Csig, \\
0,~ & \textrm{otherwise},
\end{cases}
\end{equation}
where $\lambda = \frac{150}{81} \sigma^{\gamma}$ and $\eta = \frac{15}{81} \sigma^{\gamma - 1}$.

In the top-left and top-middle, we show draws of $n = 20000$ samples from two different density functions. In the top-left panel, $\sigma = \rho = 3.2$, while in the top-middle panel $\sigma = .1$ and $\rho = 3.2$. (For both, $d = 2$).

The remaining four panels (top-right and the bottom row) in Figure \ref{fig:bounds} show the change in normalized cut and mixing time, respectively, as the parameters $\sigma$ (top-right and bottom-left) and $\rho$ (bottom-middle and bottom-right) are varied. In the top-right and bottom-left panels $\sigma = .1 \cdot \sqrt{2}^j, j = 1,\ldots,10$, and $\rho$ is fixed at $3.2$. In the bottom-middle and bottom-right panels, $\rho = .1 \cdot \sqrt{2}^j, j = 1,\ldots,10$ and $\sigma$ is fixed at $.1$.
For each panel, the solid lines show, up to constants, the theoretical upper bound, given by Theorem \ref{thm: conductance_upper_bound} for the top-right and bottom-left panels and Theorem \ref{thm: mixing_time_upper_bound} for the bottom-middle and bottom-right panels. The dashed lines show the computed empirical value, averaged over $m$ trials ($m = 100$ for the normalized cut, dashed lines in the top-right and bottom-left panels, and $m = 20$ for the mixing time, dashed lines in the bottom-middle and bottom-right panels). For each trial across all parameters, $r$, the neighborhood graph radius, is set throughout to be as small as possible such that the resulting graph is connected, for computational efficiency. Green lines correspond to dimension $d = 2$, whereas purple/pink lines correspond to $d = 3$. 

\subsection{Experimental settings for Figure \ref{fig:moons}}

To form each of the three rows in Figure \ref{fig:moons}, 800 points are independently sampled following a 'two moons plus Gaussian noise model'. Formally, the (respective) generative models for the data are
\begin{align}
Z & \sim \textrm{Bern}(1/2), \theta \sim \textrm{Unif}(0, \pi) \\
X(Z,\theta) & = 
\begin{cases}
\mu_1 + (r \cos(\theta), r \sin(\theta)) + \sigma \epsilon,~ & \text{if}~ Z = 1 \\
\mu_2 + (r \cos(\theta), - r \sin(\theta)) + \sigma \epsilon,~ & \text{if}~ Z = 0
\end{cases}
\end{align}

where 
\begin{align*}
\mu_1 & = (-.5, 0),~ \mu_2 = (0,0),~ \epsilon \sim N(0, I_2) \tag{row 1} \\
\mu_1 & = (-.5, -.07),~ \mu_2 = (0,.07),~ \epsilon \sim N(0, I_2) \tag{row 2} \\
\mu_1 & = (-.5, -.125),~ \mu_2 = (0,.125),~ \epsilon \sim N(0, I_2) \tag{row 3} 
\end{align*}
for $I_d$ the $d \times d$ identity matrix. The first column consists of the empirical density clusters $C[\Xbf]$ and $C'[\Xbf]$ for a particular threshold $\lambda$ of the density function; the second column shows the ~\pprspace plus minimum normalized sweep cut cluster, with hyperparameter $\alpha$ and all sweep cuts considered; the third column shows the global minimum normalized cut, computed according to the algorithm of \cite{szlam2010}; and the last column shows a cut of the density cluster tree estimator of \cite{chaudhuri2010}.

\subsection{Performance of \pprspace with high-dimensional noise.}

Figure \ref{fig:moons_hd} is similar to Figure \ref{fig:moons} of the main text, with parameters
\begin{equation*}
\mu_1 = (-.5, -.025),~ \mu_2 = (0,.025),~ \epsilon \sim N(0, I_{10}).
\end{equation*}
The gray dots in $(a)$ (as in the left-hand column of Figure \ref{fig:moons} in the main text) represent observations in low-density regions. While the \pprspace sweep cut $(b)$ has relatively high symmetric set difference with the chosen density cut, it still recovers $C[\Xbf]$ in the sense of Definition \ref{def: consistent_density_cluster_estimation}.

% PUT FIGURE BACK HERE

\clearpage

% \bibliographystyle{plainnat}
\bibliography{../../local_spectral_bibliography} 

\end{document}