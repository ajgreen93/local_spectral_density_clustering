\documentclass{article}

% \usepackage{icml2019}
% \usepackage[accepted]{icml2019}

\renewcommand{\thesection}{\Alph{section}}
\renewcommand{\theequation}{A.\arabic{equation}}

\usepackage{microtype}
\usepackage{graphicx}
\usepackage{subfigure}
\usepackage{booktabs}
%\usepackage{hyperref}
\usepackage{amsmath}
\usepackage{amsfonts, amsthm, amssymb}
\usepackage{graphicx}
\usepackage[parfill]{parskip}
\usepackage{enumerate}
\usepackage[shortlabels]{enumitem}
\usepackage{xr-hyper}
\usepackage{bm}
\usepackage[colorlinks=true,citecolor=blue,urlcolor=blue,linkcolor=blue]{hyperref}
\usepackage{natbib}

\externaldocument{icml_local_density_clustering}

\newcommand{\diam}{\mathrm{diam}}
\newcommand{\set}[1]{\left\{#1\right\}}
\newcommand{\defeq}{\overset{\mathrm{def}}{=}}
\newcommand{\vol}{\mathrm{vol}}
\newcommand{\abs}[1]{\left \lvert #1 \right \rvert}
\newcommand{\N}{\mathbb{N}}
\newcommand{\Reals}{\mathbb{R}}
\newcommand{\Rd}{\Reals^d}
\newcommand{\norm}[1]{\left\lVert#1\right\rVert}
\newcommand{\1}{\mathbf{1}}
\newcommand{\var}{\mathrm{Var}}
\newcommand{\Err}{\mathrm{Err}}
\newcommand{\Log}{\mathrm{Log}}
\newcommand{\TV}{\mathrm{TV}}
\newcommand{\dist}{\mathrm{dist}}
\newcommand{\Id}{\mathrm{Id}}

%%% Graph terms
\newcommand{\cut}{\mathrm{cut}}

%%% Vectors
\newcommand{\pbf}{\mathbf{p}}
\newcommand{\qbf}{\mathbf{q}}
\newcommand{\ebf}[1]{\mathbf{e}_{#1}}
\newcommand{\pibf}{\bm{\pi}}
\newcommand{\rhobf}{\bm{\rho}}
\newcommand{\Deltabf}{\bm{\Delta}}
\newcommand{\deltabf}{\bm{\delta}}
\newcommand{\zbf}{\mathbf{z}}

%%% Random walk vectors


%%% Matrices
\newcommand{\Abf}{\mathbf{A}}
\newcommand{\Xbf}{\mathbf{X}}
\newcommand{\Wbf}{\mathbf{W}}
\newcommand{\Lbf}{\mathbf{L}}
\newcommand{\Dbf}{\mathbf{D}}
\newcommand{\Ibf}[1]{\mathbf{I}_{#1}}

%%% Probability distributions (and related items)
\newcommand{\Pbb}{\mathbb{P}}
\newcommand{\Qbb}{\mathbb{Q}}
\newcommand{\Cbb}{\mathbb{C}}
\newcommand{\Ebb}{\mathbb{E}}

%%% Sets
\newcommand{\Sset}{\mathcal{S}}
\newcommand{\Cset}{\mathcal{C}}
\newcommand{\Aset}{\mathcal{A}}
\newcommand{\Asig}{\Aset_{\sigma}}
\newcommand{\Csig}{\Cset_{\sigma}}
\newcommand{\Asigr}{\Aset_{\sigma,\sigma + r}}
\newcommand{\Csigr}{\Cset_{\sigma,\sigma + r}}

%%% Operators
\DeclareMathOperator*{\argmin}{arg\,min}


%%% Algorithm notation
\newcommand{\ppr}{{\sc PPR}}
\newcommand{\pprspace}{{\sc PPR~}}

%%% Tilde notation for quantities over the expansion set 
\newcommand{\wn}{\widetilde{n}}
\newcommand{\wX}{\widetilde{\Xbf}}
\newcommand{\wx}{\widetilde{x}}
\newcommand{\wz}{\widetilde{z}}
\newcommand{\wbz}{\widetilde{\bf{z}}}
\newcommand{\wu}{\widetilde{u}}
\newcommand{\wPbb}{\widetilde{\Pbb}}
\newcommand{\wf}{\widetilde{f}}
\newcommand{\wDbf}{\widetilde{\Dbf}}


\newtheoremstyle{aldenthm}
{6pt} % Space above
{6pt} % Space below
{\itshape} % Body font
{} % Indent amount
{\bfseries} % Theorem head font
{.} % Punctuation after theorem head
{.5em} % Space after theorem head
{} % Theorem head spec (can be left empty, meaning `normal')

\theoremstyle{aldenthm}
\newtheorem{lemma}{Lemma}
\newtheorem{theorem}{Theorem}
\newtheorem{definition}{Definition}


%\newcommand{\theHalgorithm}{\arabic{algorithm}}


%\icmltitlerunning{Local clustering of density upper level sets}

\begin{document}
	
Let $\pbf = (p_u)_{u \in \Xbf}$ denote the \pprspace vector computed over $G_{n,r}$ (where for ease of reading we suppress dependence on the hyperparameter $\alpha$ and seed node $v$.) 
Recalling that $\widetilde{\pi}_{n,r}$ is the stationary distribution over $\widetilde{G}_{n,r}$, we write $\widetilde{\pi}_{n,r}(u)$ to denote the stationary distribution evaluated at $u \in \Csig[X]$. 
	
\begin{lemma} 
	\label{lem: setup}
	Consider running Algorithm \ref{alg: ppr} with any $r < \sigma$ and
	
	\begin{equation} 
	\label{eqn: upper_bound_alpha}
	\frac{\Psi_{n,r}(\Csig[\Xbf])}{10} \leq \alpha \leq \frac{\Psi_{n,r}(\Csig[\Xbf])}{9}.
	\end{equation}
	
	There exists a good set $\Csig[\Xbf]^g \subseteq \Csig[\Xbf]$ with $\vol(\Csig[\Xbf]^g) \geq \vol(\Csig[\Xbf])/2$ such that the following statements hold for all $v \in \Csig[\Xbf]^g$:
	\begin{itemize}
		\item For all $u \in \Cset[\Xbf]$,
		\begin{equation}
		\label{eqn: lower_bound_PPR_in_cluster}
		p_u \geq \frac{4}{5} \widetilde{\pi}_{n,r}(u) - \frac{2 \Phi_{n,r}(\Csig[\Xbf])}{\Psi_{n,r}(\Csig[\Xbf]) \widetilde{D}_{\min}}
		\end{equation}
		\item For all $u' \in \Csig'[\Xbf]$,
		\begin{equation}
		\label{eqn: upper_bound_PPR_in_other_cluster}
		p_{u'} \leq \frac{2 \Phi_{n,r}(\Csig[\Xbf])}{\Psi_{n,r}(\Csig[\Xbf]) \widetilde{D}_{\min}}
		\end{equation}
	\end{itemize}
\end{lemma}

\begin{proof}
	We will write $\Wbf_n = \Dbf_n \Abf_n^{-1}$ for the transition probability matrix over $G_{n,r}$, and let $\widetilde{\Dbf}_n$ and $\widetilde{\Wbf}_n$ be the degree and random walk matrices for the subgraph $\widetilde{G}_{n,r}$.
	
	We introduce \emph{leakage} and \emph{soakage} vectors, defined by
	\begin{align*}
	\ell_t & := e_v (\Wbf_n \widetilde{\mathbf{I}}_n )^t (\mathbf{I}_n - \Dbf_n^{-1} \wDbf_{n}) \\
	\ell & := \sum_{t = 0}^{\infty} (1 - \alpha)^t \ell_t \\
	s_t & := e_v (\Wbf_n \widetilde{\mathbf{I}}_n )^t (\Wbf_n \widetilde{\mathbf{I}}_n^c) \\
	s & := \sum_{t = 0}^{\infty} (1 - \alpha)^{t} s_t
	\end{align*}
	where $\mathbf{I}_n$ is the $n \times n$ identity matrix, $\widetilde{\mathbf{I}}_n$ is a diagonal matrix with $(\widetilde{\mathbf{I}}_n)_{uu} = 1$ if $u \in \Csig[\Xbf]$ and $\widetilde{\mathbf{I}}_n^c = \mathbf{I}_n - \widetilde{\mathbf{I}}_n$. 
	
	Roughly, the proof will unfold in four steps. The first two will result in the lower bound of (\ref{eqn: lower_bound_PPR_in_cluster}), while the latter two will imply the upper bound in (\ref{eqn: upper_bound_PPR_in_other_cluster}).
	
	\begin{enumerate}
		\item For $u \in \Cset'[\Xbf]$, use the results of \cite{zhu2013} to produce the lower bound 
		\begin{equation*}
		\pbf(u) \geq 4/5 \widetilde{\pi}_{n,r}(u) - \widetilde{\pbf}_{\ell}(u)
		\end{equation*}
		where 
		\begin{equation*}
		\widetilde{\pbf}_{\ell} = \alpha \ell + (1 - \alpha) \widetilde{\pbf}_{\ell} \widetilde{\Wbf}_n
		\end{equation*}
		is the \pprspace random walk over $\widetilde{G}_{n,r}$, and $\ell$ has bounded norm $||\ell||_1 \leq 2\frac{\Phi_{n,r}(\Csig[\Xbf])}{\alpha}$.
		
		\item Since $r < \sigma$, for any $u \in \Cset[\Xbf]$ there are no edges between $u$ and $G_{n,r} / \Csig[\Xbf]$. Therefore, the page-rank vector $\widetilde{\pbf}_{\ell}$ will not assign more than $||\ell||_1 / d_{\min}(\Csig[\Xbf])$ probability mass to any vertex in $\Cset'[\Xbf]$. This observation will conclude our proof of (\ref{eqn: lower_bound_PPR_in_cluster}).
		\item For vertices $u' \in G_{n,r} / \Csig[\Xbf]$, we can upper bound $p_v(u) \leq p_s(u')$. In particular, this hold for all $u' \in \Cset'[\Xbf]$.
		\item Since $r < \sigma$, there are no edges between $u'$ and $G / \Cset'[\Xbf]$. Therefore, the page-rank vector $p_{s}$ will assign no more than $||s||_1 / d_{\min}(\Csig[\Xbf])$ probability mass to any vertex in $\Cset'[\Xbf]$. Additionally, $s$ has bounded norm $||s||_1 \leq ||\ell||_1$. This will conclude our proof of (\ref{eqn: upper_bound_PPR_in_other_cluster}), and hence Lemma \ref{lem: setup}.
	\end{enumerate}
	
	\paragraph{Step 1}
	We will begin by restating some results of \cite{zhu2013}.
	
	For seed node $v$, we write
	\begin{align} \label{eqn: page_rank_body}
	\widetilde{\pbf}_v & = \alpha e_v + (1 - \alpha) \widetilde{\pbf}_v \widetilde{\Wbf}_n \\
	& = \alpha \sum_{t = 0}^{\infty} (1 - \alpha)^t \left(e_v \widetilde{\Wbf}_n^t \right)
	\end{align}
	
	From Lemma 3.1 of \cite{zhu2013} we have for all $u \in \Csig[\Xbf]$
	
	\begin{align} \label{eqn: zhu_body}
	p_u & \geq \widetilde{\pbf}_v(u) - \widetilde{\pbf}_{\ell}(u) \nonumber \\
	||\ell||_1 & \leq \frac{2 \widetilde{\Phi}_{n,r}}{\alpha}
	\end{align}
	where $\widetilde{\pbf}_v = (\widetilde{\pbf}_v(u))$ and likewise for $\widetilde{\pbf}_{\ell} = (\widetilde{\pbf}_{\ell}(u))$.
	
	Moreover if, as we have specified, $\alpha \leq \widetilde{\Psi}_{n,r}/9$, Lemma 3.2 of \cite{zhu2013} yields a lower bound on $\widetilde{p}$
	\begin{equation} \label{eqn: page_rank_mixes}
	\widetilde{\pbf}_v(u) \geq \frac{4}{5} \widetilde{\pi}_{n,r}(u).
	\end{equation}
	
	\paragraph{Step 2}
	
	We turn to upper bounding $\widetilde{\pbf}_{\ell}(u)$. For any $u \in \Cset[\Xbf]$, we have
	
	\begin{align} \label{eqn: leakage_page_rank_body}
	\widetilde{\pbf}_{\ell}(u) & = \alpha \sum_{t = 0}^{\infty} (1 - \alpha)^t \left(\ell \widetilde{\Wbf}_n^t \right)(u)  \nonumber \\
	& = \|\ell\|_1 \alpha \sum_{t = 0}^{\infty} (1 - \alpha)^t \left(\frac{\ell}{\|\ell\|_1}  \widetilde{\Wbf}_n^t \right)(u)\nonumber \\
	& \overset{\text{(i)}}{=} \|\ell\|_1 \alpha \sum_{t = 1}^{\infty} (1 - \alpha)^t \left(\frac{\ell}{\|\ell\|_1}  \widetilde{\Wbf}_n^t \right)(u)\nonumber \\
	& \overset{\text{(ii)}}{\leq} \|\ell\|_1 \frac{1}{\widetilde{D}_{\min}} 
	\end{align}
	
	where we use $\left(\ell \widetilde{\Wbf}_n^t \right)(u)$ to denote $\ell \widetilde{\Wbf}_n^te_u$.
	
	$\text{(i)}$ follows from the fact that since $r < \sigma$, $\cut(\Cset'[\Xbf], G_{n,r} / \Csig[\Xbf]; G_{n,r}) = 0$. Therefore $(\Dbf_n^{-1})_{uu} (\widetilde{\Dbf}_n)_{uu} = 1$, and as a result
	\begin{equation*}
	(\ell \widetilde{\Wbf}_n^0)(u) = \ell(u) = 0.
	\end{equation*}
	To see $\text{(ii)}$, let $q = \frac{\ell}{\|\ell\|_1}  \widetilde{\Wbf}_n^{t-1}$. Then 
	
	\begin{align*}
	\left(\frac{\ell}{\|\ell\|_1}  \widetilde{\Wbf}_n^t \right)(u) & = \left(q \widetilde{\Wbf}_n \right)(u) \\
	& \leq \|q\|_1 \|\widetilde{\Wbf}_{\cdot u}\|_{\infty} \\
	& \overset{\text{(iii)}}{\leq} \frac{1}{\widetilde{D}_{\min}}.
	\end{align*}
	where $\widetilde{\Wbf}_{\cdot u}$ is the $u$th column of $\widetilde{\Wbf}_n$. $\text{(iii)}$ then follows from the fact that any vertex in $\Cset[\Xbf]$ is connected only to vertices in $\Csig[\Xbf]$, and therefore every entry of $\widetilde{\Wbf}_{\cdot u}$ is either $0$ or at most $1 / \widetilde{D}_{\min}$.
	
	
	Combined, (\ref{eqn: leakage_page_rank_body}), (\ref{eqn: page_rank_mixes}), and \eqref{eqn: zhu_body} imply
	\begin{equation*}
	p_v(u) \geq \frac{4}{5} \widetilde{\pi}_{n,r}(u) - 18\frac{ \widetilde{\Phi}_{n,r}}{\widetilde{D}_{\min} \alpha}.
	\end{equation*}
	
	\paragraph{Step 3}
	To get the corresponding upper bound on $p_v(u')$, we will use the soakage vectors $s$ and $s_t$. We will first argue that $s$ is a worse starting distribution -- meaning it puts uniformly more mass outside the cluster -- than simply starting at $v$.
	
	\begin{lemma} \label{lem: gained mass is soaked_body}
		For all $u' \notin \Csig[\Xbf]$,
		\begin{equation}
		\pbf_v(u') \leq \pbf_{s}(u').
		\end{equation}
	\end{lemma}
	
	\begin{proof}
		
	We have
	\begin{align*}
	\pbf_v(u') & = \alpha \sum_{T=0}^{\infty} (1 - \alpha)^T (e_v \Wbf_n^T)(u)\\
	& \overset{(i)}{=} \alpha \sum_{T=1}^{\infty} (1 - \alpha)^T (e_v \Wbf_n^T)(u')
	\end{align*}
	
	where $(i)$ follows from $v \in \Csig$ , $u \not\in \Csig$ and therefore $e_v(u) = 0$. 
	
	Lemma \ref{lem: sum_of_soakages} allows us to make the transition to sums of soakage vectors. 
	\begin{lemma}
		\label{lem: sum_of_soakages}
		
		Let $G = (V,E)$ be a graph, with associated random walk matrix $W$.
		
		For any $T \geq 1$, $q$ vector, $S \subset V$, and $s_t = s_t(S^c,q)$
		\begin{equation}
		qW^T = \sum_{t = 0}^{T - 1} s_t W^{T - t - 1} + q(W I_S)^T
		\end{equation}
	\end{lemma}
	We prove Lemma \ref{lem: sum_of_soakages} after completing the proof of Lemma \ref{lem: gained mass is soaked_body}.
	
	Now, along with the fact $u \not\in \Csig$, we have
	
	\begin{equation*}
	\left(e_v \Wbf_n^T \right)(u') = \sum_{t = 0}^{T - 1} \left(s_t \Wbf_n^{T - t - 1} \right)(u')
	\end{equation*}
	
	and so
	
	\begin{align*}
	\pbf_v(u) & = \alpha \sum_{T=1}^{\infty} (1 - \alpha)^T \left( \sum_{t = 0}^{T - 1} s_t \Wbf^{T - t - 1} \right)(u') \\
	& = \alpha \sum_{t=0}^{\infty} \sum_{T = t + 1}^{\infty} (1 - \alpha)^T \left( s_t \Wbf^{T - t - 1} \right)(u')\\
	& = \alpha \sum_{t=0}^{\infty} \sum_{\Delta = 0}^{\infty} (1 - \alpha)^{\Delta + t + 1} \left( s_t \Wbf_n^{\Delta} \right)(u') \\
	& \leq \alpha \sum_{t=0}^{\infty} \sum_{\Delta = 0}^{\infty} (1 - \alpha)^{\Delta + t } \left( s_t \Wbf_n^{\Delta} \right)(u') \\
	& = \alpha \sum_{\Delta = 0}^{\infty} (1 - \alpha)^{\Delta} \left(s \Wbf_n^{\Delta}\right)(u') \\
	& = \pbf_s(u')
	\end{align*}
	\end{proof}
	
	\begin{proof}[Proof of Lemma \ref{lem: sum_of_soakages}]
	Proceed by induction. When $T = 1$,
	
	\begin{align*}
	qW & = q(WI_S) + q(WI_{S^c}) \\
	& = q(W I_S)^T + s_0 
	\end{align*}
	
	Assume true for $T_0$. For $T = T_0 + 1$,
	
	\begin{align*}
	qW^T & = qW^{T_0}W \\
	& = \left\{ \sum_{t = 0}^{T_0 - 1} s_t W^{T_0 - 1 - t} + q(WI_S)^{T_0} \right\} W \\
	& =  \sum_{t = 0}^{T_0 - 1} s_t W^{T - 1 - t} + q(WI_S)^{T_0}(WI_S + WI_{S^c}) \\
	& =  \sum_{t = 0}^{T - 1} s_t W^{T - 1 - t} + q(WI_S)^{T}
	\end{align*}
	\end{proof}
	
	\paragraph{Step 4}
	
	Just as we upper bounded the probability mass $\widetilde{\pbf}_{\ell}$ could assign to any one vertex, we can upper bound 
	
	\begin{align} \label{eqn: soakage_page_rank_body}
	\pbf_{s}(u') & = \alpha \sum_{t = 0}^{\infty} (1 - \alpha)^t \left(s \Wbf_n^t \right)(u') \nonumber \\
	& = \|s\|_1 \alpha \sum_{t = 0}^{\infty} (1 - \alpha)^t \left(\frac{s}{\|s\|_1} {\Wbf}_n^t \right)(u')\nonumber \\
	& = \|s\|_1 \alpha \sum_{t = 1}^{\infty} (1 - \alpha)^t \left(\frac{s}{\|s\|_1}  {\Wbf}_n^t \right)(u')\nonumber \\
	& \leq \|s\|_1 \frac{1}{\widetilde{D}_{\min}}.
	\end{align}
	
	Finally, letting $q_t = e_v (\Wbf_n \widetilde{\mathbf{I}}_n)^t$ for ease of notation, and writi we have
	\begin{align*}
	\|s_t\|_1 & = \|q_t (\Wbf_n \widetilde{\mathbf{I}}_n)\|_1 \\
	& = \sum_{u' \in \Xbf} \sum_{u \in \Xbf} q_t(u) (\Wbf_n \widetilde{\mathbf{I}}_n)(u, u')\\
	& = \sum_{u' \in \Xbf / \Csig[\Xbf]} \sum_{u \in \Csig[\Xbf]} \frac{q_t(u)}{(\Dbf_n)_{uu}} \1(e_{u,u'} \in G_{n,r}) \\
	& = \sum_{u \in \Csig[\Xbf]} \frac{q(u) \left((\Dbf_n)_{uu} - (\widetilde{\Dbf}_n)_{uu} \right)}{(\Dbf_n)_{uu}} \\
	& = \|q_t (I - \Dbf_n^{-1} \widetilde{\Dbf}_n)\|_1 = \|\ell_t\|_1.
	\end{align*}
	
	and as a result $\|s\|_1 = \|\ell\|_1$. Combining with $\|\ell\|_1 \leq 2 \frac{\widetilde{\Phi}_{n,r}}{\alpha}$ and (\ref{eqn: soakage_page_rank_body}) yields the desired upper bound.
	
\end{proof}

\begin{lemma}
	\label{lem: ball_bounds_in_probability}
	Let $\Csig$ satisfy the conditions of Theorem \ref{thm: consistent_recovery_of_density_clusters}. For $r < \sigma$, the following statements hold with probability tending to one as $n \to \infty$:
	\begin{align*}
	D_{\min}(\Csig[X]; \widetilde{G}_{n,r}) & \geq \frac{1}{2} \nu_d r^d \lambda_{\sigma} \\
	D_{\max}(\Csig[X]; \widetilde{G}_{n,r}) & \leq 2 \nu_d r^d \Lambda_{\sigma} \\
	\widetilde{\vol}_{n,r}(\widetilde{G}_{n,r}) & \leq 2 \nu(\Csig) \Lambda_{\sigma}
	\end{align*}
	 where $D_{\min}(\Csig[X]; \widetilde{G}_{n,r})$ is the minimum degree of any vertex $v \in \Csig[X]$ in the subgraph $\widetilde{G}_{n,r}$, and analogously for $D_{\max}(\Csig[X]; \widetilde{G}_{n,r})$.
\end{lemma}

The statement follows immediately from Lemma \ref{lem: ball_bounds}.

\subsection{Proof of Theorem \ref{thm: consistent_recovery_of_density_clusters}}

We note that by Theorems \ref{thm: conductance_upper_bound} and \ref{thm: inverse_mixing_time_lower_bound_nonconvex}, 
\begin{equation*}
\kappa_2(\Cset) \geq \frac{\Phi_{n,r}(\Csig[\Xbf])}{\Psi_{n,r}(\Csig[\Xbf])}.
\end{equation*}
As a result Lemma \ref{lem: setup} implies
\begin{align}
\label{eqn: theorem_4_1}
p_u & \geq \frac{4}{5} \widetilde{\pi}_{n,r}(u) - \frac{18 \kappa_2(\Cset)}{\widetilde{D}_{\min}} ~~~~~~ (u \in \Cset[\Xbf]) \nonumber \\
p_{u'} & \leq \frac{18 \kappa_2(\Cset)}{\widetilde{D}_{\min}} ~~~~~~~~~~~~~~~~~~~~~ (u' \in \Cset'[\Xbf])
\end{align}

We then have
\begin{align*}
\widetilde{\pi}_{n,r}(u) & \geq \frac{D_{\min}(\Csig[X]; \widetilde{G}_{n,r})}{\widetilde{\vol}_{n,r}(\widetilde{G}_{n,r})} \\
& \geq \frac{D_{\min}(\Csig[X]; \widetilde{G}_{n,r})}{\wn \widetilde{D}_{\max}}
\end{align*}
and application of Lemma \ref{lem: ball_bounds_in_probability} yields
\begin{equation}
\label{eqn: theorem_4_2}
\widetilde{\pi}_{n,r}(u) \geq 8 \frac{\lambda_{\sigma}}{\nu(\Csig) \Lambda_{\sigma}^2}
\end{equation}
and
\begin{equation}
\label{eqn: theorem_4_3}
\frac{1}{\widetilde{D}_{\min}} \leq \frac{2}{\nu_d r^d \lambda_{\sigma}}
\end{equation}
with probability tending to $1$ as $n \to \infty$, for all $u \in \Cset[\Xbf]$.

Combining \eqref{eqn: theorem_4_1}, \eqref{eqn: theorem_4_2} and \eqref{eqn: theorem_4_3}, along with the requirement on $\kappa_2(\Cset)$ given by \eqref{eqn: kappa2_ub}, we have
\begin{align*}
p_u & \geq 3/5 \frac{\lambda_{\sigma}}{\nu(\Csig) \Lambda_{\sigma}^2} \\
p_{u'} & \leq 1/5 \frac{\lambda_{\sigma}}{\nu(\Csig) \Lambda_{\sigma}^2}
\end{align*}
for any $u \in \Cset$, $u' \in \Cset'$. As a result, if $\pi_0 \in (2/5, 3/5)\cdot \frac{\lambda_{\sigma}}{\nu(\Csig) \Lambda_{\sigma}^2}$, as $n \to \infty$ with probability tending to one any sweep cut of the form of \eqref{eqn: sweep_cuts}, including the output set $\widehat{C}$, will successfully recover $\Cset$ in the sense of \eqref{eqn: consistent_density_cluster_recovery}.


\bibliography{icml_bib}
\bibliographystyle{plain}
	
\end{document}