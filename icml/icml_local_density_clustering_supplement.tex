%%%%%%%% ICML 2019 submission %%%%%%%%%%%%%%%%%

\documentclass{article}

% \usepackage{icml2019}
% \usepackage[accepted]{icml2019}

\renewcommand{\thesection}{\Alph{section}}
\renewcommand{\theequation}{A.\arabic{equation}}

\usepackage{microtype}
\usepackage{graphicx}
\usepackage{subfigure}
\usepackage{booktabs}
%\usepackage{hyperref}
\usepackage{amsmath}
\usepackage{amsfonts, amsthm, amssymb}
\usepackage{graphicx}
\usepackage[parfill]{parskip}
\usepackage{enumerate}
\usepackage[shortlabels]{enumitem}
\usepackage{xr-hyper}
\usepackage{bm}
\usepackage[colorlinks=true,citecolor=blue,urlcolor=blue,linkcolor=blue]{hyperref}

\externaldocument{icml_local_density_clustering}

\newcommand{\diam}{\mathrm{diam}}
\newcommand{\set}[1]{\left\{#1\right\}}
\newcommand{\defeq}{\overset{\mathrm{def}}{=}}
\newcommand{\vol}{\mathrm{vol}}
\newcommand{\abs}[1]{\left \lvert #1 \right \rvert}
\newcommand{\N}{\mathbb{N}}
\newcommand{\Reals}{\mathbb{R}}
\newcommand{\Rd}{\Reals^d}
\newcommand{\norm}[1]{\left\lVert#1\right\rVert}
\newcommand{\1}{\mathbf{1}}
\newcommand{\var}{\mathrm{Var}}
\newcommand{\Err}{\mathrm{Err}}
\newcommand{\Log}{\mathrm{Log}}

%%% Graph terms
\newcommand{\cut}{\mathrm{cut}}

%%% Vectors
\newcommand{\pbf}{\mathbf{p}}
\newcommand{\qbf}{\mathbf{q}}
\newcommand{\ebf}[1]{\mathbf{e}_{#1}}
\newcommand{\pibf}{\bm{\pi}}
\newcommand{\rhobf}{\bm{\rho}}
\newcommand{\Deltabf}{\bm{\Delta}}
\newcommand{\deltabf}{\bm{\delta}}

%%% Random walk vectors


%%% Matrices
\newcommand{\Abf}{\mathbf{A}}
\newcommand{\Xbf}{\mathbf{X}}
\newcommand{\Wbf}{\mathbf{W}}
\newcommand{\Lbf}{\mathbf{L}}
\newcommand{\Dbf}{\mathbf{D}}
\newcommand{\Ibf}[1]{\mathbf{I}_{#1}}

%%% Probability distributions (and related items)
\newcommand{\Pbb}{\mathbb{P}}
\newcommand{\Qbb}{\mathbb{Q}}
\newcommand{\Cbb}{\mathbb{C}}
\newcommand{\Ebb}{\mathbb{E}}

%%% Sets
\newcommand{\Sset}{\mathcal{S}}
\newcommand{\Cset}{\mathcal{C}}
\newcommand{\Aset}{\mathcal{A}}
\newcommand{\Asig}{\Aset_{\sigma}}
\newcommand{\Csig}{\Cset_{\sigma}}
\newcommand{\Asigr}{\Aset_{\sigma,\sigma + r}}
\newcommand{\Csigr}{\Cset_{\sigma,\sigma + r}}

%%% Operators
\DeclareMathOperator*{\argmin}{arg\,min}


%%% Algorithm notation
\newcommand{\ppr}{{\sc PPR}}
\newcommand{\pprspace}{{\sc PPR~}}


\newtheoremstyle{aldenthm}
{6pt} % Space above
{6pt} % Space below
{\itshape} % Body font
{} % Indent amount
{\bfseries} % Theorem head font
{.} % Punctuation after theorem head
{.5em} % Space after theorem head
{} % Theorem head spec (can be left empty, meaning `normal')

\theoremstyle{aldenthm}
\newtheorem{lemma}{Lemma}
\newtheorem{theorem}{Theorem}
\newtheorem{definition}{Definition}


%\newcommand{\theHalgorithm}{\arabic{algorithm}}


%\icmltitlerunning{Local clustering of density upper level sets}

\begin{document}

%\twocolumn[
%\icmltitle{Supplement to ``Local clustering of density upper level sets''}

%\icmlsetsymbol{equal}{*}

%\begin{icmlauthorlist}
%\icmlauthor{Alden Green}{cmu}
%\icmlauthor{Sivaraman Balakrishnan}{cmu}
%\icmlauthor{Ryan Tibshirani}{cmu}
%\end{icmlauthorlist}

%\icmlaffiliation{cmu}{Department of Statistics and Data Science, Carnegie Mellon University, Pittsburgh PA, USA}

%\icmlcorrespondingauthor{Alden Green}{ajgreen@andrew.cmu.edu}

%\icmlkeywords{local clustering}

%\vskip 0.3in
%]

%\printAffiliationsAndNotice{} % otherwise use the standard text.

\section{Proofs}

In this supplement, we present proofs for ``Local Clustering of Density Upper Level Sets''. We begin by providing technical lemmas, before moving on to proving the main results of the paper. 

Throughout, we will fix $\Aset \subset \Rd$ to be an arbitrary set. To simplify expressions, for the $\sigma$-expansion $\Asig$, we will write the set difference between $\Asig$ and the $(\sigma + r)$-expansion $\Aset_{\sigma + r}$ as 
\begin{equation*}
\Asigr := \set{x: 0 < \rho(x, \Asig) \leq r},
\end{equation*}
where $\rho(x, \Aset) = \min_{x' \in \Aset} \norm{x - x'}$.

For notational ease, we write
\begin{align*}
\cut_{n,r} = \cut(\Csig[\Xbf]; G_{n,r}), ~ \mu_K = \mathbb{E}(\cut_{n,r}), ~ p_K = \frac{\mu_K}{{n \choose 2}} \\
\vol_{n,r} = \vol(\Csig[\Xbf]; G_{n,r}), ~ \mu_V = \mathbb{E}(\vol_{n,r}), ~ p_V = \frac{\mu_V}{{n \choose 2}}
\end{align*}
for the random variable, mean, and probability of cut size and volume, respectively.

\subsection{Technical Lemmas}

We state Lemma \ref{lem: expansion_sets} without proof, as it is trivial. We formally include it mainly to comment on its (potential) suboptimality; for sets $\Aset$ with diameter much larger than $\sigma$, the volume estimate of Lemma \ref{lem: expansion_sets} will be quite poor. 

\begin{lemma}
	\label{lem: expansion_sets}
	For any $\sigma > 0$ and the $\sigma$-expansion $\Asig = \Aset + \sigma B$, 
	\begin{equation*}
	\sigma B \subset \Asig, ~~\mathrm{and~ }\nu(\Aset + \sigma B) \leq \nu((1 + \sigma)\Aset) = (1 + \sigma)^d \nu(\Aset).
	\end{equation*}
\end{lemma}

We will need to carefully control the volume of the expansion set using the above estimate; Lemma \ref{lem: Taylor_series} serves this purpose.
\begin{lemma}
	\label{lem: Taylor_series}
	For any $0 \leq x \leq 1/2d$,
	\begin{equation*}
	(1 + x)^d \leq 1 + 2dx.
	\end{equation*}
\end{lemma}
The proof of Lemma \ref{lem: Taylor_series} is based on approximation via Taylor series, and we omit it.

We will repeatedly employ Lemma \ref{lem: expansion_sets} and Lemma \ref{lem: Taylor_series} in tandem. As a first example, in Lemma \ref{lem: interior_of_expansion_sets}, we use it to bound the ratio of $\nu(\Asig)$ to $\nu(\Aset_{\sigma - r})$. This will be useful when we bound $\vol(\Csig)$.

\begin{lemma}
	\label{lem: interior_of_expansion_sets}
	For $\sigma$, $\Asig$ as in Lemma \ref{lem: expansion_sets}, let $r > 0$ satisfy $r \leq \sigma/4d$. Then,
	\begin{equation*}
	\frac{\nu(\Asig)}{\nu(\Aset_{\sigma - r})} \leq 2.
	\end{equation*}
\end{lemma}
\begin{proof}
	Fix $q = \sigma - r$. Then,
	\begin{align*}
	\nu(\Asig) & = \nu(\Aset_{q + \sigma - q}) = \nu(\Aset_q + (\sigma - q)B ) \\
	& \leq \nu(\Aset_q + \frac{(\sigma - q)}{q} \Aset_q) = \left(1 + \frac{\sigma - q}{q}\right)^d \nu(\Aset_q)
	\end{align*}
	where the inequality follows from Lemma \ref{lem: expansion_sets}. Of course, $\sigma - q = r$, and $\frac{r}{q} \leq \frac{1}{2d}$ for $r \leq \frac{1}{4d}$. The claim then follows from Lemma \ref{lem: Taylor_series}.
\end{proof}

The proof of Theorem \ref{thm: inverse_mixing_time_lower_bound} also depends on a parameter -- which we term \emph{discrete local spread} -- to handle the mixing over very small steps. Formally, the discrete local spread $\pi_1(G)$ is given by
\begin{equation}
\label{eqn: local_spread}
\pi_1(G) := \frac{d_{\min}(G)^2}{10 \vol(V; G)} 
\end{equation}
where $d_{\min}(G) = \min_{v \in V} d(v)$ is the minimum degree in $G$. Intuitively, the discrete local spread gauges how much the walk given by $\Wbf$ has mixed after one step, starting from any node $v$. We will denote $\pi_1(G_{n,r}[\Csig[\Xbf]])$ by $\widetilde{\pi}_{1,n}$. 

\textcolor{red}{Prove Lemma \ref{lem: local_spread_lower_bound}}.

\subsection{Cut and volume estimates}
\begin{lemma}
	\label{lem: expected_number_boundary_points}
	Under the conditions of Theorem \ref{thm: conductance_upper_bound}, and for any $r < \sigma/2d$,
	\begin{equation*}
	\Pbb(\Csigr) \leq 2 \nu(\Csig) \frac{rd}{\sigma}  \left(\lambda_{\sigma} - \frac{r^{\gamma}}{\gamma + 1}\right)
	\end{equation*}	
\end{lemma}
\begin{proof}
	Recalling that $f$ is the density function for $\Pbb$, we have
	\begin{equation}
	\label{eqn: integral_over_epsilon_neighborhood}
	\Pbb(\Csigr) = \int_{\Csigr} f(x) dx
	\end{equation}
	We partition $\Csigr$ into slices, based on distance from $\Csig$, as follows: for $k \in \N$,
	\begin{equation*}
	\mathcal{T}_{i,k} = \set{x \in \Rd: t_{i,k} < \frac{\rho(x, \Csig)}{r} \leq t_{i+1,k}}, ~~ \Csigr = \bigcup_{i = 0}^{k-1} \mathcal{T}_{i,k}
	\end{equation*}
	where $t_i = i/k$ for $i = 0, \ldots, k - 1$. As a result,
	\begin{equation*}
	\int_{\Csigr} f(x) dx = \sum_{i = 0}^{k-1} \int_{\mathcal{T}_{i,k}} f(x) dx \leq \sum_{i = 0}^{k-1} \nu(\mathcal{T}_{i,k}) \max_{x \in \mathcal{T}_{i,k}} f(x).
	\end{equation*}
	We substitute
	\begin{equation*}
	\nu(\mathcal{T}_{i,k}) = \nu(\Csig + rt_{i+1,k}B) - \nu(\Csig + rt_{i,k}B) := \nu_{i+1,k} - \nu_{i,k}. 
	\end{equation*}
	where for simplicity we've written $\nu_{i,k} = \nu(\Csig + rt_{i+1,k}B)$.
	This, in concert with the upper bound
	\begin{equation*}
	\max_{x \in \mathcal{T}_{i,k}} f(x) \leq \lambda_{\sigma} - (rt_{i,k})^{\gamma},
	\end{equation*}
	which follows from \ref{asmp: bounded_density} and \ref{asmp: low_noise_density}, yields
	\begin{align}
	\label{eqn: telescoping_sum}
	\sum_{i = 0}^{k-1} \nu(\mathcal{T}_{i,k}) \max_{x \in \mathcal{T}_{i,k}} f(x) & \leq \sum_{i = 0}^{k-1} \biggl\{ \nu_{i+1,k} - \nu_{i,k} \biggr\} \biggl( \lambda_{\sigma} - (rt_{i,k})^{\gamma} \biggr) \nonumber \\
	& = \sum_{i = 1}^{k} 
	\underbrace{\nu_{i,k} \biggl( \left[\lambda_{\sigma} - (rt_{i,k})^{\gamma}\right] -  \left[\lambda_{\sigma} - (rt_{i-1,k})^{\gamma}\right]\biggr)}_{:= \Sigma_k} + \underbrace{\biggl(\nu_{k,k}\left[\lambda_{\sigma} - r^{\gamma}\right] - \nu_{1,k}\lambda_{\sigma} \biggr)}_{:= \xi_k}
	\end{align}
	
	We first consider the term $\Sigma_k$. Here we use Lemma \ref{lem: expansion_sets} to upper bound
	\begin{equation*}
	\nu_{i,k} \leq \vol(\Csig)\left(1 + \frac{rt_{i,k}}{\sigma}\right)^d
	\end{equation*}
	and so we can in turn upper bound $\Sigma_k$:
	\begin{equation}
	\label{eqn: Sigmak_riemann_sum}
	\Sigma_k \leq \vol(\Csig) r^\gamma \sum_{i = 1}^{k} \left(1 + \frac{rt_{i,k}}{\sigma}\right)^d \biggl( (t_{i-1,k})^{\gamma} - (t_{i,k})^{\gamma}\biggr).
	\end{equation}
	This, of course, is a Riemann sum, and as the inequality holds for all values of $k$ it holds in the limit as well, which we compute to be
	\begin{align*}
	\lim_{k \to \infty} \sum_{i = 1}^{k} \left(1 + \frac{rt_{i,k}}{\sigma}\right)^d \biggl( (t_{i-1,k})^{\gamma} - (t_{i,k})^{\gamma}\biggr) & = \gamma \int_{0}^{1} \left(1 + \frac{rt}{\sigma}\right)^d t^{\gamma - 1} dt \\
	& \overset{(i)}{\leq} \gamma \int_{0}^{1} \left(1 + \frac{2drt}{\sigma}\right) t^{\gamma - 1} dt = \left(1 + \frac{\gamma 2dr}{\gamma + 1}\right).
	\end{align*}
	where $(i)$ follows from Lemma \ref{lem: Taylor_series}. 
	We plug this estimate in to \eqref{eqn: Sigmak_riemann_sum} and obtain
	\begin{equation*}
	\lim_{k \to \infty} \Sigma_k \leq \vol(\Csig) r^{\gamma} \left(1 + \frac{\gamma 2dr}{\gamma + 1}\right).
	\end{equation*}
	
	We now provide an upper bound on $\xi_k$. It will follow the same basic steps as the bound on $\Sigma_k$, but will not involve integration:
	\begin{align*}
	\xi_k & \overset{(ii)}{\leq} \nu(\Csig) \biggl\{ \left(1 + \frac{r}{\sigma}\right)^d(\lambda - r^{\gamma}) - \lambda \biggr\} \\
	& \overset{(iii)}{\leq} \nu(\Csig) \biggl\{ \left(1 + \frac{2dr}{\sigma}\right)(\lambda - r^{\gamma}) - \lambda \biggr\} = \nu(\Csig) \biggl\{ \frac{2dr}{\sigma}(\lambda - r^{\gamma}) - r^{\gamma} \biggr\}.
	\end{align*}
	where $(ii)$ follows from Lemma \ref{lem: expansion_sets} and $(iii)$ from Lemma \ref{lem: Taylor_series}. The final result comes from adding together the upper bounds on $\Sigma_k$ and $\xi_k$ and taking the limit as $k \to \infty$.
\end{proof}

\begin{lemma}
	\label{lem: expected_density_cut}
	Under the setup and conditions of Theorem \ref{thm: conductance_upper_bound}, and for any $r < \sigma/2d$,
	\begin{equation*}
	p_K \leq \frac{4 \lambda \nu_d r^{d+1} \nu(\Csig)d}{\sigma}  \left(\lambda_{\sigma} - \frac{r^{\gamma}}{\gamma + 1}\right)
	\end{equation*}
\end{lemma}
\begin{proof}
	We can write $\cut_{n,r}$ as the sum of indicator functions,
	\begin{equation}
	\label{eqn: density_cut_expansion}
	\cut_{n,r} = \sum_{i = 1}^{n} \sum_{j = 1}^{n} \1(x_i \in \Csigr) \1(x_j \in B(x_i,r) \cap \Csig)
	\end{equation}
	and by linearity of expectation, we can obtain
	\begin{equation*}
	p_K = \frac{\mu_K}{{n \choose 2}} = 2 \cdot \Pbb(x_i \in \Csigr, x_j \in B(x_i,r) \cap \Csig)
	\end{equation*}
	Writing this with respect to the density function $f$, we have
	\begin{align*}
	p_K & = 2 \int_{\Csigr} f(x) \left\{ \int_{B(x,r) \cap \Csig} f(x') dx' \right\} dx \\
	& \leq 2 \nu_d r^d \lambda  \int_{\Csigr} f(x) dx
	\end{align*}
	where the inequality follows from Assumption \ref{asmp: cluster_separation}, which implies that the density function $f(x') \leq \lambda$ for all $x' \in \Csig \setminus \Cset$ (otherwise, $x'$ would be in some $\Cset' \in \Cbb_f(\lambda)$, which \ref{asmp: cluster_separation} forbids). Then, upper bounding the integral using Lemma \ref{lem: expected_density_cut} gives the final result.
\end{proof}

\begin{lemma}
	\label{lem: expected_density_volume}
	Under the setup and conditions of Theorem \ref{thm: conductance_upper_bound},
	\begin{equation*}
	p_V \geq \lambda_{\sigma}^2 \nu_d r^d \nu(\Csig)
	\end{equation*}
\end{lemma}
\begin{proof}
	The proof will proceed similarly to Lemma \ref{lem: expected_density_cut}. We begin by writing $\vol_{n,r}$ as the sum of indicator functions,
	\begin{equation}
	\label{eqn: volume_expansion}
	\vol_{n,r} = \sum_{i = 1}^{n} \sum_{j = 1}^{n} \1(x_i \in \Csig) \1(x_j \in B(x_i, r))
	\end{equation}
	and by linearity of expectation we obtain
	\begin{equation*}
	p_V = \frac{\mu_V}{{n \choose 2}} = 2 \cdot \Pbb(x_i \in \Csig, x_j \in B(x_i,r)).
	\end{equation*}
	Writing this with respect to the density function $f$, we have
	\begin{align*}
	p_V & = 2 \int_{\Csig} f(x) \left\{ \int_{B(x,r)} f(x') dx' \right\} dx \\
	& \geq 2 \int_{\Cset_{\sigma - r}} f(x) \left\{ \int_{B(x,r)} f(x') dx' \right\} dx \\
	& \overset{(i)}{\geq} 2 \lambda_{\sigma}^2 \nu_d r^d \int_{\Cset_{\sigma - r}} f(x) dx
	\end{align*}
	where $(i)$ follows from the fact that $B(x,r) \subset \Csig$ for all $x \in C_{\sigma - r}$, along with the lower bound in Assumption \ref{asmp: bounded_density}. The claim then follows from Lemma \ref{lem: interior_of_expansion_sets}.
\end{proof}

We now convert from bounds on $p_K$ and $p_V$ to probabilistic bounds on $\cut_{n,r}$ and $\vol_{n,r}$ in Lemmas \ref{lem: prob_bound_cut} and \ref{lem: prob_bound_vol}. The key ingredient will be Lemma \ref{lem: bounded_difference}, Hoeffding's inequality for U-statistics; the proofs for both are nearly identical and we give only a proof for Lemma \ref{lem: prob_bound_cut}.

\begin{lemma}
	\label{lem: prob_bound_cut}
	The following statement holds for any $\delta \in (0,1]$: Under the setup and conditions of Theorem \ref{thm: conductance_upper_bound}, 
	\begin{equation}
	\label{eqn: numerator_additive_bound}
	\frac{\cut_{n,r}}{{n \choose 2}} \leq p_K + \sqrt{\frac{\log(1/\delta)}{n}}
	\end{equation}
	with probability at least $1 - \delta$. 
\end{lemma}

\begin{lemma}
	\label{lem: prob_bound_vol}
	The following statement holds for any $\delta \in (0,1]$: Under the setup and conditions of Theorem \ref{thm: conductance_upper_bound}, 
	\begin{equation}
	\label{eqn: denominator_additive_bound}
	\frac{\vol_{n,r}}{{n \choose 2}} \geq p_V - \sqrt{\frac{\log(1/\delta)}{n}}
	\end{equation}
	with probability at least $1 - \delta$. 
\end{lemma}

\begin{proof}[Proof of Lemma \ref{lem: prob_bound_cut}.]
	From \eqref{eqn: density_cut_expansion}, we see that $\cut_{n,r}$, properly scaled, can be expressed as an order-$2$ $U$-statistic,
	\begin{equation*}
	\frac{\cut_{n,r}}{{n \choose 2}} = \frac{1}{{n \choose 2}} \sum_{1 \leq i < j \leq n} \phi_K(x_i, x_j)
	\end{equation*}
	where 
	\begin{equation*}
	\phi_K(x_i,x_j) = \1(x_i \in \Asigr) \1(x_j \in B(x_i,r) \cap \Asig) + \1(x_j \in \Asigr) \1(x_i \in B(x_j,r) \cap \Asig).
	\end{equation*}
	
	From Lemma \ref{lem: bounded_difference} we therefore have
	\begin{equation*}
	\frac{\cut_{n,r}}{{n \choose 2}} \leq p_k + \sqrt{\frac{\log(1/\delta)}{n}}
	\end{equation*}
	with probability at least $1 - \delta$. 
\end{proof}

\subsection{Proof of Theorem \ref{thm: conductance_upper_bound}}
The proof of Theorem \ref{thm: conductance_upper_bound} is more or less given by Lemmas \ref{lem: expected_density_cut}, \ref{lem: expected_density_volume}, \ref{lem: prob_bound_cut}, and \ref{lem: prob_bound_vol}. All that remains is some algebra, which we take care of below.

Fix $\delta \in (0,1]$ and let $\delta' = \delta/2$. Noting that $\Phi_{n,r}(\Csig[\Xbf]) = \frac{\cut_{n,r}}{\vol_{n,r}}$, some trivial algebra gives us the expression
\begin{equation}
\label{eqn: conductance_representation_1}
\Phi_{n,r}(\Csig[\Xbf]) = \frac{p_K + \left(\frac{\cut_{n,r}}{{n \choose 2}} - p_K\right)}{p_V + \left(\frac{\vol_{n,r}}{{n \choose 2}} - p_V\right)}
\end{equation}
We assume (\ref{eqn: numerator_additive_bound}) and (\ref{eqn: denominator_additive_bound}) hold with respect to $\delta'$, keeping in mind that this will happen with probability at least $1 - \delta$. Along with (\ref{eqn: conductance_representation_1}) this means
\begin{equation*}
\Phi_{n,r}(\Csig[\mathbf{X}]) \leq \frac{p_K + \Err_n}{p_V - \Err_n}
\end{equation*}
for $\Err_n = \sqrt{\frac{\log(1/\delta')}{n}}$.
Now, some straightforward algebraic manipulations yield
\begin{align*}
\frac{p_K + \Err_n}{p_V - \Err_n} & = \frac{p_K}{p_V} + \left(\frac{p_K}{p_V - \Err_n} - \frac{p_K}{p_V}\right) + \frac{\Err_n}{p_V - \Err_n} \\
& = \frac{p_k}{p_V} + \frac{\Err_n}{p_V - \Err_n}\left(\frac{p_K}{p_V} + 1\right) \\
& \leq \frac{p_K}{p_V} + 2 \frac{\Err_n}{p_V - \Err_n}.
\end{align*}
By Lemmas \ref{lem: expected_density_cut} and Lemma \ref{lem: expected_density_volume}, we have
\begin{equation*}
\frac{p_K}{p_V} \leq \frac{4rd}{\sigma} \frac{\lambda}{\lambda_{\sigma}} \frac{\left(\lambda_{\sigma} - \frac{r^{\gamma}}{\gamma + 1}\right)}{\lambda_{\sigma}}
\end{equation*}
Then, the choice of 
\begin{equation*}
n \geq \frac{9\log(2/\delta)}{\epsilon^2}\left(\frac{1}{ \lambda_{\sigma}^2 \nu(\Csig) \nu_d r^d}\right)^2 
\end{equation*}

implies $2 \frac{\Err_n}{p_V - \Err_n} \leq \epsilon$.


\subsection{Concentration inequalities}

Given a symmetric kernel function $k: \mathcal{X}^m \to \Reals$, and data $\set{x_1, \ldots, x_n}$, we define the \textit{order-$m$ $U$ statistic} to be 
\begin{equation*}
U := \frac{1}{ {n \choose m} } \sum_{1 \leq i_1 < \ldots < i_m \leq n} k(x_{i_1},\ldots,x_{i_m})
\end{equation*}

For both Lemmas \ref{lem: bounded_difference} and \ref{lem: bernstein}, let $X_1, \ldots, X_n \in \mathcal{X}$ be independent and identically distributed. We will additionally assume the order-$m$ kernel function $k$ satisfies the boundedness property $\sup_{x_1, \ldots, x_m} \abs{k(x_1, \ldots, x_m)} \leq 1$. 

\begin{lemma}[Hoeffding's inequality for $U$-statistics.]
	\label{lem: bounded_difference}
	For any $t > 0$,
	\begin{equation*}
	\mathbb{P}(\abs{U - \mathbb{E}U} \geq t) \leq 2 \exp\left\{- \frac{2nt^2}{m}\right\}
	\end{equation*}
	Further, for any $\delta > 0$, we have
	\begin{align*}
	U & \leq \mathbb{E}U + \sqrt{\frac{m \log(1 / \delta)}{2n} }, \\
	U & \geq \mathbb{E}U - \sqrt{\frac{m \log(1 / \delta)}{2n} }
	\end{align*}
	each with probability at least $1 - \delta$. 
\end{lemma}

\subsection{Mixing time on graphs}
For $N \in \mathbb{N}$ and a set $V$ of $N$ vertices, take $G = (V,E)$ to be an undirected and unweighted graph, with associated adjacency matrix $\Abf$, random walk matrix $\Wbf$, and stationary distribution $\pibf = (\pi_u)_{u \in V}$ where $\pi_v = \frac{\Dbf_{vv}}{\vol(V; G)}$. For $v \in V$, 
\begin{equation}
\label{eqn: random_walk}
q_{vu}^{(m)} = e_v\Wbf^m e_u, ~~ \qbf_{v}^{(m)} = \left(q_{vu}^{(m)}\right)_{u \in V}, ~~ \qbf_v = (\qbf^{(1)}_{v\cdot},\qbf^{(2)}_{v\cdot}, \ldots), 
\end{equation}
denote respectively the $m$-step transition probability, distribution, and sequence distributions of the random walk over $G$ originating at $v$. Letting $\qbf = (\qbf_v)_{v \in V}$, the relative pointwise mixing time is thus
\begin{equation*}
\tau_{\infty}(\qbf; G) = \min\set{m: \forall u,v \in V, \frac{\abs{q_{vu}^{(m)} - \pibf_u}}{\pibf_u} \leq 1/4} 
\end{equation*}

Two key quantities relate the mixing time to the expansion of subsets $S$ of $V$. The \emph{local spread} is defined to be
\begin{equation*}
s(G) := \frac{9D_{\min}}{10}\pi_{\min} 
\end{equation*}
for $D_{\min} := \min_{v \in V} \Dbf_{vv}$ and $\pi_{\min} := D_{\min} / \vol(V; G)$.


%\inf \set{t: \forall S \subset V ~\text{with}~ \pi(S) = t, \beta(S) \leq 1/10}
%\end{equation*}
where $\beta(S) := \inf_{v \in S} \qbf_{v}^{(1)}(S^c)$, and by convention we let $\pbf(S) = \sum_{u \in S} p_u$ for any distribution vector $\pbf = (p_u)_{u \in V}$ over $V$. We collect some necessary facts about the local spread in Lemma \ref{lem: local_spread_G}.
\begin{lemma}
	\label{lem: local_spread_G}
	\begin{itemize}
		\item If $\pibf(S) \leq s(G)$, then for every $u \in S$, $\qbf_u^{(1)}(S^c) \geq 1/10$.
		\item For any $v, u \in V$, and $m \in N$ greater than $0$, $q_{vu}^{(m)}/ \pi_{\min} \leq 1/s(G)$.
	\end{itemize}
	
\end{lemma}

\begin{proof}
	If $t = \pibf(S) \leq \frac{9 D_{\min}}{10} \pi_{\min}$, divide both sides by $\pi_{\min}$ to obtain
	\begin{equation*}
	\abs{S} \leq \frac{9 D_{\min}}{10}
	\end{equation*}
	which implies $\qbf_{v}^{(1)}(S^c) \geq 1/10$ for all $v \in S$. This implies the first statement.
	
	The second statement follows from the fact $q_{vu}^{(m)} \leq 1/D_{\min}$ for any $m$.
\end{proof}

The local spread facilitates conversion between $\tau_{\infty}(\qbf_v; G)$ and the more easily manageable \emph{total variation} mixing time, given by
\begin{equation*}
\tau_1(\rhobf; G) = \min\biggl\{m: \forall v \in V, \norm{\rhobf_v - \pibf}_{TV} \leq 1/4 \biggr\}
\end{equation*}
where 
\begin{equation}
\label{eqn: uniform_random_walk}
\rhobf_{v}^{(m)} = \frac{1}{m}\sum_{k = 1}^{m+1} \qbf_{v}^{m}, ~~ \rhobf_v = \left( \rhobf_{v}^{(1)}, \rhobf_{v}^{(2)}, \rhobf_{v}^{(3)} \ldots \right), ~~ \rhobf = \left( \rhobf_v \right)_{v \in V}
\end{equation}
and $\norm{\pbf - \pibf}_{TV} = \sum_{v \in V}\abs{p_v - \pi_v}$ is the total variation norm between distributions $\pbf$ and $\pibf$. 
\begin{lemma}
	\label{lem: tv_mixing_to_pointwise_mixing1}
	For $\qbf$ as in \eqref{eqn: random_walk} and $\rhobf$ as in \eqref{eqn: uniform_random_walk},
	\begin{equation*}
	\tau_{\infty}(\qbf; G) \leq 2752 \tau_1(\rhobf; G) \log \left(4 \max\left\{1, \frac{1}{s(G)}\right\}\right)
	\end{equation*}
\end{lemma}
\begin{proof}
	Masking dependence on the starting vertex $v$ for the moment, let
	\begin{equation*}
	\Delta_u^{(m)} = q_{vu}^{(m)} - \pi_u, ~~ \delta_u^{(m)} = \frac{\Delta_u^{(m)}}{\pi_u}
	\end{equation*}
	and $\Deltabf^{(m)} = (\Delta_u^{(m)})_{u \in V}$, $\deltabf^{(m)} = (\delta_u^{(m)})_{u \in V}$. For a vector $\Deltabf = (\Delta_u)_{u \in V}$, the $L^{p}(\pibf)$ norm is given by
	\begin{equation*}
	\norm{\Deltabf}_{L^p(\pibf)} = \left(\sum_{u \in V} \left(\Delta_u\right)^{p} \pi_u \right)^{1/p}
	\end{equation*}
	To go between the $L^{\infty}(\pibf)$ and $L^{1}(\pibf)$ norms, we have
	\begin{align*}
	\norm{\deltabf^{(2m)}}_{L^{\infty}(\pi)} & \overset{(i)}{\leq} \norm{\deltabf^{(m)}}^2_{L^{2}(\pi)} \\
	& = \norm{(\deltabf^{(m)})^2}_{L^{1}(\pi)} \\
	& \overset{(ii)}{\leq}  \norm{(\deltabf^{(m)})}_{L^{1}(\pi)} \norm{(\deltabf^{(m)})}_{L^{\infty}(\pi)}
	\end{align*}
	where $(i)$ is a result of \textcolor{red}{Benjamini and Morris} and $(ii)$ follows from Holder's inequality. Now, we upper bound the second factor on the right hand side by observing
	\begin{align*}
	\norm{(\deltabf^{(m)})}_{L^{\infty}(\pi)} & \leq \max\left\{1, \max_{u \in V} \frac{q_{vu}^{(m)}}{\pi_u} \right\} \\
	& \overset{(iii)}{\leq} \max\left\{1, \frac{1}{s(G)}\right\}
	\end{align*}
	where $(iii)$ follows from Lemma \ref{lem: local_spread_G}.
	
	Now, we leverage the following well-known fact (\textcolor{red}{PhD thesis of Montenegro}): for any $\epsilon > 0$, if $m \geq \tau_1(\qbf_v^{(m)}; G) \cdot \log(1/\epsilon)$ then
	\begin{equation*}
	\norm{\qbf_v^{(m)} - \pibf}_{TV} \leq \epsilon.
	\end{equation*}
	But $\norm{\qbf_v^{(m)} - \pibf}_{TV}$ is exactly $\norm{(\deltabf^{(m)})}_{L^{1}(\pi)}$. Therefore, picking 
	\begin{equation*}
	m_0 = \tau_1(\qbf_v^{(m)}; G) \cdot \log \left(4 \max\left\{1, \frac{1}{s(G)}\right\} \right)
	\end{equation*} implies $\norm{(\deltabf^{(m)})}_{L^{\infty}(\pi)} \leq 1/4$ for all $m \geq 2 m_0$.  Then, 
	\begin{equation*}
	\norm{(\deltabf^{(m)})}_{L^{\infty}(\pi)} = \sup_{u}\left\{ \frac{\abs{q_{vu}^{(m)} - \pibf_u}}{\pibf_u} \right\}.
	\end{equation*}
	and since none of the above depended on a specific choice for $v$, the supremum can be taken over all starting vertices $v$ as well. Thus $\tau_{\infty}(\qbf^{(m)}; G) \leq 2m_0$. 
	
	Finally, it is known (\textcolor{red}{PhD thesis of Montenegro}) that
	\begin{equation*}
	\tau_{1}(\qbf^{(m)}; G) \leq 1376 \tau_{1}(\rhobf^{(m)}; G)
	\end{equation*}
	and so the desired result holds.	
\end{proof}

The second key quantity is the \emph{conductance function}
\begin{equation}
\label{eqn: conductance_function}
\Phi(t; G) := \min_{\substack{S \subseteq V, \\ \pibf(S) \leq t} } \Phi(S; G) ~~~~~~~ (\pi_{\min} \leq t < 1)
\end{equation}
where $\Phi(S; G)$ is the normalized cut of $S$ in $G$ given by \eqref{eqn: norm_cut}. 

Lemma \ref{lem: expansion_to_tv_distance} leverages the conductance function and local spread to produce an upper bound on the total variation distance between $\rhobf_{v}^{(m)}$ and $\pibf.$
\begin{lemma}
	\label{lem: expansion_to_tv_distance}
	If $D_{\min} > 10$, for any $v \in V$:
	\begin{equation*}
	\norm{\rhobf_v^{(m)} - \pibf}_{TV} \leq \max\left\{ \frac{1}{4}, \frac{1}{10} +  \frac{70}{m}\left(\frac{20}{9} + \int_{t = s'(G)}^{1/2} \frac{4}{t \Phi^2(t; G)} dt\right) \right\}
	\end{equation*}
	where $s'(G) = s(G) / 9$.
\end{lemma}

To prove Lemma \ref{lem: expansion_to_tv_distance} we first introduce a generalization of $\Phi(t;G) \cdot \Phi(t;G)$ known as a blocking conductance function.
\footnote{For more details, see \textcolor{red}{PhD thesis of Montenegro}}

\begin{definition}[Blocking Conductance Function of \textcolor{red}{PhD thesis of Montenegro}]
	\label{def: blocking_conductance_function}
	For $t_0 \geq \pi_{\min}$, a function $\phi(t; G): [t_0, 1/2] \to [0,1]$ is a \emph{blocking conductance function} if for all $S \subset V$ with $\pibf(S) = t \in [t_0, 1/2]$, either of the following hold: 
	\begin{enumerate}
		\item \emph{Exterior inequality.} For all $y \in \left[\frac{1}{2}t, t\right]: \phi_{int}(S) \geq \phi(\max\{t_0,y\})$
		\item \emph{Interior inequality.} For all $y  \in \left[t, \frac{3}{2}t\right]: \phi_{ext}(S) \geq \phi(\max\{y,1 - y\})$.
	\end{enumerate}
	where $\phi_{int}$ and $\phi_{ext}$ are defined respectively as
	\begin{align*}
	\phi_{int}(S) & = \sup_{\lambda \leq \pibf(S)} \min_{\substack{B \subset S \\ \pibf(B) \leq \lambda} } \frac{\lambda \cut(S \setminus B, S^c; G)}{\vol(V; G) \left[\pibf(S) \pibf(S^c)\right]^2} \\
	\phi_{ext}(S) & = \sup_{\lambda \leq \pibf(S)} \min_{\substack{B \subset S^c \\ \pibf(B) \leq \lambda} } \frac{\lambda \cut(S \setminus B, S^c; G)}{\vol(V; G) \left[\pibf(S) \pibf(S^c)\right]^2}
	\end{align*}
\end{definition}

\begin{theorem}[\textcolor{red}{PhD thesis of Montenegro} Theorem 3.2]
	\label{thm: montenegro}
	Consider $\phi(t; G): [t_0, 1/2] \to [0,1]$ a blocking conductance function. Then, letting
	\begin{equation*}
	h^m(t_0) = \sup_{S: \pibf(S) < t_0}  \bigl(\rhobf_v^{(m)}(S) - \pibf(S) \bigr)
	\end{equation*}
	the following statement holds: if $\phi$ is a blocking conductance function,
	\begin{align*}
	\norm{\rhobf_v^{(m)} - \pibf}_{TV} & \leq \max\left\{ \frac{1}{4}, h^1(t_0) +  \frac{70}{m}\left(\frac{1}{\phi(t_0; G)} + \int_{t = t_0}^{1/2} \frac{4}{t \phi(t; G)} dt\right) \right\} \\
	\end{align*}
\end{theorem}

Note that in \textcolor{red}{PhD thesis of Montenegro} this theorem is stated with respect to $h^0$. However, in the subsequent proof it holds with respect to $h^m$, and it is observed that $h^m$ is decreasing in $m$. For our purposes it is more useful to state it with respect to $h^1$, as we have done.

\begin{proof}[Proof of Lemma \ref{lem:  expansion_to_tv_distance}]
	Consider the function $\phi_0(t, G): [s(G), 1/2] \to [0,1]$ defined by
	\begin{equation}
	\label{eqn: local_spread_blocking_conductance_function}
	\phi_0(t; G) = 
	\begin{cases}
	\frac{1}{5}, ~~~~~~~~~~~~ t = s'(G) \\
	\frac{1}{4} \Phi^2(t; G), ~ t \in \left(s'(G), 1/2\right] \\
	\end{cases}
	\end{equation}
	\begin{lemma}
		\label{lem: local_spread_blocking_conductance_function}
		If $D_{\min} > 10$, $\phi_0$ is a blocking conductance function.
	\end{lemma}
	We take Lemma \ref{lem: local_spread_blocking_conductance_function} as given, and defer the proof until after the proof of Lemma \ref{lem: expansion_to_tv_distance}. 
	
	Lemma \ref{lem: local_spread_blocking_conductance_function} and Theorem \ref{thm: montenegro} together yield:
	\begin{equation*}
	\norm{\rhobf^t - \pibf}_{TV} \leq \max\left\{ \frac{1}{4}, h^1(s'(G)) +  \frac{70}{m}\left(5 + \int_{t = s'(G)}^{1/2} \frac{4}{t \Phi^2(t; G)}\right) \right\}
	\end{equation*}
	
	Then, $h^1(s'(G)) \leq 1/10$ follows exactly from the proof of Lemma \ref{lem: local_spread_G}, except now $\pibf(S) \leq s'(G)$ results in the sharper bound of $\qbf_u^{(1)}(S^c) \geq 9/10$ for every $u \in S$.
\end{proof}
\begin{proof}[Lemma \ref{lem: local_spread_blocking_conductance_function}]
	The condition $D_{\min} > 10$ ensures that $s(G) \geq \pi_{\min}$.
	
	It is known that $\frac{1}{4} \Phi^2(x;G)$ satisfies the exterior inequality for all $t \in (\pi_{\min}, 1/2]$.
	
	For $t = s'(G)$ we will instead use the interior inequality. For any $S$ such that $\pibf(S) \leq s'(G)$, the following statement holds: for every $u \in S$, $\cut(u, S^c; G) \geq 9/10 \cdot \deg(u; G)$. 
	Fixing $\lambda = \pi(S)/2$, we have
	\begin{align*}
	\phi_{int}(S) & \geq \min_{\substack{B \subset S \\ \pibf(B) \leq \lambda} } \frac{\lambda \cut(S \setminus B, S^c; G)}{\vol(V; G) \left[\lambda (1 - \lambda)\right]^2} \\
	& \geq  \min_{\substack{B \subset S \\ \pibf(B) \leq \lambda} } \frac{9 \lambda \sum_{u \in S \setminus B} \deg(u; G)}{10 \vol(V; G) \left[\lambda (1 - \lambda)\right]^2} \\
	& \geq \frac{9\lambda^2}{20[\lambda^2(1 - \lambda)^2]} \geq \frac{9}{20}. 
	\end{align*} 
\end{proof}

\subsection{Population-level conductance function.}

We will make use of the above theory with respect to the conductance function $\Phi(t; G_{n,r}[\Csig(\Xbf)])$. First, however, we introduce a population-level analogue to $\Phi(t; G_{n,r}[\Csig(\Xbf)])$ over the set $\Csig$, which we denote $\widetilde{\Phi}_{\Pbb,r}$. (In general, we will adopt the convention of using $\widetilde{f}$ to denote functionals computed with respect to $\Csig$.)

For $\Sset \subset \Rd$
\begin{equation*}
\nu_{\Pbb}(\Sset) := \int_{\Sset} f(x) dx
\end{equation*} 
is the weighted volume.

The $r$-ball walk over $\Csig$ is a Markov chain. For $x \in \Csig$ and $\Sset, \Sset' \subset \Csig$ the transition probability is given by
\begin{equation*}
\widetilde{P}_{\Pbb, r}(x;\Sset) := \frac{\nu_\Pbb(\Sset \cap B(x,r))}{\nu_\Pbb(\Csig \cap B(x,r))}, ~~~ \widetilde{Q}_{\Pbb, r}(\Sset, \Sset') := \int_{x \in \Sset} f(x) P_{\Pbb, r}(x;\Sset') dx,
\end{equation*}
stationary distribution defined by
\begin{equation*}
\ell_{\Pbb,r}(x) := \frac{\nu_\Pbb(\Csig \cap B(x,r))}{\nu_{\Pbb}(B(x,r))}, ~~~ \pi_{\Pbb,r}(\Sset) := \frac{1}{\int_{\Csig} f(x) \ell_{\Pbb,r}(x) dx} \int_{\Sset} f(x) \ell_{\Pbb,r}(x) dx
\end{equation*}
and corresponding conductance function
\begin{equation*}
\widetilde{\Phi}_{\Pbb,r}(t) := \min_{\substack{\Sset \subset \Csig, \\ \pi_{\Pbb,r}(\Sset) \leq t} } \frac{\widetilde{Q}_{\Pbb,r}(\Sset, \Csig \setminus \Sset)}{\pi_{\Pbb,r}(\Sset)}.
\end{equation*}

For $m > 0$ and $0 < t_0 < t_1 < \ldots < t_m < 1$, denote the \emph{stepwise approximation to $g$} by $\bar{g}$, defined as
\begin{equation}
\label{eqn: stepwise_approximation}
\bar{g}(t) = g(t_i), ~~ \text{ for $t \in [t_{i-1}, t_i]$ }
\end{equation}
The stepwise approximation will be important to showing the consistency results of Section \textcolor{red}{(A.7)} hold across the entire conductance function. Lemma \ref{lem: stepwise_approximation} shows that the approximation will not overly degrade our estimates of the population-level conductance function.
\begin{lemma}
	\label{lem: stepwise_approximation}
	\begin{itemize}
		\item For any function $f$ monotone decreasing in $t$ on the interval $[t_0,t_m]$, $\bar{f}(t) \leq f(t)$ for all $t \in [t_0,t_m]$.
		\item Fix
		\begin{equation*}
		g(t) = \log\left(\frac{1}{t}\right) \text{ for $x \in [t_0, 1/2]$}
		\end{equation*}
		If for all $i$ in $1,\ldots,m$, $(t_i - t_{i - 1}) \leq t_0/2$, then $\bar{g}(t)\geq g(t) / 2$.
	\end{itemize}
\end{lemma}

\begin{proof}
	The first statement is immediately obvious, and we turn to proving the second. 
	
	The upper bound $g(t) \geq \bar{g}(t)$ follows immediately from the fact that $g(t)$ is a decreasing function along with the first statement.
	
	By the concavity of the $\log$ function, 
	\begin{equation*}
	\bar{g}(t) = \log\left(\frac{1}{t_i}\right) \geq \log\left(\frac{1}{t}\right) - \frac{(t_i - t)}{t}.
	\end{equation*}
	As a result,
	\begin{equation*}
	\bar{g}(t) - \frac{g(t)}{2} \geq \frac{\log\left(\frac{1}{t}\right)}{2} - \frac{(t_i - t)}{t} \geq 1/2 - 1/2 = 0.
	\end{equation*}
\end{proof}


\begin{theorem}[Restatement of \textcolor{red}{Kannan 2004} Theorem 4.6]
	\label{thm: kannan}
	Let $K \subset \Rd$ be a convex body of diameter $D$. Then for any $\Sset \subset K$ with $\pi_{\nu,r}(\Sset) \leq 1/2$,
	\begin{equation}
	\label{eqn: kannan}
	\frac{Q_{\nu,r}(\Sset, \Csig \setminus \Sset)}{\pi_{\nu,r}(\Sset)} > \min\left\{\frac{1}{288\sqrt{d}},\frac{r}{81 \sqrt{d}D}\log\left(1 + \frac{1}{\pi_{\nu,r}(\Sset)}\right)\right\}.
	\end{equation}
\end{theorem}

\begin{lemma}
	\label{lem: continuous_conductance_lower_bound}
	Under the conditions on $\Csig$ given by Theorem \ref{thm: inverse_mixing_time_lower_bound}, the following bounds hold: 
	\begin{itemize}
		\item 
		for $0 < t < 1/2$,
		\begin{equation*}
		\widetilde{\Phi}_{\Pbb,r}(t) > \min\left\{\frac{1}{288\sqrt{d}},\frac{r}{81 \sqrt{d}D}\log\left(1 + \frac{\lambda_{\sigma}^2}{\Lambda_{\sigma}^2 t}\right)\right\} \cdot \frac{\lambda_{\sigma}^4}{\Lambda_{\sigma}^4}
		\end{equation*}
		\item
		Let
		\begin{equation*}
		M = \frac{2^{d+1}D^d \Lambda_{\sigma}^2}{r^d \lambda_{\sigma}^2}
		\end{equation*}
		and $t_i = (i + 1)/M$ for $i = 0, \ldots, m - 1$. Then, for $1/M < t < 1/2$
		\begin{equation*}
		\overline{\Phi}_{\Pbb,r}(t) > \min\left\{\frac{1}{288\sqrt{d}},\frac{r}{162 \sqrt{d}D}\Log\left( \frac{\Lambda_{\sigma}^2}{\lambda_{\sigma}^2 t}\right)\right\} \cdot \frac{\lambda_{\sigma}^4}{\Lambda_{\sigma}^4}
		\end{equation*}
		where $\overline{\Phi}_{\Pbb,r}(t)$ is defined as in \eqref{eqn: stepwise_approximation} with respect to $t_0, \ldots t_{M - 1}$, and $\Log(A/t) = \max\{\log(1 + 2A), \log(A/t)\}$.
	\end{itemize}
\end{lemma}
Before we prove Lemma \ref{lem: continuous_conductance_lower_bound}, note that the choice of $M$ is made to ensure $t_0$ is greater than the local spread of $G_{n,r}\left[\Csig[\Xbf]\right]$, as we will see in Section \ref{sec: consistency}.
\begin{proof}[Proof of Lemma \ref{lem: continuous_conductance_lower_bound}]
	We note that
	\begin{equation*}
	\pi_{\Pbb,r}(S) \leq \pi_{\nu,r}(S) \cdot \frac{\Lambda_{\sigma}^2}{\lambda_{\sigma}^2}, ~~~ Q_{\Pbb,r}(\Sset, \Csig \setminus \Sset) \geq Q_{\nu,r}(\Sset, \Csig \setminus \Sset) \cdot \frac{\lambda_{\sigma}^2}{\Lambda_{\sigma}^2}
	\end{equation*}
	
	Plugging these estimates in to \eqref{eqn: kannan} gives
	\begin{equation*}
	\frac{Q_{\Pbb,r}(\Sset, \Csig \setminus \Sset)}{\pi_{\Pbb,r}(\Sset)} > \min\left\{\frac{1}{288\sqrt{d}},\frac{r}{81 \sqrt{d}D}\log\left(1 + \frac{\lambda_{\sigma}^2}{\Lambda_{\sigma}^2 \pi_{\Pbb,r}(\Sset)}\right)\right\} \cdot \frac{\lambda_{\sigma}^4}{\Lambda_{\sigma}^4}
	\end{equation*}
	and since the right hand side is decreasing in $\pi_{\Pbb,r}(\Sset)$, the desired lower bound holds on $\widetilde{\Phi}_{\Pbb,r}(t)$.  The bound on $\overline{\Phi}_{\Pbb,r}(t)$ then follows from  $\Log(A/t) \leq \log(1 + 1/t)$ for all $0 < t < 1/2$ and application of Lemma \ref{lem: stepwise_approximation}.
\end{proof}

\subsection{Consistency of local spread and conductance function.}
\label{sec: consistency}

For notational ease, we write 
\begin{equation*}
\widetilde{\Phi}_{n,r}(t) = \Phi(t; G_{n,r}[\Csig[\Xbf]]), ~~ \widetilde{s}_{n,r} = s(G_{n,r}[\Csig[\Xbf]])
\end{equation*}

\begin{lemma}
	\label{lem: local_spread_lower_bound}
	For $\Csig$ satisfying the conditions of Theorem \ref{thm: inverse_mixing_time_lower_bound}:
	\begin{equation*}
	\liminf_{n \to \infty}~ \widetilde{s}_{n,r} \geq \frac{\lambda_{\sigma}^2}{\Lambda_{\sigma}^2} \frac{r^d}{(2D)^d}
	\end{equation*}
\end{lemma}

\textcolor{red}{Prove Lemma \ref{lem: local_spread_lower_bound}}.

The introduction of the stepwise approximation allows us to make use of Lemma \ref{lem: consistency_of_conductance_function}, which gives us (pointwise) consistency of the discrete graph functionals $\widetilde{\Phi}_{n,r}(t)$ to the continuous functionals $\widetilde{\Phi}_{\Pbb,r}(t)$. 

\begin{lemma}
	\label{lem: consistency_of_conductance_function}
	Fix $0 < t < 1/2$. Under the conditions on $\Csig$ given by Theorem \ref{thm: inverse_mixing_time_lower_bound}, the following statement holds: with probability one, as $n \to \infty$,
	\begin{equation}
	\label{eqn: pointwise_consistency_of_conductance_function}
	\liminf_{n \to \infty} \widetilde{\Phi}_{n,r}(t) \geq \widetilde{\Phi}_{\Pbb,r}(t)
	\end{equation}
	As a consequence, for $M$ and $(t_i)_{i=0}^{M-1}$ defined as in Lemma \ref{lem: continuous_conductance_lower_bound}, we have that
	\begin{equation}
	\label{eqn: consistency_of_stepapprox_conductance_function}
	\liminf_{n \to \infty} \overline{\Phi}_{n,r} \geq \overline{\Phi}_{\Pbb,r}
	\end{equation}
\end{lemma}
We defer the proof of pointwise consistency to Section \ref{subsection: proof_of_pointwise consitency_of_conductance_function}. For now, we show assume  that \eqref{eqn: consistency_of_stepapprox_conductance_function} is immediately implied by \eqref{eqn: pointwise_consistency_of_conductance_function}.

\begin{proof}[Proof of \eqref{eqn: consistency_of_stepapprox_conductance_function}]
	We take as given that for any $0 < t < 1/2$,
	\begin{equation*}
	\liminf_{n \to \infty} \widetilde{\Phi}_{n,r}(t) \geq \widetilde{\Phi}_{\Pbb,r}(t).
	\end{equation*}
	In particular, this will occur for $t_0, t_1, \ldots, t_m$ and therefore
	\begin{equation*}
	\liminf_{n \to \infty} \overline{\Phi}_{n,r} \geq \overline{\Phi}_{\Pbb,r}
	\end{equation*}
	uniformly over $[1/m,1/2]$.
\end{proof}

\subsection{Proof of pointwise consistency of conductance function.}
\label{subsection: proof_of_pointwise consitency_of_conductance_function}


\subsection{Proof of Theorem \ref{thm: inverse_mixing_time_lower_bound}}
Throughout this proof, we will refer to the subgraph $G_{n,r}\left[\Csig[\Xbf]\right]$ as $\widetilde{G}_{n,r}$.

Fix arbitrary $v = x_i \in \Csig[\Xbf]$, and let
\begin{equation*}
\widetilde{\qbf}_n^{(m)} = e_v \Wbf_{\Csig[\Xbf]}^t, ~~ \widetilde{\qbf}_n = (\widetilde{q}_n^{(1)}, \widetilde{q}_n^{(2)}, \ldots)
\end{equation*}
Our goal is to upper bound $\tau_{\infty}(\widetilde{\qbf}_n; \widetilde{G}_{n,r})$.

By Lemmas \ref{lem: local_spread_G} and \ref{lem: local_spread_lower_bound},
\begin{align*}
\tau_{\infty}(\widetilde{\qbf}_n; \widetilde{G}_{n,r}) & \leq 2752 \tau_{1}(\widetilde{\qbf}_n; \widetilde{G}_{n,r}) \max\left\{2, \log\left(\frac{4}{\widetilde{s}_{n,r}}\right) \right\} \nonumber \\
& \leq 2752 \tau_{1}(\widetilde{\qbf}_n; \widetilde{G}_{n,r}) \max\left\{2, 4d\log\left(\frac{2D \Lambda_{\sigma}^2}{\lambda_{\sigma}^2}\right) \right\}
\end{align*} 

We now upper bound $\tau_{1}(\widetilde{\qbf}_n; \widetilde{G}_{n,r})$.  From Lemma \ref{lem: expansion_to_tv_distance}, we have that
\begin{equation}
\label{eqn: mixing_time_bound_on_neighborhood_graph}
\limsup_{n \to \infty} \tau_{1}(\widetilde{\qbf}_n; \widetilde{G}_{n,r}) \leq \limsup_{n \to \infty} \frac{1400}{3}\left(5 + \int_{\widetilde{s}_{1,n}}^{1/2} \frac{4}{t\widetilde{\Phi}_{n,r}^2(t)} dt\right)
\end{equation}
(As $r$ remains constant and $n \to \infty$, $\Dbf_{xx} > C$ will be fulfilled for any $x \in \Csig[\Xbf]$, and any $C < \infty$.)
We set aside the constant term for the moment and turn to the integral. By Lemma \ref{lem: local_spread_lower_bound},
\begin{equation*}
\limsup_{n \to \infty}\int_{\widetilde{s}_{n,r}}^{1/2} \frac{4}{t\widetilde{\Phi}_{n,r}^2(t)} dt \leq \limsup_{n \to \infty}\int_{ s_{\Pbb,r}}^{1/2} \frac{4}{t\widetilde{\Phi}_{n,r}^2(t)} dt
\end{equation*}
where $s_{\Pbb,r} = \frac{\lambda_{\sigma}^2}{\Lambda_{\sigma}^2} \frac{r^d}{(2D)^d}$. We now replace the discrete conductance function $\widetilde{\Phi}_{n,r}$ by the stepwise approximation to the continuous conductance function, $\overline{\Phi}_{n,r}$:
\begin{align*}
\limsup_{n \to \infty}\int_{s_{\Pbb,r}}^{1/2} \frac{4}{t\widetilde{\Phi}_{n,r}^2(t)} dt & \overset{(i)}{\leq} \limsup_{n \to \infty}\int_{s_{\Pbb,r}}^{1/2} \frac{4}{t\overline{\Phi}_{n,r}^2(t)} dt \\
& = \int_{s_{\Pbb,r}}^{1/2} \limsup_{n \to \infty} \frac{4}{t\overline{\Phi}_{n,r}^2(t)} dt\\
& \overset{(ii)}{\leq} \int_{s_{\Pbb,r}}^{1/2} \frac{4}{t\overline{\Phi}_{\Pbb,r}^2(t)} dt \\
\end{align*}
where $(i)$ follows from Lemma \ref{lem: stepwise_approximation} and $(ii)$ from Lemma \ref{lem: consistency_of_conductance_function} (along with the continuous mapping theorem). Now, we make use of Lemma \ref{lem: continuous_conductance_lower_bound}:
\begin{align*}
& \int_{s_{\Pbb,r}}^{1/2} \frac{4}{t\overline{\Phi}_{\Pbb,r}^2(t)} dt \leq \frac{\Lambda_{\sigma}^8}{\lambda_{\sigma}^8} \cdot \left( 331776
\int_{s_{\Pbb,r}}^{1/2} \frac{d}{t} dt + \int_{s_{\Pbb,r}}^{1/2} \frac{81dD^2}{r^2t \Log(\frac{\Lambda_{\sigma}^2}{t\lambda_{\sigma}^2})}  dt\right) \\
& \leq \frac{\Lambda_{\sigma}^8}{\lambda_{\sigma}^8} \cdot \left( 331776
\underbrace{\int_{s_{\Pbb,r}}^{1/2} \frac{d}{t} dt}_{:= \mathcal{J}_1} + 
81\underbrace{\int_{s_{\Pbb,r}}^{\lambda_{\sigma}^2/(4 \Lambda_{\sigma}^2)} \frac{dD^2}{r^2t \log(\frac{\Lambda_{\sigma}^2}{t\lambda_{\sigma}^2})} dt}_{:= \mathcal{J}_2} +
81\underbrace{\int_{\lambda_{\sigma}^2/(4 \Lambda_{\sigma}^2)}^{1/2} \frac{dD^2}{r^2t \log(1 + \frac{4 \lambda_{\sigma}^2}{\Lambda_{\sigma}^2})}dt}_{:= \mathcal{J}_3}  \right)
\end{align*}

Computing a few simple integrals yields the following upper bounds on $\mathcal{J}_1, \mathcal{J}_2, \mathcal{J}_3$:
\begin{align*}
\mathcal{J}_1 & \leq d^2 \log\left( \frac{2D\Lambda_{\sigma}^2}{r\lambda_{\sigma}^2} \right) \\
\mathcal{J}_2 & \leq \frac{dD^2}{r^2} \left[ \log\left(2d\right) + \log\left(\log\left(\frac{2D}{r}\right)\right) \right] \\
\mathcal{J}_3 & \overset{(iii)}{\leq} 2 \frac{dD^2}{r^2} \frac{\Lambda_{\sigma}^2}{\lambda_{\sigma}^2} \log\left(4 \frac{\Lambda_{\sigma}^2}{\lambda_{\sigma}^2}\right)
\end{align*}
where $(iii)$ uses the upper bound $\frac{1}{\log(1 + x)} \leq \frac{1}{x}$.

Plugging these bounds in to \eqref{eqn: mixing_time_bound_on_neighborhood_graph} gives the desired upper bound on $\tau_{\infty}(\widetilde{q}_n, \widetilde{G}_{n,r})$, which translates to the lower bound of \eqref{eqn: inverse_mixing_time_lower_bound}.





\section{OTHER STUFF}

	\begin{lemma}[Bernstein's inequality for $U$-statistics]
		\label{lem: bernstein}
		Additionally, assume $\sigma^2 = \var\left(k(X_1, \ldots, X_m) \right) < \infty$. Then for any $\delta > 0$, 
		\begin{align*}
		\mathbb{P}(U - \mathbb{E}U \geq t) \leq \exp\left\{-\frac{n}{2m}\frac{t^2}{\sigma^2 + t/3}\right\},
		\end{align*}
		
		Moreover if $\sigma^2 \leq \mu/n$, 
		\begin{align*}
		U & \leq \mathbb{E}U \cdot \left(1 + \max\left\{ \sqrt{\frac{2m\log(1/\Delta)}{\mu}}, \frac{2m \log(1/\Delta)}{3\mu} \right\}\right), \\
		U & \geq \mathbb{E}U \cdot \left(1 - \max\left\{ \sqrt{\frac{2m\log(1/\Delta)}{\mu}}, \frac{2m \log(1/\Delta)}{3\mu} \right\}\right)
		\end{align*}
		each with probability at least $1 - \Delta$.
	\end{lemma}

\textcolor{red}{\textbf{Multiplicative bound}: As $\tilde{k}(x_1,x_2)$ is the sum of two Bernoulli random variables with negative covariance (since $\1(x_i \in \Asigr) \1(x_j \in B(x_i,r) \cap \Asig) = 1$ implies $\1(x_j \in \Asigr) \1(x_i \in B(x_j,r) \cap \Asig) = 0$ and vice versa), we can upper bound $\var\left(\tilde{k}(x_1, x_2)\right) \leq  \widetilde{p}$, where we recall 
	\begin{equation*}
	\widetilde{p} = 2\cdot \mathbb{P}\left(\1(x_1 \in \Asigr) \1(x_2 \in B(x_1,r) \cap \Asig)\right)
	\end{equation*}
	From Lemma \ref{lem: bernstein}, we therefore have
	\begin{equation*}
	\frac{\widetilde{\mathcal{E}}}{{n \choose 2}} \leq \widetilde{p} + \max\left\{ \sqrt{\frac{4\log(1/\Delta)\widetilde{p}}{n}}, \frac{4 \log(1/\Delta)}{3n} \right\}
	\end{equation*}
	with probability at least $1 - \Delta$.}

\textcolor{red}{Multiplicative bound: The two terms on the right hand side are both distributed $\mathrm{Bernoulli}(p/2)$. Moreover, since $\1(x_i \in A_\sigma) = 1$ implies $\1(x_j \in A_{\sigma}) = 0$, they have negative covariance. We can therefore upper bound $\var(k'(x_i,x_j)) \leq p$, and so from Lemma \ref{lem: bernstein}, we have
	\begin{equation*}
	\frac{\mathcal{V}}{{n \choose 2}} \geq p - \max\left\{ \sqrt{\frac{4\log(1/\Delta)p}{n}}, \frac{4 \log(1/\Delta)}{3n} \right\}
	\end{equation*}
	with probability at least $1 - \Delta$. }
\end{document}