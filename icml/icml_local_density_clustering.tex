%%%%%%%% ICML 2019 submission %%%%%%%%%%%%%%%%%

\documentclass{article}

\usepackage{icml2019}
% \usepackage[accepted]{icml2019}

\usepackage{microtype}
\usepackage{graphicx}
\usepackage{subfigure}
\usepackage{booktabs}
\usepackage{hyperref}
\usepackage{amsmath}
\usepackage{amsfonts, amsthm, amssymb}
\usepackage{graphicx}
\usepackage[parfill]{parskip}
\usepackage{enumerate}
\usepackage[shortlabels]{enumitem}
\usepackage{bm}

\newcommand{\diam}{\mathrm{diam}}
\newcommand{\set}[1]{\left\{#1\right\}}
\newcommand{\defeq}{\overset{\mathrm{def}}{=}}
\newcommand{\vol}{\mathrm{vol}}
\newcommand{\cut}{\mathrm{cut}}
\newcommand{\abs}[1]{\left \lvert #1 \right \rvert}
\newcommand{\N}{\mathbb{N}}
\newcommand{\Reals}{\mathbb{R}}
\newcommand{\Rd}{\Reals^d}
\newcommand{\norm}[1]{\left\lVert#1\right\rVert}
\newcommand{\1}{\mathbf{1}}
\newcommand{\Phibf}{\mathbf{\Phi}}
\newcommand{\Psibf}{\mathbf{\Psi}}
\newcommand{\dist}{\mathrm{dist}}

%%% Vectors
\newcommand{\pbf}{\mathbf{p}}
\newcommand{\qbf}{\mathbf{q}}
\newcommand{\ebf}[1]{\mathbf{e}_{#1}}
\newcommand{\pibf}{\bm{\pi}}

%%% Matrices
\newcommand{\Abf}{\mathbf{A}}
\newcommand{\Xbf}{\mathbf{X}}
\newcommand{\Wbf}{\mathbf{W}}
\newcommand{\Lbf}{\mathbf{L}}
\newcommand{\Dbf}{\mathbf{D}}
\newcommand{\Ibf}[1]{\mathbf{I}_{#1}}

%%% Probability distributions (and related items)
\newcommand{\Pbb}{\mathbb{P}}
\newcommand{\Cbb}{\mathbb{C}}

%%% Sets
\newcommand{\Cset}{\mathcal{C}}
\newcommand{\Aset}{\mathcal{A}}
\newcommand{\Asig}{\Aset_{\sigma}}
\newcommand{\Csig}{\Cset_{\sigma}}

%%% Operators
\DeclareMathOperator*{\argmin}{arg\,min}

%%% Algorithm notation
\newcommand{\ppr}{{\sc PPR}}
\newcommand{\pprspace}{{\sc PPR~}}

\newtheoremstyle{aldenthm}
{6pt} % Space above
{6pt} % Space below
{\itshape} % Body font
{} % Indent amount
{\bfseries} % Theorem head font
{.} % Punctuation after theorem head
{.5em} % Space after theorem head
{} % Theorem head spec (can be left empty, meaning `normal')

\theoremstyle{aldenthm}
\newtheorem{theorem}{Theorem}
\newtheorem{definition}{Definition}
\newtheorem{lemma}{Lemma}
\newtheorem{corollary}{Corollary}

\newtheoremstyle{aldenrmrk}
{6pt} % Space above
{6pt} % Space below
{} % Body font
{} % Indent amount
{\itshape} % Theorem head font
{.} % Punctuation after theorem head
{.5em} % Space after theorem head
{} % Theorem head spec (can be left empty, meaning `normal')

\theoremstyle{aldenrmrk}
\newtheorem{remark}{Remark}
\newcommand{\theHalgorithm}{\arabic{algorithm}}

%%%%%
\icmltitlerunning{Local Spectral Clustering of Density Upper Level Sets}

\begin{document}

\twocolumn[
\icmltitle{Local Spectral Clustering of Density Upper Level Sets}
\icmlsetsymbol{equal}{*}

\begin{icmlauthorlist}
\icmlauthor{Alden Green}{cmu}
\icmlauthor{Sivaraman Balakrishnan}{cmu}
\icmlauthor{Ryan J. Tibshirani}{cmu}
\end{icmlauthorlist}

\icmlaffiliation{cmu}{Department of Statistics and Data Science, Carnegie Mellon
  University, Pittsburgh PA, USA} 
\icmlcorrespondingauthor{Alden Green}{ajgreen@andrew.cmu.edu}
\icmlkeywords{local clustering}

\vskip 0.3in
]

\printAffiliationsAndNotice{}

\begin{abstract}
\vskip 0.1 in % RJT: there was no space here and it looked very squished 
Spectral clustering methods are a family of popular nonparametric clustering
tools.  Recent works have proposed and analyzed \emph{local} spectral methods,
which extract clusters using locally-biased random walks around a user-specified
seed node.  Several authors have shown that local methods, such as personalized
PageRank (PPR), have worst-case guarantees for certain graph-based measures of
cluster quality.  In contrast to existing works, we analyze PPR in a traditional
statistical learning setup, where we obtain samples from an unknown
distribution, and aim to identify connected regions of high-density (density
clusters).  We introduce two natural criteria for cluster quality, and derive
bounds for these criteria when evaluated on empirical analogues of density
clusters. Moreover, we prove that PPR, run on a neighborhood graph, extracts
sufficiently salient density clusters.
\end{abstract}

\section{Introduction}
\label{sec: introduction}

Let $\mathbf{X} = \{x_1, \ldots, x_n\}$ be a sample drawn i.i.d.\ from a
distribution $\Pbb$ on $\Rd$, with density $f$, and consider the problem of 
clustering: splitting the data into groups which satisfy some notion of
within-group similarity and between-group difference.  We focus on spectral
clustering methods, a family of powerful nonparametric clustering algorithms.
Roughly speaking, a spectral technique first constructs a geometric graph $G$,
where vertices are associated with samples, and edges correspond to proximities
between samples. It then learns a feature embedding based on the Laplacian of
$G$, and applies a simple clustering technique (such as k-means clustering) in
the embedded feature space.

To be more precise, let $G=(V,E,w)$ denote a weighted, undirected graph  
constructed from the samples $\mathbf{X}$, where $V=\{1,\ldots,n\}$, and $w_{uv}
= K(x_u,x_v) \geq 0$ for $u,v \in V$, and a particular kernel function $K$.
Here $(u,v) \in E$ if and only if $w_{uv} > 0$.  We denote by $\Abf \in
\Reals^{n \times n}$ the weighted adjacency matrix, which has entries
$A_{uv}=w_{uv}$, and by $\Dbf$ the degree matrix, with 
\smash{$\Dbf_{uu} = \sum_{v \in V} \Abf_{uv}$}.  We also denote by $\Wbf,\Lbf$
the random walk transition probability matrix and normalized\footnote{Other
  popular choices here include the unnormalized Laplacian, and symmetric
  normalized Laplacian.} 
Laplacian matrix, respectively, which are defined as
$$
\Wbf = \Dbf^{-1}\Abf, \quad \Lbf = \Ibf{} - \Wbf,
$$
where $\Ibf{} \in \Reals^{n\times n}$ is the identity matrix.  Classical global
spectral methods take a eigendecomposition $L=U \Sigma U^T$, use some 
number of eigenvectors (columns in $U$) as a feature representation for the
samples, and then run (say) k-means in this new feature space.

When applied to geometric graphs constructed from a large number of samples,
global spectral clustering methods can be been computationally cumbersome and   
insensitive to the local geometry of the underlying distribution
\citep{leskovec2010,mahoney2012}.  This has led to recent increased interest in
local spectral algorithms, which leverage locally-biased spectra computed using
random walks around a user-specified seed node.  A popular local clustering
algorithm is Personalized PageRank (PPR), first introduced by
\citet{haveliwala2003}, and further developed by
\citet{spielman2011,spielman2014,andersen2006,mahoney2012,zhu2013},
among others.  

Local spectral clustering techinques have been practically very successful
\citep{leskovec2010,andersen2012,gleich2012,mahoney2012,wu2012}, which has led
many authors to develop supporting theory
\citep{spielman2013,andersen2009,gharan2012,zhu2013} that gives worst-case
guarantees on traditional graph-theoretic notions of cluster quality (like
conductance).  In this paper, we adopt a more traditional statistical viewpoint,
and examine what the output of a local clustering algorithm on $\Xbf$ reveals
about the unknown density $f$.  In particular, we examine the ability of the PPR
algorithm to recover \emph{density clusters} of $f$, which are defined as the
connected components of the upper level set $\{x \in \Rd : f(x) \geq \lambda\}$
for some threshold $\lambda > 0$ (a central object of central interest in the
classical statistical literature on clustering, dating back to
\citealt{hartigan1981}).

\subsection{Graph Connectivity Criteria}

Here we define a pair of criteria that reflect the quality of a cluster with
respect to $G=(V,E,w)$.  There are many graph-based measures of cluster quality  
that one could consider; see, e.g., \citet{yang2015,fortunato2010} for 
an overview.  The pair of criteria that we focus on are (arguably) quite
natural, and moreover, they play a fundamental role in our analysis 
of the PPR algorithm.  Our two critera capture the \emph{external} and
\emph{internal} connectivity of a subset $S \subseteq V$, denoted $\Phi(S; G)$
and $\Psi(S; G)$, respectively, and defined below in turn.  

\paragraph{External Connectivity: Normalized Cut.}

Define the cut between subsets $S,S' \subseteq V$ to be
$$
\cut(S, S'; G) = \sum_{u \in S} \sum_{v \in S'} w_{uv},
$$
and define
\smash{$\vol(S; G) = \cut(S, V; G) = \sum_{u \in S} \sum_{v \in V} w_{uv}$}.  
As our notion of external connectivity, we use the \emph{normalized cut}  
of $S$, defined as 
\begin{equation}
\label{eqn: norm_cut}
\Phi(S; G) = \frac{\cut(S; G)}{\min\{\vol(S; G), \vol(S^c; G)\}},
\end{equation} 
where we abbreviate $\cut(S; G) = \cut(S; S^c; G)$.

\paragraph{Internal Connectivity: Inverse Mixing Time.}

For $S \subseteq V$, denote by $G[S] = (S, E_S, w_S)$ the subgraph induced by 
$S$ (where the edges are $E_S = E \cap (S \times S)$). Let $\Abf_S,\Dbf_S$
be the adjacency matrix and degree matrix, respectively, of $G[S]$.  Define the
random walk matrix as usual, $\Wbf = \Dbf_S^{-1} \Abf_S$, and for $v \in V$,
write 
\begin{equation}
\label{eqn: random_walk}
q_{vu}^{(t)} = e_v\Wbf_S^t e_u
\end{equation}
for the $t$-step transition probability of a random walk over $G[S]$
originating at $v$.\footnote{Given a starting node $v$ and and a random walk
  defined by transition probability matrix $\mathbf{P}$, the rotation $e_v
  \mathbf{P}^t$ is used to denote the distribution of the random walk after $t$
  steps.}   Also write \smash{$\tilde\pibf = (\tilde\pi_{u})_{u \in S}$} for the 
  stationary distribution of this random walk.  (Given the definition of
  $\Wbf_S$, it is well-known that the stationariy distribution is given by
  \smash{$\tilde\pi_{u} = (\Dbf_{S})_{uu} / \vol(S; G[S])$}.) 

Our internal connectivity parameter will capture the time it takes for the
random walk over $G[S]$ to mix (approach the stationary distribution)
uniformly over $S$.  For this, we first define \emph{relative pointwise mixing
  time} of $G[S]$ as 
\begin{equation*}
\tau_{\infty}(\qbf; G[S]) = \min\set{ t: \frac{|q_{vu}^{(m)} -
      \tilde\pi_u|}{\tilde\pi_u} \leq \frac{1}{4}, \; \text{for $u,v \in V$}}, 
\end{equation*}
where \smash{$\qbf = (\qbf_v^{(1)}, \qbf_v^{(2)}, ...)_{v \in V}$}, and
\smash{$\qbf_v^{(m)} = (q_{vu}^{(m)})_{u \in V}$}. Now our internal 
connectivity parameter is simply the inverse mixing time,
\begin{equation}
\label{eqn: inv_mixing_time}
\Psi(S; G) = \frac{1}{\tau_{\infty}(\qbf; G[S])}.
\end{equation}

If $S$ has normalized cut no greater than $\Phi$, and inverse mixing time no
less than $\Psi$, we call it as a \emph{$(\Phi,\Psi)$-cluster}. Both
local \citep{zhu2013} and global \citep{kannan04} spectral algorithms have been
shown to output clusters (or partitions) which approximate the optimal $(\Phi,
\Psi)$-cluster (or partition) for a given graph $G$.\footnote{In the case of
  \citet{kannan04}, the internal connectivity parameter $\phi$ is actually the
  conductance, i.e., the minimum normalized cut within the subgraph $G[S]$. See
  Theorem 3.1 in their paper for details; however, note that $\phi^2 /
  \log(\vol(S)) \leq O(\Psi)$, and so the lower bound on $\phi$ translates to a
  lower bound on $\Psi$.}   

\subsection{PPR on a Neighborhood Graph}

We now describe the clustering algorithm that will be our focus for the rest of 
the paper. We start with the geometric graph that we form based on the samples 
$\Xbf$: for a radius $r > 0$, we consider the \emph{$r$-neighborhood graph} of 
$\Xbf$, denoted $G_{n,r}=(V,E)$, an unweighted graph with vertices
$V=\{1,\ldots,n\}$, and an edge $(u,v) \in E$ if and only if $\norm{x_u - x_v}
\leq  r$, where $\norm{\cdot}$ denotes Euclidean norm.  Note that this is a
special case of the general construction introduced above, with 
$K(u,v) = 1(\norm{x_u - x_v} \leq r)$. 

Next, we define the PPR vector $\pbf = \pbf(v,\alpha;G_{n,r})$, with respect to  
a seed node $v \in V$ and a teleportation parameter $\alpha \in [0,1]$, to be
the solution of the following linear system:
\begin{equation}
\label{eqn: ppr_vector}
\pbf = \alpha \ebf{v} + (1 - \alpha) \pbf \Wbf,
\end{equation}
where $\Wbf$ is the random walk matrix of the underlying graph $G_{n,r}$ 
and $e_{v}$ denotes indicator vector for node $v$ (with a 1 in the $v$th
position and 0 elsewhere).  In practice, we can approximately solve the above
linear system via a simple, efficient random walk, with appropriate restarts to
$v$. 

For a level $\beta > 0$ and a target volume $\pi_0 > 0$, we define a
\emph{$\beta$-sweep cut} of $\pbf$ as  
\begin{equation}
\label{eqn: sweep_cuts}
S_\beta = \{u \in V: p_u > \beta \pi_{0}\}.
\end{equation}
Having computed sweep cuts over a range \smash{$\beta \in  
(\frac{3}{10},\frac{1}{2})$},\footnote{The choice of a specific range such as 
\smash{$(\frac{3}{10}, \frac{1}{2})$} is standard in the analysis of PPR
algorithms, see, e.g., \citet{zhu2013}.}
we output a cluster \smash{$\widehat{C} = S_{\beta^*}$}, based on the sweep cut
$S_{\beta^*}$ that minimizes the normalized cut \smash{$\Phi(S_{\beta^*};
  G_{n,r})$} as defined in \eqref{eqn: norm_cut}. For concreteness, we summarize
this procedure in Algorithm \ref{alg: ppr}.   

\begin{algorithm}
\caption{PPR on a Neighborhood Graph}
\label{alg: ppr}	
{\bfseries Input:} data $\Xbf=\{x_1,\ldots,x_n\}$, radius $r > 0$, teleportation 
parameter $\alpha \in [0,1]$, seed $v \in \Xbf$, target volume $\pi_0 >
0$. \\   
{\bfseries Output:} cluster $\widehat{C} \subseteq V$.
\begin{algorithmic}[1]
  \STATE Form the neighborhood graph $G_{n,r}$.
  \STATE Compute the PPR vector $\pbf(v, \alpha; G_{n,r})$ as in \eqref{eqn:
    ppr_vector}. 
  \STATE For $\beta \in (\frac{3}{10}, \frac{1}{2})$ compute sweep cuts
  $S_{\beta}$ as in \eqref{eqn: sweep_cuts}.
  \STATE Return \smash{$\widehat{C} = S_{\beta^*}$}, where 
  $$
  \beta^* = \argmin_{\beta \in (\frac{3}{10}, \frac{1}{2})} \Phi(S_{\beta}; G_{n,r}).
  $$
\end{algorithmic}
\end{algorithm}

\subsection{Summary of Results}

% It is worth calling attention to some other work on computing the normalized
% cut over neighborhood graphs. In this context, continuous analogues to (for
% instance) normalized cut have been defined, over the data-manifold rather than
% the graph, and convergence finite sample graph-theoretic functionals to their
% continuous counterparts has been shown 
% \cite{garciatrillos16, arias-castro12, maier11}. 
% However, in addition to the graph-minimization problem being computationally
% infeasible, these continuous analogues are not always easily interpretable --
% and their corresponding minimizers not always easily identifiable -- for the
% particular density function under consideration. Of course, relating these
% partitions to the arguably more simply defined high density clusters can be
% also challenging in general. Intuitively, however, under the right conditions
% such high-density clusters should have more edges within themselves than to
% the remainder of the graph. We formalize this intuition next. 

%% RJT: I didn't know where to put this.  It was out of place, and now I'm not
%% sure where it goes ... should it go in related work?

%It is worth pointing out that in this context, some theory has been developed
%regarding how graph theoretic quantities such as the normalized cut $\Phi$ (and
%others) relate to properties of the underlying distribution $f$ as well as the
%kernel function $k$. Such analyses typically proceed by defining a continuous
%analogue to the measure of cluster quality under consideration. Then, under
%appropriate specification of $k$ and a proper schedule of $(r_n)$, convergence
%of clusters output by spectral (and other) algorithms to the corresponding
%minima of these continuous analogues has been shown \cite{vonluxburg2008,
%garciatrillos18}. 

%% RJT: This was already commented out before

Let $\Cbb_f(\lambda)$ denote the connected components of the density upper level
set $\{x \in \Rd: f(x) > \lambda\}$.  For a given density cluster $\Cset \in
\Cbb_f(\lambda)$, we call $\Cset[\Xbf] = \Cset \cap \Xbf$ the \emph{empirical
  density cluster}. Below we define a notion of consistency in density cluster
estimation.     

\begin{definition}[Consistent density cluster estimation]
\label{def: consistent_density_cluster_estimation}
For an estimator \smash{$\widehat{\Cset} \subseteq \Xbf$} and cluster 
$\Cset \in \Cbb_f(\lambda)$, we say \smash{$\widehat{\Cset}$} is a consistent
estimator of $\Cset$ if for all $\Cset' \in \Cbb_f(\lambda)$ with $\Cset \not=
\Cset'$ the following holds as $n \to \infty$: 
\begin{equation}
\label{eqn: consistent_density_cluster_recovery}
\Cset[\Xbf] \subseteq \widehat{\Cset} \quad \text{and} \quad
\widehat{\Cset} \cap \Cset'[\Xbf] = \emptyset,
\end{equation}
with probability tending to 1.
\end{definition}

A summary of our main results (and outline for the rest of this paper) is as
follows.  

\begin{enumerate}
\item In Section \ref{sec: PhiPsi}, we derive in Theorem \ref{thm:
    conductance_upper_bound} an upper bound on the normalized cut of a
  (thickened) empirical density cluster $\Csig[\Xbf]$, under natural geometric 
  conditions (precluding clusters that are too thin and long).  

\item Under largely the same set of geometric conditions, we derive in Theorem
  \ref{thm: inverse_mixing_time_lower_bound_nonconvex} a lower bound on the
  inverse mixing time of a random walk over $\Csig[\Xbf]$.  We also provide in 
  Theorem \ref{thm: inverse_mixing_time_lower_bound} a tighter lower bound, but
  under more restrictive assumptions (convexity of $\Csig$).
	
\item In Section \ref{sec: consistent_cluster_estimation_with_ppr}, we show in
  Theorem \ref{thm: consistent_recovery_of_density_clusters} that these bounds
  in bounds in Theorems \ref{thm: conductance_upper_bound} and \ref{thm:
    inverse_mixing_time_lower_bound_nonconvex},
  on the cluster quality criteria, have algorithmic consequences for PPR:
  properly initialized, Algorithm \ref{alg: ppr} performs consistent density
  cluster estimation in the sense of \eqref{eqn:
    consistent_density_cluster_recovery}. 
	
\item We show in Corollary \ref{cor: ppr_cluster} that Theorems \ref{thm:
    conductance_upper_bound} and \ref{thm:
    inverse_mixing_time_lower_bound_nonconvex}, along with the results in 
  \citet{zhu2013}, lead to alternative, graph-theoretic guarantees on cluster 
  quality: an upper bound on the normalized cut of the estimated cluster 
  \smash{$\widehat{C}$}, and an upper bound on volume of the symmetric set
  difference between \smash{$\widehat{C}$} and $\Cset[\Xbf]$.    

\item In Section \ref{section: examples}, we discuss the implications of our
  results for some example density functions of interest, and empirically
  demonstrate that violations of the gometric conditions we require manifestly
  impact density cluster recovery. 
\end{enumerate}

On the topic of conditions, it is worth mentioning that, as density clusters
are inherently local, focusing on the PPR algorithm actually eases our analysis
and allows us to require fewer global regularity conditions relative to those
needed for more classical global spectral algorithms.    

\subsection{Related Work}

In addition to the background given above, a few related lines of work are worth
highlighting. Global spectral clustering methods were first developed in the
context of graph partitioning \citep{fiedler1973,donath1973} and their
performance is well-understood in this context (see, e.g.,
\citealt{tolliver2006,luxburg2007}).  In a similar vein, several recent works
\citep{mcsherry2001,rohe2011,kamalika2012,balakrishnan2011,lei2015,abbe2018} 
have studied the efficacy of spectral methods in successfully recovering the
community structure in the stochastic block model and variants.

\citet{vonluxburg2008,hein2005}, building on earlier work of
\citet{koltchinskii2000}, studied the limiting behaviour of spectral clustering
algorithms. These authors show that when samples are obtained from a
distribution, and we appropriately construct a geometric graph, the spectrum of
the Laplacian converges to that of the Laplace-Beltrami operator on the
data-manifold. However, relating the partition obtained using the
Laplace-Beltrami operator to the more intuitively defined high-density
clusters can be challenging in general.

Perhaps most similar to our results are the works
\citet{vempala2004,shi2009,schiebinger2015}, who study the consistency of
spectral algorithms in recovering the latent labels in certain parametric and
nonparametric mixture models. These results focus on global rather than local
algorithms, and as such impose global rather than local conditions on the nature
of the density. Moreover, they do not in general ensure recovery of density
clusters, which is the focus in our work. 

\section{Cluster Quality Critera Bounds for Density Clusters}  
\label{sec: PhiPsi}

\subsection{Assumptions}

In order to provide meaningful bounds on the normalized cut and inverse mixing
time of an empirical density cluster, we must introduce conditions on the
density $f$. Let $B(x,r) = \{y \in \Rd: \norm{y - x} \leq r\}$ be the ball of
radius $r$ centered at $x$.  Given a set $\Aset \subseteq \Rd$ and $\sigma > 0$,
define $\Asig = \Aset + B(0,\sigma) = \{y \in \Rd: \inf_{x \in \Aset} \norm{y - 
    x} \leq \sigma\}$, which we call the $\sigma$-expansion of $\Aset$. 

We are now ready to give the conditions, stated with respect to a density
cluster $\Cset \in \Cbb_f(\lambda)$, for some $\lambda > 0$, and an expansion 
parameter $\sigma > 0$. 

\begin{enumerate}[label=(A\arabic*)]
\item
\label{asmp: bounded_density}
\emph{Bounded density within cluster:} There are $0 < \lambda_{\sigma} <
\Lambda_{\sigma} < \infty$ such that
$$
\lambda_{\sigma} = \inf_{x \in \Csig} f(x) \leq \sup_{x \in \Csig} f(x) \leq
\Lambda_{\sigma}.
$$
% and 
% \begin{equation*}
% \frac{\diam \Asig}{\sigma} \leq \mu
% \end{equation*}
% where $\diam \Asig = \sup \set{d(x,y) : x,y \in \Asig}$
	
\item 
\label{asmp: low_noise_density}
\emph{Low noise density:} There exist $\gamma, c_1 > 0$ such that for all $x
\in \Rd$ with $0 < \dist(x, \Csig) \leq \sigma$,   
$$
  \inf_{x' \in \Csig} f(x') - f(x) \geq c_1 \dist(x, \Csig)^{\gamma},
$$
where \smash{$\dist(x,\Aset) = \inf_{x_0 \in \Aset} \norm{x - x_0}$}.
% RJT: should we change this to c_0?  c_1 seems to be used below and throughout
% for other constants, and c_0 is not used

\item
\label{asmp: cluster_separation}
\emph{Cluster separation:}
For all $\Cset' \in \Cbb_f(\lambda)$,
$$
\dist(\Csig,\Csig') > \sigma,
$$
where \smash{$\dist(\Aset,\Aset') = \inf_{x \in \Aset} \dist(x,\Aset')$}.  
	
\item
\label{asmp: cluster_diameter}
\emph{Cluster diameter:}
There exists $D < \infty$ such that for all $x, x' \in \Csig$,
$$
\norm{x - x'} \leq D.
$$
\end{enumerate}

Note that $\sigma$ plays several roles here, precluding arbitrarily narrow
clusters and long clusters in \ref{asmp: bounded_density} and \ref{asmp:
  cluster_diameter}, flat densities around the level set in \ref{asmp: 
  low_noise_density}, and poorly separated clusters in \ref{asmp:
  cluster_separation}. 

Assumptions \ref{asmp: bounded_density}, \ref{asmp: low_noise_density}, and
\ref{asmp: cluster_separation} are used to upper bound $\Phi(\Cset[\Xbf];
G_{n,r})$, whereas \ref{asmp: bounded_density} and \ref{asmp: cluster_diameter} 
are required to lower bound $\Psi(\Cset[\Xbf]; G_{n,r})$. We note that the
lower bound on minimum density in \ref{asmp: bounded_density} and \ref{asmp:  
cluster_separation} combined are similar to the $(\sigma,\epsilon)$-saliency of
\citet{chaudhuri2010}, a standard density clustering assumption, while
\ref{asmp: low_noise_density} is seen in, e.g., \citet{singh2009} (as well as
many other works on density clustering and level set estimation.) It is worth
highlighting that these assumptions are all local in nature, a benefit of
studying a local algorithm such as PPR.

In the next several subsections, we will derive bounds on the cluster quality
criteria evaluated on (thickened) density clusters. For notational simplicity,
hereafter for $S \subseteq V$, we will denote $\Phi(S; G_{n,r})$ by
$\Phi_{n,r}(S)$, and likewise with $\Psi(S; G_{n,r})$ and $\Psi_{n,r}(S)$. We
will also use $\nu$ for Lebesgue measure on $\Rd$, and $\nu_d = \nu(B)$ for 
the measure of the unit ball $B=B(0,1)$. 
% RJT: you had B(0,d) before ... just check this should have been B(0,1) 

\subsection{Upper Bound on Normalized Cut}

We begin by upper bounding the normalized cut of $\Cset_\sigma[\Xbf]$.  (In the
theorem, the upper bound on the density in Assumption \ref{asmp: 
  bounded_density} will not actually be needed, so we omit the parameter
$\Lambda_\sigma>0$ from the theorem statement.) 

\begin{theorem}
\label{thm: conductance_upper_bound}
Fix $\lambda > 0$, and let $\Cset \in \Cbb_f(\lambda)$ satisfy
Assumptions \ref{asmp: bounded_density}, \ref{asmp: low_noise_density}, and
\ref{asmp: cluster_separation}, for some 
$\sigma, \lambda_{\sigma}, c_1, \gamma > 0$. Then for any $0 < r <
\sigma/(4d)$, $0 < \delta < 1$, $\epsilon > 0$, and
\begin{equation}
\label{eqn: conductance_sample_complexity}
n \geq \frac{9\log(2/\delta)}{\epsilon^2}\left(\frac{1}
  { \lambda_{\sigma}^2\nu(\Csig) \nu_d r^d}\right)^2,
\end{equation}
we have
\begin{equation}
  \label{eqn: conductance_additive_error_bound}
  \frac{\Phi_{n,r}(\Csig[\mathbf{X}])}{r} \leq 4 c_{\sigma} d
  \frac{\lambda}{\lambda_{\sigma}} \frac{(\lambda_{\sigma} -
    \frac{r^{\gamma}}{\gamma+1})}{\lambda_{\sigma}} + \epsilon, 
\end{equation}
with probability at least $1-\delta$, where $c_{\sigma} = 1 / \sigma$. 
\end{theorem}

\begin{remark}
The proof of Theorem \ref{thm: conductance_upper_bound}, along with all other
proofs in this paper, can be found in the supplementary document. The key 
idea is that for any $x \in \Cset$, the simple (possibly loose) fact
$B(x,\sigma) \subseteq \Csig$ translates into the upper bound $\nu(\Cset_{\sigma
  + r}) \leq (1 + 2d r /\sigma)\nu(\Csig)$. We leverage \ref{asmp:
  low_noise_density} to find a corresponding bound on the weighted volume,
then apply standard concentration inequalities to convert from population-
to sample-based results.  
\end{remark}

\begin{remark}
The inequality in \eqref{eqn: conductance_additive_error_bound} is almost
tight. Specifically, choosing $\Asig = B(0,\sigma)$ and
$$
f(x) = \lambda - \dist(x,\Asig)^{\gamma},
$$
% RJT: is this correct (is this what you meant)?  Original formula is below
% $$
% f(x)  = \begin{cases}
%   \lambda &\text{for $x \in \Asig$}, \\
%   \lambda - \dist(x,\Asig)^{\gamma} & \text{ for $0 < \dist(x,\Asig) < r$}, 
% \end{cases}
% $$
we have that for $n$ on the order of the lower bound in \eqref{eqn: 
  conductance_sample_complexity}, 
\begin{equation*}
  \frac{\Phi_{n,r}(\Asig[\mathbf{X}])}{r} \geq c \frac{(\lambda -
    \frac{r^{\epsilon+1}}{\epsilon+1})}{\lambda} - \epsilon,
\end{equation*}
with probability at least $1 - \delta$, for some constant $c$. (Note that the
factor of $1 / \sigma$ in $c_{\sigma}$ is not replicated above.)
\end{remark}

\subsection{Lower Bound on Inverse Mixing Time}

We lower bound the inverse mixing time of $\Csig[\Xbf]$.

\begin{theorem}
\label{thm: inverse_mixing_time_lower_bound_nonconvex}
Fix $\lambda > 0$, and let $\Cset \in \Cbb_f(\lambda)$ satisfy
Assumptions \ref{asmp: bounded_density} and \ref{asmp: cluster_diameter} 
for some $\sigma, \lambda_{\sigma}, \Lambda_{\sigma}, D > 0$. Then for any $0 <
r < \sigma/(4d)$, $0 < \delta < 1$, $\epsilon > 0$, and $n$ satisfying 
$$
\sqrt{3^{d+1}\frac{(\log n + d\log \mu + \log(4/\delta))}{n \nu_d r^d
    \lambda_{\sigma}}} \leq \epsilon, 
$$
where \smash{$\mu = \log(\frac{2D}{r})$}, we have
\begin{multline}
\label{eqn: inverse_mixing_time_lower_bound_nonconvex}
\Psi_{n,r}(\Csig[\Xbf]) \geq \left(d \log \mu + c_{\lambda} \right)^{-1} \cdot \\
\left(c_1 + c_2 \frac{\Lambda_{\sigma}^4D^{2d}}{\lambda_{\sigma}^4r^{2d}}\left(d
    \log \mu + c_{\lambda}\right)\right)^{-1},
\end{multline}
% RJT: confused ... is this correct?
with probability at least $1 - \delta$, where $c_1,c_2,c_3>0$ are constants
depending only on the dimension $d$, and $c_{\lambda} = \log(\Lambda_{\sigma}^2/
\lambda_{\sigma}^2)$.   
\end{theorem}
% RJT: saying that c_1,c_2,c_3 depend on the dimension is a little odd/weak.
% c_2 multiplyies, e.g., a factor of d \log \mu.  So if it's d^{10}, then this
% will swamp the linear factor of ....

\begin{remark}
The proof of Theorem \ref{thm: inverse_mixing_time_lower_bound_nonconvex} relies
on upper bounding the mixing time using the \emph{conductance} of $G_{n,r}[\Csig[X]]$, 
$$
\widetilde{\Phi} = \min_{S \subseteq \Csig[\Xbf]} \Phi(S; G_{n,r}[\Csig[X]]).
$$
The factor $1/r^{2d}$ in the lower bound in \eqref{eqn:
  inverse_mixing_time_lower_bound_nonconvex} is suboptimal. This exponential
dependence on $d$ stems from a loose bound on the aforementioned
conductance. In particular, we assert only that any set $S \subseteq \Csig[X]$
must have $\cut(S; \Csig[\Xbf])$ on the order of $n^2 r^{2d}$, while upper bounding
$\vol(S; \Csig[\Xbf])$ by roughly $n^2 r^d$, for a bound on the conductance of
order $r^d$. The presence of $r^{2d}$ comes from upper bounding the mixing time
by about \smash{$1/\widetilde{\Phi}^2$}, this being a variant of classic results
on rapid mixing \citep{jerrum89}.  
\end{remark}

\subsection{Tigher Lower Bound Under Convexity}

It is possible to sharpen the dependency on $d$ in 
\label{eqn: inverse_mixing_time_lower_bound_nonconvex}, but at the cost of an 
additional assumption: 
\begin{enumerate}[label=(A\arabic*)]
  \setcounter{enumi}{4}
\item 
  \label{asmp: convex}
  $\Cset$ is convex.
\end{enumerate}
% RJT: do we need C to satisfy this, or Csig?  Not sure I know why the former
% matters ... 

With \ref{asmp: convex} in place, we now give a tighter lower bound on the
inverse mixing time of $C_\sigma[\Xbf]$ (albeit one that holds only
asymptotically).  

\begin{theorem}
\label{thm: inverse_mixing_time_lower_bound}
Assume the conditions of Theorem \ref{thm:
  inverse_mixing_time_lower_bound_nonconvex}, and additionally, let $\Cset$
satisfy \ref{asmp: convex}. Then, 
for any $r < \sigma/4d$, the following statement holds with probability 1: 
\begin{align}
\label{eqn: inverse_mixing_time_lower_bound}
  \liminf_{n \to \infty} \, &\Psi_{n,r}(\Csig[\Xbf]) \geq \nonumber \\ 
&\quad c_1 \frac{\Lambda_{\sigma}^8}{\lambda_{\sigma}^8} (\log \mu +
  c_{\lambda}) \biggl( c_{\lambda}(d^3 \log\mu + c_2) +{} \nonumber \\  
&\quad \frac{d^2D^2}{r^2} \bigl( \log d + c_2
  \frac{\Lambda_{\sigma}^{2}}{\lambda_{\sigma}^{2}}c_{\lambda} +{} \nonumber \\  
&\quad c_3  + \log \log \mu \bigr) \biggr) + c_d \frac{o_r(1)}{r^2},
\end{align}
% RJT: confused ... is this correct?
where \smash{$\mu = \log(\frac{2D}{r})$}, $c_1,c_2,c_3>0$ are universal
constants, $c_d$ is constant in $r$ but depends on dimension $d$, and
$c_{\lambda} = \log(\Lambda_{\sigma}^2/ \lambda_{\sigma}^2)$.  
% Additionally, $\lim_{r \to 0} o_r(1) = 0$.  
\end{theorem}
% RJT: this was a lower bound on 1/\Psi and I changed it to \Psi ... just
% checking 

\begin{remark}
The only potentially exponential dependence on dimension comes from the
factor of $c_d$. However, for sufficiently small values of $r$ this will
be dominated by the preceding factors, due to the presence of the
$o_r(1)$ term. 
\end{remark}
% RJT: Don't get this comment.  You have o_r(1) / r^2, so this might still blow
% up as r -> 0, right?
 
\begin{remark}
We achieve superior rates in Theorem \ref{thm: inverse_mixing_time_lower_bound}
to those in Theorem \ref{thm: inverse_mixing_time_lower_bound_nonconvex} in part
by working with a generalization of the conductance, the \emph{conductance
  function},
\begin{equation*}
\widetilde{\Phi}_n(t) = 
\min_{\substack{S \subseteq \Csig[\Xbf] \\ \tilde\pi(S) \leq t}} \Phi(S; G_{n,r}[\Csig[X]]) 
\end{equation*}
where \smash{$\tilde\pi$} is the stationary distribution over
$G_{n,r}[\Csig[\Xbf]]$. The utility of the conductance function comes from the
known upper bound of mixing time by  
\begin{equation}
\label{eqn: average_conductance}
\int \frac{1}{t \widetilde{\Phi}_n(t)^2} dt,
\end{equation}
which results in a tighter bound than merely using the conductance when
\smash{$\widetilde{\Phi}_n(t)$} is large for small values of $t$. 
 
Our proof relies on a novel (to the best of our knowledge) uniform lower bound
on this conductance function over $G_{n,r}$ in terms of a population-level
analogue, which we term \smash{$\widetilde{\Phi}_{\Pbb,r}$}. Plugging
\smash{$\widetilde{\Phi}_{\Pbb,r}$} into \eqref{eqn: average_conductance} yields
a bound on mixing time (with respect to total variation distance) of order $d
D^2/r^2 + d^2\log(D/r)$ over convex sets. By contrast, \eqref{eqn:
  inverse_mixing_time_lower_bound} is of order $d^2 D^2/r^2 + d^3\log(D/r)$ 
(ignoring the $o_r(1)$ term) but handles mixing time with respect to relative
pointwise distance (which is known to be a stricter metric). 
\end{remark}

\begin{remark}
  The convexity requirement is necessary only to lower bound the
  \smash{$\widetilde{\Phi}_{\Pbb,r}$}; any bound on $\widetilde{\Phi}_{\Pbb,r}$
  which   does not require convexity could immediately be plugged into the
  machinery of   the proof of Theorem \ref{thm: inverse_mixing_time_lower_bound}
  to achieve a corresponding result. Recent work \citep{abbasi17} has developed
  such population-level bounds on the conductance function, however the Markov
  chain dealt with there is somewhat different than the one we consider.  
\end{remark}

\section{Consistent Cluster Estimation}
\label{sec: consistent_cluster_estimation_with_ppr}

For PPR to successfully recover density clusters, the ratio $\Phi_{n,r}(\Csig[\Xbf])/\Psi_{n,r}(\Csig[\Xbf])$ must be small. 

We introduce
\begin{align*}
\mathbf{\Phi}(\sigma, \lambda,\lambda_{\sigma},\gamma) & := \frac{\lambda}{\lambda_{\sigma}} \frac{(\lambda_{\sigma} - \frac{\sigma^{\gamma}}{\gamma+1})}{\lambda_{\sigma}} \\
1 / \mathbf{\Psi}(\sigma, \lambda_{\sigma}, \Lambda_{\sigma}, D) & := \left(d \log \mu + c_{\lambda} \right) \cdot \nonumber \\ 
& \left(c_1 + c_2 \frac{\Lambda_{\sigma}^4D^{2d}}{\lambda_{\sigma}^4r^{2d}}\left(d \log \mu + c_{\lambda}\right)\right)
\end{align*}

Well-conditioned density clusters satisfy all of the given assumptions, for parameters which results in `good' values of $\Phibf$ and $\Psibf$.
\begin{definition}[Well-conditioned density clusters]
	For $\lambda > 0$ and $\Cset \in \Cbb_f(\lambda)$, let $\Cset$ satisfy \ref{asmp: cluster_separation} - \ref{asmp: low_noise_density} with respect to parameters $\sigma, \lambda_{\sigma}, \gamma > 0$ and $\Lambda_{\sigma}, D < \infty.$ Letting $\kappa_1(\Cset)$ and $\kappa_2(\Cset)$ be given by
	\begin{align*}
	\kappa_1(\Cset) & := \frac{\mathbf{\Phi}(\sigma, \lambda,\lambda_{\sigma},\gamma)}{\mathbf{\Psi}(\sigma, \lambda_{\sigma}, \Lambda_{\sigma}, D)} \\
	\kappa_2(\Cset) & := \kappa_1(\Cset) \cdot \sqrt{\mathbf{\Psi}(\sigma, \lambda_{\sigma}, \Lambda_{\sigma}, D)},
	\end{align*}
	we call $\Cset$ a \textrm{$(\kappa_1, \kappa_2)$-well-conditioned density cluster (with respect to $\sigma, \lambda_{\sigma}, \gamma, \Lambda_{\sigma}$ and $D$).}
\end{definition}

$\Phibf$ and $\Psibf$ are familiar; they are exactly the upper and lower bounds on $\Phi_{n,r}(\Csig[\Xbf])$ and $\Psi_{n,r}(\Csig[\Xbf])$ derived in Theorems \ref{thm: conductance_upper_bound} and \ref{thm: inverse_mixing_time_lower_bound_nonconvex}, respectively.

\begin{remark}
	For convenience and maximum generality, we define $\mathbf{\Psi}(\sigma, \lambda_{\sigma}, \Lambda_{\sigma}, D)$ to correspond with the bound given by \eqref{eqn: inverse_mixing_time_lower_bound_nonconvex}, and assume only \ref{asmp: cluster_separation} - \ref{asmp: low_noise_density}. However, if we additionally have \ref{asmp: convex}, then we could sharpen $\mathbf{\Psi}(\sigma, \lambda_{\sigma}, \Lambda_{\sigma}, D)$ to the tighter rate of \eqref{eqn: inverse_mixing_time_lower_bound}, with nothing changing hereafter. 
\end{remark}

As is typical in the local clustering literature, our results will be stated with respect to specific choices or ranges of each of the user-specified parameters, which in this case may depend on the underlying (unknown) density. 

In particular, for a well conditioned density cluster $\Cset$ (with respect to some $\sigma, \lambda_{\sigma}, \gamma, \Lambda_{\sigma}$ and $D$), we require
\begin{align}
\label{eqn: initialization}
\alpha \in [1/10, 1/9] \cdot \mathbf{\Psi}(\sigma, \lambda_{\sigma}, \Lambda_{\sigma}, D), & ~r \leq \sigma/4d  \nonumber \\
\pi_0 \in [2/3, 6/5] \frac{\lambda_{\sigma}}{\nu(\Csig) \Lambda_{\sigma}^2}, & ~v \in \Csig[\Xbf]^g
\end{align}
where $\Csig[\Xbf]^g \subseteq \Csig[\Xbf]$ is some 'good' subset of $\Csig[\Xbf]$ which, as we will see, satisfies $\vol(\Csig[\Xbf]^g) \geq \vol(\Csig[\Xbf])/2$. (Intuitively one can think of $\Csig[\Xbf]^g$ as being the nodes sufficiently close to the center of $\Csig[\Xbf]$, although we provide no formal justification to this effect.)

\begin{definition}
	If the input parameters to Algorithm \ref{alg: ppr} satisfy \ref{eqn: initialization} with respect to some $\Csig[\Xbf]$, we say the algorithm is  \emph{well-initialized}.
\end{definition}

\begin{theorem}
	\label{thm: consistent_recovery_of_density_clusters}
	Fix $\lambda > 0$, and let $\Cset \in \Cbb_f(\lambda)$ be a $(\kappa_1,\kappa_2)$-well conditioned cluster (with respect to some $\sigma, \lambda_{\sigma}, \gamma, \Lambda_{\sigma}$ and $D$). If
	\begin{equation}
	\label{eqn: kappa2_ub}
	\kappa_2 \leq \frac{1}{40 \cdot 36} \frac{\lambda_{\sigma}^2}{\Lambda_{\sigma}^2} \frac{r^d \nu_d}{\nu(\Csig)}.
	\end{equation}
	and Algorithm \ref{alg: ppr} is well-initialized, the output set $\widehat{C} \subseteq \Xbf$ is a consistent estimator for $\Cset$, in the sense of Definition \ref{def: consistent_density_cluster_estimation}.
\end{theorem}

\begin{remark}
	We note that larger constants than $40 \cdot 36$ allow for wider range of the parameters in $\eqref{eqn: initialization}$. 
\end{remark}
\paragraph{Approximate cluster recovery via PPR.}

In \cite{zhu2013}, building on the work of \cite{andersen2006} and others, theory is developed which links algorithmic performance of PPR to the normalized cut and mixing time parameters. Although not the primary focus of our work, it is perhaps worth noting that these results, coupled with Theorems \ref{thm: conductance_upper_bound}-\ref{thm: inverse_mixing_time_lower_bound}, translate immediately into bounds on the normalized cut and symmetric set difference of $\widehat{C}$.  

We collect some of the main results of \cite{zhu2013} in Lemma \ref{lem: ppr_cluster}.

For $G = (V,E)$ consider some $A \subseteq V$, and let $\Phi(A; G)$ and $\Psi(A; G)$ be defined as in (\ref{eqn: norm_cut}) and (\ref{eqn: inv_mixing_time}), respectively.
\begin{lemma}[PPR clustering]
	\label{lem: ppr_cluster}
	There exists a set $A^g \subseteq A$ with $\vol(A^g;G) \geq \vol(A;G)/2$ such that the following statement holds: Choose any $v \in A^g$, fix $\alpha = 9 / 10 \Psi(A; G)$, and compute the page rank vector $\pbf(v,\alpha; G)$. Letting 
	\begin{equation*}
	\widehat{C} = \argmin_{\beta \in [\frac{1}{8}, \frac{1}{2}]} \Phi(S_{\beta}; G)
	\end{equation*}
	the following guarantees hold:
	\begin{align*}
	\vol(\widehat{C} \setminus A) & \leq \frac{24 \Phi(A; G)}{\Psi(A; G)} \vol(A) \\
	\vol(A \setminus \widehat{C})  & \leq \frac{30 \Phi(A; G)}{\Psi(A; G)} \vol(A) \\
	\Phi(\widehat{C}; G) & = O\left(\frac{\Phi(A; G)}{\sqrt{\Psi(A; G)}}\right)
	\end{align*}
\end{lemma}

Corollary \ref{cor: ppr_cluster} immediately follows.
\begin{corollary}
	\label{cor: ppr_cluster}
	Fix $\lambda > 0$, and let $\Cset \in \Cbb_f(\lambda)$ be a $(\kappa_1,\kappa_2)$-well conditioned cluster (with respect to some $\sigma, \lambda_{\sigma}, \gamma, \Lambda_{\sigma}$ and $D$). Then, if Algorithm \ref{alg: ppr} is well-initialized (in the sense that the choices of input parameters satisfy (\ref{eqn: initialization})), the following guarantees hold for output set $\widehat{C} \subseteq \Xbf$:
	\begin{align*}
	\vol(\widehat{C} \setminus \Csig[\Xbf]), \vol(\Csig[\Xbf] \setminus \widehat{C})  & \leq 30 \kappa_1(\Cset) \vol(\Csig[X]) \\
	\Phi_{n,r}(\widehat{C}) & = O\left(\kappa_2(\Cset)\right)
	\end{align*}
\end{corollary} 

\section{Examples}
\label{section: examples}

Example 1 is intended to show how the machinery developed above translates in a specific, common mixture model, and the extent to which bounds are (or are not) tight.  Example 2 \textcolor{red}{will try to} delve into some of the details of how \pprspace interpolates the conductance and density cut, and will show a case where a poorly conditioned density cluster is not recovered by \pprspace. Example 3 will emphasize finite sample cluster recovery, for a well-conditioned but non-convex mixture model

Examples 1 and 2 should be thought of as shedding light on the population performance of \ppr, whereas Example 3 shows performance on a finite sample.

\begin{enumerate}
	\item 
	\textit{Gaussian Mixture Model:} 
	We will compute optimal $\Phibf$ and $\Psibf$ for given $\lambda$, and show the following
	\begin{itemize}
		\item A graph comparing $\Phibf$ to $\Phi_{n,r}$ as the value of $\lambda$ changes.
		\item A graph comparing $\Psibf$ to $\Psi_{n,r}$ as the value of $\lambda$ changes.
		\item That for some values of $\lambda$, the conditions required for Theorem \ref{thm: consistent_recovery_of_density_clusters} hold.
	\end{itemize}

	\item
	\textit{Thin and long parallel clusters, with $\epsilon$-uniform noise: }
	We will \textcolor{red}{(try to)} show that the set outputted by \pprspace interpolates between the minimum normalized cut solution (fatter) and the density cluster (thinner). The \emph{conductance} is
	\begin{equation*}
	\Phi^{\star}(G_{n,r}) := \min_{C \subseteq \Xbf} \Phi_{n,r}(C)
	\end{equation*}
	and the conductance cut is $C^{\star} \subseteq \Xbf$ which achieves the minimum.
	
	We will show that
	\begin{itemize}
		\item For sufficiently small $\epsilon$, all three of the conductance cut, \pprspace cut, and density cut agree.
		\item For an intermediate value of $\epsilon$, the conductance cut and the density cut disagree. The \pprspace cut interpolates between the two.
		\item For a sufficiently large value of $\epsilon$, the \pprspace cut fails to recover the density cut, and draws closer to the conductance cut.
	\end{itemize}

	\item 
	\textit{Non-convex mixture model:} We will show that, for well-conditioned non-convex mixture model, and a finite sample size $n$, cluster recovery is achieved with high probability over repeated simulations.
\end{enumerate}

\section{Discussion}
\label{sec: discussion}
For a clustering algorithm and a given object (such as a graph or set of points), there are an almost limitless number of ways to define what the 'right' clustering is. We have considered a few such ways -- density level sets, and the bicriteria of normalized cut, inverse mixing time -- and shown that under the right conditions, the latter agree with the former, with resulting algorithmic consequences.

There are still many directions worth pursuing in this area. Concretely, we might wish to generalize our results to hold over a wider range of kernel functions, and hyperparameter inputs to the \pprspace algorithm. More broadly, we do not provide any sort of theoretical lower bound, although we give empirical evidence in Example 2 that poorly conditioned density clusters are not consistently estimated by \pprspace. Example 2 also hints at a way of understanding local spectral algorithms -- or, at least, \ppr-- as interpolating between normalized and density cut. Exploring this connection is an avenue for future work.

\clearpage

\bibliography{icml_bib}
\bibliographystyle{icml2019}


\end{document}


% This document was modified from the file originally made available by
% Pat Langley and Andrea Danyluk for ICML-2K. This version was created
% by Iain Murray in 2018, and modified by Alexandre Bouchard in
% 2019. Previous contributors include Dan Roy, Lise Getoor and Tobias
% Scheffer, which was slightly modified from the 2010 version by
% Thorsten Joachims & Johannes Fuernkranz, slightly modified from the
% 2009 version by Kiri Wagstaff and Sam Roweis's 2008 version, which is
% slightly modified from Prasad Tadepalli's 2007 version which is a
% lightly changed version of the previous year's version by Andrew
% Moore, which was in turn edited from those of Kristian Kersting and
% Codrina Lauth. Alex Smola contributed to the algorithmic style files.
